\part{Consequentialism}
\label{ch.mod5}
\addtocontents{toc}{\protect\mbox{}\protect\hrulefill\par}
\chapter{Part 13: Consequentialism}
So far, we have covered 3 different theories about the morality of actions, these were Natural Law Theory, Divine Command Theory, and Ethical Egoism. For normative ethical theories, there are, in general, two camps. There are the Consequentialists, who hold that the consequences determine whether an action is right or wrong. On the other side, there are the Non-Consequentialists. These guys hold that the consequences don’t matter. In general, cultures tend to follow one of these two ideas when it comes to morality, though in practice, it becomes a bit mixed.

\section{The main command of Consequentialism}

This is the idea behind consequentialism:
\begin{center}
Do all the good you can, by all the means you can, in all the ways you can, at all the times you can, as long as you can.
\end{center}
This stems from the notion that we are here on earth to do good. Our moral duty is to leave the world better than how we found it. But, we shouldn’t be ethical egoists about it, our moral duty extends beyond ourselves. We have a duty to help others too. G.E. Moore (the guy who wrote an early version of Porky) (1873-1958) declared that what is right is whatever produces the most good. If you have the choice between two options, the one which will cause the most good is the one you ought to go with.
\begin{center}
An action is moral IFF it causes the most good of your options.
\end{center}
Moore, and others, have thought that this is too obvious to deny. I mean, shouldn’t I always take the ‘goodest’ option? 

Consequentialism, itself, is very basic. You can think of it as a stance about what the correct moral theory will look like. First, this stance does not define what the `good' is, rather it just says we need to cause the most of it (whatever it may be). Second, this stance does not tell us how we should promote this good, whatever it may be. These two points come together to make two questions which ever Consequentialist theory needs to answer:

\begin{enumerate}
\item What is the good?
\item How should one promote that?
\end{enumerate}

As I hinted at in the onset, Consequentialism is a family of theories which all share the primary command that we should cause the most good we can. The theories in this family differ, however, in how they answer these two questions. In response to the first question, some theories say that happiness is good (Utilitarianism, which we will cover in this module), others say that family and societal relations are good\footnote{And that the good in them doesn't reduce down to some other more primative good.}, others still might claim that having one's desires satisfied is good. How these theories respond to the second question leads to even more diversity in the family. In general, there are two types of responses to this second question. A theory could say that we need to promote the good \emph{directly} with every action we take. This is to say that the morality of the action is determined by how much good it causes, no other factors are really considered. We call theories which answer in this way \emph{Direct Consequentialist}.\footnote{We will see a Direct Consequentialist theory in this module (Act-Utilitarianism).} Because of the simplicity of these theories, some Philosophers have taken to calling them \emph{Na\"ive}. Other theories seem to take on a more sophisticated approach to this second question. Rather than saying that we are to directly promote the good, these other theories say that we are to \emph{indirectly} promote the good. We call these theories \emph{Indirect Consequentialist}. One can promote the good indirectly in multiple different ways: By following a rule which, when followed, tends to produce the best outcomes;\footnote{this is Rule Consequentialism, a sketch for which is found in this module} by acting on certain character traits which tend to be benevolent or good-producing;\footnote{This is Virtue Consequentialism, a sketch for this one is found in the Virtue Ethics module} or by having an overarching pattern or code in one's life which tends to promote the goods.\footnote{This is sometimes called Dao Consequentialism, Way Consequentialism, or even Pattern Consequentialism. Mohist Consequentialism, which is covered in this module, follows this line of thought.} Because of the wide array of different responses one can take to these questions, Consequentialism tends to be the most vibrant and inclusive area of research in Ethics.\footnote{For example, there is Feminist Consequentialism, sketched in the Feminist Ethics module, which is, in my opinion, a very plausible theory combining all of the best the Indirect Consequentialist theories have to offer.}   

\section{Consequentialism At Work: The Death Penalty and The Trolley Problem}
Despite having numerous different forms, we can give a basic outline of the way in which Consequentialism chooses which action in the morally right one. The different answers to those first two questions determine the modifications which the individual theory makes to this overarching system. That said, here are the five steps: 

\begin{enumerate}
    \item Figure out what’s intrinsically good.
    \item Figure out what’s intrinsically bad.
    \item Survey your options, what can you do right now.
    \item For each option, take the good caused and subtract the bad.
    \item Go with the one which has the highest amount.
\end{enumerate}

The first two steps come directly from the first question. The consequentialist theories should have an answer to them and then we, when applying the theory, just gloss over them and move on to the third step. The last three steps flow from the second question. They relate to how we actually calculate or figure out the action we should take to promote the good in practice. Direct Consequentialists will leave these steps the same, while the Indirect Consequentialists will modify the process by better fitting it into the method they describe in how we should promote the good. 

These five steps, though simple, can be applied and are applied in various situations, especially high-stakes ones. For example, there is always a lively debate over whether we should have the death penalty. The Consequentialist methods can be used to give reasons to support either side. The Direct Consequentialist would \emph{not} propose an overarching ban on the death penalty, but rather would look at each case on a case-by-case basis. They ask, for each case, what is the best means to promote the goods overall. In some cases, this might be to absolutely remove the person from the world (through the death penalty) and in other cases, this could be to have forms of rehab and job-training within the prison system. The Indirect Consequentialist, on the other hand, would ask about the precedent which having the death penalty sets. Would having set trends which promote the good? Is having it acting on benevolent traits which promote the good? The answer to these questions depends on what we take the good to be but the theories will clearly provide advice and direction moving forward. 

In other cases, Consequentialism can easily be applied without too many modifications. Let's apply this to a very common and simple case, The Trolley Problem:

\factoidbox{\noindent  \fontsize{20pt}{0pt}\textbf{The Trolley Problem}

There is a run away trolley. You are at a switch which will change the course of the trolley. If you pull the switch, then you have essentially killed one person. If, on the other hand, you do nothing, then the trolley will run over and kill 5 people. What should you do?}

The consequentialist says to pull the lever, and their reasoning goes something like this:
\begin{earg}
    \item[1] Figure out what’s intrinsically good. (Human Lives (this is an example))
    \item[2] Figure out what’s intrinsically bad. (Dead Humans (this is an example))
    \item[3] Survey your options, what can you do right now. (Do nothing, Pull the lever)
    \item[4] For each option, take the good caused and subtract the bad. (Do nothing = -4 human lives, pull the lever = +4 human lives)
    \item[5] Go with the one with the highest amount. (Pull the lever)
\end{earg}

We are assuming here that the theory we are using has Human Lives as an intrinsic good which we are to promote. In our actions, we need to save as many as possible. In refusing to pull the lever, we are, through our inaction, causing 5 people to die while saving one. This comes out to a net loss of 4 lives. In pulling the lever, on the other hand, we are savng 5 lives while causing one to die. This is a net gain of 4 lives. A net gain is better than a net loss, so Consequentialism says to pull the lever. 

\section{Another Example and a Potential Worry} 
This na\"ive form of Consequentialism is not always sunshine and rainbows, however, in the correct situations, it can justify certain actions which really turn people's stomachs. For example,


\factoidbox{\noindent  \fontsize{20pt}{0pt}\textbf{The Origami Frog Killer}

    In the city of Light, a serial-killer has been on the loose. This person is particularly gruesome, leaving a distinctive calling card, on the mangled body of their victim, on top of what once their chest, the killer leaves an origami frog. Your team has been chasing this, as the papers have been calling them, ‘origami frog killer’ for months and have finally located them. You and your team bust in to find that the killer has committed suicide and she left a note mocking you. If that news comes out, people will be very angry and there will be riots, more people may do similar (because they can get away with it by killing themselves before getting caught) etc. There is a homeless man in the cell, who you can frame for the crimes and make believe that he did it. Your team is very loyal to you and know that if you go down, they go down with you. Doing a public execution of the man will deter many from committing similar crimes, restore faith in the justice system for catching this person, give some closure to the families who lost loved ones, etc.

Do you frame him?}

We now apply the five steps which were outlined to this case to determine whether the theory will tell you to frame the person in the cell as the Origami Frog Killer. In the first step, we need to determine the goods. Some of these are obvious; lives, closure (the end of grief), societal order, and others. The second step tells us to figure out the bads. These too can be obvious; deaths, continued grief, chaos, and others. Now that we have these, what are our options? Well, we only have two, frame or not. In framing, we \emph{directly} promote the goods while in not, we are not promoting those goods. This means that this form of Consequentialism would have us frame the person in the cell. 

\begin{earg}
    \item[1] Figure out what’s intrinsically good. (Human Lives, Closure, Order (maybe, these are examples))
    \item[2] Figure out what’s intrinsically bad. (Dead Humans, Continued Suffering, Chaos)
    \item[3] Survey your options, what can you do right now. (Frame, Don’t Frame)
    \item[4] For each option, take the good caused and subtract the bad. (Frame = lives saved, closure, order restored. Don’t Frame = continued suffering, riots (chaos), more dead people)
    \item[5] Go with the one which has the highest total. (Frame)
\end{earg}

Most of us will likely think that this is incorrect. It's wrong to frame someone for a crime. We could even argue that it sets a precedent which is harmful in the long run. This gives us a very basic argument against consequentialism: 
\subsection{The Origami Frog Killer Argument}

\begin{earg}
    \item[1] If consequentialism is correct, it is morally right to frame the homeless man.
    \item[2] It is not morally right to frame the homeless man.
    \item[3] Therefore, consequentialism is not correct
\end{earg}

This argument is far from conclusive. Consequentialists are constantly exploring the different reasons why the stance comes to this conclusion and modifying the steps accordingly and some theories can completely avoid this worry while others can't. In general, the Direct Consequentialist theories have a far harder time with this than the Indirect theories. 

\chapter{Part 14: Utilitarianism}

As I have mentioned, Consequentialism is very basic. It’s not all that complex and only has a roadmap for the general ways one could develop a completed theory of morality. The basic idea behind consequentialism is that we should max out the good. There are many, many different theories which fall into this family. As I mentioned, also, some can get out of the Origami Frog Killer case, other’s can’t. One classical Consequentialist theory is called Utilitarianism.

As you may recall, there are two questions which the Consequentialist theory needs to answer, what's the good? And how should I promote it. Utilitarianism answers the first question by taking notes from Hedonism and says that happiness is the good. Utilitarian thinkers have traditionally understood happiness in terms of pleasure and the absence of pain. This simple form of Consequentialism also answers the second question in the most simple way, by saying that our actions must \emph{directly} promote happiness, every action we take must max out the happiness and minimize the suffering. Since this combination is so simple,  the general ideas behind this theory are found the world over, for example, you can find this theory being discussed and spread in Ancient Greece. It was first really formalized by Jeremy Bentham (for a fun fact about him, look up the Auto-Icon) and then was further developed by John Stuart Mill, who is its best known advocate, and Harriot Taylor Mill.\autocite{Driver3} J.S. Mill, characterizes utilitarianism as the view that “an action is right in-so-far as it tends to produce pleasure and the absence of pain.” Mill, also, has a fascinating life story as well, if you care to look into it.  If happiness, conceived of as pleasure and the absence of pain, is the one thing that has positive value and pain/sadness being pleasure's opposite (the one thing with negative value), then this criterion of right action should follow fairly quickly. This leads us to \emph{the Principle of Utility.}

In any given scenario, the actions we make in that scenario will have consequences. We assign those consequences value based on the amount of happiness (pleasure) caused and the amount of sadness (pain) caused, for all beings affected, that action's utility. The utility of an action is the net total of pleasure caused by the action minus any pain caused by that action. In calculating the utility of an action, we need to consider all of the effects of the action, both long run and short run. Given the utilities of all available courses of action, utilitarianism says that the correct course of action is the one that has the greatest utility. So an action is right if it produces the greatest net total of pleasure over pain of any available alternative action. Note that sometimes no possible course of action will produce more pleasure than pain. This is not a problem for utilitarianism as we’ve formulated it. Utilitarianism will simply require us to pursue the lesser evil. The action with the highest utility can still have negative utility.

There are a few things which are worth noting about Utilitarianism. First, this has no room for self-interest bias. You are not special in the moral figuring. To use this theory, you need to count your own wants and desires equally to all other beings who could be affected.\footnote{This kind of impartiality as been rather revolutionary in the areas where Utilitarianism has sprung up.} For example, suppose that I really want a million dollars, so I rob a bank. If I only counted myself in the equation, then it would tell me that this is the right course of action (assuming I don't get caught). However, there are other people and I am not, as it were, an island. The grief and suffering caused to others by my action would greatly outweigh the benefit to myself, and that makes the action wrong.

Second, this theory only requires that you cause the most good (once the bad is subtracted). This good does not need to be evenly distributed. For example, suppose that a government has a surplus and the president wants to use that money to help the citizens. The Pres. has a couple of choices. She could distribute the money in the form of gas-cards to the top 90\% of the population. This would cause some happiness for the majority. The other option is to use the money to help fight starvation, provide education/work training, and otherwise help the bottom 10\%. While the first option causes good things for the most people, in the grand-scheme of things, dedicating the money to the bottom 10\% is the better option because the suffering removed is greater than the pleasure which would have been added. 

Third, this theory does not make a distinction between long term and short term consequences. This makes sense most of the time, but it could be an issue for the theory when it comes to practical use. For example, allowing your child to eat all of the candy and junk food they want will, likely, maximize their happiness in the short term. In the long-term, however, there are a bunch of different health problems and behavioral issues which arise from that sort of diet, making it wrong to feed a kid a ton of junk food.

Act-Utilitarianism AU is one version of consequentialism and it is the most simple and has its origins in Hedonism/ethical egoism. Like hedonism, it takes well-being/happiness as the core good in the world. Like egoism,  it takes that we should take the action which maxes it out. However, the point isn’t to max it out for just you, but for the world in general.AU tends to use as units (these are just useful tools, in real life, it gets more complicated) ‘hedons’ and ‘dolors’. ‘Hedon’ is from Greek meaning pleasure and ‘dolor’ is from Latin meaning pain. So, you take the number of hedons, subtract the dolors, and then you go with the one with the highest. NOTE: This is for all people affected, not just you. It does not always require us to cause the most pleasure period, sometimes the action with the most hedons is also the one with the most dolors. It does not require us to distribute the pleasure evenly. Sometimes, giving one person all the pleasure is the way to go.  Here is an example of this general idea at work. Remember that Act-Utilitarianism is, essentially, consequentialism with a hedonism bend.
\factoidbox{
    The US has a massive surplus, Pres. Bill, to help the people, has two options. First, he could give 90\% of the population a \$50 gas coupon. Second, he could use that money to fight starvation among the poorest 10\%. The poor will more than likely benefit more from the money being spent that way than the 90\% would benefit were it spent the other way.
}
So, using Act-Utilitarianism, we find that the money would best be spent, not on the majority of people, giving them a gas coupon, but rather on the poorest 10\%. This is because the benefit in helping that small group a ton is greater than the benefit of helping 90\% of people just a little bit.

\section{Reasons to like this theory}
\subsection{Reason 1: Impartiality}

Utilitarianism in almost all its forms (since there are many, it is hard to say all) and consequentialism in general (same reason) have a doctrine of impartiality. It tells us that the welfare of all people (and sometimes animals) is equally morally valuable. Whether you are rich, poor, black, white, male, female, man woman, part of some religious sect, or none at all, it doesn’t matter. Sometimes we take this sort of equality for granted, but when this theory was being formulated, it was not the norm, and even today, equality of this sort is fought for. You don’t need to argue for it in utilitarianism, it comes built in.

\subsection{Reason 2: Justification}

Utilitarians are no strangers to being controversial. Classical Utilitarians have been on the front lines of many periods of history: the abolishing of slavery (Bentham, 1748-1832), equal rights for women (John Stuart Mill, 1806-1873), and animal rights (Peter Singer 1946-present) to name a few. That said, utilitarians can give reasons why many of our deeply held moral beliefs are correct. They see this as a major plus for the view. For example, many of the things which we find repugnant morally, slavery, killing innocent people for no reason, and others, all tend to do more harm than good. At the same time, things we have strong moral feelings in favor of, helping others, telling the truth, bravery, and others, all tend to make the best outcomes. The tendencies for both of these line up with the utilitarian mindset.
\subsection{Reason 3: Conflict Resolution}

A really important feature in a theory is that it can provide practical advice about what to do in a given situation and how to resolve conflicts. When you are in a situation where you are not sure what to do, the utilitarian can apply their theory and tell you what the best option is, there’s something in the world which they can point to, the outcomes, and say which one is the best. When two groups or people disagree about the morality of something or what they should do, the utilitarian can point to some base level facts and say what the right way to go is. So long as everyone agrees about the facts, the utilitarian can settle disputes.

\subsection{Reason 4: Flexibility}

As I mentioned in the part about justification, certain rules tend to produce the best outcomes, so the utilitarian supports them. But, those rules are not absolutes, they are there because they tend to get the best result. Sometimes, we need to break those rules in extreme situations. Utilitarians are just fine with this and can easily give the reasons why those rules should be broken. Its flexibility lets it work in most any situation. Take this example:
\factoidbox{\noindent  \fontsize{20pt}{0pt}\textbf{The Donner Party}

    In the winter between 1846-1847, members of the Donner Party, traveling west, found themselves in heavy mountain snows. About half of the 87 members of the party died after food and supplies ran out. Those left had a terrible choice: Starve to Death or Eat the Remains of their fellows.}

Some may think that the rules against cannibalism are absolutes. Not to be violated under any conditions. The utilitarian disagrees. This disagreement is not based on cannibalism, but rather that no moral rules (other than maxing the good while minimizing the bad) are absolutes. While it’s wonderful that many of us are against these sorts of actions, utilitarians understand that desperate times call for desperate measures.

\subsection{Reason 5: Moral Community}

Like I mentioned before, utilitarians are the most likely to support rights and equalities for all people, but also, they can account for animal rights and well-being. A special term, which we will encounter again, is ‘moral community’. This is the set or group of things which are morally important all on their own. Most limit these things to people. The utilitarian has no problem extending this to animals. Bentham stated the following: “The question is not ‘can they reason?’ Nor ‘can they talk?’ But ‘can they suffer?’” (I fixed the grammar to fit our modern ways).

Tables and chairs can’t suffer nor feel joy, so they don’t count in our moral community, but animals are a different story. They do suffer and so we can say that torturing a little puppy is horrible because it cause more suffering than anything else.

\section{Problems for Utilitarianism}
With all of these positives, one might think that Utilitarianism is the way to go. Personally, as a utilitarian myself, I tend to agree, but there are some problems which keep even me up at night.
\subsection{Problem 1: Measuring Well-Being}

Remember that for the steps to apply this theory, we need to add up the good caused. This means that well-being would need to be in units, something I can measure. But this is completely implausible. This gives us this argument:
\begin{earg}
    \item[1] If Utilitarianism is true, then there’s a precise unit of measure the outcomes of our actions.
    \item[2] There is no such measurement.
    \item[3] Therefore, Utilitarianism can’t be true.
\end{earg}
Most utilitarians don’t fight back against the second line, though there may be someone out there, smarter than me who does. Most utilitarians argue against the first line. They say that yep, it’s true, but we don’t need a measurement like you claim. As I had mentioned when it comes to the hedons and dolors, these are not actual units of measure, with a precise value. Rather they are just useful tools to help conceptualize the situations, remove the fluff. In cases where it’s obvious which one is the better, they are totally irrelevant, just go with that one. Otherwise, in cases which are harder, we can use them to make sort of equalities between different outcomes/results. But even then, without a clear cut unit to measure things out, sometimes it still will be unclear. 
\subsection{Problem 2: It’s Demanding}

One of the issues with utilitarianism is that it’s very demanding of us in many different ways. First, we need to think about things, it requires a ton of mental power and effort to apply. The standard reply here is that we don’t, most of the time, need to do all of the math. We have personal experience and knowledge of history to rely on. We have the wisdom of those who came before us to help. This gets us simple ways to handle every day situations and then we have the more extreme cases, which we would need to think on. Second, we have a problem of motivation. Remember what utilitarianism claims, the morally right one is the best one. If I am not getting the best results, I am not being moral. This means that I have to be a moral saint, all of the time. Third, we have our action. First, we need the mental power to use the theory. Then we need to always be the best saint ever. And now, I actually need to do it, which is extremely demanding, if I don’t, I’m a moral monster. 

The reply to this is to, well, bite the bullet. Morality, most of the time, does not require much from us, but sometimes it absolutely requires a ton from us. Many of us think that we have a moral obligation to make the world a better place. This is a demanding thing, and one which should require a ton from us.
\subsection{Problem 3: Impartiality}

This was initially a good thing, but it can be seen equally, as a bad thing. Morality, it seems intuitive, may require us to be bias in some ways. For example, I could spend my time in a soup-kitchen helping the poor or I could spend my time caring for my children. If morality were always impartial, I would have more of a duty to the homeless than my children. But that doesn’t seem right. A person needing to chose between their lover and two strangers seems right in saving her lover, but that’s also bias. This sort of bias is very explicitly built in to the Feminist Ethics which we will cover in this class.

The response to this is to claim that this bias is a mental bias and that we should be impartial in our ethical concerns. 
\subsection{Problem 4: No Intrinsic Wrongness or Rightness}

Some actions/situations we want to say are always morally wrong, period, forever. This is a distinction between act-tokens and act-types. Act-tokens are individual instances of an action. Act-types, on the other hand, are the classes or categories of actions. Utilitarianism, on its own, does not handle act-types, only act-tokens. Some want to that certain act-tyoes are always  good, period, forever. Some act-types we want to say are always morally wrong, period, forever. Utilitarianism can’t account for that. Remember with the Donner Party? The utilitarian was able to explain why it’s good that we consistently follow certain rules but also can/does give exceptions to those rules. Any thing, aside from the core principle of the theory, can be made OK to do, if the situation is right.
\subsection{Problem 5: The Injustice Problem}

This is an insanely hard problem and one which can keep me up at night. It is put pretty simply: Utilitarianism says to max out my actions, but sometimes doing this leads to some serious problems of injustice. Take this examples: Vicarious Punishment: This can backfire sometimes, but sometimes it is very effective. To prevent terrorists or enemy combatants, you punish/torture/kill their relatives, the people who shelter them, or other innocent people who they care about. These people are innocent. Justice does not allow for us to punish innocent people. But there we are.
\begin{earg}
    \item[1] The correct moral theory would never allow us to commit injustice.
    \item[2] Utilitarianism sometimes allows/commands us to commit injustice.
    \item[3] Therefore, Utilitarianism just can’t be right.
\end{earg}
\subsubsection{Reply 1: Justice is also intrinsically valuable}

This is a bit odd for a utilitarian to claim. Basically, for that theory, only the outcomes matter, as measured by happiness. So, injustice seems to follow, under certain circumstances. These utilitarians claim that we need to include justice into our measures and figure out how the two compare. Some say we ought to favor justice over pleasure, saying that no amount of pleasure would ever outweigh injustice. Others say that there are times where it can, but those are so rare and extreme that they align with our intuitions.
\subsubsection{Reply 2: Injustice is never gets the best outcomes}

This one too seems odd for the utilitarian to claim. Take the Origami Frog Killer case, it seems like that that one gets us the best while violating justice. The general idea that these people go with is that some injustice may work in the short term, but in the long term it comes out as worse than otherwise. Setting moral precedents which can be used as short cuts which lead to worse outcomes. But this too doesn’t really line up with all cases of injustice having the best outcome in the short run. EG Robin Hood
\subsubsection{Reply 3: Sometimes Justice must be sacrificed}

This one is the last one which we will cover today and the utilitarians here are basically saying ‘nope, sorry, but sometimes the correct moral theory will command us to commit an injustice or two.’ They are basically biting the bullet. We should be just as much as possible, because this tends to be the best choice. But, there are times where we just have to say ‘it sucks, but this is the way it needs to be’.

\subsection{Porky The Pig Farmer}

When we speak of utility as pleasure and the absence of pain, we need to take “pleasure” and “pain” in the broadest sense possible. There are social, intellectual and aesthetic pleasures to consider as well as sensual pleasures. Recognizing this is important to answering what Mill calls the “doctrine of swine” objection to Utilitarianism. This objection takes the Utilitarianism to be unfit for humans because it recognizes no higher purpose to life than the mere pursuit of pleasure. This objection only applies to treating pleasure and pain in their most basic, animalistic, senses.  This was not originally mine, but a version of it was given as a two day lecture when I was in community college. The very original version is not as fun to give and it was published by G.E. Moore in 1903. If you want to see one other objection, one which you could write about in the assignment as well as the original objection to hedonism (which is a more primitive form of utilitarianism, without special amendments, utilitarianism falls into both of these problems):
\factoidbox{\noindent  \fontsize{20pt}{0pt}\textbf{Porky the Pig Farmer}

    Way deep in the back woods, there lives a pig farmer named Porky. Porky raises prize winning swine and sells them to cover his basic needs. He has no wife/husband and doesn’t really want one. One day, when he was particularly bored, he noticed his swine having fun wallowing in the mud. Thinking that this looked like a good time, he stripped down and jumped in, having a whale of a time. As time goes on, Porky needs more entertainment than merely wallowing with his sows. He starts thinking “man, that’s a pretty little piggy”. Over time, late in the evening, his neighbors start hearing strange sounds coming from the direction of Porky’s shack. They think nothing of it, maybe Porky is just doing some late night breeding to get better piggies for market. Porky is doing some late night breeding of a sort. He is engaged in bestiality!! And O! Is he enjoying himself!} 

With all this information about how awesome and pleasurable Porky's life is with his piggy time, we now have to ask "is what he is doing moral?" Well, according to Utilitarianism, if we take pleasure to be in the animalistic sense, it is.
\begin{earg}
    \item[1] The morally right action is the one which produces the greatest amount of happiness/pleasure for the greatest number of people.
    \item[2] If Porky doesn’t engage with his pigs in this way, he will have very few pleasures.
    \item[3] If Porky does engage with his pigs in this way, he has a lot of pleasures.
    \item[4] If he has a lot of pleasures, then the greatest amount of happiness over the greatest number will be served.
    \item[5] Therefore, the morally right action is to engage with his pigs in this way.
\end{earg}
But this can't be right. The core of it is that there is no consent (among other things), which means that it can't be right.
\begin{earg}
    \item[1] If Utilitarianism is correct, Porky’s actions would be morally permissible.
    \item[2] Porky’s actions are not morally permissible.
    \item[3] Therefore, Utilitarianism is not correct.
\end{earg}
\subsubsection{The Reply to Porky}

Because of this objection, Mill and others don't take pleasure in the animalistic sense, but rather in a far more broad sense, where the other intellectual and emotional pleasures are taken into account. Mill responds that it is the person who raises this objection that portrays human nature in a degrading light, not the utilitarian theory of right action. People are capable of pleasures beyond mere sensual indulgences and the utilitarian theory concerns these as well. Mill then argues that social and intellectual pleasures are of an intrinsically higher quality than sensual pleasure. This response seems OK to some, but others argue that a sufficient amount of physical pleasures can, in principle, outweigh the intellectual. 

\subsection{The Utility Monster Objection}

One objection to Utilitarianism isn't concerned with what it measures to determine morality or even how it is measured, rather this objection concerned how it determines the morally right action given the outcomes. As Act-Utilitarianism sits right now, it claims that the morally right action is the one with the highest outcome given the available options. This, however, leads to the possibility of a utility monster. A utility monster is a being which receives a massive quantity of pleasure (happiness, the good) from consuming resources, higher than any other being, by a significant margin, often at the expense of others. Just a cursory glance over the numbers would have Act-Utilitarianism claim that such a utility monster is doing the right thing in exploiting or harming others for their own benefit, because of the massive amount of good which they receive. To put this idea as an example, take the following case:

\factoidbox{\noindent  \fontsize{20pt}{0pt}\textbf{Singer's Pipes}
    Peter Singer, late in the evening, while he is resting at home, hears loud banging and commotion coming from his basement. So, he grabs his flashlight and goes down to investigate. He sees a green, round, being with arms and legs tearing apart Singer's plumbing causing massive flooding. "What are you?" Singer exclaims, "and what are you doing to my pipes?" The creature pauses their destruction for a moment and turns to Singer. "I am destroying your plumbing" they explain, "you see, I love wrecking pipes, far more than any suffering caused to you. I am a utility monster."}

In this case, most of us would say that it's wrong for the utility monster to destroy Singer's pipes. There are concerns about personal property and there are concerns about the harm done to Singer.\footnote{Peter Singer is an Australian philosopher and is best known for his, seemingly, extreme views regarding Utilitarianism. He holds that the theory is correct and applies it to many contemporary issues. For example, one of his foundations, The Life You Can Save, tracks the spending habits of various charities and connects donors with the one which will get the most 'bang for their buck' in the issue which they are concerned with. Singer, holding true to this belief, lives well below his means and donates substantial amounts to charity. He is mostly concerned with world hunger and poverty, but he has been known to be outspoken about animal 'rights' and welfare. } However, Act-Utilitarianism, without any modifications, measures pleasure and pain with the same metric. One unit of happiness cancels out one unit of sadness. As a result, so long as the Utility Monster didn't have any other options which would cause a higher total, Act-Utilitarianism would say that they did the right thing. This does not square with general intuitions about morality. 

Act-Utilitarianism gets the wrong result in this sort of case. And this objection is so basic, to the core of Act-Utilitarianism, that it might lead us to a change in theory, amend the theory to better fit our intuitions. One possible alteration is to change what is measured to determine morality. In this case, rather than measuring happiness and sadness equally, you only measure the sadness caused by the action or inaction. This is called negative utilitarianism. Negative Utilitarianism was proposed, most notably, by Karl Popper in his work On the Open Society and Its Enemies. Popper states that the morally right action is the one which minimizes suffering, rather than maximizing pleasure. Going this route avoids both the Utility Monster Objection as well as the Porky the Pig Problem but it also leads to certain other problems, if you take the letter, rather than the spirit, of the moral theory.  

\subsection{The Organ Harvest Problem}

This problem is often seen as a more gruesome version of the Trolley Problem for Ethics. This objection to Act-Utilitarianism stems from the idea that only the results matter, the ends justify the means. I personally have other versions of problems like this which involve framing an innocent person, forging evidence, and rigging elections, all of which have (due to the situation that they are in) the best consequences. Consider the following case:

\factoidbox{\noindent  \fontsize{20pt}{0pt}\textbf{A Simple Check-Up}

    Bob goes to the doctor for a check up. His doctor finds that Bob is in perfect health. And his doctor also finds that Bob is biologically compatible with six other patients she has who are all dying of various sorts of organ failure. Let’s assume that if Bob lives out his days he will live a typically good life, one that is pleasant to Bob and also brings happiness to his friends and family. But we will assume that Bob will not discover a cure for AIDs or bring about world peace. And let us make similar assumptions about the six people suffering from organ failure.} 

The question for the Act-Utilitarian is "what should the doctor do?" According to the theory, it seems that the good doctor should quickly and as painlessly as possible kill Bob and harvest his organs, getting them to the 6 other patients as quickly as possible. This is because, to quote Spock, "the needs of the many outweigh the needs of the few." The overall outcome of letting Bob go is the value of one average good life minus the values of six average good lives and the overall outcome of killing Bob is the value of six average good lives minus the value of one. But this can't be right. Our intuitions clearly speak to the immorality of this. 

\chapter{Part 15: Rule Consequentialism/Utilitarianism}
In response to these and other worries for Utilitarianism, many philosophers have opted to amend Consequentialism on a base-level. It still takes the consequences as what matters but indirectly.\footnote{As you may recall, this is a form of Indirect Consequentialism.} Rather than having the actions directly promote the good, some theories have it that they need to follow a rule which promotes the good. This stance is called Rule Consequentialism or RC. Rule Consequentialism, if it takes happiness to be the good, is called ‘Rule Utilitarianism’ from time to time. It promises to take on the problems, give reasoning for why they are problems, and even claim that our intuitions are right. That’s a pretty big goal!

This is the stance that an action is moral because it follows a social rule which if (nearly) everyone were to follow it, it would have the best results (out of competing rules).
\begin{center}
An action is moral IFF it follows an optimific social rule.
\end{center}
Yes, ‘optimific’ is spelled right. An optimific social rule is one which were (nearly) everyone to follow it, it would have the best results, compared to other potential social rules. The basic idea is that rather than thinking about morality of an action in terms of its consequences, think about it in terms of moral rules. These rules are there because they get the best outcomes. Thinking in terms of moral rules is a pretty common practice already and most moral theories operate in this way, just with different ways of making rules.

\section{How to Use It:}
When you are in a new situation and are not sure about what to do morally speaking, follow these steps:
\begin{earg}
    \item[1] Formulate a few rules which apply to your case, things like ‘don’t punish innocent people without a fair trial’.
    \item[2] Imagine what a society would look like if just about everyone followed that rule.
    \item[3] Ask your self whether the society is better than if they followed some other rule (I will give examples).
\end{earg}
Consequentialists differ on what makes a society better, some are hedonists (rule-utilitarians), some are desire-satisfaction theorists (the good is having desires satisfied (I find my self in this camp often)), and others still generate a list of goods and ways of comparing them. Basically, if you get the answer ‘yes’, then follow that rule.Here are some basic rules which one can follow: 

\begin{center}
Don’t punish people without a fair trial. 
\end{center}
This is because a society which does punish people without a fair trial is worse than otherwise. 
\begin{center}
If you are an instructor (teacher/professor), give students the grade they deserve rather than the grade they want. 
\end{center}
This is because the results otherwise, where a professor just hands out A’s is detrimental to the value of a degree and grades are measures of how well a student knows something. If the student gets a job which requires A-level knowledge, claims to have it because of this professor, but doesn’t actually have it, the results could be really bad. 
\begin{center}
Don’t steal, even when the items will not be missed or you are never caught. 
\end{center}
This is because a society where theft is rampant is worse than one where it’s not.

\section{Responses to the various worries:}
This theory, because of its method for formulating rules and because the goals for the rules is to make the world a better place, can respond to most, if not all of the problems which faced simple Act-Utilitarianism and Consequentialism. 

\subsection{Response to the Injustice Problem}

For the act-utilitarian and consequentialist in general, the goal is to take the action which is optimific, has the best results. For the rule-consequentialist, they are thinking more in terms of policies. They are saying follow the policy which leads to the best results over all people. This is because that, in the long run, following the policies lead to better results even if in isolated cases, the results aren’t the best. Even philosophers who disagree with the intrinsic value of justice are prone to claim that always following the rules about justice will lead to a better result overall.

\subsection{Response to the Impartiality Problem}

Along with giving an answer to the injustice problem, it also gives us a reason why we can be partial. For example, you have to chose between saving your SO and two strangers, who do you save? The rule which could be the one which gives us the best outcomes is the one which, under the right circumstances, tells us to pick our SO. Being loyal in this way to our partners, our children, and our friends is very beneficial to a society.

\subsection{Response to the Intrinsic Wrongness Problem}

We often have very deep moral problems with things like torture, even in cases where it could lead to the information leading to the best consequences (worth noting that there are far more effective ways than torture, I had videos, would need to find again). The rule would be something like ‘don’t torture’ period, and we can say that this rule is right because a society with it, in the long haul, is better than one without it.

\section{A Problem: Pete The Pervert}

There are very few philosophers, of any stripe, which actually like this theory. Why? Because it is basically ‘rule-worship’. It tells us to follow the rule even if, in that situation, it doesn’t have the best outcome. The Donner Party should not have eaten their fellows to survive because the general rule is ‘don’t eat people.’ Sometimes we are required, by this theory, to do things which leave the world worse than how we found it.

I generally just use this theory as an example all on its own, but here I will be using it as an example for a rule in RC. The golden rule states:
\begin{center}
Do unto others as you would have done unto you.
\end{center}
Many people think that this is the right way to think about ethics, and we could see that this rule, if followed by everyone, leads to the best results. But when we think about it, it implies some pretty nasty things.

\factoidbox{\noindent  \fontsize{20pt}{0pt}\textbf{Sally and Pete}

    Pete is a homeless person who likes to hang around this campus. He can’t remember the last time he bathed. The smell of Swisher Sweets clings to his clothes and his hair. Chunks of Wintergreen Long-Cut freely flow between the gaps in his mostly missing teeth. Pete’s favorite time of the year is the spring and summer, as the young ladies around campus wear more scanty attire. One day, a particular young lady catches his eye, Sally Student, who arrives early each day and leaves late in the evening.

    Pete has slowly become enamored with Sally, knowing her route each and every day. One morning, watching her, he realizes how much he would like to have her come up behind him, spin him around and plant a wet kiss on him, tongue and all. Remembering elementary school, he recalls the golden rule, and formulates a plan…

    That evening, as Sally is fumbling for her keys to get into her car, Pete rushes up behind her, spins her around, and Frenches her. The smell is nauseating, the chunks of long finished Swishers and unspit chew flow freely into the young girl’s mouth.}
    
The Pete The Pervert Argument
\begin{earg}
    \item[1] If RC were correct, then Pete’s actions would be morally permissible.
    \item[2] But, Pete’s actions weren’t morally permissible.
    \item[3] Therefore, the RC is not correct.
\end{earg}


There are a few questions which the Rule Consequentialists have about this case: First, is the Golden Rule an optimific one? There may be good reason to think that it is not, which would get them out of this worry. Another response to this problem might be to say that, while Pete is following the letter of the rule, he has fundamentally missed the spirit of the rule. That said, RC has other problems too. For example, why follow the rule when you know you are in an extra ordinary situation? Why not commit a small injustice when you know that not doing so will lead to an even greater one?

\chapter{Part 16: Mohist Consequentialism}


In `Ancient' China, during the Warring States period (between 479 BCE and 221 BCE), there was an influential philosophical, social, and political movement which we, today, call `Mohism' after its founder Mo Di.\footnote{Calling Mo Di the founder might be mischaracterizing things. Mohism originates from the writings and follows the teachings of Mo Di.} Mo Di is also called `Mozi' and `Master Mo'. In the Chinese Philosophical Tradition, Mozi and his followers were the first to engage in explicit argumentation and structured reasoning to search for objective moral standards. Even by today's standards, their reasoning is still well done. They also formulated China's first ethical and political theories, advancing the earliest form of Consequentialism. One might expect this consequentialism to be simplistic, like Act-Utilitarianism, but, in fact, this form of Consequentialism is remarkably sophisticated.

Using critical thinking and philosophical argumentation, the Mohists also developed sophisticated theories of knowledge, language, ontology,\footnote{Ontology, in Philosophy, is the study of what exists and what conditions does it take for a thing to exist. The Mohists also engaged in a branch of this subject called `Mereology' which is primarily concerned with composition, as in, under what conditions to two or more smaller objects `come together' to make a larger one.} and even branched out into economics, geometry, optics, and mechanics.\autocite{sep-mohism}

\section{The Mohist Doctrines}

Entire libraries could be filled with books and entire courses taught explaining and exploring the ideas and accomplishments which this school of thought achieved. For this text, however, we must focus solely on the ethical theory which they developed. As you should recall, all Consequentialist theories have the core command to promote the good\footnote{Leave the world a better place} and from this command, there are two questions which the theories must answer; to reiterate':

\begin{enumerate}
\item What is the good?
\item How should one promote that?
\end{enumerate}

We know how the Mohists answer these questions because, as a political movement, they offered ten core doctrines (collections of short books and essays), divided into five pairs. For all but Heaven's Intent, the Mohists gave sophisticated Consequentialist-style arguments and for that one, they gave arguments for their form of Consequentialism. These doctrines are:

\begin{enumerate}
\item \textbf{Promote the Worthy} and \textbf{Exalting Unity}
\item \textbf{Inclusive Care} and \textbf{Condemning Aggression}
\item \textbf{Moderation in Use} and \textbf{Moderation in Burial}
\item \textbf{Heaven's Intent} and \textbf{Understanding Ghosts}
\item \textbf{Condemning Music} and \textbf{Condemning Fatalism}
\end{enumerate}

The answers they give for the two fundamental questions are scattered throughout these texts. Interestingly, when they were seeking to advocate for their ideals with political leaders, different doctrines would be emphasized over others, depending on the state of, well, the state. 

\factoidbox{
 If the country is disordered and confused, then one speaks about
[promoting the worthy] and [exalting unity]: If the country is poor,
then one speaks about moderation in use and moderation in
[burials]. If the country has a liking for music and depravity, then
one speaks about [condemning music] and [condemning fate]. If
the country has fallen into licentiousness and lacks propriety, then
one speaks about honouring Heaven and serving ghosts. If the
country is dedicated to invasion and oppression, then one speaks
about [inclusive care] and condemning aggression. Therefore I say,
pick out what is fundamental and devote one’s attention to it.
\autocite[49.14]{Johnston2}}

Fundamentally, the Mohists saw themselves as a moral advocacy group devoted to realizing the kind of morally right society which would promote the benefit and eleminate harm across the world. After we extract the relevant aspects of their thoughts, we will return to how they answer the two questions. 

\subsection{Promote the Worthy ad Exalting Unity}

Promote the Worthy and Exalting Unity\footnote{this is also translated as `Identify Upward'} concern the structure of the society which they seek to develop. This has two fundamental aspects, which are addressed in the books. In Exalting Unity, the Mohists give an argument claiming that the good society must be one with a unified sense of morality, right and wrong. As they claim, having different senses of right and wrong will lead to internal conflict and the break down of society, disorder and chaos.\footnote{It should be noted that Mozi's comments here are a pretty blunt rejection of moral relativism as being destructive to a society.} As Mozi put it: 

\factoidbox{Master Mo Zi spoke, saying: “Ancient times, when people first
came into being, were times when there were as yet no laws or
government, so it was said that people had differing principles.
This meant that, if there was one person, there was one principle;
if there were two people, there were two principles; and if there
were ten people, there were ten principles. The more people there
were, the more things there were that were spoken of as principles.
This was a case of people affirming their own principles and
condemning those of other people. The consequence of this was
mutual condemnation. In this way, within a household, fathers and
sons, and older and younger brothers were resentful and hostile,
separated and dispersed, and unable to reach agreement and
accord with each other. Throughout the world, people all used
water and fire, and poisons and potions to injure and harm one
another. As a result, those with strength to spare did not use it to
help each other in their work, surplus goods rotted and decayed
and were not used for mutual distribution, and good doctrines
were hidden and obscured and not used for mutual teaching. So
the world was in a state of disorder comparable to that amongst
birds and beasts.”\autocite[11.1]{Johnston2}}

According to Mohism, the best way to do this is to have the population `identify upwards', that is, to look at the good examples that their superiors in society have set and to have those superiors reward virtuousness and punish viciousness. This is closely related to the content in Promote the Worthy. This book gives arguments saying that the rank in society one has should not be related to one's caste, origin, or family, rather it should be determined by their worthiness, their qualifications and virtues. The leader of the state is the most virtuous person and then the government is structured with a hierarchy of officials, each appointed according to their qualifications and moral merit. Remember, the social status of one's family does not play a part in this. If you are worthy, you are promoted. This inclusion and rejection of discrimination was a very revolutionary idea and one which is foundational to almost all Consequentialist theories. 

\subsection{Inclusive Care}

In Inclusive Care, the Mohist argue that all of great harms in the world all stem from a common source; bias and discrimination.\footnote{More particularly, having greater concern for those close to you or like you than you have for those distant from you or unlike you} The world which the Mohists invision is one where all people must display a virtue of benevelence. This, according to Mohism, is inclusively caring about all people.\footnote{``jian'ai'' is the term used for `inclusive care' by the Mohists, it is often also translated as `universal love', such as by the translator used in this textbook, or as `impartial concern'.} You must have concern for the welfare of all people and have just as much concern for their welfare as you do for your own. 

\factoidbox{
Master Mo Zi said: “ ‘Universal’ is the means of changing
‘discriminating’.” If this is the case, how can ‘universal’ change
‘discriminating’? [He] said: “If people were to regard others’
states as they regard their own state, then who would still mobilise
their own state to attack the states of others? \ldots If people were to regard
the capital cities of others as they regard their own capital city,
then who would still raise their own capital city to strike at the
capital cities of others? \ldots If people were to regard the houses of
others as they regard their own house, who would still stir their
own house to bring disorder to the houses of others? \ldots Now if states and
cities did not attack and strike at each other, and if people’s houses
did not bring disorder to and damage each other, would this be
harmful to the world? Or would it be beneficial to the world? This
must be said to be beneficial to the world. For the moment let us
think about the origin of these many benefits, what it is they arise
from. And what is this from which they arise? Is it from hating
people and harming people that they arise? We must certainly say
it is not. We must say that it is from loving people and benefiting
people that they arise. And, if we were to distinguish and name
those in the world who love people and benefit people, would it be
as ‘discriminating’ or as ‘universal’? We must certainly say it
would be as ‘universal’. In this case, then, it is ‘mutual and
universal’ which gives rise to the world’s great benefits.”\autocite[16.2]{Johnston2}}

\factoidbox{
It was for this reason that Master Mo Zi said: “‘Universal’ is right.
Moreover, as I originally said, the business of the benevolent man
must be to seek diligently to promote the world’s benefits and
eliminate the world’s harms. Now I [have established] what
‘universal’ gives rise to — it is the world’s great benefits. And
I [have established] what ‘discriminating’ gives rise to — it is
the world’s great harms.” This is why Master Mo Zi said:
“ ‘Discriminating’ being wrong and ‘universal’ being right comes
from this principle.” \autocite[16.3]{Johnston2}}

These passages point to a very basic idea; if people cared about each other, then they would have concern for their well-being and this will prevent the great harms in the world because people would not attack or harm others just as much as they would not attack or harm themselves. In your interactions with others, you must act on that concern for their well-being and promote their benefit. One must be careful, however, in how one understands this passage and the Modists' commitment to Inclusive Care. This \emph{does not} mean that you have an equal moral duty to all people. All it means is that you should have an equal level of concern for the well-being of all people. The Mohists recognized, as opposed to the na\"ive Utilitarians, that social relationships come with special obligations. Book 16 of the Mohists' texts is full of examples of how one must fulfil these special obligations to others because of their relationship with them.\autocite{BackYoungsun1} For example: 

\factoidbox{Master Mo Zi spoke, saying: “The business of the benevolent man
must be to seek assiduously to promote the world’s benefits and to
eliminate the world’s harms.” This being so, of the world’s harms
what, at the present time, are the greatest? [Master Mo Zi] said:
“They are great states attacking small states, great houses bringing
disorder to small houses, the strong plundering the weak, the many
ill-treating the few, the cunning scheming against the foolish, and
the noble being arrogant towards the lowly. These are the world’s
harms. Also, it is rulers not being kind, ministers not being loyal,
fathers not being compassionate and sons not being filial. These
too are the world’s harms. \autocite[16.1]{Johnston2}}

Notice that Mozi listed many different actions which harm the world, from his perspective, and the first few, the strong harrassing the weak, can all be explained as harms because they flow from a lack of inclusive care. Having this `inclusive care' would make you see people the same as those close to you and would prevent you from taking advantage of them in this way. The last few listed, however, are different. Rulers are to be kind to \emph{their} subjects, fathers compassionate to \emph{their} sons, sons filial to \emph{their} parents. These are traits which do flow from inclusive care but entail a \emph{greater} duty to the person you have the relationship with. Caring about another person forces you to recognize the duties that you have towards them, the special obligations which you have because of their relationship with you. 

Closely tied to this, in Condemning Aggression, the Mohists argue that we need to be opposed to millitary agression and seek peace whenever possible. At the same time, Mohism is very much opposed to military aggression. While it is true that some Mohists did end up in the military for their states, as they rose through the ranks, they assumed exclusively defensive attitudes. Aggression is wrong for the same reason as theft and murder: They promote selfish interests at the cost of benefits to society. Later Mohists modified this slightly, recognizing that sometimes they needed to use millitary aggression to overthrow tyrrants and unjust rulers. Aggression merely to benefit yourself (your society) is never justified but aggression to protect others and bring about a better society is justified. 

\subsection{Moderation in Use}

To benefit society, one must care about others and this care must be displayed through one's actions. The wealthy have the power to benefit a large portion of society and promote the goods on a large scale. This, however, is in opposition to how some spend their money. They spend it on wasteful luxuries and useless ventures. It is argued in Moderation in Use that such wasteful luxuries  must be removed. The wealthy and those in charge should be benevolent and care about the welfare of those `below' them on the social ladder. As a result, wasteful luxuries is a sign of a vicious person.\footnote{A vicious person is, literally, the opposite of a virtuous one.} Or, as the Mozi put it:

\factoidbox{In evaluating elaborate funerals and prolonged
mourning, how do they accord with these three benefits? Perhaps
it is the case that, if we take their words as a model and make use
of their plans, elaborate funerals and prolonged mourning really
can make the poor rich and make the few many, resolve danger
and bring order to disorder. If so, they are benevolent and
righteous, and the duty of a filial son, and their use in planning for
people must be encouraged. Those who are benevolent will seek
to promote them in the world, establish them and cause the people
to praise them, so forever they will not be done away with.
Perhaps, on the other hand, if we take their words as a model and
make use of their plans, elaborate funerals and prolonged
mourning really cannot make the poor rich and the few many, or
resolve danger and bring order to disorder. If so, they are not
benevolent and not righteous, and not the duty of a filial son, and
their use in planning for people must be resisted. \autocite[25.3]{Johnston2}}


In Moderation in Burials, a similar line of thought is followed, showing that, as quoted above, wasteful funerals and prolonged mourning does not benefit society and help people. It shows too much care for the dead and not enough for the living. 

\subsection{Heaven's Intent}

Many people today and then believe(d) in some form of an after-life. According to many religions, the good people are rewarded and the bad people are punished.\footnote{This small subsection should remind you of Divine Command Theory.} Social order and the benefits of a good society can be promoted by encouraging people to believe in such a divine reward and punishment system (at least at first, eventually, people will grow to understand that they should do good because it's good).  At the same time, the Mohists believe that Heaven, as a personal deity, is just and right. Doing as Heaven intends will cause good things to follow and refusing will cause bad things. 

\factoidbox{This being so, then what does Heaven desire and what does it
abhor? Heaven desires righteousness and abhors unrighteousness.
In this case, then, if I lead the ordinary people of the world to
conduct their affairs with righteousness, I will, in fact, be doing
what Heaven desires. If I do what Heaven desires, Heaven will
also do what I desire. What, then, do I desire and what do I abhor?
I desire good fortune and prosperity and I abhor bad fortune and
calamity.
If I do not do what Heaven desires, but do what Heaven
does not desire, then I will lead the ordinary people of the world to
land themselves in misfortune and calamity in the conduct of
affairs. This being so, how do I know that Heaven desires
righteousness and abhors unrighteousness? I say that when the
world is righteous, it ‘lives’, and when it is not righteous, it ‘dies’.
When it is righteous, it is rich. When it is not righteous, it is poor.
When it is righteous, it is well ordered. When it is not righteous,
it is disordered. So then, Heaven desires its (the world’s) ‘life’ and
abhors its ‘death’. It desires its wealth and abhors its poverty. It
desires its order and abhors its disorder. This is how I know that
Heaven desires righteous and abhors unrighteousness.”\autocite[25.2]{Johnston2}}

\subsection{Condemning Music/Fatalism}

Much like with Moderation in Use, Mohism frowns upon extravagant and wasteful uses of resources which could be used to benefit the people. In condemning music, the Mohists are actually condemning the wasteful use of resources by the rulers and officials in society for minute, temporary, benefits. On the other side, the Mohists condemn Fatalism, or at least the promotion of the belief in it. This, in ordinary terms, is the stance that our status and station in life is predetermined and human effort against it is useless. Promoting this depressing outlook is very good for keeping people in their place and forcing a totalitarian society. But, it fundamentally contradicts the good which Mohist Consequentialism seeks to promote.

\section{Mohist Conception of `The Good'}

It is worth remembering that the ethical system proposed and defended by the Mohists is, fundamentally, Consequentialist. This means that it is, on the most basic level, concerned with causing the most good one can while preventing the most bad.\footnote{And also causing the least bad one can.} This leads us to two fundamental questions which all Consequentialists must grapple with: 

\begin{enumerate}
\item What is the good?
\item How should one promote that?
\end{enumerate}

To the first question, Utilitarians respond by saying that happiness is the one and only good and then use that to say that the inverse, sadness/suffering, is the one and only bad. Other ethical theories have followed suit and identified only one good and used that to identify only one bad. Mohism, however, did not identify only one good. Rather, the Mohists gave a list of goods which one must promote.\footnote{Each of these goods has an inverse, an opposite, which Mohist Consequentialism commands that one minimize or prevent.} This list is based on a loose notion of human welfare and each one is, according to Mohism, necessary for our well-being. The Mohists openly admit that the theory will not automatically solve all ethical problems because, as they admit, there will likely need to be trade-offs between the various goods\footnote{That is, there will likely be cases where one must promote one good at the cost of another. For example, one might need to sacrifice tastiness for healthiness.} and one needs practical wisdom, world experience, to know when the trade-offs are necessary and worth it.  

\factoidbox{Master Mo Zi spoke, saying: “The business of the benevolent
[man] must be to seek diligently to promote what benefits the
world and eliminate what harms it so he will be a model for the
world. If he is benefiting people, then he acts. If he is not
benefiting people, then he stops.\autocite[32.1]{Johnston2}} 

In the above passage, the most important word which we need to understand is `benefit'. In various contexts and situation, being beneficial could mean promoting or providing different things, in a na\"ive Utilitarian account, this would be simply the promotion of happiness and the prevention of suffering. Mohism, on the other hand, provides three different goods which we need to promote to be beneficial:\footnote{As a histotical note, the early Mohists were mostly concerned with convincing leaders and gentlemen to adopt their teachings and implement their ideas. With that in mind, having wealth and large families as two primary goods would make this very pursuasive.}

\begin{enumerate}
\item Material Wealth
\item Large Family
\item Social Order
\end{enumerate}

\subsection{Material Wealth}

At first pass, this might make you think that Mohism is promoting greed and the hoarding of resources. This is incorrect. On the most basic level, all people need food, water, and shelter. The Mohists are commanding us to engage in behaviors which promote those basic necessities for life.\autocite[52]{VanNorden1} More over, because of inclusive care, we should also promote those behaviors/programs which provide those basic necessities for other members of our society, as we care about their well-being as well. Once the basic necessities are met, we can start to promote material wealth in other ways, like providing other resources which further elevates the quality of life in our society. A flourishing society requires resources (material wealth), so we must promote the programs which promote having those resources.

\subsection{Large Family}

This might be more aptly called ``a booming population''. Remember that Mohism developed and came into its own during the Chinese Warring States period. Wars during this time period (for China and any other society at that level of technological development) brought famine, disease, and devistation to cities. Having a growing population was necessary for survival. More than that, though, having more people means that you have the man-power to increase your resources, having more resources means that you can provide for more, and so on.

\subsection{Social Order}

The first two are easily measurable and can easily be understood.\footnote{Though I don't see them as strong contenders for actual intrinsic goods.} The third requires us to dig a little deeper, because it is far more complicated. This is not the sort of good which one can have individually, rather, it is a social good. Order, in this sense, is having a good `social life'. By this, I don't mean that you regularly go to parties or hang out with friends,\footnote{Though, individually, that likely will happen.} rather having a good social life means that the society you live in is good. This is a collective goal, one where we must all do our parts. There are at least four conditions for social order:

\begin{enumerate}
\item Unified Moral Standards (administered by virtuous leaders)
\item An absence of crime, injury, harassment, and conflict (peace, security, and love (harmony) prevail)
\item Members of society have virtues proper to their social role
\item The community members engage in assistance and charity, sharing information, labor, education, and food (and other material goods), and helping the less fortunate
\end{enumerate}

We can take all of these notions and distill them into a few basic points: The main command of Mohist Consequentialism is to benefit the world and to benefit the world, one must promote social harmony, public security, economic prosperity, increases in population, cooperation, charity, and good social relations. Though the texts do call on us, as a duty, to benefit the world, they do not call upon us to engage in selfless altruism. The idea here is that we need to respect others' status, lives, and property and be cooperative with other community members. Selfless acts of charity are only called for in cases of special hardship.\footnote{You do not, for example, need to give the shirt off of your own back every time someone else faces the slightest hardship. This is an interesting reply to the demandingness problem faced by simple Act-Utilitarianism.} 

It is also worth noting that there is a special attention given to `virtues'. As we will encounter later, in the module concerning Virtue Ethics, these are character traits of a good person. But, unlike Aristotle, Mohism does not seek for you to have these virues to promote your own flourishing, rather, having these virtues promotes the flourishing of the society you are in. Also, unlike Aristotle, these virtues are not `one-size-fits-all'. The Mohist virtues depend on the position you have in society. A flourishing society will be one where every social role is performed wholeheartedly. These social roles include those related to your social relationships. For example, it is virtuous for a parent to care about their own children more than they care about another person's.\footnote{Simple Act-Utilitarianism would have it that you need to care for all children equally, regardless of their relation to you.} This means that, despite calling for us to care about all people, Mohism has it that we have more of a responsibility towards those we have close personal relationships with.\autocite{Robins1} \footnote{See Feminist Consequentialism for a modern, rediscovering, of this idea.}  

One interesting aspect of this is that these three goods, material wealth, large family, and social order, all build into each other, in a feedback loop. For example, if a society has a large amount of material wealth, then the population will increase, increasing the amount of material wealth and also the driving forces behind crime and conflict, at least from the Mohist perspective, are poverty and desperation. Having virtuous, caring, leaders utilize their resources to promote the well-being of all members of the society will increase the amount of harmony by decreasing the amount of poverty and desperation. 

\section{How Do We Promote These Goods?}

Mohism does not demand that every single action we take be the one with the best outcomes. Like other classical Chinese thinkers, the most relevant aspect of our activities is not the individual actions, but rather the \emph{dao} or way, manner, style or pattern of doing the actions. \emph{Dao} is a way of life, the dispositions we have (our virtues). As a result, this system of ethics has a lot of similarties with Rule Consequentialism. Rather than asking you to formulate a rule which would be most beneficial to society, Mohism asks you to structure your habits and dispositions to promote the good society.\footnote{Like Virtue Consequentialism.} In practice, this is done through being mindful about your place in society and the place of another, having a disposition to be kind and helpful, and caring about all people. 

`Caring about all people' is especially worth noting. In the original Mohist texts, this is known as \emph{jian ai} and translated as `inclusive care' and sometimes as `universal love'. Mohists recognize this as the most fundamental disposition (virtue) and all of the others flow from it. In the texts, this notion refers to a concern about the welfare of another. Having a disposition towards inclusive care will open your eyes to the needs of other people and motivate you to help them when you are able.  You must care about everyone and, in practice, benefit those you actually interact with. Social harm arises from the exclusion of people and a disposition towards disregarding their interests.

As you may recall from the section concerning this inclusive care, I stressed that this does not mean that you have an equal duty towards all people. The Mohists recognized that you have a greater duty to some people than to others. They described this with a metaphor, the `thicker' your care is (or should be) for another, the greater your duties are to them. 

\factoidbox{If, according to duty, it is permissible to love [someone] “thickly”,
then love them “thickly”. If, according to duty, it is permissible to
love [someone] “thinly”, then love them “thinly”. This is to speak
of “proper sequence”. Virtuous rulers, elders and parents all are
those one should love “thickly”. [However], loving one’s elders
“thickly” does not entail loving those who are young “thinly”. If
relations are close they should be loved “thickly”; if they are
distant they should be loved “thinly”. One should be on close
terms with one’s parents whereas, with respect to those other than
parents, one may love “thinly”. It is in accord with principle to
love one’s parents “thickly”. One must look closely at their
conduct, but hope only to see virtues.\autocite[44.6]{Johnston2}} 

All people deserve a degree of care, you must have some concern for their well-being and welfare, but there are virtues proper to certain roles which one has in society, those of a parent, child, boss, employee, and so on. These roles and virtues entail that we must have a greater duty towards those with whom we are close (`thicker love') than with those who are distant (`thinner love').  On a very basic level, this theory is asking you to lead a life which benefits the society you are in.
