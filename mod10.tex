\part{Environmental/Intergenerational Ethics}
\label{ch.mod10}
\addtocontents{toc}{\protect\mbox{}\protect\hrulefill\par}
\chapter{Part 29: Duties to the Environment}

Environmental Ethics is the study of the moral relationship between humans and the environment (including non-human animals). For example, the question ‘are certain animals members of our moral community?’ would be discussed in this field.\footnote{ Remember that a member of our moral community is the sort of thing that we have some sort of moral obligation to.} For example, if you think that a fetus is a member of our moral community, then you must be against abortion. In connection to the previous stances which we have seen, the consequentialist will have no problem thinking that animals are members of the moral community, because the only requirement for moral consideration there is that they can suffer. The Non-consequentialist and Feminist Ethicist, in general, will have a harder time determining the moral duties which we have to the environment and the animals in it because they limit the scope of the moral community.\footnote{This can be because either they only take human persons as members of the moral community or they say that the moral community only contains those we care about.}

As global warming and human caused crisises actively threaten our planet and all those who live on it,\footnote{Including animals.} we must face the moral questions concerning our duties to animals and the environment which they live in. To deal with those questions we need to figure out what the moral value of a non-human animal is. If the value of an animal life is equal to that of the life of a human person, then we have just as much a duty to ensure that they can lead a good life as we do to another human person. If, on the other hand, they have a greater value than the life of a human person, we would have a greater duty to ensure that they can lead a good life. And finally, to finish off all the possibilities, if an animal life is worth less than that of a human person, we have a lesser duty to them. 

Most people come to this question and immediately jump onto a stance which many people find intuitive, The Unequal Value Thesis. This is a stance which says that we have some duties to animals but we have a greater duty to human persons. Moreover, it has it that there are certain animals who have more valuable lives than others. This stance can be summed up as three claims: 

\begin{enumerate}
\item Animal Life Has Some Value
\item Some Animal Lives Are Worth More Than Others
\item Human Lives Are The Most Valuable
\end{enumerate}

But, each of these three claims needs to be defended. We need cases and arguments which really show that these are intuitive. The cases we will look at are three different variations on the classic Trolley Problem. For spoilers, if you think that you should flip the switch for all three, then you agree with this unequal value thesis and this will have interesting implications regarding our duties to animals. 

\section{Animal Life Has Some Value}

This is, roughly, claiming that we have some kind of duty towards non-human animals. For an example, consider this case:  

\factoidbox{\noindent  \fontsize{20pt}{0pt}\textbf{Kitten Vs Nothing}

There is a run away trolley and if you do nothing, then it will run over and kill a kitten who wandered onto the tracks. If, however, you flip the switch to divert the trolley, then nothing will happen, the trolley will curve around the little fluff ball.  

Do you flip the switch?
}

There are three different ways one could respond to this case. If you have a strong urge to flip the switch, then you are saying that at least some animals have moral value.\footnote{And that value is greater than the effort to flip the switch.} If you have it that one should do nothing, then you might hold that animal lives have a negative value or have a value so low that it doesn't compensate the effort to flip the switch. Finally, if you find yourself needing to flip a coin, then you hold that animal life is so low in value that it's equal to the effort of flipping the switch. 

Given this case, I would argue that most people would hold that flipping the switch is the right thing to do. Doing nothing is rather monsterous, from the ordinary person's perspective, and needing to flip a coin shows an inhuman level of discompassion. This means that we have some duties to non-human animals. This claim does not have it that we have the same duties to animals as we do to human persons. If we were to add the extra, stronger, claim that all animal lives are equal in value (your life and an earthworm's are the same value), then this would entail that we have the same duties across the board. This, however, will be into question by the other thought experiments. A simple argument can be made for this aspect of the Unequal Value Thesis: 

\begin{enumerate}
    \item If animal lives have no value, then in cases of saving an animal vs doing nothing, we would probably do nothing (or at least flip a coin).
    \item We don’t do nothing (nor do we flip a coin). All else being equal, we save the animal.
    \item So, animal lives have some value (not necessarily the same value, but some).
\end{enumerate}

\subsection{Some Animal Lives Are Worth More Than Others}

\factoidbox{\noindent  \fontsize{20pt}{0pt}\textbf{Chimp Vs Earthworm}

There is a run away trolley and if you do nothing, then it will run over and kill a chimp who wandered onto the tracks. If, however, you flip the switch to divert the trolley, it will kill an earthworm who has also wiggled onto the tracks. It is not possible for you to save both, one must die so the other may live. 

Do you flip the switch?
}

For this scenario, if you claim that one ought to flip the switch, then you are saying that the chimp's life is worth more than that of the earthworm.\footnote{I chose two animals which are extremely different in what most take to be the relevant aspects in order to make this very clear.} If you say that we ought not flip the switch, then you are making the claim that the worm's life is more valuable than that of the chimp. And, finally, if you must flip a coin about what to do, then you hold that the lives are of equal value.

From my experience, the vast majority of people would say that refusing to flip the switch or needing to flip a coin is just plain wrong. It is obviously the case that the chimp's life is more valuable than that of the earthworm.


This is the second stance which needs to be supported. This is to reject the possible answer to the question concerning the value of animal life in comparison to human life which claims that all life is equal. It should, at first glance, seem pretty intuitive. It does not make a claim that human life is more valuable than all other animal life, rather it merely makes the claim that there is an inequality in the values. It is perfectly possible for these last two stances to be true and humans be the least valuable (debunking that option is the last stance). Here is a quick argument concerning this: 
\begin{enumerate}
    \item If all animal lives are equal in value, then in cases of saving an kitten vs an earthworm, we would need to flip a coin.
    \item But we don’t flip a coin. All else being equal, we save the kitten.
    \item So, animal lives are not equal in value (some are worth more than others).
\end{enumerate}


\subsection{Human life is more valuable than animal life.}

This final stance builds off of the others, it requires that there be a discrepancy in value between the different kinds of life and that there be some kind of value to begin with. But it does add in something interesting, namely that human life is at the top of the ladder, so to speak. Whatever we take the value of puppies and kittens to be, it seems clear that we would prefer to save a human over saving them. If we took human life as less valuable, then we would save the animal over a human. If we took human life as equal to animal life, we would need to flip a coin. But we don’t do that, we take human life as more valuable. Take this case as an example:

\factoidbox{\noindent  \fontsize{20pt}{0pt}\textbf{Baby Vs Chimp}

There is a run away trolley and if you do nothing, then it will run over and kill an toddler who crawled onto the tracks. If, however, you flip the switch to divert the trolley, it will kill a chimp who has also wandered onto the tracks. It is not possible for you to save both, one must die so the other may live. 

Do you flip the switch?
}

In this case, if you have a strong preference to flipping the switch, then you would say that human lives have greater value than that of a chimp.\footnote{Since chimps are the most like us when it comes to non-human animals, this may also show that you hold that human lives are more valuable than any animal life.} If, however, you have a strong opposition to flipping the switch, this shows that you hold that the chimp's life is more valuable than that of a human's. And, finally, if you find yourself needing to flip a coin about whether to flip the switch, then you likely hold that the lives are of equal value. 

I would wager that most people, if faced with this choice would flip the switch, most without question. This is because we, naturally, hold that human lives are more valuable. There are several possible explanations for why this is, which we will explore in this module. Here is this put as a clear argument:

\begin{enumerate}
    \item If human lives aren’t more valuable than animals lives, then in cases of saving a baby vs an earthworm, we would save the earthworm, or, at least, need to flip a coin.
    \item But we don’t save the earthworm nor do we flip a coin. All else being equal, we save the baby.
    \item So, human lives are more valuable than animal lives.
\end{enumerate}
 
These three stances, in combination, lead us to the Unequal Value Thesis. This is the stance that humans are more valuable than animals. We hold this stance, in general, despite us being opposed to animal cruelty and recognize the value of animal life, sometimes extreme value. Many animal rights activists today seem to be inclined to accept that human life is more valuable than animals life. Our intuitions about this can be seen in the trolley problem I gave not moments ago, but this could be seen as an extreme case. In ordinary cases, this intuition also applies. This intuition can be seen in how most of us treat animals vs how we treat humans. We, consistently, oppose animal and human cruelty, but we don’t take them to be equal. We also can be against the suffering of any sentient creature without thinking at they are all equal (all three of the interrelated stances can be held together). 

\section{Consequences of the Unequal Value Thesis}

While many people find this stance intuitive and, if you have followed the thought experiments, you likely do too, there are some very serious consequences to holding this view. This is because one's beliefs about value fundamentally shape and direct their actions and one a large scale, can cause very drastic things to happen. The first consequence of the Unequal Value Thesis is that we feel justified in making life for animals worse in order to make life for humans better, even by the same degree. For example, we regularly use rabbits to test beauty products, especially those which go near the eyes, because rabbits cannot wash their eyes out naturally, like humans. Using rabbits shows the worst-case scenario for a reaction in humans. This is extremely painful for the bunny. Even if your life is made better by having the perfect shade of eye-shadow, the suffering caused on the rabbit might not be worth it.\footnote{Using the Consequentialist style reasoning, we an come up with a worst-case for this argument, where the happiness people get from the eye-shadow balances out the suffering to the rabbit. The Unequal Value Thesis would have it that we are still justified in doing so because human lives are worth more.}

The next consequence of the Unequal Value Thesis is that we are justified in killing animals for food.\footnote{we will be returning to this point in the Argument from Marginal Cases.} Since we take animal lives to be worth less than human lives, we are justified in killing animals for our own benefit. For example, we are justified in, immediately after birth, locking calves away in white huts, without the ability to move, and force feeding them large quantities of food in order to fatten them up and produce veal. Similarly, we are justified in having our livestock contained in large warehouses, so crowded that it's impossible to move, and then kill them when they are nice and fat.\footnote{This is the process for large scale chicken farming. Even those which claim to give them `free range' only need to provide access to the outdoors for a few hours a day. Often, the chickens are so fat that eventually they couldn't go outside even if they wanted to.} 


\section{Can we defend the Unequal Value Thesis?}

The consequences of the Unequal Value Thesis might make your gut turn, and it rightly should. We can be opposed to animal cruelty without rejecting the Unequal Value Thesis, but how can we do this? What justification can the defenders of this stance provide which makes such cases permissible? The defenders of the Unequal Value Thesis would need to show that the gap between the value of a human life and that of an animal is so great that it clearly justifies us using them in the way we do. To do this, the defenders will need to argue against two competing views which might be reasonable: First, that animal life is equal to human life (this is the one which gives us a real run for our money). Second, that animal life has no value whatsoever (this one would be very hard to defend). But, we also need to ask where the burden of proof is, who needs to give evidence for their stance? Often it seems that we don’t think that the stance that animal and human lives are the same in value requires the proof but we think that the stance that animal lives have no value does require the proof. This is interesting because we should think that both require proof. 

When it comes to pain and suffering, we tend to think of the human and animal cases in the same way. Cruelty to a child and cruelty to a dog are wrong for the same reason.  Pain is pain; it is evil, it is as much an evil for dogs as for humans. Furthermore, autonomy (what people often take to say that animal life is less valuable) does not seem a relevant factor here, since the pains of nonautonomous creatures count as well as the pains of autonomous ones. Neither the child nor the dog is autonomous (yet); but the pains of both child and dog count and affect our judgments of rightness and wrongness with respect to what is done to them. But, when it comes to life, we don’t tend to think of the cases similarly. A human life, we take it, is always more valuable than an animal life. We regularly experiment and kill animals for the benefit of humans and the justification for this tends to follow suit. The autonomy of a human now seems to play a part in our decisions.

We are now moving on to ways which we can defend the UVT. There are several ways in which one could do this and we will finish out with the one which might have the greatest potential of success.

\subsection{Appeals to God}

This does not just apply in Christianity, but also in most religions which have a similar historical origin (Islam, Judeism, and others). In such belief systems, you have a passage or belief that God created man and gave them dominion over the world, as in God created humans with the explicit feature that they are greater than all other beings, and as a result all humans are worth more than animals. Morally speaking, no amount of animal suffering (if they actually have it) counteracts human pleasure.  But this stance does have its issues, as we will see.
\begin{enumerate}
    \item God gave man dominion over the animals.
    \item If (1), then they aren't equal.
    \item If they aren't equal, then animal suffering, if beneficial to humans, doesn't count morally.
    \item Therefore, animal suffering, if beneficial to humans, doesn't count morally. 
\end{enumerate}
This argument, however, does have its issues. For example, this assumes the existence of God, which is debatable, but it also assumes that dominion means that we can use them as we see fit without moral consideration. This might not be the case. For example, just because I own a dog, I have dominion over them, but that doesn't mean that I can treat the puppy poorly.  

\subsection{Tu Quoque Fallacy}

This is a Latin term for ``you also". Think about it this way, imagine that a parent is a smoker and gets really angry when they catch their 21+ year old child smoking; a natural response is for the person to say ``well, you smoke, why can't I?" Basically, if you do it, I can too. This is a fallacy, just because another being does something, it doesn't make it OK. In the animal kingdom, we don't see other animals treating other species without favorism towards their own (as in, we see animal favoring members of their own species over members of different ones). This raises the question ``why shouldn't we favor our own, solely because they are our own?" 

The issue with this line of thinking is that it's a fallacy, the core root of the reasoning doesn't generalize well. For example, (this is an example given to me by former military students) in some cases, the enemy combatants in war will not treat our fallen soldiers with respect. The question here is ``are we allowed to treat their fallen disrespectfully?" Unless you are particularly hardened, the reply is ``no, we are better than that." And that's the point, we are better than that, we need to hold ourselves to a higher moral standard.

\subsection{The Richness of Human Lives}

When looked at on an abstract level, killing is wrong because there is the loss of something of value. The wrongness of killing seems to be deeply connected with the value of the life taken. As we have seen, very few people think that animal lives are worthless and very few people think that they are equal in value to human lives. This means that there must be something inherent to human lives which makes them more valuable than those of other animals. We are saying that human lives have certain qualities, are of a higher quality which makes killing a human worse than killing an animal. This is not because we are humans, but rather that our lives have a ‘richness’ to them.

When we, at first glance, compare a human life to that of some animal, we notice that there are some fundamental aspects which seem to make our lives more valuable. These aspects are things which humans can do but the other living creatures on this planet cannot. First, humans have far deeper and more complex intellectual and emotional lives. This complexity allows for humans to experience the world in a far richer way, which could be argued makes our lives more valuable than those had by other animals. Second, humans are capable of love. Other creatures may be capable of emotions similar to love, but only humans are capable of the real experience of love and many hold that love is the most valuable thing in the world. This means that creatures capable of love have more valuable lives than those incapable of it. Third, and most importantly, humans have autonomy. Autonomy is the fundamental aspect of a human life which allows for all of the other riches to arise.  We can mold our lives to suit our conceptions of the good life. Some of us like playing sports, others like more cultural activities, others still like intellectual endeavors; some of us are good with our hands and making/building things; and all of us see a job as an important part of a full life. The emphasis is upon agency: we can make ourselves into repairmen, pianists, and accountants; by exercising our autonomy, we can impose upon our lives a conception of the good life that we have for the moment embraced. We can then try to live out this conception, with the consequent sense of fulfillment and achievement that this makes possible. Even failure can be part of the picture: a woman can try to make herself into an Olympic athlete and fail; but her efforts to develop and shape her talents and to take control of and to mold her life in the appropriate ways can enrich her life. Thus, by exercising our autonomy and trying to live out some conception of how we want to live, we make possible further, important dimensions of value to our lives.

For the person who thinks that animal lives are worth just as much as a human life, we would need to have some kind of story, some kind of picture, which gives those animals capacities which are far beyond what we see in them or forces us into a very serious case of skepticism about the value of human life or that of the animal. For example, we don’t actually know what it is like to be a rabbit, so we have no clue how good of a life it really is. But, given what we see in the capacities of little bunny foofoo, the odds of it being as good as a human life are very slim. 

This argument, case, is not without its issues, same as before. Humans may be, as far as we know, the only creatures able to richly mold our own lives but not all humans have this capability and we run into an issue when we try to divide up who could lead a rich life and who can't. If we claim that all and only humans have this quality, then we also run into issues.

\section{Argument from Marginal Cases}

To start us off, we will be discussing the Argument from Marginal Cases. Many people don't like the term `marginal', but frankly explaining the argument without using that term is difficult. \footnote{Tyler Doggett of the University of Vermont made a video making a similar argument to this without using the term `marginal'.\autocite{Doggett1}} The term `marginal' here is used to refer to an individual who, due to various factors, is severely mentally handicapped. Due to this, their life lacks the `richness' which makes normal human life more valuable. For example, a human who is not capable of having the mental life of a `normal' human over the age of three. The argument from marginal cases goes like this:

\begin{enumerate}
    \item If we are justified in denying the moral status of non-human animals (killing them for food, using them for our benefit, exploiting them for medical research, etc), then we are justified in doing so in the case of marginal humans.
    \item We are not justified in doing so in the case of marginal humans.
    \item Therefore, we are not justified in denying the moral status of non-human animals.
\end{enumerate}


The second line is the intuition pull. Basically, ask yourself whether it would be OK to kill and eat a mentally handicapped human, someone with the mental life of a 3 year old. I hope that most of you will say that this is just wrong. It is wrong to kill and eat humans (in non-dire situations) in the same way which we do livestock (more especially pigs). But, this leads us into the first line. What is the moral difference between non-human animals and humans which justifies our using them for our benefit in this way?

\subsection{Why is it OK to eat animals and not marginal humans?}
 
First, one could claim that the moral difference between humans and non-human animals (which makes their use permissible) is that humans are humans and animals are animals. This is to claim that the species of the creature is the morally relevant factor. But, this can't do. Imagine that in a couple years an advanced alien society comes down and starts to farm humans for food and starts to use humans for their medical research, etc. If one was to ask you whether this is permissible, I'd wager that you'd say that it's wrong for the aliens to farm and use humans in this way. But, this goes against the initial claim that the moral difference between humans and non-human animals is the species. These aliens are aliens and humans are humans. So, the same reasoning used  to argue that humans can use animals in this way is used here to show that aliens could use us in this way. This means that this just could not be right.

Second, one could claim that the moral difference between humans and animals which justifies us farming/using them is that humans are special,  we have reached this certain level of development which makes it so that we aren't the sort of things which can be used in this way. Pigs and other non-human animals aren't like that. The aliens in the above example, we can assume, are that sort of thing too. But, what makes us special, in this way. We can't point to the intelligence alone for this, otherwise one could just reply with the argument from marginal cases again, because those humans don't have the intelligence necessary to qualify. But, one could reply with the claim that we care about humans. This caring about them is the moral difference. But, there is an issue with this. In claiming that the moral difference is this act of caring about them, you are claiming that the reason it is wrong to kill and eat a marginal human is that another cares about them, it has nothing to do with them. So, ask yourself whether it would be OK to kill and eat a marginal human which no one cared about. If you claim that it is still wrong, then this does not have a leg to stand on.

This leaves us with the following argument: 

\begin{enumerate}
\item If the Unequal Value Thesis is true, then there would need to be some kind of relevant moral difference between humans and animals which makes all human lives (even the marginal ones) more valuable than animal life. 
\item There is no such moral difference. 
\item Therefore the Unequal Value Thesis must be false. 
\end{enumerate}

\chapter{Part 30: Duties to Future Generations}
When I say ‘future people’, I mean people who aren’t born yet, who, unless something terrible happens, will exist a few generations down the road. People who aren’t even a twinkle in their pappy’s eyes yet. Although I can’t be sure that any one of them will exist, I can be sure that some will exist. One way to think about this interesting point is that there, more than  likely, will be people in the future, but I can't say with certainty who those people will be, what individuals will make up the collection of people out there. We can point to things like chaos theory\footnote{Chaos Theory implies determinism, but it's an epistemic claim about whether we can predict the future.} to explain why we can't know who those people will be.  Now, our question is ``do those people, people who don't exist, but likely will, have rights? Are they the sort of things worthy of our moral consideration? Do we need to be concerned about their welfare?"

Some of you, I am willing to wager, will say that this is a no-duh sort of question, but, like with the Abortion Debate, we need to look at the reasons. Just as in that case, both sides will say that it's obvious. Some say, the consequentialists being the strongest voices there, that we have the same obligations to future people as we do currently existing people. Just because they don't exist, they will, so we need to ensure that the best future is there for them. Others will say that we don't have any obligations to future people. Here we could find the non-consequentialists. They would say that, for example, the being must exist in order to be treated as a mere-means; so we don't have obligations to them. We can only have duties to contemporaneous persons.

\section{Intergenerational Justice/Ethics}

Trying to figure out the morality of actions concerning future people is an area of philosophy called \gls{Intergenerational Ethics}. When it comes to their rights and the duties we have to them, this is `intergenerational justice'. Since the consequentialist is not too much of a fan of rights and doesn't, fundementally, think about morality in terms of duties, the consequentialist will tend to work in the more general intergenerational ethics, while the non-consequentialist will tend to work in the more particular intergenerational justice. However, there are certain powers which the current generation has over the future generations which are not had by them over us. These powers add variables into the typical equations which we would use for the morality of actions which lead to both epistemic and metaphysical questions which need to be addressed in order to figure out what the right course of action is. There are three such powers, with the third being the greatest and most perplexing problem, especially for the consequentialist.

\newglossaryentry{Intergenerational Ethics}
{
  name=Intergenerational Ethics,
  description={The study of the moral relationship, the duties and obligations, contemporary people have towards generations in a far distant future},
}


\subsection{Limiting choices:}

The current generation can set up a system which would be very costly for the future generations to change and, essentially, force them to continue with that system. For example, what if we made various choices which placed the future generations into an extreme economic debt, this would force them to maintain certain policies in order to pay off said debt. Or, on a smaller scale, what if we bought houses which were tied to familial wealth with a 200 year long repayment plan? This would force the future generations to live in that house, unless they got very wealthy rather quickly. Future generations are not able to place that sort of burden on us, they can't force us to do anything (with the exception of if time-travel happens, then they could). But the limitations don't need to just be economic, they can also, for example, be intellectual. In the case of intellectual limitations, we can greatly advance in one direction which would make the alternatives woefully under-researched, making switching to the alternatives very hard because it would require the future generation to back-track and start the research from scratch. Here are two examples: 
\factoidbox{\noindent  \fontsize{20pt}{0pt}\textbf{Green Dictators}

Suppose that we know that the future generation will need to continue pursuing alternative energy sources and not transition back to coal/oil power for electricity/transportation. As a result, the current generation signs multiple treaties which make it very difficult for the countries to back out and have extreme penalties for refusing to comply with this green agenda.}

\factoidbox{\noindent  \fontsize{20pt}{0pt}\textbf{The Cheapest Route}

Suppose that we build an infrastructure, roads, electrical lines, waterways, etc. with the easiest materials and power-sources available to progress very quickly and have great advancements. This infrastructure gets so ingrained that the transition to other sources of power and other infrastructure methods is very expensive and underdeveloped compared to where it would have been.}


For both of these cases, we have constrained the future generations to take a path which they did not choose for themselves. They could not have given consent or a voice in the process. So, we could, maybe, say that they were coerced into the system. But, some might say that we did the right thing when it comes to The Green Dictators and the wrong thing in The Cheapest Route. This difference must be because of the consequences. But, we are still forcing another group to follow our choices, much like the Cultural Imperialist. As an interesting side note, regardless of the path we choose, we are still imposing our wishes onto the future generations.
\subsection{Unidirectional Benefit:}

This is a sort of reiteration of the `Limiting choices' ability which our generation has on the future ones. It is possible for the current generation to benefit themselves at the expense of the future generation and the future generation will experience all of the cost and, in some cases, none of the benefits. In such cases, the current generation will enact policies or engage in behaviors which will benefit them in the short term, be costly in the long term, and be long dead by the time those bills come due.  The Cheapest Route case works for this example if we add in that the benefits of the rapid expansion are less than the costs to the future generations. But there are other cases:
\factoidbox{\noindent  \fontsize{20pt}{0pt}\textbf{The Origins of the Automobile}

When automobiles (cars) were first becoming a product which the average American could buy, there were three different generic kinds. First, they had the steam-powered car. There were several companies producing these and they had the benefits of being familiar technology as well as able to go long distances. They weren't bought up as much because they didn't have the `get up and go' and the speed of the other options. Second, there were the electric cars. These did not have the speed nor could they go long distances. And finally, there were the gasoline cars. These, with the introduction of the electric starter, had the quick start, the speed, and the ability to go long distances. The predominate buyers really liked these. That generation collectively greatly benefited gasoline cars,  but it resulted in great costs to the future generations, in the form of Global Warming. 
}

\factoidbox{\noindent  \fontsize{20pt}{0pt}\textbf{Multi-Generational Mortgages}

In some areas of the world, there are multi-generational mortgages 100+ year plans (the most famous are Japan’s 100 year mortgages, but Sweden has 105 year plans). In such cases, the future generations will be forced to pay the debt off of the house or the loan with the current generation getting the money from the loan. This will also force those generations to live in those houses until the house is paid off and able to be sold, some cases they are forced to sell the home to pay it off.}

As I said previously, there are three, general, powers which the current generation has over the future generations which they don't have towards us. But this third problem deserves a page all to itself, because it leads us into a very complex and mind-bending problem, especially for certain kinds of consequentialists.

\subsection{Power to bring them into being:}

Not only do we have the power, without any resistance from them, to essentially make them our slaves long after we have taken to dust, but we have the power to actually create them. We have the power to influence and alter what individuals will come into being and how many of them will come into being. We could, though unlikely, completely destroy ourselves, making no future people possible, or we could have a giant baby-boom which makes a ton of them. Even little things, which are totally coincidental or seemingly unrelated can result in an entirely different crop of future people being produced than if it hadn't happened. Take these three cases:

\factoidbox{\noindent  \fontsize{20pt}{0pt}\textbf{The Black-Out Baby Boom}

In certain areas in New York, there was a time when the places would shut down because everyone was watching Friends. On one such day, a young grad student and his wife were settling down to watch and there was a power outage, so, they grab a few blankets, light the candles… 9 months later, a baby is born. This was, to be exact, the August 14th Northeast Blackout in 2003.}
	
\factoidbox{\noindent  \fontsize{20pt}{0pt}\textbf{The Iron Maiden Baby Boom}

In Brazil and several South American Countries, when Iron Maiden did their world tour, on Flight 666, there was a spike in births in the cities which they visited 9 months later, in order of their visit. (This is a bit exaggerated, but Woodstock had a similar effect.)}
	
\factoidbox{\noindent  \fontsize{20pt}{0pt}\textbf{The Sport Event Baby Boom}

In 2005, the Red Sox beat the Cardinals in a match after a 85 year losing streak. This resulted in a spike in the number of births roughly 9 months later. FC Barcelona's win in the 2009 UEFA Champions League semi-finals caused a 16.1\% jump in the birth rate in Barcelona 9 months later.\autocite{baby2} This was directly attributed to the victory. On Nov. 2nd 2016, the Chicago Cubs, a Baseball team, won the world series after a 108 year losing streak.\autocite{baby1} There was a lot of celebrations in the city. As a result, roughly 40 weeks later, there was a massive increase in births.}

When we are talking about intergenerational ethics, we do run into a bit of a mess. Though Consequentialism does have the better methods for handling these sort of cases, it falls flat in a few regards. The reason is that, for all of the ethical theories we have discussed, there is always an individual, or group of persons, which is/are being affected. Keeping with Consequentialism, since it has the easiest time handling these cases, we will say that an action is wrong only if it makes the well-being of a person/group of persons worse than otherwise. For example, when we are talking about current people, we say that something is wrong when they are in a worse situation than some other action which could have been taken. When thinking about future generations, sometimes, the actions we take directly cause them to exists. Those actions, so long as their lives are better than not existing, can't be said to make their lives worse off than otherwise. But, we still, sometimes, want to say that the acting generation did something wrong. This is a paradox, each line seems intuitive, but in combination, they can't work. Put clearly, we have The Non-Identity Problem.

\section{The Non-Identity Problem}
As I said in the previous page, there are three, general, powers which the current generation has over the future generations which they don't have towards us. There powers, collectively, lead to a uniquely difficult situation to think about morally. But, the third power, the power to bring them into being, generates a very complex and mind-bending problem, especially for certain kinds of consequentialists. 

\begin{enumerate}
    \item An action is wrong only if it makes a person (morally relevant being) worse off than otherwise.
    \item An action which brings a person into existence such that they couldn’t have been made otherwise and their life is at least marginally better than never being born, can’t have made them worse off than otherwise.
    \item There are some actions which bring such a being into existence and are still wrong.
\end{enumerate}

Since this is a paradox, each of these lines needs to seem intuitive. So, I will give some examples or arguments for the first two and in doing so, show that the third line is intuitive, just makes sense or seems true. 
\subsection{An action is wrong only if it makes a person (morally relevant being) worse off than otherwise.}

This is the first line of the paradox and to make it seem intuitive, I could reuse ton of the examples which I have already given in the cases concerning Utilitarianism (consequentialism), but for this one, I have chosen to given another example. This example is quasi-historical, as in all of the events took place, but there were other factors involved (which, frankly, only make the actions more wrong). This particular line, however, does have an element of a person-oriented nature to Ethics. It's basically saying that there needs to be a person (current or future) who is made worse by the action for it to be wrong. If there isn't a person (future or otherwise) made worse off, then the action isn't wrong. 

\factoidbox{\noindent  \fontsize{20pt}{0pt}\textbf{The Potato Famine}

George Forde\footnote{Yes, George Forde was a real person, in real life, he was a preacher.} was a farmer in Ireland in 1845. Due to the oppressive practices of the British, he and his family survived on a regular potato crop and the milk of a single cow. This was quite common for his region and in fact, this diet made the people in his region healthier in several regards than their British counterparts. Then the potato famine struck. This was a blight which rotted the potatoes and made them inedible. Removing almost all of the food sources in the region. Other countries and regions throughout Europe were also affected, however, due to relief measures, it did not escalate very far. The British, for a time, did similar, and the Irish were doing quite well. But, a few years into the blight, the British powers saw this as an act of God and removed all support and relief measures.\footnote{For a more rounded explanation for the ending of support, the British Government saw this, in private, as a way to reform the moral character of the Irish people and used a sanctimonious cover to end the aid (which was, through convoluted machinations, paying for itself).} This lead to Forde contracting diseases due to malnourishment and starving to death, like approximately 1 million others.}

The British ending their relief programs and aid lead to mass starvation and death. They could have, easily, continued those aid programs and saved more than a million people from a slow death. The vast majority of people will say that the British were wrong in ending their aid. This is, in part, because we know the results of it looking back and, even in that time period, they could have looked at how the other countries were handling the blight. We can generalize this, as we have done in other parts of this class, to any action, which, if it follows, makes this line of the paradox seem correct. 

\subsection{An action which brings a person into existence such that they couldn’t have been made otherwise and their life is at least marginally better than never being born, can’t have made them worse off than otherwise.}

I know that this line of the paradox is a bit long and seemingly convoluted, but this is because I need to give some examples for it to make sense. Essentially it is saying that a person can't be made worse than otherwise by an action which was necessary for their creation, so long as their life is better than never being born. Because of the precariousness of a person's existence, as can be seen in my above examples, there are several events (and the actions/situations which caused them) which need to go just right for any person to exist. However, if those events don't happen just right then they will not exist. This means that if we choose to take a different route than the one which will result in a person, their `good life points' would be 0, rather than whatever they would have been had we chosen to take the other route which would have made them. So long as their life is better than 0, they could not have been made worse by the choice. Take this as an example:
\factoidbox{\noindent  \fontsize{20pt}{0pt}\textbf{Bio-dome Bob}

Bob is a person living on Earth in a bio-dome in the far off future, breathing recycled air, unable to go outside because the O-zone layer has long since been whipped out. He eats rationed food because farming is impossible in this environment. Earth has essentially become Venus. Bob’s life is not the best, but there are enough pleasures to make it better than non-existing. In this future, the past generations chose to continue with the non-green policies, not caring about global warming or the environmental effects. Bob’s ancestors, in particular, were coal-mine executives, who were able to ensure that their kids meet certain people because of the profits. Bob could not have existed if the green policies were enacted.}

We can give Bob's life a score, this is an arbitrary choice which we can use as a metric for later, Bob's life has 10 good-life points. In order to exist, the previous generations needed to not enact those green policies. If the previous generations had gone with the green policies, then Bob's good-life points would have been 0 (because he would not have existed). So, since 10 > 0, Bob's life is not worse than it would have been otherwise. As more generations take place, I can generalize this to all people in the future because of Chaos Theory and things like the Butterfly Effect. This means that for an action which makes a person who couldn't have been otherwise and has a life better than non-existence, can't have made that person worse off than otherwise. 
There are some actions which bring such a being into existence and are still wrong.

This is our third line to the paradox and it seems like the most intuitive of the bunch, but maybe less intuitive than the first line. For this one, we can look at several examples to justify it. But, really, it's just the underlying intuition behind things like wanting to prevent global warming and why we would think that the previous generations in the Bio-dome Bob case did something wrong. To start on potential ways out of this paradox, take this case which is based on my Bio-dome Bob case:

\factoidbox{\noindent  \fontsize{20pt}{0pt}\textbf{Sunshine Sally}

Sally is a person living on Earth in the forest 100 years from now, breathing air from the trees, happily playing outside. She eats food grown from her garden and can drink the water from the tap. Sally’s life is awesome. In this future, the past generations chose to change their policies to green ones, caring deeply about the environmental effects and global warming.  Sally’s ancestors were pioneering wind-farmers, the move to green policies caused them to make a bunch of money and be able to send their kids to different places and meet different people.}

We can say that Sally's life is, roughly, 10x better than Bob's. If you were to choose which life you would like to be born into, you would choose one like Sally's over one like Bob's. In this case, we will give Sally's life a score of 100 good-life points. But, same as Bob, if the previous generation had chosen differently, she would not have existed. And in this case, she would have had a good-life score of 0, because she was never born. 

\section{Two Potential Ways Out (there are problems with each)}

If we take Sunshine Sally and Bio-dome Bob as our token cases of people in these two far distant futures, we can’t say that the people individually were worse off than otherwise, we can only say that they are better off than otherwise. But, which do we choose? Neither is wrong (given our situation). The trick is to say that what makes something wrong is not, necessarily, making an individual worse, but rather making the collection, the group, worse. Sometimes this is a group of only one person, in which case it’s the same individualistic intuition, other times it’s more than one. So, for this, we are going to reject the first line of the paradox, but not become non-consequentialists, rather say that we need to take the collective well-being. 
\subsection{Averagism}

One way is to take the overall average of the future people and compare the averages. In Sunshine Sally’s world, the average person has a better life than the average person in Bio-dome Bob’s world, so we can say that this abstract, average person, is worse off than otherwise and that makes the choice the wrong one. But, there is an issue with this sort of account for the outcomes, although it works for cases like Bob's and Sally's, it gets the wrong answer for more down-to-Earth cases, like those concerning inequalities. 

\factoidbox{\noindent  \fontsize{20pt}{0pt}\textbf{Great Inequality}
    
Suppose that the average across the board for life-points is 100 (Sally's life) in some future, but this is because the top 1 percent of the population has all the points, while the lower groups have very few, if any, points, maxing out at 10 (Bob's life). But it’s a far worse world than one where everyone in fact had 100 (Sally's world). To drive this home, imagine that you needed to choose which world you would want to be born into, by chance. Would you take the bet on a world with  a 1\% chance of having a good life or a pretty close to 100\% chance of having a good life?\footnote{This case/thought experiment is a variation on John Rawls' Veil of Ignorance thought experiment. I generalized his thought experiment to concern future generations.}}

The vast majority of people, if they are sensible, would not choose to take the 1\% bet, that's just crazy. So, we can say that Sally's world is better. But, at the same point, what if I just increased the majority's max by just a few points? This would increase the average by a tiny amount, but make it better than Sally's world. This is an issue because a sensible person still would not take that bet. 

\subsection{Totalism}

Another way we could go about this is to have it just be the net total of the people in each world and say that the abstract group “Future People” is better or worse off because of that. In Sunshine Sally’s world, the total is higher than the total in Bio-dome Bob’s world, so, that’s what makes it wrong. This is called Totalism. But this one too does have its issues as well. For this case, the more people in the world, the better the world would look from an outsider's perspective, even though the individual lives in it are far worse.

\factoidbox{\noindent  \fontsize{20pt}{0pt}\textbf{Large vs Small Population}

Imaging that you have two possible futures, one where A) there are 1 million people each with 100 good-life points (a million Sallys) and another where B) there are 10 million people each with 10 good-life points (10 million Bobs). By totalism, we could not tell the difference between these two futures, and, in fact, if I changed B so that everyone had 20 good life points, totalism would claim that it’s the better world, even though everyone, individually, is worse off than in A.}

As before, we would not want to take this bet, if I were to give you a choice to gamble on where to be born and I told you to choose which of these two worlds you would `roll the dice' in, you would certainly choose A, even if I made B have everyone with 20 points rather than 10. This is the sort of issue which we saw with the Utility Monster Objection in the past.