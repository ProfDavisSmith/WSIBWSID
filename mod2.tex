\part{Moral Objectivism (Non-Relativism)}
\label{ch.mod2}
\addtocontents{toc}{\protect\mbox{}\protect\hrulefill\par}
\chapter{Part 6: Objectivism and Absolutism}
Since there are some very serious problems with Relativism, philosophers and others have moved to a different way of thinking about Ethics. They asked themselves "what if ethics is absolute? What if there are certain things which are wrong regardless of the culture?" This gives us \gls{Moral Objectivism} (not to be confused with Ayn Rand's Objectivism, which should just be called Stuff-ism, we will cover this when we cover Ethical Egoism). Moral Objectivism is a meta-ethical stance, like Moral Relativism. It says that there are absolute, objective, moral truths. These truths apply to everything, like the truths in mathematics and the other sciences.  I personally prefer the term ‘objectivism’ rather than ‘absolutism’ for this subject (all of them actually), but there is a slight difference. Like the other stances, it can be divided into global and limited. However, it does not need to be further broken down from there. For this class, we are focusing on moral objectivism. This is the stance that there are absolute, objective, rules about morality. They aren’t relative and apply to ALL people. When we make a claim like ‘lying is wrong’, we are saying something which is either true or false and whether it is true or false is not relative.

\newglossaryentry{Objectivism}
{
  name=Objectivism,
  description={The stance that there are absolute, objective facts about a given topic.}
}

\newglossaryentry{Moral Objectivism}
{
  name=Moral Objectivism,
  description={The stance that there are some absolute, objective facts about morality, right and wrong.}
}


Moral Objectivism could be divided into two different, closely related, stances. One could be a \gls{Moral Absolutism}. This is the stance that the moral truths, the rules, are objective but it goes one step further and claims that the moral truths are without exceptions. They are absolute. There are many moral rules which we apply in our daily lives. One should always obey the law, one should not lie, one should not steal, one should not cheat, and one should not harm others. If moral absolutism is correct, then these rules (if they are the real moral rules) are without exceptions. There is never a case where lying is the morally right thing to do, for example. For example, if the claim “killing is wrong” is true, then it is true regardless of context. So, killing one person to save 5 is wrong. 

\newglossaryentry{Moral Absolutism}
{
  name=Moral Absolutist,
  description={The stance that there are absolute, objective moral rules which are exceptionless and absolute. For any, if they hold in one case, they hold in all cases}
}


Moral Situationalism is different. This holds that moral truths are objective, but they apply differently according to context/situation. \gls{Moral Situationalism} is happy to say that most of the time the moral rules we apply in our daily lives are accurate, but they are not absolute or without exceptions. If we take the previous list for examples, Moral Situationalism is willing to say that most of the time one should obey the laws of the land but there are or could be cases where the laws are unjust and should not be followed. If obeying the law were an absolute moral rule, then it would be impossible for a law to be unjust. So either moral absolutism is not correct or obeying the law is not an absolute moral rule. Typically, a situationalist theory has at least one absolute moral rule involved, but this is typically very abstract and not all that useful in a practical situation. It’s used to determine which rules apply in a context. So, for example, most of the time, stealing is wrong, but there could be cases where it’s the right thing to do because the rule doesn’t apply. For another example, lying to your spouse about what you did over the weekend is wrong but lying to an axe-murderer about the whereabouts of your friend is right. Situationalism can be seen in other fields aside from Ethics. If a little kid points to a silver car and says that it’s silver, they would right. But if they point to a different red car and say that it’s silver, they would be wrong. The situation, the car in this case, is different so whether the kid is right could be different. Theories about what makes the situation different are numerous and some, but not all, are quite intuitive, but this is not a class about Philosophy of Language.

\newglossaryentry{Moral Situationalism}
{
  name=Moral Situationalism,
  description={The stance that there are objective moral rules but they apply differently in different situations. There is typically an absolute rule which determines whether the more practical ones apply}
}


There are several reasons to be a moral objectivist aside from getting out of the problems with Relativism and still being able to claim that there are certain things which you shouldn't do. But those problems are negative arguments. They say "well Relativism is out, so you must go with Objectivism." This doesn't really convince most people, however, there are some tests to tell whether you actually, in your heart of hearts, think that a subject is relative. These are our more positive arguments, our evidence that people really think morals are objective.

\chapter{Why I am an Objectivist about Ethics (And Why You Are, Too) by David Enoch }\autocite{Enoch1}

You may think that you're a moral relativist or subjectivist - many people today seem to. 
But I don't think you are. In fact, when we start doing metaethics - when we start, that 
is, thinking philosophically about our moral discourse and practice - thoughts about 
morality's objectivity become almost irresistible. Now, as is always the case in 
philosophy, that some thoughts seem irresistible is only the starting point for the 
discussion, and under argumentative pressure we may need to revise our relevant 
beliefs. Still, it's important to get the starting points right. So it's important to 
understand the deep ways in which rejecting morality's objectivity are unappealing. 
What I want to do, then, is to highlight the ways in which accepting morality's objectivity 
is appealing, and to briefly address some common worries about it, worries that may 
lead some to reject - or to think they reject - such objectivity. In the final section, I 
comment on the (not obvious) relation between the underlying concerns about 
morality's objectivity and the directions in which current discussion in metaethics are 
developing. As it will emerge, things are not (even) as simple as the discussion below 
seems to suggest. This is just one reason why metaethics is so worth doing. 
 
 \section{Why Objectivity? Three (Related) Reasons} 
In the next section we're going to have to say a little more about what objectivity is. But 
sometimes it's helpful to start by engaging the underlying concerns, and return to more 
abstract, perhaps conceptual, issues later on.
\subsection{The Spinach Test} 
Consider the following joke (which I borrow from Christine Korsgaard): A child hates 
spinach. He then reports that he's glad he hates spinach. To the question "Why?" he 
responds: "Because if I liked it, I would have eaten it; and it's yucky!". 

In a minute we're going to have to annoyingly ask why the joke is funny. For 
now, though, I want to highlight the fact that similar jokes are not always similarly 
funny. Consider, for instance, someone who grew up in the twentieth-century West, and 
who believes that the earth revolves around the sun. Also, she reports to be happy she 
wasn't born in the Middle Ages, "because had I grown up in the Middle Ages, I would 
have believed that the earth is in the center of the universe, and that belief is false!".  
To my ears, the joke doesn't work in this latter version (try it on your friends!). The 
response in the earth-revolves-around-the-sun case sounds perfectly sensible, precisely 
in a way in which the analogous response does not sound sensible in the spinach case. 

We need one last case. Suppose someone grew up in the US in the late 
twentieth century, and rejects any manifestation of racism as morally wrong. He then 
reports that he's happy that that's when and where he grew up, "because had I grown 
up in the 18th century, I would have accepted slavery and racism. And these things are 
wrong!" How funny is this third, last version of the joke? To my ears, it's about as 
(un)funny as the second one, and nowhere nearly as amusing as the first. The response 
to the question in this last case (why he is happy that he grew up in the 20th century) 
seems to me to make perfect sense, and I suspect it makes sense to you too. And this is 
why there's nothing funny about it.  

OK, then, why is the spinach version funny and the others are not? Usually, our 
attitude towards our own likings and dislikings (when it comes to food, for instance) is 
that it's all about us. If you don't like spinach, the reason you shouldn't have it is 
precisely that you don't like it. So if we're imagining a hypothetical scenario in which you 
do like it, then you no longer have any reason not to eat it. This is what the child in the 
first example gets wrong: He's holding fixed his dislike for spinach, even in thinking 
about the hypothetical case in which he likes spinach. But because these issues are all 
about him and what he likes and dislikes, this makes no sense.  

But physics is different: What we want, believe or do – none of this affects the 
earth’s orbit. The fact that the earth revolves around the sun is just not about us at all. 
So it makes sense to hold this truth fixed even when thinking about hypothetical cases 
in which you don't believe it. And so it makes sense to be happy that you aren’t in the 
Middle Ages, since you’d then be in a situation in which your beliefs abut the earth’s 
orbit would be false (even if you couldn’t know that it is). And because this makes sense, 
the joke isn't funny.  

And so we have the spinach test: About any relevant subject matter, formulate 
an analogue of the spinach joke. If the joke works, this seems to indicate that the 
subject matter is all about us and our responses, our likings and dislikings, our 
preferences, and so on. If the joke doesn't work, the subject matter is much more 
objective than that, as in the astronomy case. And when we apply the spinach test to 
moral issue (like the moral status of racism), it seems to fall squarely on the objective 
side.   
(Exercise: Think about your taste in music, and formulate the spinach test for it. Is the 
joke funny?) 
 
\subsection{Disagreement and Deliberation} 
We sometimes engage in all sorts of disagreements. Sometimes, for instance, we may 
engage in a disagreement about even such silly things as whether bitter chocolate is 
better than milk chocolate. Sometimes we disagree about such things as whether 
human actions influence global warming. But these two kinds of disagreement are very 
different. One way of seeing this is thinking about what it feels like from the inside to 
engage in such disagreements. In the chocolate case, it feels like stating one's own 
preference, and perhaps trying to influence the listener into getting his own preferences 
in line. In the global warming case, though, it feels like trying to get at an objective truth, 
one that is there anyway, independently of our beliefs and preferences. (Either human 
actions contribute to global warming, or they don't, right?) 

And so another test suggests itself, a test having to do with what it feels like to 
engage in disagreement (or, as we sometimes say, with the phenomenology of 
disagreement).  

But now think of some serious moral disagreement - about the moral status of 
abortion, say. Suppose, then, that you are engaged in such disagreement. (It's important 
to imagine this from the inside, as it were - don't imagine looking from the outside a
two people arguing over abortion; think what it's like to be engaged in such argument 
yourself, if not about abortion, then about some other issue you care deeply about). 
Perhaps you think that there is nothing wrong with abortion, and you're arguing with 
someone who thinks that abortion is morally wrong. What does such disagreement feel 
like? In particular, does it feel more like disagreeing over which chocolate is better, or 
like disagreeing over factual matters, like whether human actions contribute to global 
warming?  

Because this question is a phenomenological one (that is, it's about what 
something feels like from the inside), I can't answer this question for you. You have to 
think about what it feels like for you when you are engaged in moral disagreement. But I 
can say that in my case such moral disagreement feels exactly like the one about global 
warming - it's about an objective matter of fact, that exists independently of us and our 
disagreement. It is in no way like disagreeing over the merits of different kinds of 
chocolate. And I think I can rather safely predict that this is how it feels for you too.  
So on the phenomenology-of-disagreement test as well, morality seems to fall on the 
objective side.  

In fact, we may be able to take disagreement out of the picture entirely. Suppose 
there is no disagreement - perhaps because you're all by yourself trying to make your 
mind about what to do next. In one case, you're thinking about what kind of chocolate 
to get. In another, you're choosing between buying a standard car and a somewhat 
more expensive hybrid car (whose effect on global warming, if human actions 
contribute to global warming, is less destructive). Here too there's a difference: In the 
first case, you seem to be asking questions about yourself and what you like more (in 
general, or right now). In the second case, you need to make up your mind about your 
own action, of course, but you're asking yourself questions about objective matters of 
fact that do not depend on you at all - in particular, about whether human actions affect 
global warming.  

Now consider a third case, in which you're tying to make up your mind about 
having an abortion, or advising a friend who is considering an abortion. So you're 
wondering whether abortion is wrong. Does it feel like asking about your own 
preferences, or like an objective matter of fact? Is it more like the chocolate case or like 
the hybrid car case? If, like me, you answer that it's much more like the hybrid car case, 
then you think, like me, that the phenomenology of deliberation too indicates that 
morality is objective.   

(Exercise: think about your taste in music again. In terms of the phenomenology of 
disagreement and deliberation, is it on the objective side?) 
 
\subsection{Would It Still Have Been Wrong If...?} 
Top hats are out of fashion. This may be an interesting, perhaps even practically 
relevant, fact - it may, for instance, give you reason to wear a top hat (if you want to be 
special) or not to (if not). But think about the following question: Had our fashion 
practices been very different - had we all worn top hats, thought they were cool, and so 
on - would it still have been true that top hats are out of fashion? The answer, it seems 
safe to assume, is "no".

Smoking causes cancer. This is an interesting, practically relevant, fact - it most 
certainly gives you a reason not to smoke, or perhaps to stop smoking. Now, had our 
relevant practices and beliefs regarding smoking been different - had we been ok with it, 
had we not banned it, had we thought smoking was actually quite harmless - would it 
still have been true that smoking causes cancer? I take it to be uncontroversial that the 
answer is "yes". The effects of smoking on our health do not depend on our beliefs and 
practices in anything like the way in which the fashionability of top hats does. Rather, it 
is an objective matter of fact.  

And so we have a third objectivity test: One in terms of the relevant “what if” 
sentences (or \emph{counterfactuals}, as they are often called), such as "Had our beliefs and 
practices been very different, would it still have been true that so-and-so?". Let's apply 
this test to morality, then.  

Gender-based discrimination is wrong. I hope you agree with me on this (if you 
don't, replace this with a moral judgment you're rather confident in). Would it still have 
been wrong had our relevant practices and beliefs been different? Had we been all for 
gender-based discrimination, would that have made gender-based discrimination 
morally acceptable? Of course, in such a case we would have believed that there's 
nothing wrong with gender-based discrimination. But would it be wrong? To me it 
seems very clear that the answer is "Yes!" Gender-based discrimination is just as wrong 
in a society where everyone believes it's morally permissible. (This, after all, is why we 
would want such a society to change, and why, if we are members, we would fight for 
reform.) The problem in such a society is precisely that its members miss something so
important - namely, the wrongness of gender-based discrimination. Had we thought 
gender-based discrimination was okay, we would have been mistaken. The morality of 
such discrimination does not depend on our opinion of it. The people in that 
hypothetical society may accept gender-based discrimination, but that doesn’t make 
such discrimination acceptable.  

In this respect too, then, morality falls on the objective side. When it comes to 
the counterfactual test, moral truths behave more like objective, factual truths (as 
whether smoking causes cancer) than like purely subjective, perhaps conventional 
claims (say, that top hats are unfashionable).  
(Exercises: Can you see how the counterfactual test relates to the spinach test? And 
think about your favorite music, the kind of music that you don't just like, but that you 
think is good. Had you not liked it, would it still have been good?) 
 
\section{What's At Issue?}  
We have, then, three tests for objectivity - the spinach test, the phenomenology-of-
disagreement-and-deliberation test, and the counterfactual test. And though we haven't 
yet said much about what objectivity comes to, these tests test for something that is 
recognizably in the vicinity of what we're after with our term "objectivity". 

Objectivity, like many interesting philosophical terms, can be understood in 
more than one way. As a result, when philosophers affirm or deny the objectivity of 
some subject matter, it's not to be taken for granted that they're asserting or denying 
the same thing. But we don't have to go through a long list of what may be meant by
morality's objectivity. It will be more productive, I think, to go about things in a different 
way. We can start by asking - why does it matter whether morality is objective? If we 
have a good enough feel for the answer to this question, we can then use it to find the 
sense of objectivity that we care about.  

I suggest that we care about the objectivity of morality for roughly the reasons 
specified in the previous section: We want morality's objectivity to support our 
responses in those cases. We want morality's objectivity to vindicate the 
phenomenology of deliberation and disagreement, and our relevant counterfactual 
judgments. We want morality’s objectivity to explain why the moral analogue of the 
spinach test isn’t funny.  

Very well, then, in what sense must morality be objective, for the 
phenomenology of disagreement and deliberation and our counterfactual judgments to 
be justified? The answer, it seems to me, is that a subject matter is objective, if the 
truths or facts in it exist independently of what we think or feel about them.  

This notion of objectivity nicely supports the counterfactual test. If a certain 
truth (say, that smoking causes cancer) doesn't depend on our views about it, then it 
would have been true even had we not believed it. Not so for truths that do depend on 
our beliefs, practices, emotions (such as the truth that top hats are unfashionable). And 
if moral truths are similarly independent of our beliefs, desires, preferences, emotions, 
points of view, and so on - if, as is sometimes said, moral truths are \emph{response-independent} – then it's clear why gender-based discrimination would have been wrong 
even had we approved of it.

Similarly, if it's our responses that make moral claims true, then in a case of 
disagreement, it seems natural to suppose that both sides may be right. Perhaps, in 
other words, your responses make it the case that abortion is morally permissible ("for 
you", in some sense of this very weird phrase?), and your friend's responses make it the 
case that abortion is morally wrong ("for her"?). But if the moral status of abortion is 
response-independent, we understand why moral disagreement feels like factual 
disagreement - one is right, one is wrong, and it's important to find out who. And of 
course, the whole point of the spinach test was to distinguish between caring about 
things just because we care about them (such as not eating spinach, if you find it yucky), 
and caring about things that seem to us important independently of us caring about 
them (such as the wrongness of racism).  

Another way of making the same point is as follows: Objective facts are those we 
seek to discover, not those we make true. And in this respect too, when it comes to 
moral truths, we are in a position more like that of the scientist who tries to discover the 
laws of nature (which exist independently of her investigations), than that of the 
legislator (who creates laws).  

Now, in insisting that morality is objective in this sense - for instance, by relying 
on the reasons given in the previous section - it's important to see what has and what 
has not been established. In order to see this, it may help to draw an analogy with 
religious discourse. So think of your deeply held religious beliefs, if you have any. (If, like 
me, you do not, try to think what it's like to be deeply committed to a religious belief, or 
perhaps think of your commitment to atheism). And try to run our tests - does it make
sense to be happy that you were brought up under the religion in which you deeply 
believe, even assuming that with a different education you would have believed another 
religion, or no religion at all?  What do you think of the phenomenology of religious 
deliberation and disagreement? And had you stopped believing, would the doctrines of 
your faith still have been true?  

Now, perhaps things are not obvious here, but it seems to me that for many 
religious people, religious discourse passes all these objectivity tests. But from this it 
does not follow that atheism is false, much less that a specific religion is true. When 
they are applied to some specific religious discourse, the objectivity tests show that such 
discourse aspires to objectivity. In other words, the tests show what the world must be 
like for the commitments of the discourse to be vindicated: If (say) a Catholic's religious 
beliefs are to be true, what must be the case is that the doctrines of the Catholic Church 
hold objectively, that is, response-independently. This leaves entirely open the question 
whether these doctrines do in fact hold.  

Back to morality, then. Here too, what the discussion of objectivity (tentatively) 
establishes is just something about the aspirations of moral discourse – namely, that it 
aspires to objectivity. If our moral judgments are to be true, it must be the case that 
things have value, that people have rights and duties, that there are better and worse 
ways to live our lives - and all of this must hold objectively, that is, response-
independently. But establishing that moral discourse aspires to objectivity is one thing. 
Whether there actually are objective moral truths is quite another.

And now you may be worried: Why does it matter, you may wonder, what 
morality's aspirations are, if (for all I’ve said so far) they may not be met? I want to offer 
two replies here. First, precisely in order to check whether morality’s aspirations are in 
fact fulfilled, we should understand them better. If you are trying to decide, for instance, 
whether the commitments of Catholicism are true, you had better understand them 
first. Second, and more importantly, one of the things we are trying to do here is to gain 
a better understanding of what we are already committed to. You may recall that I 
started with the hypothesis that you may think you're a relativist or a subjectivist. But if 
the discussion so far gets things right (if, that is, morality aspires to this kind of 
objectivity), and if you have any moral beliefs at all (don't you think that some things are 
wrong? Do we really need to give gruesome examples?), then it follows that you 
yourself are already committed to morality's objectivity. And this is already an 
interesting result, at least for you. 

That morality aspires in this way to objectivity also has the implication that any 
full metaethical theory - any theory, that is, that offers a full description and explanation 
of moral discourse and practice - has to take this aspiration into account. Most likely, it 
has to accommodate it. Less likely, but still possibly, such a theory may tell us that this 
aspiration is futile, explaining why even though morality is not objective, we tend to 
think that it is, why it manifests the marks of objectivity that the tests above catch on, 
and so on. What no metaethical theory can do, however, is ignore the very strong 
appearance that morality is objective. I get back to this in the final section, below. 

\section{Why Not?}

As I already mentioned, we cannot rule out the possibility that under argumentative 
pressure we're going to have to revise even some of our most deeply held beliefs. 
Philosophy, in other words, is hard. And as you can imagine, claims about morality's 
objectivity have not escaped criticism. Indeed, perhaps some such objections have 
already occurred to you. In this section, I quickly mention some of them, and hint at the 
ways in which I think they can be coped with. But let me note how incomplete the 
discussion here is: There are, of course, other objections, objections that I don't discuss 
here. More importantly, there are many more things to say - on both sides - regarding 
the objections that I do discuss. The discussion here is meant as an introduction to these 
further discussions, no more than that. (Have I mentioned that philosophy is hard?) 
 
\subsection{Disagreement} 
I have been emphasizing ways in which moral disagreement may motivate the thought 
that morality is objective. But it's very common to think that something about moral 
disagreement actually goes the other way. For if there are perfectly objective moral 
truths, why is there so much disagreement about them? Wouldn't we expect, if there 
are such objective truths, to see everyone converging on them? Perhaps such 
convergence cannot be expected to be perfect and quick, but still - why is there so much 
persistent, apparently irreconcilable disagreement in morality, but not in subject 
matters whose objectivity is less controversial? If there is no answer to this question, 
doesn't this count heavily against morality's objectivity? 

It is not easy to see exactly what this objection comes to. (Exercise: Can you try 
and formulate a precise argument here?) It may be necessary to distinguish between 
several possible arguments. Naturally, different ways of understanding the objection 
will call for different responses. But there are some things that can be said in general 
here. First, the objection seems to underrate the extent of disagreement in subject 
matters whose objectivity is pretty much uncontroversial (think of the causes and 
effects of global warming again). It may also overrate the extent of disagreement in 
morality. Still, the requirement to explain the scope and nature of moral disagreements 
seems legitimate. But objectivity-friendly explanations seem possible.  

Perhaps, for instance, moral disagreement is sometimes best explained by noting 
that people tend to accept the moral judgments that it's in their interest to accept, or 
that tend to show their lives and practices in good light. Perhaps this is why the poor 
tend to believe in the welfare state, and the rich tend to believe in property rights.  
Perhaps the most important general lesson here is that not all disagreements 
count against the objectivity of the relevant discourse. So what we need is a criterion to 
distinguish between objectivity-undermining and non-objectivity-undermining 
disagreements. And then we need an argument showing that moral disagreement is of 
the former kind. I don't know of a fully successful way of filling in these details here.  

Notice, by the way, that such attempts are going to have to overcome a natural 
worry about self-defeat. Some theories defeat themselves, that is, roughly, fail even by 
their own lights. Consider, for instance, the theory “All theories are false”, or the belief 
“No belief is justified”. (Exercise: Can you think of other self-defeating theories?). Now,
disagreement in philosophy has many of the features that moral disagreement seems to 
have. In particular, so does metaethical disagreement. Even more in particular, so does 
disagreement about whether disagreement undermines objectivity. If moral 
disagreement undermines the objectivity of moral conclusions, metaethical 
disagreement seems to undermine the objectivity of metaethical conclusions, including 
the conclusion that disagreement of this kind undermines objectivity. And this starts to 
look like self-defeat. So if some disagreement-objection to the objectivity of morality is 
going to succeed, it must show how moral disagreement undermines the objectivity of 
morality, but metaethical disagreement does not undermine the objectivity of 
metaethical claims. Perhaps it's possible to do so. But it's not going to be easy. 

\subsection{But How Do We Know?} 
Even if there are these objective moral truths - for instance, the kind of objective moral 
truth that both sides to a moral disagreement typically lay a claim to - how can we ever 
come to know them? In the astronomical case of disagreement about the relative 
position and motion of the earth and the sun, there are things we can say in response to 
a similar question - we can talk about perception, and scientific methodology, and 
progress. Similarly in other subject matters where we are very confident that objective 
truths await our discovery. Can anything at all be said in the moral case? We do not, 
after all, seem to possess something worth calling moral perception, a direct perception 
of the moral status of things. And in the moral case it's hard to argue that we have an
established, much less uncontroversial, methodology either. (Whether there is moral 
progress is, I'm sure you've already realized, highly controversial.) 

In other words, what we need is a moral epistemology, an account of how moral 
knowledge is possible, of how moral beliefs can be more or less justified, and the like. 
And I do not want to belittle the need for a moral epistemology, in particular, an 
epistemology that fits well with an objectivist understanding of moral judgments. But 
the objectivist is not without resources here. After all, morality is not the only subject 
matter where perception and empirical methodology do not seem to be relevant. Think, 
for instance, of mathematics, and indeed of philosophy. But we do not often doubt the 
reality of mathematical knowledge (philosophical knowledge is a harder case, perhaps; 
but, Exercise: can you see how claiming that we do not have philosophical knowledge 
may again give rise to a worry about self-defeat?). 

Perhaps, then, what is really needed is a general epistemology of the a priori - of 
those areas, roughly, where the empirical method seems out of place. And perhaps it's 
not overly optimistic to think that any plausible epistemology of the a priori will 
vindicate moral knowledge as well.  

Also, to say that there is no methodology of doing ethics is at the very least an 
exaggeration. Typically, when facing a moral question, we do not just stare at it 
helplessly. Perhaps we're not always very good at morality. But this doesn't mean that 
we never are. And perhaps at our best, when we employ our best ways of moral 
reasoning, we manage to attain moral knowledge.  

(Exercise: There is no uncontroversial method of doing ethics. What, if anything, follows 
from this?) 
 
\subsection{Who Decides?} 
Still, even if moral knowledge is not especially problematic, even if moral disagreement 
can be explained in objectivity-friendly ways, and even if there are perfectly objective 
moral truths, what should we do in cases of disagreement and conflict? Who gets to 
decide what the right way of proceeding is? Especially in the case of inter-cultural 
disagreement and conflict, isn't saying something like "We're right and you're wrong 
about what is objectively morally required" objectionably dogmatic, intolerant, perhaps 
an invitation to fanaticism? 
 
Well, in a sense, no one decides. In another sense, everyone does. The situation 
here is precisely as it is everywhere else: No one gets to decide whether smoking causes 
cancer, whether human actions contribute to global warming, whether the earth 
revolves around the sun. Our decisions do not make these claims true or false. But 
everyone gets (roughly speaking) to decide what they are going to believe about these 
matters. And this is so for moral claims as well.  

How about intolerance and fanaticism? If the worry is that people are likely to 
become dangerously intolerant if they believe in objective morality, then first, such a 
predictions would have to be established. After all, many social reformers (think, for 
instance, of Martin Luther King, Jr.) who fought against intolerance and bigotry seem to 
have been inspired by the thought that their vision of equality and justice was
objectively correct. Further, even if it's very dangerous for people to believe in the 
objectivity of their moral convictions, this doesn't mean that morality isn’t objective. 
Such danger would give us reasons not to let people know about morality's objectivity. 
It would not give us a reason to believe that morality is not objective. (Compare: even if 
it were the case that things would go rapidly downhill if atheism were widely believed, 
this wouldn’t prove that atheism is false.) 

More importantly, though, it's one thing to believe in the objectivity of morality, it's 
quite another to decide what to do about it. And it's quite possible that the right thing 
to do, given morality's objectivity, is hardly ever to respond with "I am simply right and 
you are simply wrong!", or to be intolerant. In fact, if you think that it's wrong to be 
intolerant, aren't you committed to the objectivity of this very claim? (Want to run the 
three tests again?) So it seems as if the only way of accommodating the importance of 
toleration is actually to accept morality's objectivity, not to reject it. 

\section{Conclusion}
As already noted, much more can be said - about what objectivity is, about the reasons 
to think that morality is objective, and about these (and many other) objections to 
morality's objectivity. Much more work remains to be done.  

And one of the ways in which current literature addresses some of these issues 
may sound surprising, for a major part of the debate assumes something like morality's 
aspiration to objectivity in the sense above, but refuses to infer from such observations 
quick conclusions about the nature of moral truths and facts. In other words, many
metaethicists today deny the most straightforward objectivist view of morality - 
according to which moral facts are a part of response-independent reality, much like 
mathematical and physical facts. But they do not deny morality's objectivity - they care, 
for instance, about passing the three tests above. And so they attempt to show how 
even on other metaethical views, morality's objectivity can be accommodated. As you 
can imagine, philosophers disagree about the success (actual and potential) of such 
accommodation projects.  

Naturally, such controversies also lead to attempts to better understand what 
the objectivity at stake exactly is, and why it matters (if it matters) whether morality is 
objective. As is often the case, attempts to evaluate answers to a question make us 
better understand - or wonder about - the question itself.  

Nothing here, then, is simple. But I hope that you now see how you are probably 
a moral objectivist, at least in your intuitive starting point. Perhaps further philosophical 
reflection will require that you abandon this starting point. But this will be an 
abandoning, and a very strong reason is needed to justify it. Until we get such a 
conclusive argument against moral objectivity, then, objectivism should be the view to 
beat.
\chapter{Part 7: The Tests for Objectivity}
\section{The Spinach Test For Objectivity}

This is a pretty straight forward test for whether you actually are a moral relativist. But, as a fair warning, philosophy kills jokes. As an example, and where the test gets its name, lets look at the following joke:

    \factoidbox{A kid hates spinach, later, he says that he’s glad he hates spinach. When asked why, he says “because if I liked it, I would eat it; and it’s yucky!”}

More often than not, we will take this as funny. And, in a lecture format, if I present it right, I will get a laugh. But, the test is not whether or not you laugh at this joke, but rather whether you laugh at jokes of this form/structure, with different things put in the place of 'spinach'. Whether you laugh is the answer to whether you think the topic is relative. You can take the joke and put it into this general form:

\factoidbox{A person believes X. Later, they say that they are glad/happy that they believe X. When asked why, they say "if I believed otherwise then I would Y, but that's just wrong/yucky/boring.}

For example, let's try it with entertainment and other foods:
\factoidbox{
    A kid hates watching golf. He says that he’s glad he hates watching golf because “if I liked it, I would watch it, and it’s boring!”

    A metal-head says that he’s glad he hates country. When asked why, he replies ‘because if I liked country, I would listen to country, and that’s just bad music.’

    George H.W. Bush hates broccoli and says that he’s glad he hates broccoli. The reason? “Because if I liked broccoli, I would eat it, and it’s gross.”}

These are the same, if presented right, they will get a laugh. But, will this joke work for anything which I plug into the joke's formula? What if I plug in something which is more absolute? Take these examples:
\factoidbox{
    A 20th century man believes that the Earth is round. He claims that he’s glad that he’s a 20th century man because “If I grew up in the first century, I would have believed that the Earth is flat, and that's just wrong.”

    A 20th century man believes that women should have the right to vote. He claims that he’s glad that he’s a 20th century man because “If I grew up in the 19th century, I would have believed that women shouldn’t vote, and this is wrong.”

    An Iraqi woman who grew up in the 2000s believes that women are not equal to men. She claims that she’s glad about this because “if I grew up in the 60s and 70s, I would have believed that we were equal, and this is wrong.”}

There, people tend not to laugh, the joke doesn't work, so the topic in question (whether the earth is the center of the solar system) must be objective. Usually, our stances about liking or disliking something are totally up to us. If we dislike something, the reason we shouldn’t have it is exactly because we don’t like it. So if we're imagining a hypothetical scenario in which you do like it, then you no longer have any reason not to eat it. This is what the child in the first example gets wrong: He's holding fixed his dislike for spinach, even in thinking about the hypothetical case in which he likes spinach. But because these issues are all about him and what he likes and dislikes, this makes no sense. But physics is different: What we want, believe or do – none of this affects the earth’s orbit. The fact that the earth revolves around the sun is just not about us at all. So it makes sense to hold this truth fixed even when thinking about hypothetical cases in which you don't believe it. And so it makes sense to be happy that you aren’t in the Middle Ages, since you’d then be in a situation in which your beliefs abut the earth’s orbit would be false (even if you couldn’t know that it is). And because this makes sense, the joke isn't funny.

The Spinach Test is where we take the joke and change the thing the person believes. If it’s funny, then we have evidence that it is subjective, totally up to us. If it isn’t funny, then we have evidence that it is not subjective, but objective.

\section{The Disagreement Test For Objectivity}

Often, in real life, we get into disagreements with people. Sometimes these are serious, like whether refugees should be allowed in the country or whether global warming is occurring. Other times they are just silly, like whether broccoli is yucky or whether dark chocolate is better than milk. The second test for relativism about a subject concerns the nature of such disagreements. For this test, you need to  imagine, or actually get into, a disagreement about something. Ask yourself what the disagreement feels like, what are you using in the disagreement (are you calling on facts or preferences?). For example, take these examples:

\begin{tabular}{p{1.5in}|p{1.5in}}
Arguments A&Arguments B\\\hline
A disagreement about whether the Earth is flat.&A disagreement about whether pineapple should go on pizza.\\
\hline
A disagreement about some mathematical principle.&A disagreement about how enjoyable Buffy the Vampire Slayer was.\\
\hline
A disagreement about whether refugees should be allowed in the country.&A disagreement about the best Metal band.\\
\hline 
A disagreement about whether slavery was bad.&A disagreement about the direction of a toilet paper roll.\\
\hline
A disagreement about whether God exists.&A disagreement about whether dark chocolate is better than milk.\\
\end{tabular}

These disagreements are very different. Think about what it’s like to be in those kinds of disagreements. In the case of broccoli or chocolate, you are trying to state your own preference and then, maybe, trying to get the listener to come to their senses and change theirs. However, when we are engaged in a disagreement about, say, global warming, the point of the argument is not to change the other person’s mind or to state how you feel. Rather, it is to get at the truth. Looking above at the examples, do the ones in the left column feel the same? Do they all feel like facts? Similarly, do the ones in the right column all feel the same? Do they feel like preferences? If the feeling of the disagreements in the left column all feel the same, then you are an objectivist about morality, in your heart. You can also ask yourself what are the things you are using in the disagreement, are you calling on your preferences or are you calling on the facts about the world? If, when you are having the disagreement, you are calling upon facts and/or data, then you think that the subject is objective, it's just not up to you. If you are calling upon preferences or feelings, then you think that it's relative or subjective. 

We can be even more particular about the case, in the sense of morality. If you are calling upon facts, then you think it's objective but if you are also calling upon facts in the particular case then you think the it's situational. So, for example, if you are in court for stealing a loaf of bread and you use facts about your individual situation to mitigate or relinquish the guilt, then you are a moral situationalist, because the fact or norm 'stealing is wrong' changes according to context/situation (with some more general principle determining whether it applies). If, on the other hand, your judge/prosecution use general facts and ignore the contextual factors, then they are an absolutist about stealing. Using another example, if you are in a disagreement about refugees being allowed into the country and say "no, absolutely not ever" or say "yes, always, let them pour in", then you hold that the rules about refugees being allowed into the country are absolute. If you say something like "in this particular situation, yes" or "in this case, no" or "it depends on who we are letting in", then you hold that the rules about refugees are situational. This becomes more obvious as the disagreement drags on and the sorts of facts you call upon to defend your stance.  

So, for this test about whether you are a relativist about some subject, you need to ask yourself what it feels like to be in a disagreement about the subject. Imagine that you are having a disagreement with someone about the morality of abortion. How does that feel? Does it feel like you are having a discussion about pineapple and green-olive pizza? Or does it feel like you are having a discussion about global warming?

Yet again, very rarely do we feel like disagreements about morality are disagreements about preference. When we have those debates, it feels like they are about facts.

\section{The Counterfactual (What-if) Test for Objectivity}

Like the other two, this one is a bit different. It involves what are called counterfactuals. They are called this because they are counter to the facts. We use these sort of statements all the time without realizing it. Any time you ask yourself what would happen if something were the case, you are using a counterfactual. These are insanely useful in philosophy as well as in science. There's an entire cottage industry in philosophy dedicated to coming up with how we use, accept, and reject these kinds of statements. That being said, since we use them so much, your gut intuition will suffice. For this test, we will present the subjects as yes-no 'what if' style questions. Such as, ‘what if I drove a 120MPH on a 60MPH highway, would that be dangerous?’ It is not true that I drive that fast, but I am trying to figure out what would be the case if I did. For this test, ask your self what it would be like if the vast majority of people believe opposite to you, would you still be right/correct? Here are some examples of what-if style questions which show this test at work.

\factoidbox{
If the vast majority of people believed that politicians were actually aliens, would they still be human?

If the vast majority of people believed that global warming wasn't happening, would it still?

If the vast majority of people thought that gender discrimination was fine, would it still be wrong? 

If the vast majority of people thought that the Earth was flat, would it still be round? 

If the vast majority of people thought that Philosophy was pointless, would it still be valuable? 

If the vast majority of people thought that pineapple on pizza was delicious, would it still be gross? 

If the vast majority of people wore top hats and thought they were cool, would they still be out of fashion?

If the vast majority of people thought that Buffy was the worst show ever, would it still be great?

If the vast majority of people thought that sweet potatoes were tasty, would they still be gross?

If the vast majority of people thought that spinach was gross, would it still be tasty?}

For each of these, there should be a yes or no answer, one which you can easily give. In some versions of this, I have presented it as "all people" rather than "the vast majority of people", but that has lead to some confusion. As before, the normal responses to the questions in the left column are all 'yes', while the answers to the questions in the right are normally 'no'. This test, if the question is formulated correctly, is the most definitive, I think. It points out the fact that objective truths are not up-to-us, they will be the case regardless of whether or not people discover them. 

The steps to preform this test on yourself are pretty straight forward. First, take a stance about something which you think is true. For example, you can have that slavery is bad, that pineapple on pizza is good, that the Kaelons shouldn't force their elderly to commit suicide, that the Earth is round, or something like that. Next, you imagine a case where the vast majority of people believe the opposite. And, finally, ask yourself whether, in this strange world, you would still be correct.  If the question is phrased correctly (it's possible to phrase them so that you will get, consistently, the opposite answer) and if you get the answer 'yes', then the subject matter is objective and if you get the answer 'no', then the subject is relative. So, for example, would top hats be out of fashion if the vast majority of people wore them and thought they looked cool? No, so fashion is relative. There was a time in this country where the vast majority of people thought that indigenous peoples were less than those of other descent, were they wrong? Yes, they were, so the morality about racism is objective. If the vast majority of people thought that Philosophy was pointless, critical thinking would decline and people would end up in a great intellectual regression. Those facts alone show that the value of philosophy isn't determined by what people believe about it, people can be mistaken about the value of a subject. 

\chapter{Part 8: Why Does It Matter? (Moral Skepticism)}

So far, for the three tests for objectivity, we have that morality is objective. We don't laugh at jokes about morality (at least serious cases of it), we feel like we are getting at facts when we disagree about morality, and ultimately, morality doesn't change with the whims of the majority (perspectives do and those could be more or less accurate). But, why would we want morality to be objective? And can we know about moral truths?

On its face, we want morality's objectivity to conform to our deeply held intuitions. We want it to explain why we don't laugh when jokes are made about it (it's too serious and not up to us). We want it to explain why we point to facts when we disagree about morality (moral facts are facts, not preferences, and you don't bring feelings to a fact fight). And we want it to explain why it doesn't change with the whims of the majority. Objectivity does all of those things, morality is just not up to us, we don't determine it based on our feelings. For example, whether the Earth revolves around the sun is not determined or changed by what people believe. Moral truths are independent to how we feel about them. We seek to discover moral facts, we don't invent them.

There are some other reasons we want morality to be objective, which relate to the objections against moral relativism. First, objective morality gives us a means to compare different cultures and people and then say how one is better or worse morally than another. We can use this same comparison to determine how one should improve. We can use it to look inward and determine how our culture or ourselves need to improve morally. We are not sheep to the views of our culture and Moral Objectivism gives us something beyond our culture to guide us. And third, Moral Objectivism gives us a standard by which to judge our progress, we can use this to figure out where we need to go and what we need to do morally to be better.  
How Do We Know?

As I said, we do seek to discover moral truths, but how do we know when we get them? The scientific method, for example, is a relatively new creation for discovering truths about the world and yet people have been trying to figure out morality for far longer than that. Discovering moral truths is hard, but not impossible. Those who hold that it is impossible are called \gls{Moral Skepticism}. Skepticism, in Philosophy, is the stance that knowledge about some subject is impossible. You could be a global skeptic or a limited skeptic, much like how you could be a global or limited relativist. It is worth noting that Moral Skeptics are not making a claim about whether there are moral truths, rather they are making a claim about whether we can know them, which is different. For example, there are certain numbers in mathematics which are so large that one can't possibly know what the first digit is, but there are still facts about what number it is.

\newglossaryentry{Moral Skepticism}
{
  name=Moral Skeptics,
  description={The stance that knowing whether an action is right or wrong is impossible}
}


People become moral skeptics for several different reasons. It should be noted that all of these are bad arguments and each can be easily pushed aside. They point to disagreements, empirical science, and degrees of certainty. To start off with, here is an argument against the idea we can know anything in Ethics using disagreements:

    \factoidbox{People and societies disagree about ethics. If it was possible to know the truth about this, then really smart people would come to it eventually. They haven’t. So, it must not be possible for us to know about it.}

This should remind you of the Cultural Differences Argument. The core intuition which is being used by those who pose this objection is that if we haven't found the answer to a question (yet), there must not be an answer to it. This is flawed on many different levels. First, right now we disagree about ethics, and if we can know about it, smart people will get to it someday, and, sure, we haven’t yet but that doesn’t mean it’s impossible. It’s just a really hard problem. For most of human history, people disagreed about what the stars were, but we now know what they are, it was a hard problem and we found ways to figure out the answer. Ethics is no different, for most of human history, we have disagreed about certain moral rules. Those rules will be settled, not because we get grinded down and all agree, but rather because we discover a moral truth. 

The next argument against the possibility of moral knowledge comes from a robust, strong, faith in the empirical sciences. Philosophy itself is a science, just not empirical. Here is a basic outline of the sort of argument used:
\factoidbox{
    The empirical sciences are supposed to give us knowledge about the world. They say nothing about ethics. So it must not be possible for us to know about it.}

For this one, there are a few different replies, and this is my preferred: The empirical sciences are SUPPOSED to give us knowledge about the world and you’re right, they don’t say anything about ethics. But, there’s more than one way of getting knowledge about the world and the empirical sciences are just one of the ways. The empirical sciences can't answer questions about the nature of God, can't answer questions about beauty, can't answer questions about truth (because that's meta to their fields), and they can't answer questions about morality. Empirical science is just one way of getting truths about the world, but it isn't the only way. The history of philosophy is one of specialization. There are some questions which have objective answers but can't be answered using the methods in the empirical sciences. Philosophy, as a field, handles the questions which her children can't.

The last argument gets a little closer to home. Skepticism concerns knowledge and as such arguments for it for some subject should touch on the standards for knowledge. 

    \factoidbox{Knowledge is a really high degree of certainty. With any action we take, there’s always some uncertainty about whether it’s the right one. This means that we can’t know whether the action we are taking is the right one.}

For the third issue, there is a simple response: Alright, I agree that knowledge is a high degree of certainty, and there’s always going to be some uncertainty about the action I take. But, how much certainty is enough for knowledge? What if I am 99\% sure that this is the right route? I am still unsure, but I am very certain. We can't be absolutely certain about anything in life, morality is no different, we can only be really really sure. Going this route not only denies the possibility of knowing moral truths, but it also messes with the possibility of knowing anything else.
