\part{Non-Consequentialism and Kantianism}
\label{ch.mod6}
\addtocontents{toc}{\protect\mbox{}\protect\hrulefill\par}
\chapter{Part 17: Kantianism}
In the previous module, we looked at the content of the consequentialist line of thinking. This, remember, is the idea that the consequences of an action determine the morality of the action. The opposing side to this line of thinking is the \glspl{Non-consequentialism}. They claim that the morality of an action is not determined by the consequences. This sort of thinking, like with consequentialism, is found all over the globe, but we will be focusing on the theory of actions given by Immanual Kant, called Kantianism. But to get us started, take a look at this example:

\newglossaryentry{Non-consequentialism}
{
  name=Non-consequentialism,
  description={The general stance that the morality or permissibility of an action is not determined by the consequences of that action. This is a family of theories which all share the basic idea that some actions are just wrong regardless of the good they might (will) produce; typically, these theories are duty-based, meaning that an action is moral if and only if you have a duty to do it and immoral if and only if you have a duty to refrain from doing it},
  plural=Non-consequentialist
}


\thoughtex{A Trip to Disneyland}{Imagine a person who is a brilliant tax lawyer. They know fully and perfectly how to cheat on their taxes and never get caught. Now, if they cheat on their taxes, they will have enough money to take their family on a wonderful vacation to Disneyland. No one is really going to suffer, because we are only talking about a few grand in taxes. That money will cause far more happiness in their hands than in the governments.}{disneyland.jpg}

In this case, the consequentialist, by and large, would say that evading your taxes would be the right thing to do. They, here, point to the consequences. But this does not seem to sit too well with some people. In some way, the evader is doing something wrong in evading their taxes. To explain this, we can't point to just the consequences. In fact, pointing to the consequences, according to the non-consequentialists, is the wrong thing to do. They think that morality is not determined by the consequences; the ends never justify the means.

Like Utilitarianism, Imannual Kant’s moral theory states that there is an absolute moral rule. But, unlike Utilitarianism, it states that the consequences of the action don't matter morally. It is, however, grounded in a theory of intrinsic value. Rather than going with the Principle of Utility, Kant's theory has it that the only thing with moral worth is the \gls{Good Will}, which we find in persons. Persons, for this theory are  autonomous rational moral agents. This theory, from the very start, makes a certain metaphysical assumption: People have free will. This can't be the kind of free will proposed by the \glspl{Compatibilism}, as Kant was fundamentally opposed to it, so the sense of free will must be the \glspl{Libertarianism} sense. There have been attempts to make a \glspl{Kantianism} Compatibilism, but those seem to have been unsuccessful.\autocite{Vilhauer1}

\newglossaryentry{Good Will}
{
  name=Good Will,
  description={The one intrinsic good according to Kant's non-consequentialist theory of morality. All rational autonomous people have this `good will'. It comes from our ability to rationally and freely choose to act on our duties. According to Kantianism, anything which thwarts or hinders either another's free will or ratonality is wrong},
  plural=Good Wills
}

\newglossaryentry{Compatibilism}
{
  name=Compatibilism,
  description={In Metaphysics, this is the stance that we humans, at least, have free will (are responsible for some of our actions) and that the world is deterministic (there is no randomness, all of our actions are determined, we coud not do otherwise than how we do)},
  plural=Libertarian
}


\newglossaryentry{Libertarianism}
{
  name=Libertarianism,
  description={In Metaphysics, this is the stance that we humans, at least, have free will (are responsible for some of our actions), that the world is non-deterministic (there is a degree of randomness), and this non-determinacy increases the responsibility an agent has over their actions},
  plural=Libertarian
}

\newglossaryentry{Kantianism}
{
  name=Kantianism,
  description={The philosophical theories proposed and defended by Immanual Kant},
  plural=Kantian
}


Kant's Moral Theory rests on this notion of the Good Will, so we should be clear about what that is. The opening passage of Immanuel Kant’s Groundwork for a Metaphysic of Morals proclaims that “it is impossible to conceive of anything in the world, or indeed beyond it, that can be understood as good without qualification except for a good will.” This eloquent sentence places this `Good Will' at the center of Kant's value theory. The one thing that has intrinsic value, for Kant, is the autonomous good will of a person. But we can't understand Kant's Good Will in the ordinary sense. In everyday discourse we might speak of someone being a person of good will if they want to do good things. But this would be a Consequentialist take. On Kant’s view, the person with Good Will wants  good things out of a sense of moral duty, not just some habit or tendency.

The naturally generous philanthropist doesn’t demonstrate their good will through their giving according to Kant, but the selfish greedy person does show their good will when they give to the poor out of a recognition of their moral duty to do so even though they’d really rather not.

So, to be worthy of dignity and moral regard, for Kant, we need to have the ability to see what our moral duty is and act according to it. For Kant, then, in order to be worthy of moral consideration, we need to be able to act in a way which is opposed to our desires/conditioning. Having an autonomous good will with the capacity to act from moral duty is central to being a person in the moral sense and it is the basis, the metaphysical grounding, for an ethics of respect for persons. Now what it is to respect a person merits some further analysis.

Kant's version of the Principle of Utility, his fundamental principle to guide our actions and thereby give people the dignity and respect worthy of their Good Will is called The \gls{Categorical Imperative}. An imperative is a command or order given. Kant, and others, explain this kind of imperative by contrasting it with another kind, a hypothetical imperative. A hypothetical imperative is a command, but it only applies conditionally according to your desires, what your goal is. For example, ``if you want to avoid traffic, leave 15min early." This imperative tells you what to do if you want to avoid traffic, but it fails to tell you what to do in a case where you don't want to avoid traffic. A categorical imperative (though Kant thinks there's only one) is different in that it tells you what to do regardless of your goal or desires. It applies according to the kind of action you are taking, not why you are taking it. Kant also holds that if there is a moral law, it will be the Categorical Imperative. Treating it that way, moral reasons must always overshadow any other sort of reasons which can be given. You might, for instance, think you have a self-interested reason to cheat on exam. But if morality is grounded in a categorical imperative, then your moral reason against cheating overrides your self-interested reason for cheating. If we think considerations of moral obligation trump self-interested considerations, Kant’s idea that the fundamental law of morality is a categorical imperative accounts for this nicely.

\newglossaryentry{Categorical Imperative}
{
  name=Categorical Imperative,
  description={The imperative, command, which determines the morally right action according to Kantianism. There are four different formulations of this, two of which clearly reduce to the other two},
}


Although Kant gives 4 different statements of the Categorical Imperative and claims that they all boil down to the same basic idea, there is some debate about whether they do. Two of the formulations just seem to be restatements of the other two, and the debate is over whether these two can be formed together. Once you see them, you will see that they certainly don't look like they are expressing the same idea. The first formulation which we will cover is called The Principle from Humanity, for this class, though other names for it are out there. The second formulation is called The Principle from Universalizability for this class, though, again, other names are out there.

\chapter{Part 18: The Universality Principle}
\begin{center}
Act only on the maxim that you can consistently will to be a universal law
\end{center}
The \gls{Universality Principle} is also known as the formula of the universal law. The \Gls{Maxim} of our action is the base-level reason or principle that determines what we are doing. We act for our own reasons and different goals might lead to similar actions. For example, a person might wash their clothes regularly because they don't want to smell bad while another person might do the same because they don't want their significant other to complain about the stack of clothes near the closet. Though they have the same action attached to them, the maxim behind the action will be different.  For Kant, intentions matter and this formulation really gets at this point. He evaluates the moral status of actions not according to the action itself or according to its consequences, but according to the maxim of the action. Whether an action is right or wrong  is determined by the actor’s intentions or reasons for acting.

\newglossaryentry{Maxim}
{
  name=Maxim,
  description={The reason and goal for which you are acting. This is typically of the form ``whenever I am A then I will E''}
}


\newglossaryentry{Universality Principle}
{
  name=Universality Principle,
  description={This is one of the two formulations of Kant's Categorical Imperative. This commands us to act only on the maxim which can be consistently willed to be a universal law}
}


According to this formulation, what makes an action morally acceptable is that its maxim is \Gls{Universalizable}. That is, morally permissible action is action that is motivated by an intention that we can rationally will that others act on similarly. A morally prohibited action is just one where we can’t rationally will that our maxim is universally followed. Basically, ask yourself ``am I making a special exception for myself?" ``could anyone in my situation do this?" If anyone with similar desires could do what you are doing and accomplish them, then you are morally in the clear, otherwise you are morally up a creek.

\newglossaryentry{Universalizable}
{
  name=Universalizable,
  description={Able to be generalized to the point that it could be could be followed consistently in any circumstance (you can follow it even if it is immoral)}
}


Here is an example of this thought at work:

\factoidbox{\noindent  \fontsize{20pt}{0pt}\textbf{Cheat to Win}

    Suppose that I really want to win at a game, so I think about cheating. The principle I go on is “whenever I want to win, I will cheat in order to do so”. But, if everyone did this, the concept of a game would be mute, no one would play the game by the rules, so cheating would not be cheating. This is a contradiction, so cheating is always wrong.}

There is no higher moral authority than the rational autonomous person according to Kant. Morality is not a matter of following rules laid down by some higher authority. It is rather a matter of writing rules for ourselves that are compatible with the rational autonomous nature we share with other persons. We show respect for others through restraining our own will in ways that demonstrate our recognition of them as moral equals.

For another example, check out this thought experiment:

\factoidbox{\noindent  \fontsize{20pt}{0pt}\textbf{The Good Partner}

    Suppose that I am a very faithful partner, but, one morning after a night of quite a few too many, I find myself next to a woman who is not my partner. Realizing what I have done, and knowing how my SO will react, I plan to lie.}

The maxim which I am acting on in this case is something like the following:
\begin{center}
Whenever telling the truth will hurt another, people will lie to save them from that pain.
\end{center}
Now, what would happen if this was universal? Well, lying would be commonplace, people would not trust each other, and I obviously would not get away with the lie because my SO would not trust me.
\section{Problems for Universalizability}
\subsection{The Traffic Jam Problem}

Suppose that I regularly get caught in traffic at 6:45AM. I know that if I leave 15 minutes earlier, I will arrive where there’s traffic at 6:45AM at 6:30AM, missing the traffic. So, I ponder the following:
\begin{center}
    Whenever I want to avoid traffic, I will leave 15min earlier.
\end{center}
But, everyone wants to avoid traffic, so what would happen if everyone did this? If everyone wanted to avoid traffic and left 15min earlier, it is reasonable to say that the traffic would not be at 6:45AM, but now at 6:30AM. This means that if everyone left early to avoid traffic, they would not avoid traffic. This is a contradiction. If I leave early, I avoid traffic and if everyone leaves early, no one avoids traffic. Therefore, according to Kant, me leaving early is morally wrong.
\subsection{The Breaking Promises Problem}

Suppose that I promise my mother on her deathbed to sing and play a certain song on my banjo at her funeral. She asks me to play/sing In Hell I'll Be In Good Company, by The Dead South. But, on the day of the funeral, I can't bring myself to play such an inappropriate song, the lyrics in this context would make everyone's grieving worse. So, I leave the banjo to the side. The principle you are going with is “whenever I make a promise that I don’t want to keep, I will break that promise.” Now, what would happen if everyone broke promises they didn’t want to keep? If everyone broke promises they did not want to keep, then the very notion of promising would go out the window. If there are no promises, you can’t break them. That is a contradiction. If you promise, you break it. If everyone broke promises, there would be no promises. If there are no promises, you can’t break them. Therefore, if everyone broke promises, they can’t break promises. So, breaking promises is always morally wrong.

But in not playing that song, did I really do something wrong? Most will say no.

\chapter{Part 19: The Humanity Principle}
\begin{center}
Always treat persons (including yourself) as ends in and of themselves and never as a mere-means
\end{center}
This \glspl{Humanity Principle} tells us to treat individuals as ends in themselves, but what does that mean? How do I use people as mere-means? How could I use myself that way? This formulation or principle is noted for really highlighting the notion that people are intrinsically valuable. To say that persons have intrinsic value is to say that they have value independent of their usefulness for this or that purpose.  They are not a tool or resource which you can use without consideration of their worth. the Principle from Humanity does not say that you can never use a person for your own purposes (using them as a means). If this were the case, you taking a class from me would be morally wrong. It tells us never to use others as a mere-means.
\newglossaryentry{Humanity Principle}
{
  name=Humanity Principle,
  description={A formulation of the Categorical Imperative which phrases it as the command to always treat persons (yourself included) as ends and never as mere-means},
  plural=formulation
}


\section{Means Vs Mere-Means}

We treat people as a means to our own ends in ways that are not morally problematic all the time. When I go to a grocery store to pick up some food, I treat the clerk as a means to my end of buying food. But I do not treat that person as a mere-means to my own end. I accomplish my end of buying food through my interaction with the clerk only with the understanding that the clerk is acting autonomously in serving me. My interaction with the clerk is morally acceptable so long as the clerk is serving me voluntarily, or acting autonomously for his own reasons.

By contrast, we use someone as a mere-means to our own ends if we force them to do our will, or if we deceive them into doing our will. Sometimes we use people as mere-means when we don't take their goals into account in our interactions with them.

\thoughtex{Smith Tower}{Suppose that I have the goal of building a tower with my name emblazoned on the top.  I have the money to do this and I hire workers to build it (they have the understanding that they will get paid when the job is done). Just before they finish the work and would expect to get paid, I take all of my money, put it in an off-shore account under my son's name and declare bankruptcy. Using the laws in this regard, I get the contracts waved so that I don't pay the workers.  Later, when the dust settles, I transfer the money back into my name.}{SmithTower.jpg}

In doing this, I have used those workers as mere-means. I have the goal of getting the building, they have the goal of getting paid. In my interactions with them, according to Kantianism, I should have, in a sense, made their goals my own and had the intermediate end of paying them for the work. 

Coercion and deception are paradigm violations of the categorical imperative. In coercing or deceiving another person, we disrupt their autonomy and violate their will. This is what the categorical imperative forbids. Respecting persons requires refraining from violating their autonomy.

Here is an example of this sort of theory at work, where it seems to get the right answer:
\begin{earg}
    \item[1] To take a person’s life, liberty, or legitimately acquired property without that person’s consent is to use that person as a means to an end (if they give consent, then they are being treated as an end). It is to treat a person as a tool or resource placed here for your convenience.
    \item[2] It is always wrong to treat a person in that way.
    \item[3] Therefore, each person has a moral right to life, liberty, and property, regardless of the consequences.
\end{earg}
So, killing is always wrong, so is stealing. Another example of this formulation getting things right comes up when we talk about slavery (though the Utilitarians were on the front-lines against slavery before the Kantians):
\begin{earg}
    \item[1] Each person is intrinsically valuable regardless of race, religion, ethnicity, or national origin and deserves to be treated as such.
    \item[2] Slavery does not treat people as an intrinsically valuable being.
    \item[3] Therefore, slavery is morally wrong.
\end{earg}
\section{Problems for Humanity}

\subsection{Taxation}
\begin{earg}
    \item[1] Taxation is the taking of a person’s property and using it to benefit others, without their consent.
    \item[2] Taking a person’s property without their consent is treating them as a mere means to an end.
    \item[3] Treating a person as a mere means to an end is always morally wrong.
    \item[4] Therefore, taxation is always morally wrong.
\end{earg}
Some of my more conservative students may like the sound of that, ``Taxation is Theft" they tend to shout. But many of you will have a problem with the consequences of taxation always being morally wrong. Public schools, as they are paid by taxes, gone. Most public roads, if they do not lead to some person's business and they did not pay for it, gone. The cost of your college education will sky-rocket, as they are subsidized by taxes. And many others.

\subsection{Lying}

This is an interesting case, and one which Kant himself thought was right, as in he thought that the theory got the correct answer here, however, this does not line up with most people's moral intuitions about the case. Take this case as an example (which will appear in the discussion for this module):

\thoughtex{The Ax-Murderer}{Your roommate is in the shower after having a late night with a recently gained boy/girlfriend, who had gone through a recent bad break-up (their former significant other is ``the ex", this is not the ex of your roommate, for clarity) . You hear a knock on the door and go to answer it. You find the ex standing at the door with an axe, murder in his/her eyes. The ex asks you “Where is [insert roommate name]?” Neither of you can hear the shower running. You know that if you tell the ex, then they will rush right through you/knock you out/whatever and get to the roommate killing them.}{KantsAxe.jpg}

According to The Humanity Principle, we should never use others as a mere-means. But, does lying count as a mere-means? According to Kant, it does. Kant says that in lying, you are misinforming one person for the benefit of another (or yourself), without appreciating their intrinsic worth. This is always wrong, according to the principle, so lying, no matter what, is always wrong. Sure, you could refuse to answer, slam the door, and so on, but you can never lie. This means that lying to the axe-murderer is wrong. So, if you need to speak, then you must tell them that your roommate is in the shower, and more than likely, deal with the Psycho scene later. 

However, this is an issue, because it certainly seem right that there are cases where lying is permissible, so this seems wrong, too hard lined. 

\subsection{The Rendering Aid Problem}

This particular problem addresses something interesting in this formulation of the Categorical Imperative (and this feature should be found in any rephrasing of it, as it's core). In it, it states that we should not use ourselves as mere-means. Since Kant explicitly stated that we shouldn't, it must thereby be possible to use ourselves as mere-means. This was quite the puzzle for me, personally, how could I possibly do something without taking my own goals into account? How could I force myself to do something against my will? The examples which Kant himself gives have not aged well, they involve sexual acts and suicide, which I don't want to use for this course. 

I have hence asked around and some examples do seem problematic and I will look at the issue of rendering aid.\footnote{This example comes from Clayton Littlejohn through personal corespondence} Often we glorify people who help others at the cost of their own wants and goals. In such a case, they are not taking their own goals into account and using themselves to benefit others. This, by its very nature, would be the person using themselves as a mere-means. Here is an example, in an ordinary case:


\thoughtex{Helping Change a Tire}{Suppose that I have the goal of buying a new video game, The Elder Scrolls 20 or some such, and I know that given the shear demand for the game, if I am not there just as the shop opens, it will be sold out. As I am driving to the shop early,  I notice a young man having a hard time changing a tire. Feeling as though I should help, I pull over to render aid. This prevents me from making it to the shop on time and violates my end.}{changingtire.jpg}

In this case, it would seem that I have used myself as a mere-means and cases of rendering aid like this would be morally wrong.
\input{bloodbending}