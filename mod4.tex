\part{Hedonism and Egoism}
\label{ch.mod4}
\addtocontents{toc}{\protect\mbox{}\protect\hrulefill\par}
\chapter{Part 11: Hedonism and The Good Life}

\section{Happiness and intrinsic value}

If you are anything like me or anyone else I know, you spend a good amount of time thinking about how to make your life better. You may be doing alright, or you may be worse off, or you may be somewhere in the middle, but there’s always room for improvement. But, you figure out how our lives could be better, we need to know what would make them good. The standard measurement is ‘welfare’ or ‘well-being’.
\subsection{Extrinsic vs intrinsic value}

Generally, when philosophers talk about whether or not something is ‘good’, they start by asking whether it’s extrinsically good (or has \gls{extrinsic value}) or intrinsically good (or has \gls{intrinsic value}). Something is extrinsically good when it’s good because it gets us something else. For example, money is extrinsically good because we use it as a means to get something which is good. Money can get you on a cruise, which can be very fun. Something is intrinsically good when it’s good ‘in and of itself’, it is not used to get you anything else, having more of it in your life makes it better. Generally, extrinsic goods are good because they get you something intrinsically good. For example, watching a football game is good because it makes you happy. Now, what we need to do is figure out what the intrinsic goods are so we can get them and have a good life.

\newglossaryentry{extrinsic value}
{
  name=extrinsic value,
  description={The value something has in virtue of its ability to get something else}
}

\newglossaryentry{intrinsic value}
{
  name=intrinsic value,
  description={The value something has in an of itself. It is good for no other reason than it is good. Things are extrinsically valuable in virtue of their ability to get this}
}

In trying to answer the question ‘what sort of things are intrinsically good?’, the most popular answer, by far, is happiness. On this view, the good life for us would be one filled with happiness and as little unhappiness as possible. Something is extrinsically good when it makes us happy (it’s a means to happiness). Money can’t buy happiness, but it can put you on a jet ski, which will make you happy. For this view, happiness = the one intrinsic good, unhappiness = the one intrinsic bad. The more happiness you have, the better your life is.

\gls{Hedonism} is the name for the view which I just described, but it takes ‘happiness’ to be ‘pleasure’. According to the hedonists, a life is as good as the amount of pleasure which fills it and how free of pain it is. 

\newglossaryentry{Hedonism}
{
  name=Hedonism,
  description={The stance that happiness is the one intrinsic good, thing of positive intrinsic value, and suffering is the one thing with negative intrinsic value. Things are extrinsically good or bad insofar as they promote happiness or suffering (respectively)}
}


\section{Hedonism}

Understanding hedonism as the idea that a life is better relative to the amount of enjoyment it has (the more the merrier) is the first step to see the appeal of hedonism and also some more serious problems with it. Various notions of hedonism are found around the globe, but the major development and formalization of it was first done in ancient Greece. The philosopher Epicurus (341-270 BCE, born and raised in Samos, which is just off the coast of Turkey, died in Athens) was the first to really defend it. He argued that pleasure was the only thing worth going after but did not call for us to indulge in all of the cardinal pleasures, but stated that the most pleasant condition for a person is inner peace. This came from two sources, moderation in the physical and mental clarity.

For Epicurus, the way to get this pleasure, inner peace, was to do philosophy. Philosophy clears up the mind and let’s you reveal the false beliefs which you have (false beliefs, for Epicurus is one source of unhappiness). For example, Epicurus thought that the following beliefs are false and bad for us to have:
\begin{earg}
    \item[$\bullet$] Death is bad.
    \item[$\bullet$] The Gods are mean.
    \item[$\bullet$] Sex, Drugs, and Rock and Roll are key to a good life.
\end{earg}

Philosophy will explain away these errors and lead to a better life.

The next great hedonist was a guy named John Stuart Mill, who had a fascinating political and romantic life (for example, he was the first person to argue for women’s suffrage in England in parliament). We will return to fun facts about him in the next module, as his major contribution was a theory of ethics based on hedonism (we now have evidence that his wife was the main producer of many of his ideas, but she would not let him put her name on the books). During Mill’s lifetime, people had taken to calling his Hedonism ‘the doctrine of swine’. Claiming that he wanted us to all live like animals and pursue one the physical pleasures.  Mill actually argued that there are many kinds and qualities of pleasure. Mill claimed that the best kind of pleasure were those which came from hard work (the satisfaction after a good day’s work, the feeling of accomplishment), and the highest quality of those pleasures were those which were gotten through mental labor (but that’s a point for another time). Mill believed that people who have had a good amount of both physical and mental pleasures in their life will always prefer the mental. As you might expect for a view which is found all over the world and spanning thousands of years there’s a lot to be said about it. 

\subsection{There are many models for a good life}

What’s a good life for me might not be a good life for you. And hedonism explains why this would be. There are many ways of getting a happy life. So, can woodcutters, stonemasons, or musicians have a good life? Not according to Plato and Aristotle, who thought that only through philosophic contemplation can you get the best life. Today, we generally reject this as snobby and elitist. Generally, hedonists think that anyone has a good chance at a happy life. Since the sources of happiness are many, and happiness is the key to a good life, there are many different good lifes.
\subsection{Flexibility}

Many theories in ethics aren’t all that flexible, they have strict hard and fast rules which can’t really fit into various contexts and situations. Hedonism is not like that. For example, ‘follow all of these rules and your life will be good’, but there are many cases where a person can’t follow those rules or are unable to engage in some activity claimed to be necessary for a good life (EG doing philosophy). Hedonism allows for a good life to be in many different shapes. My recipe is, probably, different than yours.
\subsection{Personal authority}

The diversity of good lives leads to an interesting take-away: Hedonism gives us a huge say in what a good life is for us. Although happiness is an objective, absolute, good according to this theory, what makes us happy is a matter of taste and as a result, we each get plenty of input from our experiences on what is a good life for us and what is not.
\subsection{Misery hampers a good life, happiness improves it}

The hedonist tells us that misery, unhappiness, takes away from a good life. To test this, imagine a life with just sadness, and no happiness whatsoever. Obviously this seems bad for the person who leads it. This person may be a genius or a brilliant artist, but the life would still not be good for them. Hedonism also tells us that happiness improves a life. To test this, imagine two identical lives. But, one person likes what they do and the other person hates it. Obviously the first person’s life is better than the other’s.

\chapter{Part 12: Ethical Egoism}

Here are some examples to get us started:

\factoidbox{\noindent  \fontsize{20pt}{0pt}\textbf{Wells Fraudo}

Suppose that you are a bank manager. You know that opening fraudulent accounts in your customers’ names will make you millions (if you don’t get caught) and also that if you get caught, the penalty is so small that you will still make millions (though slightly less). Since the pay off is so great, why let morality stand in the way?}

\factoidbox{\noindent  \fontsize{20pt}{0pt}\textbf{Cheating on a Test}

Suppose that cheating on a test will land you an awesome job after you graduate. Why not cheat?}


\factoidbox{\noindent  \fontsize{20pt}{0pt}\textbf{Lie vs Jail}

Suppose that lying to the cops will save you from going to jail. Why not lie?}

\factoidbox{\noindent  \fontsize{20pt}{0pt}\textbf{The Panama Papers }

Suppose that you can hide money in off-shore accounts to avoid paying taxes on it. You stand to keep millions/billions instead of giving it to the man. Why not hide the money?}

In a perfect world, greed like in these examples would never flourish, but we aren’t in a perfect world, greed does flourish. Ethical egoism seeks to show that actually, because we are in this dog-eat-dog world, the examples in the above cases are all moral. \gls{Ethical Egoism}:
\begin{center}
Something is moral IFF it best promotes our individual self-interest.
\end{center}

\newglossaryentry{Ethical Egoism}
{
  name=Ethical Egoism,
  description={The stance that an action is moral if and only if it promotes the agent's self-interest}
}


On this view, conflicts between your self-interest and morality are impossible. So, you should open those accounts, cheat, lie, and evade taxes.

\section{Popular Arguments for Ethical Egoism}

We will now be moving on to some popular reasons given to be an egoist. There are three, the self-reliance argument, the libertarian argument (in this case, for those who have taken me before, this is the political ideology, not the metaphysical stance), and the best argument for egoism.

\subsection{The Self-Reliance Argument}

This argument was given by Ayn Rand (you may have heard of her) and her writing arguing for egoism have been insanely influential in contemporary political culture.

\begin{earg}
\item[1] The most effective way of making everyone better off is for each person to mind her own business and only tend to her needs.
\item[2] We ought to take the most effective path to making everyone better off.
\item[3] Therefore, we each ought to mind our own business and only tend to our own needs.
\end{earg}

\subsubsection{Problem one: The First line}

The first line was “The most effective way of making everyone better off is for each person to mind her own business and only tend to her needs”, this line seems just false. Take for example a person in dire need of help; they would not be better off if no person went out of their way to help. More over, it would be wrong to celebrate a person who did.

\subsubsection{Problem two: The Second Line}

The second line is “We ought to take the most effective path to making everyone better off”, but this does not line up with egoism, the egoist, proper, can’t accept this. For them, our only moral job is to make life better for us, individually. Why should we care about other people?

\subsection{The Libertarian Argument}

This is a little less formal of an ‘argument’, and it tends, in real life, to appear in conversations with those who identify as this political stance. It goes something like this:

\factoidbox{Our moral jobs, what we need to do to stay moral, come from two sources, either consent or reparation. It comes from consent when a person freely agrees to accept the job. It comes from reparation when the person violated another’s rights and needs to repair it. So, if I don’t owe them anything and haven’t agreed to anything freely, I have no duty to help others.}

\subsubsection{The Problem with this one}

The Egoist can’t support it. The problem is that for egoism, the only duty we have is to best serve our self interest. The libertarians are claiming that we have two moral duties, following through on our agreements and not violating another’s rights. But, there may be (and probably are) cases where violating a contract will serve your self-interest (I am sure you can think of examples, just watch the news today/yesterday) and obviously, there are times when violating another’s rights would serve you selfishly (take voter suppression tactics).

\subsection{The Best Argument for Egoism}

This is the last argument in favor of this stance, and it is the best one you are likely to find. Though it does have some major problems with it.
\begin{earg}
\item[1] If you are morally required to do something, you have good reason to do it.
\item[2] If you have good reason to do it, then doing it must serve your self-interest.
\item[3] So, If you are morally required to do something, it must serve your self-interest.
\end{earg}
Now, this is a valid argument, we can only get around it by looking at facts. Also, each of these lines have a ton of support, so we might be stuck… If something doesn’t serve my self-interest, I don’t have to do it.

\subsubsection{The Problem}

Ultimately, this argument is unpersuasive (which is an iffy part of whether it’s a good argument). Once we look into the reasons why we like these lines, we start seeing it unravel. Take these two similar lines:
\begin{center}
\textbf{If it serves my self-interest, I have good reason to do it.}

\textbf{If I have good reason to do it, it serves my self-interest.}
\end{center}
Many people who like the second line of the argument confuse these two. It is really common, however, there’s a giant difference. The first one looks alright, it is basically saying that my self-interest is always a good reason, no trouble there. But the second is a bit different. I have good reason to save a drowning kid, but that doesn’t serve my self-interest. I have good reason to donate money to help others, but that doesn’t serve my self-interest.

\section{Problems for Egoism}

Like with Relativism, just because the arguments for it don't work, the stance could still be true. So, we will look at the real consequences/problems with the stance itself.
\subsection{Problem 1: Violates core moral beliefs}

One solid cultural universal (almost required for a culture to survive more than a few generations) is that you should help others (either all people or those in your ‘tribe’). Egoism gives us a duty which says that we shouldn’t help others. You should help others only if it would promote your self-interest. Many cases of us helping others fail to do so. That may seem like enough to reject it.
\subsection{Problem 2: Egoism can’t allow for the existence of human rights.}

The vast majority of people the world over want to claim that there are certain universal human rights. Human rights apply to all people (though there may be cases of exceptions) and it is always morally wrong to violate those things. Egoism can’t allow for a moral duty like that because not violating another’s rights is often against a person’s self-interest. 
\section{Problem 3: Why are my interests the only important ones?}

This one cuts to the core of egoism. What makes my interests the only important ones? Why should I totally discredit the basic needs of others and only care about my own? The egoist can have a kind of reply:

 \factoidbox{Suppose that I am hiking in the woods and then a stray arrow misses a deer and clips me in the leg. I drive to the hospital but know that the stitches and all that will cost me all that I have extra (an emergency fund). Another person is there with a very similar injury. But for them, there’s no way that they can afford to have it fixed. Should I forgo my treatment and pay for hers? Am I doing something wrong if I don’t? Add to this, suppose that this lady is as smart a person as I am, community-minded and over all very nice (just in a rough spot). What then?}

Various different versions of that reply to the third problem for egoism are very very hard in contemporary ethics (applied differently to different theories). The problem with them is that they weigh our preferences against ethics and there is something to be said about them. For example, three people are drowning, a woman can save either the two strangers or her significant other. Who should she save? Is it wrong for her to save her S.o.? One of my old professors, Douglas Portmore,  wrote a book where he spent a solid chunk trying to piece that together… The book is called Commonsense Consequentialism.