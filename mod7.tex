\part{Virtue Ethics}
\label{ch.mod7}
\addtocontents{toc}{\protect\mbox{}\protect\hrulefill\par}
\chapter{Part 20: Virtue Ethics}

All of the theories which we have seen so far have been directed towards getting an answer to the question "what should I do?". There seems to be something missing in the ethical picture when we only consider this question. There is another, potentially equally important question about the sort of person I should be: "Who should I be?". If we applied the theories we have seen to this question, then they would say something like I should be a person who maxes out happiness or treats others fairly. But that’s a pretty short list. To see what’s missing, take the following example:

\factoidbox{\noindent  \fontsize{20pt}{0pt}\textbf{The Good-Bad Cop}

    A cop who obeys and enforces the law, but only reluctantly. He does only the minimum required. He would happily extort money, beat prisoners, or doctor evidence, but doesn’t because he’s afraid of getting caught.}

If we focus only on what the cop has or hasn’t done or why he does/doesn’t do those things, we are missing a huge chunk of the picture. The appropriate ethical theory about actions would say that this cop was a good person, but this can't be correct.  We want to say that he is not a good person, despite doing all of the right things. We don’t like people like this. This shows why we need to expand our thinking to explain this. To fill things out, we need to ask about his character, the sort of person he is.

To handle this massive hole in the theories, we need to move into another way of thinking. This is focusing not on the rightness of actions, but rather the rightness of a person’s character. Rather than asking what makes an action good, we should be asking what makes a person good. This is virtue ethics’ main goal. For all other ethical theories which we have covered, the core question was ‘what should I do?’ Once we have an answer to that, we can piece together the sort of person we should be. But, what if we started with a different question, what if we started by asking ‘what should I be?’? We start by looking at what makes a good person and then figure out what we should do from there.

Virtue ethics is not a single theory, but a family of theories, all starting with that basic question. Various versions of it have sprung up around the globe. In the west, this traces its origins to the Nicomachean Ethics written around 2,400 years ago by Aristotle. In the east, this way of doing ethics traces back to Mencius (around 2,400 years ago (400BCE)) and Confucius (around 2,500 years ago (500BCE)). The influence of these thinkers and this way of doing ethics has been very strong and persistent. Aristotle’s work, in particular, defines many of the themes found in virtue ethics at large today.  We will mostly be focusing on Aristotle’s virtue ethics, but if you are interested in Confucius’ virtue ethics, I do have resources.

Aristotle, for his part, started with the closely related question "what makes a good person?". Expanding on this a bit, he saw that there are certain kinds of things which we can judge as good, average, or bad based on how well the thing performs its function. For example, take a computer. A bad computer would be one which is slow at its computations, unreliable in its results, and spotty in activation. An average computer would be one which is decently quick in its computations, fairly reliable in its results, and predicable in activation. A good computer, on the other hand, would be very quick at computing, very reliable, and dependable in activation. Aristotle thought that humans would be the same way; we just need to figure out what the function of humans, the role we play, is and then use that to figure out what a good person would be. This reasoning should sound similar to the sort we saw with Natural Law Theory. Aristotle thought that a good person would be the one which exhibited the human excellences, which we now call virtues. 

\section{Why should I want to be a good person?}

This is an interesting question. What motivation should I have, or do I have, to try to be a good person? In Aristotle's mind, being a good person would be to exhibit the human excellences, but why should I want to be excellent? His reply is to say that we want to achieve eudaimonia (no that is not a typo). This is often translated as "happiness" or "flourishing", but that doesn't get at the right picture. This is the state of mind that you are doing what you were meant to do, a satisfaction with your placement in life, a sense of accomplishment. Aristotle held that this is the good life for any being, not just humans, as you are fulfilling your function. He also held that, for humans, exhibiting the virtues, these excellences, are necessary to reach eudaimonia and thereby lead a good life. So, long story short, we should want to be a good person, be virtuous, because it gives us a good life. 

\section{A Theory of Right Actions}

Remember, that virtue ethics is fundamentally different than the other theories which we discuss. So, applying it to a theory of right action is going to be different. It is a by-product of the question that they are trying to answer. The theory of right action looks like this:
\begin{center}
An action is moral if and only if it is what a virtuous person, acting on their virtues, would do in that situation.
\end{center}
For the virtue ethicist, an action isn’t right because of the results or follows some rule, rather it’s right because it would be done by a virtuous person. Virtue ethics asks us to do what the virtuous person would do. But who is that? Virtue ethics calls such a person a ‘moral exemplar’, someone who sets a fine example and serves as a role model for the rest of us. This ideal person provides us with a goal to aim for, even if we tend to fall short.
\section{Ethical Pluralism and Virtue Ethics}

Ethical pluralism is the stance that there are two or more moral rules which apply under different circumstances. For example, one could think that in certain cases consequentialism is the correct theory to apply, but in other cases, it’s non-consequentialism. And then define how to choose. For example, when the stakes are low, non-consequentialism, when the stakes are high, consequentialism. We could, as some cultures do, flip those and say that when the stakes are low, consequentialism, but when the stakes are high, non-consequentialism. Virtue ethics is like this. It gives us one hard and fast rule, do what the virtuous person would, but this is far to general to be helpful in daily life. So, what it does is says that there’s a set of more particular rules which are more tailored for the case at hand. Virtue ethics gives us these rules as well and tells us which ones apply to a case.

\section{Moral Understanding}

Moral understanding is basically how we come to understand right and wrong actions. According to the virtue ethicist, our moral understanding is not just knowing a bunch of facts about morality. If this were the case, then a child could be the morally wisest of us all. But how could that be? Can you imagine a child giving advice about how to deal with a difficult coworker? How to help an addict? How to end a toxic relationship?

There are two ways, at least, of having knowledge, there’s knowing-that and there’s knowing-how. Knowing-that is the sort of book knowledge you get. ‘I know that 2 and 2 is 4’, ‘I know that Caesar was assassinated’, and others. Knowing-how is a bit of a different game, this is us having the ability to DO something, it’s the hands on aspect. ‘I know how to fix a car’, ‘I know how to read’, and so on. For the virtue ethicist, moral understanding is a knowing-how sort of thing, you need to experience things to learn about morality. Knowing-how does require some book-knowledge, but there’s more to it than that.  The other theories which we have looked at have it more as a knowing-that sort of thing.

Moral understanding is a kind of wisdom and to get it, it requires, according to VE, experience, emotional maturity, reflection, and training. We have to know how to read a room, understand common problems facing people and understand their personalities. You just can’t get that from a book. The Emotional maturity is a big one for VE, and We will go into why having a rounded emotional life is good (not too extreme on either side, you don’t over react, but you don’t underreact).
\subsection{1. Emotions tell us what matters}

Fear signals danger, guilt is a sign that we did something wrong, compassion that someone needs our help. It’s no use knowing that you ought to help people and never get the hint that a person needs help. A compassionate, kind, sympathetic person will see the things that others miss.
\subsection{2. Emotions can tell us when something is right or wrong}

If the person is virtuous, then a feeling of anxiety is a pretty good sign that what you are about to do is wrong. We often have a good gauge in our emotions about whether something is OK or not. Before we even have the intellectual reason behind it (why do you think I use examples which gross you out?).
\subsection{3. Emotions give us motivation to do the right thing.}

When we experience certain emotions, we are given motivation to do something about it, either to make more of it or to fight against it. When something makes you sad, you have motivation to make it stop. When something makes you mad, you have motivation to fight against it. When something makes you happy, you have motivation to support it. The morally wise person has her emotions in line with the right reactions to the events, because of their experience.

\section{Moral Education and The Nature of Virtue}

Moral Education for the virtue ethicist is the process by which people get the know-how for morality. They develop the virtues. It’s true that many people are born naturally kinder or more generous, but without the proper training, developing those traits, they won’t get the right kind of action for the situation at all times. A person who is naturally generous will be more lenient in certain cases, but sometimes the other would need some tough-love. The wise person will know when to give and when to withhold generosity.

Getting the virtues requires time and patience. It also requires the right kind of teacher. A world-wise person who can help you see when you have gone too far in either direction. According to Aristotle, there is a solid chunk of being virtuous which can be accounted for by moral luck. This is basically that you happened to be in the right place at the right time to come out good morally speaking. Moral Luck is a more common problem for consequentialism (cases where you tried to do something wrong, but by a freak chance it came out as moral). But for the virtue ethicist, you kind of need to be lucky in getting the right kind of teacher/people around you.

\subsection{The Nature of Virtue}

I have mentioned a few times, and it has been seen in the videos a few times, that virtues are the mean between two vices. A ‘vice’ is the opposite of a virtue, in an interesting way. For both Aristotle and Confucius, the virtues have several interesting features:
\begin{center}
Virtues are character traits, not habits.
\end{center}
A habit is just something you do. A person could be habitually generous but lack the virtue, because they lack the understanding of why this is the thing to do. Virtues in this way require wisdom and experience to develop. The virtuous person thinks about the situation differently than the person who does the right thing out of habit. There’s an actual process in the virtuous person’s head that there’s not in the habitual person. The virtuous person is actually motivated to do the things because they are right, the habitual person does not have this, they just do it.
\begin{center}
The virtuous person is not defined by their deeds but by their inner life.
\end{center}
The virtuous person is the kind of person who sees, believe, and feel things differently than the non-virtuous person. They are trained to see what’s important, do what is right, and be motivated to do what’s right.

\subsubsection{The Golden Mean/The Doctrine of the Mean}

So, what are the traits which are virtuous? This is defined by both Aristotle and Confucius as the middle between two different vices. A vice is a bad character trait. What is it to be in the middle of two vices?  Well, a vice is an excess or deficiency in some trait. When you are in the middle, you are in the proper level for that trait. For example, For cautiousness, the extreme would be overly fearful, but the deficiency would be being foolhardy. Cautiousness is right in the middle. 

There are many other examples which Confucius and Aristotle give to illustrate this point. To start with, we have courage. Take this case as an example: 

\factoidbox{\noindent  \fontsize{20pt}{0pt}\textbf{Rushing In}

Ralph is walking home to his apartment one day and he notices a building on fire with several people inside and a fire-truck out front. Ralph does not have the training, physical strength, or necessary skills to be of any assistance, but, despite this, he rushes in an attempts to save people.
}

It is reasonable to assume that Ralph put himself into danger and added more victims to the blaze. This case shows that Ralph has an excess of the trait relevant to courage, he is foolhardy or rash. A similar case can also be used to illustrate the deficiency: 

\factoidbox{\noindent  \fontsize{20pt}{0pt}\textbf{Frozen Stiff}

Frank is a firefighter who has all of the training and equipment necessary to safely save people in a burning building. When Frank and his team arrive at the blaze, Frank allows his fear to get the better of him and he can't bring himself to go in and sabe the people. 
}

Here, Frank is showing a deficiency of the trait associated with courage. The proper amount is the middle between these two extremes. Acting despite your fear when the conditions are such that it is safe\footnote{or the risk is reasonable} for you to do so. This is not the only example where this applies: 

\factoidbox{\noindent  \fontsize{20pt}{0pt}\textbf{The Party Don't Stop}

Paul is a college student who enjoys blowing off steam and going to a party a little too much. He constantly attends the most `lit' parties, over indulges in alcohol, experiments with harder drugs, and otherwise fails to complete his other obligations because of his partying.
}

In this case, Paul is showing an excess of the trait related to the pleasures of life. This excess causes him to not be the sort of person we would want. At the same time, a deficiency can be just as bad: 

\factoidbox{\noindent  \fontsize{20pt}{0pt}\textbf{Poopy Pants}

Paula is a college student who focuses and grinds away at her work. She has scant few friends because whenever people invite her out to go clubbing or dancing, she either refuses or, if they manage to drag her out, she never lets the music move her and is always cold, for lack of a better term. 
}

Paula's insensibility towards enjoyment is not what we would want in a good person. A good person knows how to blow off steam in a reasonable way and when to do so. They have temperance.

For a final example, let's consider the use of humor in a conversation. Humor and laughter loosens us up, it allows for the creative juices to flow, so to speak, and helps us relax in order to come up with creative solutions to problems at hand. A topic may be seen as too serious to joke about, but this is inaccurate. One could be discussing very serious topics, like the economy, politics, and ethics, and a well-placed joke or witty remark would break the tension and allow for the members of the conversation to explore the topic in a new way. All that said, an excess of humor in a conversation is likely to derail or otherwise hinder the progress which one is hoping to get out of the discussion. Cracking a joke or a witty remark at every chance is not the quality of a good person, we do not want them to be a buffoon. At the same time, if one never breaks the tension with a targetted remark, then the conversation will miss-out on the alternative views and may end up just spinning its wheels. A good conversational partner is not boorish. The middle is the key. The quality is being witty. Knowing when to make a joke and what the joke should concern. 

\section{Problems for Virtue Ethics}

\subsection{Choosing Between Virtues}

\factoidbox{\noindent  \fontsize{20pt}{0pt}\textbf{A Weekend Vay-Kay}

Suppose that you are on a vacation with some of your friends, your bestie brought her hubby along. You are walking along the beach and you spot your bestie’s husband cuddled up, intimately, with another woman. Would the virtuous person reveal what she has seen? Would she stay quiet?}

Well, there’s the virtue of honesty. You should inform your friend, because that would, probably, prevent her from having the pain and anguish later on from being lied to for a long time. But, there’s the virtue of minding-your-own-business. Being a busybody and poking your nose into other people’s business is a vice. It’s not your marriage, there could be many factors there.

What would you do? The Virtue Ethicist provides very little information about choosing between the virtues. This means that in cases like these, it does not give us a practical way to use the theory.
\subsection{Tragic Dilemmas}

Virtue Ethics claims that the right action is the one which the virtuous person would do acting in character. And that those actions are worthy of praise. However, if those stances lead to a problem, then VE is in deep trouble. Tragic Dilemmas, also called ‘moral dilemmas’, are situations where there are no right answers. Damned if you do and damned if you don’t. Virtuous people will tend to avoid these situations as they normally come from mistakes that they have made. But this is not always true. Take for example Sophie’s Choice:

\factoidbox{\noindent  \fontsize{20pt}{0pt}\textbf{Sophie's Choice}

Sophie is detained in a concentration camp during WW2, separated from her children. The head of the camp comes in and tells her that one of her two children will be killed and the other will be freed. She must choose which one, either her son or her daughter. If she refuses, both die. What should she do?}

Sophie’s life is going to be ruined regardless of what she does or doesn’t do. A virtuous person, acting in character, would have to make a choice. According to VE, Sophie choosing one of her children and letting the other be killed is morally right and praise-worthy. But this seems wrong.
\begin{earg}
    \item[1] If Virtue Ethics is correct, then Sophie choosing is morally correct and praise-worthy.
    \item[2] It’s neither.
    \item[3] Therefore Virtue Ethics is not correct.
\end{earg}
\subsubsection{A reply (the virtuous person would choose)}

That first line, that the virtuous person would choose in this case, is suspect. The VE person could claim that the virtuous person would not choose. This is their only way out. The virtuous person might just refuse to make a deal with evil, keeping her hands clean. But would the virtuous person really be more motivated to refuse such a deal than to save a life? This is quite up in the air.
\subsubsection{Another reply (that it’s neither)}

A safer route for the virtue ethicist in this case is to reject the idea that choosing is neither moral nor praise-worthy. In this case, they need to say that yes, it is both. Very counter intuitive. Basically, under these circumstances, the virtuous person would try and save as many innocent lives as they could, so they would need to try. Also, since they were able to make a choice, they should be praised, they saved a life.
\subsection{Demandingness}

Virtue Ethics tells us to do/be as the virtuous person does/is. But what if this is an impossibly high standard for us to handle? Morality sometimes puts really high standards on us. For example Gandhi performed hunger-strikes to change the rules, which almost killed him. Protesters against injustice have died to change thing. If we suppose that at least some of the people who have died in the process of their fights were virtuous people, then VE would have us follow suit. This might be going a step to far.
\subsubsection{Two replies}

There are a few ways that VE can get out of this. First, they can claim that, actually, in the right cases, morality does require that level of sacrifice. This is similar to some of the lines which the consequentialist has made in the past. They could say that the expectations which we have about what is required are too lax. If we had the right up-bringing, then we would know the value of noble self-sacrifice and not see this as overly demanding.

Another reply that they could try is to change the theory slightly. Say that such sacrifices are normally not what the virtuous person would do, but there are cases where they are morally required. If I went on a hunger-strike to protest my rent being too high, people would probably not pay much attention to it. The change would be to emphasize that the circumstances matter. In the rent case, my circumstances are quite different than those of Gandhi, so the virtuous person would not protest in this way.
\subsection{Who are these virtuous people?}

According to VE, we can solve our moral puzzles by looking at what the virtuous person would do. But we need to ask who these people are? What if people endorse different candidates for the morally perfect person? This is a hard problem because we often take people as virtuous based on our pre-existing beliefs about virtues. Some people out there may hold that suicide-bombers are the moral exemplars. Others may think that this is horrible.
\subsubsection{Some replies}

A resistance to this could be that we should stick with relativism. The ideal moral exemplar in this case is relative to either the person or to the culture. They could also claim that what the virtues are, what makes a good person is relative to the person or to the culture. Since the right action  is defined relative to the moral exemplar, going this route leads to the same problems which we have seen with relativism.

Another way to reply is to say that our failure to notice the moral exemplars and choose appropriately is actually a failure of our own virtues. We have not gone through right cultivation of our virtues to notice when others lack one or more. For example, Winston Churchill was a remarkably virtuous person, but he had a very deep hatred for the people of India. To the point that he failed to see the virtues in Gandhi, claiming that he deserved to die. This failure of virtue is why he could not see the virtues of another.  
\subsection{Conflict and contradiction}

If a theory tells us both to do something and not to do something, then that is a very serious problem for the theory. For example, if I have a theory which claims that lying is always wrong and that allowing another person to be killed is always wrong (even in cases of letting die), then my theory would have a very hard time in cases where I need to lie to save a life. The problem for this theory is simple. Suppose that there are many perfectly virtuous people and, for the same case, some say that I should do one thing and others say that I should do something else. It might seem that in such a situation, the act is both moral and immoral. Uh-Oh! For example, some perfectly virtuous people might claim that saying nothing in the case of the ax-murderer is moral. Others might say that it’s to lie. Others, still, might claim that you need to tell the truth. How do we handle this?
\subsubsection{Some replies}

One of the ways out of this problem is to claim that there’s only a single, truly virtuous person. They could claim that this person is hypothetical or exists/existed. If there’s only one truly virtuous person, then we don’t have the issue of having the conflict. But this becomes an issue about how to ID this person.

Another way out is to say that every truly virtuous person, acting in character, would actually make the same call. This seems less plausible.

I personally take that it’s what the hypothetically perfectly virtuous person would do in that situation is the stronger option, and then we have use the golden mean to get at all of the traits, but this is not the route the book takes.

Another, third way out is to change the stance slightly, yet again. The new version of the theory gives us three different statements each handling a different state (required, permitted, and forbidden):
\factoidbox{
An action is morally required because all virtuous people, under those circumstances, would act that way.

An action is permitted because some (but not necessarily all) virtuous people, under those circumstances, would act that way.

An action is forbidden because no virtuous people, under those circumstances, would act that way.}

Permitted is the interesting case, this means that you aren’t doing wrong if you go either way with it. 
\subsection{The Priority Problem}

Another, final, problem for virtue ethics is that of priority. What do they take as most important. The standard way is to have the actions be the most important and then define good people according to those. VE rejects this, claiming that we should hold the person’s character as more important than their deeds. Under certain contexts, this seems backwards. All other theories take the action as the most important. For example, if we look at Porky the Pig Farmer, VE would claim that what he is doing is wrong because it’s not what the virtuous person would do. It seems better to say that the virtuous person wouldn’t do those things because they’re wrong. The same thing is true for right actions. VE tells us that saving a drowning child is moral because it’s what the virtuous person would do. But we want to say that the virtuous person would do it because it’s moral.
\subsection{Euthyphro on Virtue Ethics}
\begin{earg}
    \item[1] Virtuous people either have or lack reasons for their choices on actions.
    \item[2] If they have reasons, then it’s those reasons which make the action right, not that they are choosing it.
    \item[3] If they lack reasons for their actions, then those actions are arbitrary and can’t be the basis for morality.
    \item[4] So, either virtue ethics is misleading (missing the point) or is arbitrary (not a basis for morality).
\end{earg}

\chapter{Part 21: Nahuatl Virtue Ethics}

The Nahuas, today, are the descendants of the indigenous peoples who inhabited what is today Mexico, Guatemala, Honduras, El Salvador, and Nicaragua. The most famous of these peoples are the Mexica,\footnote{Better known by the term `Aztecs'} with an approximation of their characters for the word 'Tenochtitlan' appearing on the Mexican flag. Prior to colonization by the Spanish, these peoples had philosophical traditions, mirroring the traditions which are found in Europe and Asia; they used similar methodologies and pursued answers to many of the same or equivalent questions. Due to the colonial practices of the Spanish (and others) and also due to an under-appreciation of their writing systems and lack of understanding of their languages, until fairly recently, pre-colonial Nahuatl Philosophy has not received the respect and adequate understanding it is due. Various contemporary works are seeking to change this by uncovering, translating, and reevaluating their thoughts and methodologies. 
\section{Differences in The Metaphilosophies}\footnote{The content for this section is derived, in part, from Aztec Philosophy by James Maffie.\autocite{Maffie1}}

Metaphilosophy is the study of Philosophy itself. It concerns the role of Philosophy, the methods used in Philosophy, and some of the basic assumptions which are made when one starts doing it. For the most part, pre-colonial Nahuatl Philosophy is similar in methodology to the traditions found in the rest of the world, as I have mentioned. There is, however, one fundamental aspect of pre-colonial Nahuatl Philosophy which makes it strikingly different from the Philosophy originating from other parts of the world and which makes teaching about something isolated like this especially difficult. In the other philosophical traditions, different questions are delineated into 'spheres of interests' or 'fields of study'. These spheres tend not to overlap, with some exceptions, and, for most, it is possible to teach about one sphere without robust reference to another. Nahuatl Philosophy is not like that. The tlamatinime (the Nahuatl noun translating roughly to "philosophers", literally "knowers of things", the singular is tlamatini) did not break up their stances or questions into different fields. Rather, they treated their philosophical pursuits as a unified whole. So, in the Nahuatl philosophical tradition, seeking the answer to, say, "what makes a good person?" would also be seeking the answer to, say, "what is beautiful?", "what does it take to know something?", or "where did the universe come from?". As a result, I will seek to be respectful of this overarching methodology, focusing primarily on their ethical views but freely referencing the stances in what we take to be different spheres which overlap with this. Though I will often treat the spheres as separate, appreciation for Nahuatl Philosophy requires that we understand that this is meant to be a continuous whole, without divisions.

The methodological pursuits also differ from other philosophical traditions in the study of truth and the nature of their pursuits differ because of this. The tlamatinime did not take knowledge to be about factual things, which we could call "knowing-that". Rather, on a very fundamental level, the tlamatinime took knowledge to be 'knowing-how'. Rather than asking questions like "is this accurate to how the world is?", the tlamatinime asked questions like "is this the way to do it?" "is this the right path?" and so on. Nahuatl Philosophy is pragmatic in this regard.\footnote{This may stem, in part, from the fact that the Nahuatl ontology, what they take to exist fundamentally, is a form of universal monism, all things we see as separate are actually one thing and also from the fact that this universe is constantly in flux, as we will see later.} Other philosophical traditions pursue knowing-that while this tradition pursues knowing-how. This is because the central unifying notion underlining all of Nahuatl Philosophy is one of unity. Not only is Philosophy treated as a unified whole but the universe itself is. The idea here is that everything which we see as separate and distinct is actually just different aspects of one thing, \emph{teotl}. Teotl is the universe itself, all things, ourselves included, are just teotl. We see teotl as a computer, as a color, as a shape, as weighing a certain amount, as sounding a certain way and this makes us think that there are many different things in the universe. Actually, according to this thought, it is all one thing. Teotl, however, is not stagnant. It is constantly flocculating, changing, flowing like a river. It would be foolish, therefore, to try to pursue absolute accuracy about the world as it will change drastically. As a result, the best we can hope for is accuracy in how we should navigate the world and how to do things in it. In the history of Western Philosophy, we do see something like this. For the pre-Socratics, the philosophers prior to the death of Socrates, there was a debate about whether the world was unified in this way or whether it was in flux. It was assumed that these two stances were incompatible and the ripple of this assumption is seen even today. The tlamatinime took this debate and saw that this was not a contradiction and actually assumed that both were true. 

\section{Similarities in Conclusions and Ethical Thought}\footnote{The content for this section and most of what follows is from Eudaimonia and Neltiliztli: Aristotle and the Aztecs on the Good Life
by L. Sebastian Purcell.\autocite{Purcell1}} 

These fundamental differences in the methods and assumptions does not make teaching about their views impossible nor did it radically change the conclusions about Ethics; that is, the Nahuatl ethical thought is strikingly similar to that of Aristotle and Confucius. We will start with a discussion of why Aristotle thought we should be virtuous and then move to the similar reasoning from the tlamatinime. After this, we will move to a discussion of the accounts of virtue and the similarities there.

\subsection{Aristotle's Eudaimonia and The Tlamatinime's Neltiliztli} 

Eudaimonia is Aristotle's conception of the good life, the sort of life which we should seek as humans. Aristotle also held that this good life was what every single action, question, or decision we make was striving towards. He later argued that there is a highest good which one can strive for and that leading a good life was that highest good. This, frankly, at first brush, does not seem related to the idea of being virtuous or being a good person. It seems possible, at least, that a good person could have a bad life. Aristotle and others, however, did see a connection and the reasoning from this is similar to the reasoning which we saw in Natural Law Theory. The thought is that, for certain kinds of things, we deem them good because they fulfill their function. For example, we judge a TV as good when it shows a clear picture and plays clear sound. It is fulfilling its function. Persons, roughly, could also be judged in this way generally. A good person would be one which fulfills their function as a person. For Aristotle and others, this is where they exhibit human excellence and, to exhibit this excellence, one must be virtuous. In other words, if you exhibit human excellence, then you are truly virtuous (and vice versa). A good human life is one which is had by a good person, so to have a good life, you need to be virtuous. This is why you should want to be virtuous, to lead a good life. 


\emph{Neltiliztli} is the Nahuatl Philosophers' conception of the good life. In the same way as Aristotle, they believed that it is the sort of life we should want as humans but they did not start by asking about our function as humans or persons: Rather, the tlamatinime started by asking about the character of our circumstances as humans on Earth (life on tlalticpac). This shows, in an interesting way, how they easily moved from a metaphysical question about the nature of our world to an ethical one about the nature of a good person. Life on tlalticpac, for the tlamatinime, is what determines why we should be virtuous and how we can lead a good life.\footnote{A good person is one who can function as a person on Earth, as they conceived it.} The tlamatinime conceptualized tlalticpac as having three distinct features, each of which deserves going over in detail.

\subsection{Tlalticpac is Slippery, Transitory, and Not a Happy Place} 

The three distinct, though maybe related, features of the kind of world we live in, according to the Nahuas from this time, are that the world is `slippery', transitory, and, fundamentally, not a happy place. A good person, according to the views developed here, is going to be one which can lead a good life and to do that one must be able to handle these three features of the world. 

\subsubsection{Tlalticpac is Slippery}

The first feature of the world we live in, which the Nahuatl Philosophers used to determine what makes a good life for a person, is that it is slippery. The Florentine Codex, a collection of works and research by the Nahuas which was collected and translated by a friar in the 16th century,\footnote{Fray Bernardino de Sahag\'un} gives us some details about what was meant by calling the world slippery in this context. Here are two, disjointed passages which describe this:
\factoidbox{
    How is this? Look well to thyself, thou fish of gold.
    
    It is said at this time: if one some
    
    time ago lived a good life [and] later
    
    fell onto some [other one]—perhaps he took
    
    a paramour, or he knocked someone
    
    down so that he took sick or even died;
    
    and for that he was thrust into jail:
    
    so at that time it is said: “How is this?
    
    Look well to thyself, thou fish of gold.”

    Slippery, slick is the earth.
    
    It is the same as the one mentioned
    
    Perhaps at one time one was of good
    
    life; later he fell into some wrong, as if
    
    he had slipped in the mud.\autocite[Book 6 page 228]{Sahagun1}}

There are a few interesting things worth noting about how the world is slippery. First off, a person of good character, leading a good life, could morally slip and do something immoral through voluntary action, such as taking a paramour. But, an otherwise good person, leading a good life, could slip through something accidental. Take for example a case where a person accidentally trips another and that person becomes sick or dies. This is a bad outcome but we could forgive them because it was accidental. These sort of accidents are a part of the world we live in, it's slippery in the sense that we can't avoid falling or failing to be rooted all of the time. There comes a point where, regardless of our individual choices, lapses will occur. We also see that appealing to the slipperiness of the world is not an excuse, accidents happen and we need to pay for the consequences of them, even if it wasn't our fault. For a second point worth mentioning about this slippery world, we can't avoid falling by good reasoning. One could slip because of a fallacy or some incorrect reasoning, but perfect reasoning and purging yourself of fallacies is not an antidote to slipping, but rather is a preventative measure. From this, we see that a good life for a person is not one without any errors, as Aristotle would have conceived it, rather it is one which recognizes that we will fall down occasionally and manage these failings as best we can when it happens.  

\subsubsection{Tlalticpac is Transitory}

The next feature of the world we live in is that it is transitory. Something is transitory when not permanent, finite, going to end and be forgotten. A thought found throughout the Americas is that the world we live in was created from a previous one and will end to give rise to the next. The Nahua hold that this is the fifth such world and it will end and give rise to the sixth. Not only will we die and other humans will replace us, but the world itself will die and the next one will replace it. This is closely linked to the idea that the world is in flux, as I have mentioned previously.


\subsubsection{Tlalticpac is Not A Happy Place}

The third, and final, feature of tlalticpac, relevant to us here, is that it's not all that fun of a place. We can see this general line of thought in the following short passage:
\factoidbox{
    Is it true that we are happy,

    that we live on earth?

    It is not certain that we live

    and have come on earth to be happy.

    We are all sorely lacking.

    Is there any who does not suffer

    here, next to the people?\autocite[p. 91]{Leon2}}

Fundamentally speaking, because of the slippery and transitory nature of the world we live in, it's foolish to try to make happiness your life's goal. For another example, consider this passage, which was a father speaking to his daughter when she was old enough (biologically) to have children of her own:\footnote{Similar passages are recorded in a mother having the same conversation with her daughter}

\factoidbox{And now thou hast become knowledgeable, already thou observest how things are. There is no rejoicing, no contentment; there is torment, there is pain, there is fatigue, there is want; torment, pain dominate. Difficult is the world, a place where one is caused to weep, a place where one is caused pain. Afflication is known. And the cold wind passeth, glideth by.\autocite[Book 6 page 93]{Sahagun1}} 

\section{Rootedness and The Good Life}

The general aim for Nahuatl Ethics and more particularly a good life from their perspective, as I hinted at, is not happiness or some sense of satisfaction, as we saw with Aristotle, rather it is to achieve a sense of rootedness. The idea of rootedness can be found in a short piece of poetry called Xochi Cuicatl (translated, literally, as Flower-Song, or Flower and Song, but metaphorically it means something like poetry). The piece concerns how one could some sort of permanence in teotl,  which remember is constantly in flux. The author in the process of writing the piece realizes that one can achieve a sense of permanence by writing poetry. 
\factoidbox{
    So this is how that lord, the vaunted one,

    comes creating them. Yes, with plume like

    bracelet beads he pleases the only-being.

    Is that what pleases the Life Giver?

    Is that the only truth [nelli] on earth [tlalticpac]?\autocite[p. 13]{Purcell1}}

It is worth noting that nelli is often translated as 'truth' but it is actually closer related to the concept of being rooted. The concept of something being true, in the sense of Western Philosophy, has built into it a sense that the world is static, which is in opposition to the Nahuatl concept of teotl. Rather than slipping and sliding one must be rooted, not apt to fall. This is because, put frankly, for you to remain you, you must be rooted, fixed in the world and doing your part in it. 

This can be seen in other Nahuatl writings as well. The early anthropologists asked the Nahuas questions about what makes a good person for a certain field, what makes a bad one, as so on. While they did not, necessarily, formulate their answers explicitly in terms of good and bad, the approval vs disapproval is very evident. Those who were bad were hardly even described as being human, being called "a lump of flesh with two eyes".  One aspect of our condition on Earth is that we are communal creatures and the good life, drawn from an understanding of our condition, must be one where we do our part in the community, otherwise, in a sense, we are not humans at all. 

Bringing these points together, the Nahuatl conception of the good life is derived from looking at the nature of the human condition being creatures on Earth. We should pursue a rooted life as it is a basic condition for leading a life in a community in teotl and a reasonable response to our circumstances.
\section{The Virtues and Action Guidance}

For all of Virtue Ethics, the idea is to create a good person and then the good or morally right actions will follow. For Aristotle, this good person was one which exhibited human excellence, reaching eudaimonia. For the Nahuas, this was a person who was rooted, in the sense that I have been building up.  The virtues, then, for these thinkers, are character traits which show this excellence. The difference between them, however, is where they derived this sense of excellence. 

The Nahuatl Virtues, then, would be the character traits which make one best suited for being rooted and would need to be cultivated. 

\factoidbox{They went on saying that on earth we travel, we live along a mountain peak. Over here is an abyss, over there is an abyss. Whenever thou art to deviate, there wilt thou plunge into the deep. That is to say, it is necessary that thou always act with discretion in that which is done, which is said, which is seen, which is heard, which is thought, etc.\autocite[Book 6 page 125]{Sahagun1}}

But, does this, then, turn into a theory of right action? Does this system tell us what to do? One can easily say yes. Both theories have access to the idea that an action is right if and only if it's what a virtuous person would do, though they may get different responses. In the case of the Nahuas, a good person can be judged, to a degree, by the content of their actions just as much as by the content of their character. This is because a good person is also determined by how they fill their social roles. Take this translated quote for example:

\factoidbox{The good middle-aged man [is] a doer. a worker, agile, active, solicitous.

The bad [middle-aged man is] lazy, negligent, slothful, indolent, idle, languid, a lump of flesh, a lump of flesh with two eyes, a thief. He absconds; he is a petty thief; he kills one by treachery; he steals from one.\autocite[Book 10 page 11]{Sahagun1}} 

Your average man is, just as today, going to be a worker. In that social role, there are certain virtues, such as being active, attentive, and honest. Those are both actions one might take as well as dispositions which one may have. The bad traits, those which we should not want, are those which fail to have one fulfill the role in society which one is expected to fill. Take, as another example, the qualities of a good mother as compared to a bad mother: 

\factoidbox{One's mother has children, she suckles them. Sincere, vigilant, agile, an energetic worker- diligent, watchful, solicitous, full of anxiety...

One's bad mother [is] evil, dull, stupid, sleepy, lazy; a squanderer, a petty theif, a deceiver, a fraud. Unreliable, one who loses things through neglect or anger, who heeds no one. She is disrespectful, inconsiderate, disregarding, careless...\autocite[Book 10 page 2]{Sahagun1}}

The role of a mother, at least as described here, is to care for her children. There are certain qualities which are proper in that role, such as vigilance and diligent. A good mother must have the disposition to \emph{care.}\footnote{This will appear again in the content concerning Feminist Ethics and we have seen this previously in the content concerning Mohist Consequentialism.} 
\factoidbox{
    The good [tlamatini] is a physician, a person

    of trust, a counselor; an instructor worthy of

    confidence, deserving of credibility, deserving

    of faith; a teacher. [He is] an advisor, a counselor,

    a good example; a teacher of prudence, of

    discretion; a light, a guide who lays out one’s path,

    who goes accompanying one. . . . 
    
    The bad  [tlamatini] is a stupid physician, silly, decrepit,

    pretending to be a person of trust, a counselor,

    advised. . . . [He is] a soothsayer, a deluder, he

    deceives, confounds, causes ills, leads into evil..\autocite[Book 10 page 29]{Sahagun1}}

We see here that the quality of one's fulfillment of a social role is the source of praise and blame which is directly tied to character traits, virtues, and this gives us a system of guidance for actions, in the same way as Aristotle thought of it. It can also be added, with a lot of textual evidence, that the Nahuas at large think that the good person will exhibit the human excellences, similar to Aristotle's thinking, though they differ in how they go about determining the excellencies.

\chapter{Part 22: Virtue Consequentialism}

This is a relatively modern stance (as in it was recently formalized into a stance as I give them in this class) but it does have historical backing (Hume, Bentham, and others). This is a different kind of virtue ethics. Like VE which we have been covering, it asks what makes a good person and says that it is a person who has certain virtues, but how it defines the virtues is different. For this content, I am mostly going to be using Ben Bradley’s Virtue Consequentialism,\autocite{Bradley1} but since the stance is still in its relative infancy, there’s a lot of room to grow.\footnote{Bradley is using Julia Driver's Uneasy Virtue, where she proposes a form of Virtue Consequentialism. In it, she argues that we should not use the doctrine of the mean but rather use consequentialism to determine virtues. I use Bradley's work as the primary source here.}

VC (Virtue Consequentialism) has a core principle about virtues in mind. As with Virtue Ethics, generally, it is seeking to answer the question ``who should I be?'' rather than trying to answer the question ``what should I do?'' This means that the Virtue Consequentialism is concerned with the character traits of a good person.\footnote{their virtues.} Rather than thinking that the virtues are the middle between two extremes,\footnote{which is how Aristotle's Ethics and Confucian Ethics thinks of them.} Virtue Consequentialism thinks of virtues as character traits which, when acting on them, produce actions with the best outcomes.\footnote{This might sound like Rule Consequentialism. The two are related; Mohist, Rule, and this Virtue Consequentialism are all indirect Consequentialist theories, but Virtue Consequentialism is after the character traits of a good person rather than the rules for right action.} This is a form of externalism. The theory claims that what makes a trait a virtue is external to, outside of, the person who has it. With the other kind of Virtue Ethics, the kind proposed by Aristotle and Confucius, whether a trait is a virtue is determined internally. One can arrive at the conclusion just through careful thought. Kantianism holds that what makes an action permissible is internal. Utilitarianism holds that it's external. Since this stance is very new, I am able to give series of different stances trying to define virtues in this way and how they evolved from each other.

\section{The First Definition of ‘virtue’ for VC}

As I just mentioned, Virtue Ethics does not take factors external to the bearer of a trait into account when determining whether the trait is a virtue. The consequences of an action produced by acting on that trait are external.  To remain consequentialist, VC will need to incorporate consequences into the definition somewhere. It might seem natural to take a page out of Rule Consequentialism and say that a trait is a virtue when, if one consistently acts on that trait, it produces the best outcomes when compared to other traits. So here is the first pass:
\begin{center}
VC1: A trait is a virtue if and only if, in the actual world, acting on that character trait systematically produces more good than not in the actual world.
\end{center}
It is worth noting that ‘systematic’ here is meant to minimizes the effects of moral luck in a situation. The more often one acts on a trait, the more chances we have to see whether it actually tends to produce the best outcomes. For example, in the actual world, a person who is honest\footnote{but not brutally so} tends to cause more good than harm, so honesty may be a virtue according to this account. This definition comes from Julia Driver’s work Uneasy Virtue.\autocite{Driver1} This definition focuses on the real world situations that the person may be in and states that the trait, when acted on, produces good things.

\subsection{Issues with this first pass at a definition}
The first issue with this account of virtues is that it seems to focus on just one person. There may be and likely are character traits which tend to produce the best outcome in the actual world if the majority of people act on them but there could be, at the same time, people who never get to act on those traits or are in a situation where them acting on it systematically would be harmful to the world. For example:

\factoidbox{\noindent  \fontsize{20pt}{0pt}\textbf{Overly Honest Diplomat}

Harry is a diplomat tasked with meeting and discussing things with representatives from a country hostile to his own. He knows various secrets and aspects within his country which it would be best for the world if the other country did not learn about. Given Harry's experiences and station in the world, he knows that being honest would be systematically harmful to the world. So, for Harry, this is not a virtue.  
}

On the other side, there may be situations which give a false positive, rather than a false negative, when we are trying to figure out what the virtues are. 

\factoidbox{\noindent  \fontsize{20pt}{0pt}\textbf{Overly Obedient Oliver}

Oliver has the trait `obedient', which he takes to mean that he must follow his superiors' orders without question and without delay. As it happens, all of Oliver's superiors have been, are, and will be good people who would never order Oliver to do anything which would not produce the best outcome; the orders always produce more good than not. In the actual world, for Oliver, thoughtless obedience is a virtue; but actually, thoughtless obedience can be disasterous to the world.
}


Second, ‘more good than not’ does not mean that the trait should max-out the consequences, rather it leaves it open to the issue of doing a minimal amount of good and still being a virtue.  So, we might try to change this to make it so that it will max-it-out.

\section{Second Definition}

This definition is an improvement on Driver’s definition, given by Ben Bradley (2005), but should be seen more as a step towards the best option, rather than a stance. He gives further definitions and examples as he goes through, but this was to make the initial evaluation as strong as possible:
\begin{center}
VC2: A trait is a virtue if and only if acting on that trait, in the actual world,  produces more good in the actual world than it produces evil in the actual world.
\end{center}
This tries to get the trait to max-out the consequences, but it also has many of the issues which were had in the previous. For example, it still focuses on the actual world and it does not tell us a good way of telling which traits may be better than others. 

\subsection{A problem for this second attempt}
As mentioned, this definition, though a slight improvement, does fall into the same issues as before. Take, for example, this counterexample:

\factoidbox{\noindent  \fontsize{20pt}{0pt}\textbf{Debbie Downer}

Imagine a world with a ton of happy people and in it there’s guy named Downer. Downer spends her days systematically making others happy, but less happy than they would have been had he done something else. For example, she invents new, less tasty foods and hoodwinks people into thinking those foods are healthier than the tasty foods they like better. She never causes them any pain but she does cause them to be less happy than they would have been otherwise.}

In this case, Downer’s traits do cause good things, but they aren’t the best things she could have done, for example, she could have, with likely the same amount of effort, made healthy and tasty food, rather than tricking people. The traits which Downer is acting on could be construed as virtues according to this pass at a definition, but we want something stronger. We would not say, even in a consequentialist mindset, that Downer was acting as a good person.

\section{A Third Definition}

This one again comes from Bradley and goes like this:
\begin{center}
VC3: A trait is a virtue if and only if acting on that trait, in the actual world, systematically produces a greater balance of intrinsic good over intrinsic evil in the actual world than the absence of that trait would systematically produce.
\end{center}
For this one, we are still focused on the actual world, but this does give us a way of saying that a virtue must max out the consequences of the actions that they produce, so we get out of Downer and less than optimal people. This should remind you of how the consequentialist determines the right action. It's the one which produces the greatest balance of good over bad. This definition really plays well with the consequentialist mindset. We still have some worries, however.

\subsection{Problems for this third pass}

For this one, a major issue is that it now does not have a way of determining a better virtue, there are no comparisons which are made explicit. Essentially, a world where some trait is not present would have some other trait take its place, but which one? There are several traits which Downer could have had rather than her minor malice. Whether or not the trait is bad would depend on the one we are comparing it to and how they are compared.  The next step for this theory is to figure out a way for the virtues to be compared. Basically, this issue is “Hey, you say that the trait is a virtue when it’s better to have it than not, but what trait would fill the void when it’s not there?”

Another issue is that it is still focusing on the actual world, this gives us a problem as a person could be very, very lucky in the outcomes when acting on some trait. Consider this case:

\factoidbox{\noindent  \fontsize{20pt}{0pt}\textbf{Lucky The Failing Murderer}

Lucky is a murderous psychopath who repeatedly attempts to kill people. Lucky, however, constantly fails at the task. She slips on a banana peel, gets bumped at just the wrong time, slips on some ice, and so on. Lucky acting on her murderous traits never produce negative results in the real world (and we can suppose that she enjoys planning the murder but if she succeeded, the pain and all that would make the acts negative).
}

Lucky’s traits are not virtuous and a reasonable Virtue Consequentialism would recognize that. This means that we need to expand our thinking and have the virtues be determined by counterfactuals and reasonable expectations.

\section{Counterfactualism and Contrastivism}

To handle the worries we have seen with the attempts at defining virtuous traits using Consequentialism, we need to turn to two different approaches. These approaches are often used separately, but we will use them both together in order to get a robust theory of virtues. 

\subsection{Counterfactualism}

In order to avoid cases like Lucky's, we need to broaden our horizon's beyond merely the actual world and start thinking about cases which are non-actual but could happen. As you may recall from the `What If' test from the Objectivism module, sometimes counterfactuals have something to say about whether a thing as a feature or not. In the What If test, the counterfactuals told us whether our cultural beliefs had bearing on the truth of a claim and, if they didn't, then we knew that the truth or falsity of the claim was objective. Here, we are using the same kind of idea, using counterfactuals to tell us whether a trait is actually the best one, according to Consequentialism. 

There are two ways in which we could use counterfactuals in order to determine whether a trait is a virtue. First, we could ask what the world would be like if everyone had this trait.\footnote{This should seem similar to Rule Consequentialism.} Going this route would be a form of Universalism about virtues. If everyone acting on the trait systematically brings about the best outcomes, then it is a virtue, according to this theory. Another option is to ask what the world would be like if people \emph{in that position or status} acted on the trait. This is a more limited, individualist, account of virtues. There are equally strong arguments on either side of this debate and, until the dust settles, it would be wise for our theory to remain neutral. For example, consider the Overly Honest Diplomat case which we saw previously. If everyone were always honest, this would systematically produce the best outcomes but, as the world is, there are certain positions and stations where it would not be the best outcome if the \emph{individual} were completely honest. 

\subsection{Contrastivism}

A contrastivist stance or being contrastivist about some topic is to claim that attributions of some kind or another are actually disguised comparisons. In this case, Bradley says that whether or not a character trait is a virtue is actually a comparison between it and other traits which one could have in its place. One way to think about this is that there are many different traits which a person could have but some of them can’t be had together. For example, one can’t be both generous and greedy. The set of these mutually exclusive traits should be kind of intuitive. These traits are the ones one could have in place of another in that set. It is similar to how in Aristotle there were the spheres. A set or collection of mutually exclusive things is sometimes called a `contrast class'. When we are determining which traits are virtues, we look at whether it would causes the most good, if systematically applied, compared to the other traits one could have had in its place.\footnote{That is, we look at whether it would cause the most good out of the traits in its contrast class.}  

\section{The Final Definition of Virtues for VC}

With Counterfactualism and Contrastivism in our toolkits, we can use them to better refine the notion of virtues from a Consequentialist mindset:

\begin{center}
VC4: A trait is a virtue IFF, when compared to the other traits which one could have in its place, were a person to act on that trait, it would systematically produce more good and less evil than the alternatives. 
\end{center}

Think about it this way, of the different traits which a person could have in a given context, having one of them will cause you to produce better outcomes than the others, and that trait is the virtue in that sphere. Going back to the issue involving obedience, one could say that blind obedience is not a virtue because systematically acting on it would not always produce the best outcomes when compared to the alternatives. One alternative could be thinking for yourself when you have all of the relevant information at your disposal. There needs to be a willingness to disobey an order when you know that it is not the best option. If you don't have all of the information (or not enough information to make an informed choice), then following the order would be acting on this trait.  

\subsection{Some Virtues and Examples according to Virtue Consequentialism}

Generosity is commonly touted as a virtue. There are many reasons to think that it is. We can use Virtue Consequentialism to give an explanation for it being a virtue. Generosity is a character trait and there are several other traits which one could have in its place. For example, one could be greedy or selfless (in this context) or any other trait in between. People consistently acting generously would tend to produce more good than them consistently acting greedy (egoism) or them consistently acting for others without regard to themselves (totally selfless) (because then they would be the one needing help and it would be a vicious cycle). 

Courage is a common example found in all virtue theories. Virtue Consequentialism can also accommodate this as a virtue. There are many different traits which could fill the place of courage, like foolhardiness and cowardice. People consistently acting foolhardy would, one can expect, lead to many unfortunate circumstances and unnecessary additional suffering in various situations (this is similar to selflessness). People consistently acting cowardly, on the other hand, will have a lot of unnecessary suffering as well because people would be too timid to give assistance. The best trait is courage because of the consequences.    

There are many traits one could have when it comes to their willingness to crack jokes and make fun in a situation. One could be willing to make jokes regardless of the circumstance or one could be overly formal and never crack jokes. The virtue, in this case, would be the disposition to crack jokes when the expected outcome of making a joke in that circumstance is higher (or the highest) than not cracking a joke. A world where people consistently cracked jokes would have a lower value than this because that world would likely have many instances of making jokes at other people’s expense or ‘hurtful humor’ not present in a world with this disposition. Similarly, a world where people consistently failed to make jokes would be lower because it would lack many of the positives which a good laugh provides.

There are many different traits one could have when it comes to their work-ethic. One could be tenacious (always will to work, never stopping), lazy (never wanting to do anything), fickle (always switching jobs for the more fun one), and so on. The optimal trait would be one where a person has a tendency to work and be willing to do the unfun jobs, but make sure that there is time for having fun outside of work, not working themselves to death. A world where everyone consistently switched to more fun jobs would crumble pretty quick because the unfun ones tend to be necessary, same would likely happen in the case of laziness. A world where everyone worked without stopping would lack the happiness of having fun. This trait would have the best of all the worlds.
