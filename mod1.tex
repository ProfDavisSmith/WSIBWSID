\part{What is Ethics? Moral Relativism Vs Moral Objectivism}
\label{ch.modone}
\addtocontents{toc}{\protect\mbox{}\protect\hrulefill\par}
\chapter{Part 1: What is Ethics? What is Relativism?}
Much like the other sciences, Philosophy is broken up in to fields of study which share certain aspects concerning methodology. There are many different fields which one could specialize in but each one has three distinct levels of inquiry. These levels, roughly, in order of abstraction, are 'meta', 'normative/theoretical', and 'applied'. Starting with the most abstract, the 'meta' level concerns questions about the questions in the field. So, for example, metabiology would be an area of biology asking questions such as 'what exactly is a species?', 'where is the line between biology and chemestry?' and others of that sort. Meta-ethics, which is the content for this module and the next, asks questions of the form 'what do right and wrong even mean?', 'what are we doing when we make ethical claims?', 'is any action, regardless of circumstance, absolutely wrong?', and 'concerning actions, what is the difference between 'right and wrong' and 'good and bad'?'. The theoretical (for Ethics, the term is "normative") level concerns the hypotheses and explanations for various things in the field. Generally, these questions/explanations are tested and used in the applied level. Normative Ethics concerns questions like "what general feature(s) make some action right or wrong?", "what does it take to be a good person?", and others like that.  The final, lowest level, is the applied level. This is where they look at real world cases, actually test the hypotheses in the meta and normative levels, to see what happens. For Ethics, this is found in questions like "what should I do in this situation?". The majority of this class concerns Normative Ethics, with some adventures into the other levels. 

Ethics, fundamentally speaking, is a field of study which concerns value, duty, right, wrong, justice, fairness, obligations, and so on.  Within Normative Ethics, there are two 'branches', so to speak, which we will explore in this textbook. The first concerns questions regarding individual actions, so, basically, "what general feature makes an action right or wrong?". This is where we will spend most of our time as it's more practically useful and many of the motivations behind contemporary civil rights movements can be found there. The second branch concerns what makes a good or bad person. Basically, "what general features make a good person?". You will likely see me use this terminology again but the first branch concerns what you should do and the second concerns who you should be. These are the titular questions of this textbook: `Who Should I Be?' and `What Should I Do?'.

\section{Going Meta: The Ethics Family Tree}  

You may have sometimes heard  the term “meta” used and attached it to a field of study, such as “metaphysics”, “metaethics”, “metabiology”, and even “metaphilosophy”. The prefix “\gls{meta-}” changes the level of questioning in the field. Physics is trying to answer certain questions, metaphysics is trying to answer questions about the questions in physics. If it helps, the prefix “meta-” means “asking questions about the questions in”, and then every additional “meta-” moves the level up a peg. The 'lowest' level of abstraction for a field is its applied version. So, you have physics and then you have applied physics, mathematics and applied mathematics. The next level up concerns abstractions from the findings in the applied field. This is either called 'theoretical' or (in the case of Ethics) 'normative'. And, finally, the most abstract level which most people need to be concerned about is 'meta-'. So, using this way of understanding, I can make the following chart: 

\begin{center}
\begin{tikzpicture}[modal]
		\node[world] (w1) [label=right:What is Ethics?] {Meta-Ethics};
		\node[world] (w2) [label=right:What Makes Actions Right or Wrong?, below=of w1]{Normative Ethics};
		\node[world] (w3) [label=right:What About These Real-World Cases?, below=of w2]{Applied Ethics};
		\path[->][label=right:What do `Right' and `Wrong' even mean?] (w1) edge (w2);
		\path[->] (w2) edge (w3);
	\end{tikzpicture}
\end{center}

\newglossaryentry{meta-}
{
  name=meta-,
  description={a prefix which changes the level of abstraction for the questions in the field, namely by moving it up. This means, roughly, `asking questions about the questions in...'}
}


Relativism and Objectivism are stances in the meta-level of Ethics. But there are two others, Nihilism and Skepticism, which are on the same level. One can think of the stances in Ethics as falling on a family tree. The stances are related to each other according to how they answer certain questions. For simplicity's sake, I have arranged these according to yes-no questions, but it is possible to get similar relations by going with degrees. The first question is 'are there moral facts?', that is 'are moral statements either true or false?'. Both Relativism and Objectivism answer the first question in the same way, they say 'yes' so they are called 'realist' stances. Nihilism says 'no' and Skepticism answers it by saying 'I don't know'. The Objectivist and the Relativist differ in that the objectivist claims that the morality of something doesn't change according to the culture or the person. Opinions about what is right or wrong may change, but the Objectivist says that what is actually moral/immoral is constant. People were just wrong about what was OK to do. The Relativist says that morality is determined by the culture/person, so when perspectives change, morality itself changes (this will be a problem for the theory later). 

\begin{center}
\begin{tikzpicture}[modal]
		\node[world] (w1) [label=right:] {\tiny{Are There Moral Facts?}};
		\node[world] (w2) [below left=of w1]{Yes};
		\node[world] (w3) [below=of w1]{No};
		\node[world] (w4) [below right=of w1]{No Clue};
		\node[world] (w5) [below=of w2]{\tiny{Moral Realism}};
		\node[world] (w6) [below=of w3]{\tiny{Moral Nihilism}};
		\node[world] (w7) [below=of w4]{\tiny{Moral Skepticism}};
		\node[world] (w8) [below=of w5]{\tiny{Do they change?}};
		\node[world] (w9) [below left=of w8]{No};
		\node[world] (w11) [below=of w9]{\tiny{Moral Objectivism}};
		\path[->] (w9) edge (w11);
		\node[world] (w10) [below right=of w8]{Yes};
		\node[world] (w12) [below=of w10]{\tiny{Moral Relativism}};
		\path[->] (w8) edge (w9);
		\path[->] (w8) edge (w10);
		\path[->] (w9) edge (w11);
		\path[->] (w10) edge (w12);
		\path[->] (w5) edge (w8);

		\path[->] (w1) edge (w2);
		\path[->] (w1) edge (w3);
		\path[->] (w1) edge (w4);
		\path[->] (w2) edge (w5);
		\path[->] (w3) edge (w6);
		\path[->] (w4) edge (w7);

	\end{tikzpicture}
\end{center}


\section{Relativism, in General}

Though we will be spending most of our time handling questions concerning Ethical/Moral Relativism, it is useful to go through exactly what that term '\Gls{relativism}' means. This sort of idea is found in several areas of philosophy and the notion (applied differently) is found in physics. This, in its broad sense, is the stance that there is no absolute truth (objective truth), but rather the truth about things varies from person to person or from culture to culture, depending on the version of relativism which is held. Much like any other theory, there are several different 'relativisms' which differ according to their scope as well as the aspect which the facts in that scope are relative to.  This may be that there are no objective truths at all, or that there aren’t any in some particular area. Relativism comes in many different flavors. The first is that everything is relative, this is global relativism. The other is that only certain things are relative, this is limited relativism. From this, we have to ask what are they relative to? These are either to your culture or to you individually. These two divisions into relativism gives us a total of 4 different 'classes' of relativism which one could find themselves in.

\newglossaryentry{relativism}
{
  name=relativism,
  description={the stance that there are no absolute, objective truths about something; rather the `truths' are \emph{relative}, changing according to what either the individual or the culture believes. Typically, this term appears after the field or area which is claimed to have only relative truths}
}


First, we have \gls{Global Cultural Relativism}. This says that all truths, all facts, are determined by what your culture believes. So, for example, if this were true, there would be no fact of the matter about whether the Earth is flat, rather it would be flat for one culture (The flat earthers) and round for another. This same thing would apply to all other facts, such as mathematics. In a more applied case, there is no way for one culture to (correctly/accurately) tell another that they are wrong about something, because regardless of the belief, they would be correct and the other culture would be correct too.

\newglossaryentry{Global Cultural Relativism}
{
  name=Global Cultural Relativism,
  description={the stance that there are no absolute, objective truths about anything at all, not just ethics; rather the `truths' are \emph{relative}, changing according to what the culture believes}
}


The second class of Relativism is \gls{Global Individual Relativism}. This says that all facts are relative to and determined by what the individual believes. And I do mean all of them. So, for example, if this were true and a little kid believed, with all their little heart, that unicorns existed, then, for them, unicorns exist. There would be no way for another person to tell them that they were mistaken because the fact about whether unicorns exist is determined by what each individual believes. This is the ultimate, extreme, form of the phrase 'you do you, bro'.

\newglossaryentry{Global Individual Relativism}
{
  name=Global Individual Relativism,
  description={the stance that there are no absolute, objective truths about anything at all, not just ethics; rather the `truths' are \emph{relative}, changing according to what the individual believes}
}


Those last two are a little too extreme for anyone to take seriously any more. Global Individual Relativism was popular during Socrates' time, but it quickly fell out of favor because of how, frankly, crazy it is. The next version is a lot more popular now a days and it's easy to find examples of it which people could find appealing. This is \gls{Limited Cultural Relativism}. 'Limited' means that it doesn't apply to all facts, rather just a small, limited variety of them. Normally, when discussing the stance, 'Limited' is replaced with the area/class of facts which are claimed to be relative. For example, one could be a Aesthetic Cultural Relativist; this means that one thinks that facts about beauty/art are determined by the culture's beliefs in those things. But, this should not be confused with 'beauty is in the eye of the beholder' because that would be Aesthetic Individual Relativism. 

\newglossaryentry{Limited Cultural Relativism}
{
  name=Limited Cultural Relativism,
  description={the stance that there are no absolute, objective truths about anything in a particular area of context; the `truths' in this area are relative to what the culture believes. Typically, the term `limited' is replaced with a term indicating which area is claimed to have only culturally relative truths}
}


The fourth and final classification of relativism is similar to the previous. This is Limited Individual Relativism. Like before, it doesn't say that all facts are relative, rather it says that only a small section of them are. As before also, 'limited' is replaced with the area which it concerns. The examples for this could be reasonable or just plain silly.  To use an example from before, Aesthetic Individual Relativism is the stance that there's no fact of the matter regarding art or beauty, rather whether or not something is beautiful is determined by whether or not the individual finds it pretty.

\newglossaryentry{Limited Individual Relativism}
{
  name=Limited Individual Relativism,
  description={the stance that there are no absolute, objective truths about anything in a particular area of context; the `truths' in this area are relative to what the individual believes. Typically, the term `limited' is replaced with the term for the area claimed to have only individually relative truths}
}


\begin{center}
\begin{tikzpicture}[modal]
		\node[world] (w1) [label=left:\tiny{At least some truths are relative}] {\tiny{Relativism}};
		\node[world] (w2) [label= above: \tiny{All truths are relative}, above=of w1]{\tiny{Global}};
		\node[world] (w3) [label=below: \tiny{Some truths are relative}, below=1cm of w1]{\tiny{Limited}};
		\node[world] (w4) [label=below: \tiny{All truths are relative to your culture},right=1cm of w2]{\tiny{Cultural}};
		\node[world] (w5) [label=below: \tiny{all truths are relative to you},left=1cm of w2]{\tiny{Individual}};
		\node[world] (w6) [label=above: \tiny{Some truths are relative to your culture}, left=1cm of w3]{\tiny{Cultural}};
		\node[world] (w7) [label=above: \tiny{Some truths are relative to you},right=1cm of w3]{\tiny{Individual}};

		\path[->] (w1) edge (w2);
		\path[->] (w1) edge (w3);
		\path[->] (w2) edge (w4);
		\path[->] (w2) edge (w5);
		\path[->] (w3) edge (w6);
		\path[->] (w3) edge (w7);

	\end{tikzpicture}
\end{center}


\chapter{The Challenge of Cultural Relativism by James Rachels}\autocite{Rachels1}

\factoidbox{“Morality  differs  in  every  society,  and  is  a  convenient  term  for 
socially  approved habits.” Ruth  Benedict,  Patterns  of Culture
(1934)}

\section{2.1 How Different Cultures Have Different Moral Codes} 
Darius, a king of ancient Persia, was intrigued by the variety of cultures 
he  encountered  in  his  travels.  He  had  found,  for  example,  that  the 
Callatians  (a  tribe  of  Indians)  customarily  ate  the  bodies  of  their  dead 
fathers.  The  Greeks,  of  course,  did  not  do  that—the Greeks  practiced 
cremation and regarded the funeral pyre as the natural and fitting way to 
dispose  of  the  dead.  Darius  thought  that  a  sophisticated  understanding 
of  the  world  must  include  an  appreciation  of  such  differences  between 
cultures. One day, to teach this lesson, he summoned some Greeks who 
happened  to  be  present  at  his  court  and  asked  them  what  they  would 
take  to  eat  the  bodies  of  their  dead  fathers.  They  were  shocked,  as 
Darius knew they would be, and replied that no amount of money could 
persuade them to do such a thing. Then Darius called in some 
Callatians,  and  while  the  Greeks  listened  asked  them  what  they  would 
take to burn their dead fathers' bodies. The Callatians were horrified and 
told Darius not even to mention such a dreadful thing.

This  story,  recounted  by  Herodotus  in  his  History  illustrates  a  recurring 
theme in the literature of social science: Different cultures have different 
moral  codes.  What  is  thought  right  within  one  group  may  be  utterly 
abhorrent  to the members  of another  group, and  vice versa. Should  we 
eat  the  bodies  of  the  dead  or  burn  them?  If  you  were  a  Greek,  one 
answer  would  seem  obviously  correct;  but  if  you  were  a  Callatian,  the 
opposite would seem equally certain. 

It  is  easy  to  give  additional  examples  of  the  same  kind.  Consider  the 
Eskimos.  They  are  a  remote  and  inaccessible  people.  Numbering  only 
about  25,000,  they  live  in  small,  isolated  settlements  scattered  mostly 
along  the  northern  fringes  of  North  America  and  Greenland.  Until  the 
beginning  of  the  20th  century,  the  outside  world knew  little  about  them. 
Then explorers began to bring back strange tales. 

Eskimos customs turned out to be very different from our own. The men 
often  had  more  than  one  wife,  and  they  would  share  their  wives  with 
guests,  lending  them  for  the  night  as  a  sign  of  hospitality.  Moreover, 
within  a  community,  a  dominant  male  might  demand  and  get  regular 
sexual access to other men's wives. The women, however, were free to 
break  these  arrangements  simply  by  leaving  their  husbands  and  taking 
up  with  new  partners—free,  that  is,  so  long  as  their  former  husbands  
chose  not  to  make  trouble.  All  in  all,  the  Eskimo  practice  was  a  volatile 
scheme that bore little resemblance to what we call marriage. 

But  it  was  not only  their  marriage  and sexual  practices  that  were 
different.  The  Eskimos also  seemed  to  have less  regard for  human  life. 
Infanticide,  for  example,  was  common.  Knud  Rasmussen,  one  of  the 
most famous early explorers, reported that be met one woman who bad 
borne  20  children  but had  killed  10  of  them  at  birth.  Female  babies,  he 
found,  were  especially  liable  to  be  destroyed,  and  this  was  permitted 
simply at the parents' discretion, with no social stigma attached to it. Old 
people  also,  when  they  became  too  feeble  to  contribute  to  the  family, 
were left out in the snow .to die. So there seemed to be, in this society, 
remarkably little respect for life. 

To the general public, these were disturbing revelations. Our own way of 
living  seems  so  natural  and  right  that  for  many  of  us  it  is  hard  to 
conceive of  others  living  so  differently.  And  when  we  do  hear  of  such 
things, we tend immediately to categorize those other peoples as 
"backward"  or  "primitive."  But  to  anthropologists  and  sociologists,  there 
was nothing particularly surprising about the Eskimos. Since the time of 
Herodotus,  enlightened  observers  have  been  accustomed  to  the  idea 
that  conceptions  of  right  and  wrong  differ  from  culture  to  culture.  If  we 
assume that our ideas of right and wrong will be shared by all peoples as 
all times, we are merely naive.

\section{2.2 Cultural Relativism} 
To  many  thinkers,  this  observation—"Different  cultures  have  different 
moral  codes"— has  seemed  to  be  the  key  to  understanding  morality. 
The idea of universal truth in ethics, they say, is a myth. The customs of 
different societies are all that exist. These customs cannot be said to be 
"correct" or "incorrect," for that implies we have an independent standard 
of  right  and  wrong  by  which  they  may  be  judged.  But  there  is  no  such 
independent standard; every standard is culture-bound. The great 
pioneering  sociologist  William  Graham  Sumner, writing in  1906,  put  the 
point like this: 
\factoidbox{The  "right"  way  is  the  way  which  the  ancestors  used  and  which 
has  been handed down. The tradition  is its own warrant. It  is not 
held subject to verification by experience. The notion of right is in 
the folkways. It is not outside of them, of independent origin, and 
brought to test them. In the folkways, whatever is, is right. This is 
because they are traditional, and therefore contain in themselves 
the  authority  of  the  ancestral  ghosts.  When  we  come  to  the 
folkways we are at the end of our analysis}

This line of thought has probably persuaded more people to be skeptical 
about  ethics  than  any  other  single  thing.  Cultural  Relativism,  as  it  has 
been called, challenges our ordinary belief in the objectivity and 
universality  of  moral  truth.  It  says,  in  effect,  that  there  is  not  such  thing 
as universal truth in ethics; there are only the various cultural codes, and 
nothing more. Moreover, our own code has no special status; it is merely 
one among many. 

As we shall see, this basic idea is really a compound of several different 
thoughts.  It  is  important  to  separate  the  various  elements  of  the  theory 
because,  on  analysis,  some  parts  turn  out  to  be  correct,  while  others 
seem  to  be mistaken.  As  a  beginning,  we may  distinguish  the  following 
claims, all of which have been made by cultural relativists: 

\begin{enumerate}
\item[1] Different societies have different moral codes.
\item[2] There  is  no  objective  standard  that  can  be  used  to  judge 
one societal code better than another.
\item[3] The moral code of our own society has no special status; it 
is merely one among many. 
\item[4] There  is no "universal truth"  in ethics; that  is, there are no 
moral truths that hold for all peoples at all times. 
\item[5] The moral code of a society determines what is right within 
that society; that is, if the moral code of a society says that 
a  certain  action  is  right,  then  that  action  is  right,  at  least 
within that society. 
\item[6] It  is  mere  arrogance  for  us  to  try  to  judge  the  conduct  of 
other  peoples.  We  should  adopt  an  attitude  of  tolerance 
toward the practices of other cultures.
\end{enumerate}

Although  it  may  seem  that  these  six  propositions  go  naturally  together, 
they  are  independent  of  one  another,  in  the  sense  that  some  of  them 
might  be  false  even  if  others  are  true.  In  what  follows,  we  will  try  to 
identify what  is  correct in Cultural Relativism, but  we will  also be 
concerned to expose what is mistaken about it.

\section{2.3 The Cultural Differences Argument} 
Cultural Relativism is a theory about the nature of morality. At first blush 
it  seems  quite  plausible.  However,  like  all  such  theories,  it  may  be 
evaluated  by  subjecting  it  to  rational  analysis;  and  when  we  analyze 
Cultural Relativism we find that it is not so plausible as it first appears to 
be. 

The first thing we need to notice is that at the heart of Cultural Relativism 
there  is  a  certain  form  of  argument.  The  strategy  used  by  cultural 
relativists  is  to  argue  from  facts  about  the  differences  between  cultural 
outlooks to a conclusion about the status of morality. Thus we are invited 
to accept this reasoning: 
\begin{enumerate}
\item[1] The Greeks believed it was wrong to eat the dead, 
whereas  the  Callatians  believed  it  was  right  to  eat  the 
dead. 
\item[2] Therefore,  eating  the  dead  is  neither  objectively  fight  nor 
objectively  wrong.  It  is  merely  a  matter  of  opinion,  which 
varies from culture to culture. 
\end{enumerate}
Or, alternatively:
\begin{enumerate} 
\item[1] The  Eskimos  see  nothing  wrong  with  infanticide,  whereas 
Americans believe infanticide is immoral. 
\item[2] Therefore, infanticide is neither objectively right nor 
objectively  wrong.  It  is  merely  a  matter  of  opinion,  which 
varies from culture to culture. 
\end{enumerate}
Clearly,  these  arguments  are  variations  of  one  fundamental  idea  They 
are both special cases of a more general argument, which says: 
\begin{enumerate}
\item[1] Different cultures have different moral codes. 
\item[2] Therefore,  there  is  no  objective  "truth"  in  morality.  Right 
and  wrong  are  only  matters  of  opinion,  and  opinions  vary 
from culture to culture. 
\end{enumerate}
We may call this  the Cultural Differences  Argument. To many people, it 
is persuasive. But from a logical point of view, is it sound?

It  is  not  sound.  The  trouble  is  that  the  conclusion  does  not  follow  from 
the  premise— that  is,  even  if  the  premise  is  true,  the  conclusion  still 
might  be  false.  The  premise  concerns  what  people  believe.  In  some 
societies,  people  believe  one  thing;  in  other  societies,  people  believe 
differently.  The  conclusion,  however,  concerns  what  really  is  the  case. 
The trouble is that this sort conclusion does not follow logically from this 
sort of premise. 

Consider  again  the  example  of  the  Greeks  and  Callatians.  The  Greeks 
believed  it  was  wrong  to  eat  the  dead;  the  Callatians  believed  it  was 
right. Does it follow, from the mere fact that they disagreed, that there is 
no objective truth in the matter? No, it does not follow; for it could be that 
the practice was objectively right (or wrong) and that one or the other of 
them was simply mistaken. 

To make the point clearer, consider a different matter In some societies, 
people  believe  the  earth  is  flat  In  other  societies,  such  as  our  own,
people believe  the  earth  is  (roughly)  spherical.  Does  it  follow,  from  the 
mere  fact  that  people  disagree,  that  there  is  no  "objective  truth"  in 
geography?  Of  course  not;  we  would  never  draw  such  a  conclusion 
because we realize that, in their beliefs about the world, the members of 
some societies might simply be wrong. There is no reason to think that if 
the world is round everyone must know it. Similarly, there is no reason to 
think that if there is moral truth everyone must know it. The fundamental 
mistake in the Cultural Differences Argument is that it attempts to derive 
a substantive conclusion about a subject from the mere fact that people 
disagree about it. 

This is a simple point of logic, and it is important not to misunderstand it. 
We are not saying (not yet, anyway) that the conclusion of the argument 
is false. It is still an open question whether the conclusion is true or false. 
The  logical  point  is  just  that  the  conclusion  does  not  follow  from  the 
premise.  This  is  important,  because  in  order  to  determine  whether  the 
conclusion is true, we need arguments in its support. Cultural Relativism 
proposes  this  argument,  but  unfortunately  the  argument  turns  out to  be 
fallacious. So it proves nothing.  

\section{2.4 The Consequences of Taking Cultural Relativism Seriously} 
Even  if  the  Cultural  Differences  Argument  is  invalid,  Cultural  Relativism 
might still be true. What would it be like if it were true? 

In the passage quoted above, William Graham Sumner summarizes the 
essence of Cultural Relativism. He says that there is no measure of right 
and wrong other than the standards of one's society: "The notion of right 
is  in  the  folkways.  It  is  not  outside  of  them,  of  independent  origin,  and 
brought to test them. In the folkways, whatever is, is right." Suppose we 
took this seriously. What would be some of the consequences? 
\subsection{1. We could no longer say that the customs of other societies are morally 
inferior to our own.} 
This, of course, is one of the main points stressed by 
Cultural  Relativism. We  would  have to  stop condemning  other  societies 
merely  because  they  are  "different:'  So  long  as  we  concentrate  on 
certain  examples,  such  as  the  funerary  practices  of  the  Greeks  and 
Callatians, this may seem to be a sophisticated, enlightened attitude. 

However,  we  would  also  be  stopped  from  criticizing  other,  less  benign 
practices. Suppose a society waged war on its neighbors for the purpose 
of taking slaves. Or suppose a society was violently anti-Semitic and its 
leaders  set  out  to  destroy the  Jews.  Cultural  Relativism  would  preclude 
us  from  saying that  either  of  these  practices  was  wrong. We  would  not 
even be able to say that a society tolerant of Jews is better than the anti-
Semitic society, for  that would imply some  sort of transcultural standard 
of  comparison.  The  failure  to  condemn  these  practices  does  not  seem 
enlightened;  on  the  contrary,  slavery  and  anti-Semitism  seem  wrong 
wherever they occur. Nevertheless, if we took Cultural Relativism 
seriously,  we  would  have  to  regard  these social  practices  as  also 
immune from criticism. 

\subsection{2. We could decide whether actions are right or wrong just by consulting 
the  standards of  our  society.}
Cultural  Relativism  suggests  a  simple test 
for  determining  what  is  right  and  what  is  wrong:  All  one  need  do  is  ask 
whether  the  action  is  in  accordance  with  the  code  of  one's  society. 
Suppose in 1975, a resident of South Africa was wondering whether his 
country's policy of apartheid—a rigidly racist system—was morally 
correct.  All  he  has  to  do  is  ask  whether  this  policy  conformed  to  his 
society's  moral  code.  If  it  did,  there  would  have  been  nothing  to  worry 
about, at least from a moral point of view. 

This  implication  of  Cultural  Relativism  is  disturbing  because  few  of  us 
think that our society's code is perfect; we can think of ways it might be 
improved. Yet Cultural Relativism would not only forbid us from criticizing 
the  codes  of  other  societies;  it  would  stop  us  from  criticizing  our  own. 
After  all,  if  right  and  wrong  are  relative  to  culture,  this  must  be  true  for 
our own culture just as much as for other cultures. 

\subsection{3. The idea of moral progress is called into doubt.}
Usually, we think that at  least  some  social  changes  are  for  the  better.  (Although,  of  course, 
other  changes  may  be  for  the  worse.)  Throughout  most  of  Western 
history the place of women in society was narrowly circumscribed. They 
could  not  own  property;  they  could  not  vote  or  hold  political  office;  and 
generally they were under the almost absolute control of their husbands. 
Recently  much  of  this  has  changed,  and  most  people  think  of  it  as 
progress.

If  Cultural  Relativism  is  correct,  can  we  legitimately  think  of  this  as 
progress? Progress means replacing a way of doing things with a better 
way.  But  by  what  standard  do  we  judge  the  new  ways as  better?  If  the 
old ways were in accordance with the social standards of their time, then 
Cultural  Relativism  would  say  it  is  a  mistake  to  judge  them  by  the 
standards of a different time. Eighteenth-century society was, in effect, a 
different society from the one we have now. To  say that  we have made 
progress implies  a  judgment  that  present-day society  is better,  and that 
is  just  the  sort  of  transcultural  judgment  that,  according  to  Cultural 
Relativism, is impermissible. 

Our  idea  of  social  reform  will  also  have  to  be  reconsidered.  Reformers 
such as Martin Luther King, Jr., have sought to change their societies for 
the better. Within the constraints imposed by Cultural Relativism, there is 
one way this might be done. If a society is not living up to its own ideals, 
the  reformer  may  be  regarded  as  acting  for  the  best:  The  ideals  of  the 
society  are  the  standard  by  which  we  judge  his  or  her  proposals  as 
worthwhile. But the "reformer" may not challenge the ideals themselves, 
for those ideals are by definition correct. According to Cultural 
Relativism,  then,  the  idea  of  social  reform  makes  sense  only  in  this 
limited way. 

These three consequences of Cultural Relativism have led many 
thinkers  to reject it as implausible on  its face. It does make  sense, they 
say,  to  condemn  some  practices,  such  as  slavery  and  anti-Semitism, 
wherever  they  occur.  It  makes  sense  to  think  that  our  own  society  has 
made some moral progress, while admitting that it is still imperfect and in 
need  of  reform.  Because  Cultural  Relativism says  that  these  judgments 
make no sense, the argument goes, it cannot be right. 

\section{2.5 Why There Is Less Disagreement Than It Seems} 
The original impetus for Cultural Relativism comes from the observation 
that cultures differ dramatically in their views of right and wrong. But just 
how much do they differ? It is true that there are differences. However, it 
is easy to overestimate  the extent of those differences, Often, when we 
examine  what  seems  to  be  a  dramatic  difference,  we  find  that  the 
cultures do not differ nearly as much as it appears. 

Consider a culture in which people  believe it is wrong to eat cows. This 
may even be a poor culture, in which there is not enough food; still, the 
cows are not to be touched. Such a society would appear to have values 
very  different  from  our  own.  But  does  it?  We  have  not  yet  asked why 
these  people  will not  eat  cows. Suppose  it  is because  they believe  that 
after death the souls of humans inhabit the bodies of animals, especially 
cows, so  that  a  cow  may be  someone's  grandmother.  Now do  we want 
to  say  that  their  values  are  different  from  ours?  No;  the  difference  lies 
elsewhere. The difference is in our belief systems, not in our values. We 
agree that we shouldn't eat Grandma; we simply disagree about whether 
the cow is (or could be) Grandma. 

The point is that many factors work together to produce the customs of a 
society. The  society's values are  only one  of them. Other matters, such 
as the religions and factual beliefs held by its members, and the physical 
circumstances  in  which  they  must  live,  are  also  important.  We  cannot 
conclude, then, merely because customs differ, that there is a 
disagreement about values. The difference in customs may be 
attributable to some other aspects of social life. Thus there may be less 
disagreement about values than there appears to be. 

Consider  again the  Eskimos,  who  often  kill  perfectly  normal  infants, 
especially girls. We do not approve of such things; a parent who killed a 
baby  in  our  society  would  be  locked  up.  Thus  there  appears  to  be  a 
great  difference  in  the  values  of  our  two  cultures.  But  suppose  we  ask 
why  the  Eskimos  do  this.  The  explanation  is  not  that  they  have  less 
affection  for  their  children  or  less  respect  for  human  life.  An  Eskimo 
family will always protect its babies if conditions permit. But they live in a 
harsh environment, where food is in short supply. A fundamental 
postulate  of  Eskimos  thought  is:  "Life  is  hard,  and  the  margin  of  safety 
small.” A family may want to nourish its babies but be unable to do so. 
As  in many "primitive" societies,  Eskimo mothers will nurse  their  infants 
over a much longer period of time than mothers in our culture. The child 
will  take  nourishment  from  its  mother's  breast  for  four  years,  perhaps 
even longer. So even in the best of times there are limits to the number 
of  infants  that  one  mother  can  sustain.  Moreover,  the  Eskimos  are  a 
nomadic  people—unable  to  farm,  they  must  move  about  in  search  of 
food.  Infants  must  be carried,  and  a  mother  can  carry only  one  baby in 
her parka as she travels and goes about her outdoor work. Other family 
members help whenever they can. 

Infant girls are more readily disposed of because, first, in this society the 
males are the primary food providers—they are the hunters, according to 
the traditional division of labor—and it is obviously important to maintain 
a  sufficient  number  of  food  providers.  But  there  is  an  important  second 
reason as well. Because the hunters suffer a high casualty rate, the adult 
men who die prematurely far outnumber the women who die early. Thus 
if  male  and  female  infants  survived  in  equal  numbers,  the  female  adult 
population would greatly outnumber the male adult population. 
Examining the available statistics, one writer concluded that "were it not 
for female infanticide...there  would  be  approximately  one-and-a-half 
times  as  many  females  in  the  average Eskimo  local  group  as  there are 
food-producing males." 

So  among  the  Eskimos,  infanticide  does  not  signal  a  fundamentally 
different  attitude  toward  children.  Instead,  it  is  a  recognition  that  drastic 
measures  are  sometimes  needed  to  ensure  the  family's  survival.  Even 
then, however, killing the baby is not the first option considered. 
Adoption  is  common;  childless  couples  are  especially  happy  to  take  a 
more fertile couple's "surplus." Killing is only the last resort. I emphasize 
this  in  order  to  show  that  the  raw  data  of  the  anthropologists  can  be 
misleading;  it  can  make  the  differences  in  values  between  cultures 
appear  greater  than  they  are.  The  Eskimos'  values  are  not  all  that 
different from our values. It is only that life forces upon them choices that 
we do not have to make. 

\section{2.6 How All Cultures Have Some Values in Common} 
It  should  not  be  surprising  that,  despite  appearances,  the  Eskimos  are 
protective  of  their  children.  How  could  it  be  otherwise?  How  could  a 
group survive that did not value its young? It is easy to see that, in fact, 
all cultural groups must protect their infants: 
\begin{enumerate}
\item[1] Human  infants  are  helpless  and  cannot  survive  if  they are 
not given extensive care for a period of years. 
\item[2] Therefore,  if  a  group  did  not  care  for  its  young,  the  young 
would  not  survive,  and  the  older  members  of  the  group 
would  not  be  replaced.  After  a  while  the  group  would  die 
out. 
\item[3] Therefore,  any  cultural  group  that  continues  to  exist  must 
care for its young. infants that are not cared for must be the 
exception rather than the rule. 
\end{enumerate}
Similar reasoning shows that other values must be more or less 
universal. Imagine what it would be like for a society to place no value at 
all on truth telling. When one person spoke to another, there would be no 
presumption at all that he was telling the truth for he could just as easily 
be speaking falsely. Within that society, there would be no reason to pay 
attention  to  what  anyone  says.  (I  ask  you  what  time  it  is,  and  you  say 
"Four  o'clock:'  But  there  is  no  presumption  that  you  are  speaking  truly; 
you  could  just  as  easily  have  said  the  first  thing  that  came  into  your 
head. So I have no reason to pay attention to your answer; in fact, there 
was no point  in my asking you in the first place.) Communication would 
then  be  extremely  difficult,  if  not  impossible.  And  because  complex 
societies  cannot  exist  without  communication  among  their  members, 
society would become impossible. It follows that in any complex society 
there  must  be  a  presumption  in  favor  of  truthfulness.  There  may  of 
course be exceptions to  this rule: There may be  situations in which it is 
thought to be permissible to lie. Nevertheless, there will be exceptions to 
a rule that is in force in the society.
 
Here  is  one  further  example  of  the  same  type.  Could  a  society  exist  in 
which  there  was  no  prohibition  on  murder?  What  would  this  be  like? 
Suppose people were free to kill other people at will, and no one thought 
there  was anything  wrong  with  it.  In  such a  "society,"  no  one  could feel 
secure.  Everyone  would  have  to  be  constantly  on  guard.  People  who 
wanted to survive would have to avoid other people as much as 
possible.  This  would  inevitably  result  in  individuals  trying  to  become  as 
self-sufficient  as  possible— after  all,  associating  with  others  would  be 
dangerous. Society on any large scale would collapse. Of course, people 
might  band  together  in  smaller  groups  with  others  that  they  could  trust 
not  to  harm  them.  But  notice  what  this  means:  They  would  be  forming 
smaller  societies  that  did  acknowledge  a  rule  against  murder:  The 
prohibition of murder, then, is a necessary feature of all societies. 

There  is  a  general  theoretical  point  here,  namely,  that  \emph{there  are  some 
moral  rules  that  all  societies  will  have  in  common,  because  those  rules 
are necessary for society to exist.} The rules against lying and murder are 
two  examples.  And  in  fact,  we  do  find  these  rules  in  force  in  all  viable 
cultures. Cultures may differ in what they regard as legitimate exceptions 
to  the  rules,  but  this  disagreement  exists  against  a  background  of 
agreement on the larger issues. Therefore, it is a mistake to 
overestimate the amount of difference between cultures. Not every moral 
rule can vary from society to society. 
\section{2.7 Judging a Cultural Practice to Be Undesirable} 
In  1996,  a  17-year-old  girl  named  Fauziya Kassindja  arrived  at  Newark 
International  Airport  and  asked  for  asylum.  She  had  fled  her  native 
country  of  Togo,  a  small  west  African  nation,  to  escape  what  people 
there call excision. 

Excision is a permanently disfiguring procedure that is sometimes called 
"female  circumcision," although it bears little resemblance to the Jewish 
ritual.  More  commonly,  at  least  in Western newspapers,  it is  referred  to 
as  "genital  mutilation."  According  to  the World  Health  Organization,  the 
practice  is  widespread  in  26  African  nations,  and  two  million  girls  each 
year  are  "excised."  In  some  instances,  excision  is  part  of  an  elaborate 
tribal ritual, performed in small traditional villages, and girls look forward 
to  it  because  it  signals  their  acceptance  into  the  adult  world.  In  other 
instances, the practice is carried out by families living in cities on young 
women who desperately resist. 

Fauziya  Kassindja  was  the  youngest  of  five  daughters  in  a  devoutly 
Muslim  family.  Her  father,  who  owned  a  successful  trucking  business, 
was opposed to excision, and he was able to defy the tradition because 
of his wealth. His first four daughters were married without being 
mutilated.  But  when  Fauziya  was  16,  he  suddenly  died.  Fauziya  then 
came under the authority of his father, who arranged a marriage for her 
and prepared to have her excised. Fauziya was terrified, and her mother 
and oldest sister helped her to escape. Her mother, left without 
resources, eventually had to formally apologize and submit to the 
authority of the patriarch she had offended. 

Meanwhile, in America, Fauziya was imprisoned for  two years while the 
authorities decided what to do with her. She was finally granted asylum, 
but  not  before  she  became  the  center  of  a  controversy  about  how 
foreigners should regard the cultural practices of other peoples. A series 
of articles in the New York Times encouraged the idea that excision is a 
barbaric  practice  that  should  be  condemned.  Other  observers  were 
reluctant  to  be  so  judgmental—live  and  let  live,  they  said;  after  all,  our 
practices probably seem just as strange to them. 

Suppose we are inclined to say that excision is bad. Would we merely be 
applying  the  standards  of  our  own  culture?  If  Cultural  Relativism  is 
correct,  that  is  all  we  can  do,  for  there  is  no  cultural-neutral  moral 
standard to which we may appeal. Is that true? 

\section{Is There a Culture-Neutral Standard of Right and Wrong?} 
There  is,  of  course,  a  lot  that  can  be  said  against  the  practice  of 
excision. Excision is painful and it results in the permanent loss of sexual 
pleasure. Its short-term effects include hemorrhage, tetanus, and 
septicemia. Sometimes the woman dies. Longterm effects include 
chronic infection, scars that hinder walking, and continuing pain. 

Why, then, has it become a widespread social practice? It is not easy to 
say. Excision has no obvious social benefits. Unlike Eskimo infanticide, it 
is  not  necessary  for  the  group's  survival.  Nor  is  it  a  matter  of  religion. 
Excision  is  practiced  by  groups  with  various  religions,  including  Islam 
and Christianity, neither of which commend it. 

Nevertheless, a number of reasons are given in its defense. Women who 
are incapable of sexual pleasure are said to be less likely  to be 
promiscuous; thus there will be fewer unwanted pregnancies in 
unmarried women. Moreover, wives for whom sex is only a duty are less 
likely  to  be  unfaithful  to  their  husbands;  and  because  they  will  not  be 
thinking  about  sex,  they  will  be  more  attentive  to  the  needs  of  their 
husbands  and  children.  Husbands,  for  their  part,  are  said  to  enjoy  sex 
more  with  wives  who  have  been  excised.  (The  women's  own  lack  of 
enjoyment  is  said  to  be  unimportant.)  Men  will  not  want  unexcised 
women,  as  they  are  unclean  and  immature.  And  above  all,  it  has  been 
done since antiquity, and we may not change the ancient ways. 

It would be easy, and perhaps a bit arrogant, to ridicule these 
arguments. But we may notice an important feature of this whole line of 
reasoning:  it  attempts  to  justify  excision  by  showing  that  excision  is 
beneficial— men,  women,  and their  families  are  all  said  to be  better  off 
when women are excised. Thus we might approach this reasoning, and 
excision itself, by asking which is true: Is excision, on the whole, helpful 
or harmful? 

Here,  then,  is  the  standard  that  might  most  reasonably  be  used  in 
thinking  about  excision:  We  may  ask  whether  the  practice  promotes  or 
hinders the welfare of the people whose lives are affected by it. And, as 
a corollary, we may ask if there is an alternative set of social 
arrangements that would do a better job of promoting their welfare. If so, 
we may conclude that the existing practice is deficient. 

But  this  looks  like  just  the  sort  of  independent  moral  standard  that 
Cultural Relativism says cannot exist. It is a single standard that may be 
brought  to  bear  in  judging  the practices  of  any  culture,  at  any  time, 
including our own. Of course, people will not usually see this principle as 
being "brought in from the outside" to judge them, because, like the rules 
against lying and homicide, the welfare of its members is a value internal 
to all viable cultures. 

\section{Why Thoughtful People  May Nevertheless  Be Reluctant to Criticize 
Other Cultures.}

Although  they  are  personally  horrified  by  excision,  many  thoughtful 
people  are  reluctant  to  say  it  is  wrong,  for  at  least  three  reasons.  First, 
there  is  an  understandable  nervousness  about  "interfering  in  the  social 
customs of other peoples." Europeans and their cultural descendents in 
America have a shabby history of destroying native cultures in the name 
of Christianity and Enlightenment, not to mention self-interest. Recoiling 
from  this  record,  some  people  refuse  to  make  any  negative  judgments 
about  other  cultures,  especially  cultures  that  resemble  those  that  have 
been  wronged  in  the  past.  We  should  notice,  however,  that  there  is  a 
difference between (a) judging a cultural practice to be morally deficient 
and (b) thinking that we should announce the fact, conduct a campaign, 
apply diplomatic pressure, or send in the army to do something about it. 
The  first  is  just  a  matter  of  trying to  see  the world clearly,  from  a  moral 
point  of  view.  The  second  is  another  matter  altogether.  Sometimes  it 
may be right to "do something about it," but often it will not be. 

People  also  feel,  rightly  enough,  that  they  should  be  tolerant  of  other 
cultures. Tolerance is, no doubt, a virtue—a tolerant  person is willing to 
live  in  peaceful  cooperation  with  those  who  see  things  differently.  But 
there is nothing in the nature of tolerance that requires you to say that all 
beliefs,  all  religions,  and  all  social  practices  are  equally  admirable.  On 
the contrary, if you did not think that some were better than others, there 
would be nothing for you to tolerate. 

Finally,  people  may  be  reluctant  to  judge  because  they  do  not  want  to 
express  contempt  for  the  society  being  criticized.  But  again,  this  is 
misguided: To condemn a particular practice is not to say that the culture 
is  on  the  whole  contemptible  or  that  it  is  generally  inferior  to  any  other 
culture,  including  one's  own.  It  could  have  many  admirable  features.  In 
fact, we should expect this to be true of most human societies— they are 
mixes of good and bad practices. Excision happens to be one of the bad 
ones

\chapter{Part 2: Moral Relativism}

Moral/Ethical Relativism is a kind of limited relativism. It does not say that all facts are relative, rather it says that (only) facts concerning morality are. This stance might concern questions about whether a person is good or bad (it might say that the standards for being a good person are relative) but that stance is not popular and not the ordinary view. Moral Relativism is a very common stance that the 'woke' people might find appealing. It says that there's no absolute, objective facts about morality, what makes actions right or wrong, rather it says that the standards for rectitude are relative. They are determined by (depending on the version held) either the beliefs of the person doing the action or the beliefs of that doer's culture. Moral Individual Relativism is a rare stance these days, you are likely not going to find it on the street, so to speak, but you will likely come across  \gls{Moral Cultural Relativism} (often just called 'relativism'). We will focus our attention on that latter stance, but most, if not all, of the problems regarding Moral Cultural Relativism applies to the former too. MCR is not without support, for example take these two examples:

\newglossaryentry{Moral Cultural Relativism}
{
  name=Moral Cultural Relativism,
  description={the stance that there are no absolute, objective truths about morality or ethics; the moral `truths' are relative to what the culture believes. No other culture or could tell them that they are mistaken because they determine right from wrong},
}


\factoidbox{\noindent  \fontsize{20pt}{0pt}\textbf{King Darius}

Darius, a king of ancient Persia, was intrigued by the variety of cultures he encountered in his travels. The Callatians (a tribe in India) customarily ate the bodies of their dead fathers. The Greeks practiced cremation. Darius thought that an understanding of the world must include an appreciation of such differences. One day, he brought some Greeks who happened to be present and asked them what they would take to eat the bodies of their dead fathers. They were shocked and replied that no amount of money could persuade them to do such a thing. Then Darius called in some Callatians, and while the Greeks listened asked them what they would take to burn their dead fathers' bodies. The Callatians were horrified and told Darius not even to mention such a dreadful thing.\autocite{Herodotus1}}

Ask yourself, would you be willing to eat the body of your dead parent? Many of us, today, would be horrified at this notion, just as the Greeks were. But, it should also be noted that the Callatians had the same reaction to, from our perspective, perfectly normal practice of cremation.  This might make you think that morality and ethics is nothing more than norms of behavior which a society agrees to. To drive this idea home, consider this example: 

\factoidbox{\noindent  \fontsize{20pt}{0pt}\textbf{The Dip}

In some Inuit cultures, there is a common practice know as ‘the dip’ (my translation). When a child is born in the dead of winter, it is common for the mother to carve out a hole in the ice and place the child in the water. This kills the child instantly (infanticide). This is done purely at the parents’ discretion and there is no negative stigma about it. Old people also, when they became too feeble to contribute to the family, were left out in the snow to die.}

Just as before, this is, seemingly, a perfectly normal practice for those inuit cultures but from our cultural perspective, this is horrifying. In many cultures, caring for the elderly and those to young to fend for themselves is seen as a very important moral duty. Something must be very different if this culture does not have the same regards. 

These practices, which I have given, are radically different than the practices which most cultures have and, as I have hinted, might lead one to think that ethics is just a set of cultural norms, rules of behavior, with no one system better than any other. One might think that if there were a fixed set of moral rules, there would be no disagreements about morality. But, there are disagreements, so there must be no absolute moral rules. 

That line of reasoning is an argument. An argument is different from a debate in that a debate requires disagreement and the drive to prove the other side wrong. An argument is just a line of reasoning which is intended to prove a conclusion. Some arguments are faulty, they fail to adequately prove their conclusion. To see how and why, for this case, let's break it down into the most relevant parts, the premises and the conclusion:

\section{The Cultural Differences Argument and Its Problems}

Not moments ago, I gave the `ordinary language' version of the Cultural Differences Argument. But, ordinary language is full of flowers and distracting prose. To avoid those and to get to what actually matters, we extract only the supporting sentences (the premises) and the supported sentence (the conclusion). For the Cultural Differences Argument, it looks like this: 

\begin{earg}
\item[1] Different Cultures have different moral codes.
\item[2] Therefore, there's no fact of the matter regarding morality (it's culturally relative).
\end{earg}

To start us off, do you think that this is a good argument? I am going to be frank, this argument is not good. It may be persuasive, but it does not get at reality. For an argument to really prove that some fact is really the case (the 'therefore' part in this case), there is a pretty high standard. First, the argument needs to be valid, that is, if we assume that the premises (the support is true), then the conclusion needs to follow necessarily from them. For this argument, the support (the premises) concern what people believe and the conclusion concerns reality. Take this example, which has the exact same form as the argument above:
\begin{earg}
\item[1] Different cultures have different views about the shape of the Earth.
\item[2] Therefore, there's no fact of the matter about the shape of the Earth (it's culturally relative).
\end{earg}
If you accept the Cultural Differences Argument, then you will need to also claim that there's no fact of the matter about anything where there's the slightest disagreement. But, this is not the only problem for the Cultural Differences Argument and Moral Cultural Relativism, there are more:

\subsection{Cultural Oppression}

Though the argument may not be any good, all that shows is that the argument isn't good. It does not show that morality is objective. But, if we look at the last example I gave, involving the Kaelons, we see something interesting. If morality is relative to a culture, there's nothing stopping one culture, morally speaking, from oppressing people in their culture and oppressing people in other cultures. For example, if there are no objective moral facts, then there would be nothing wrong with the Kaelons declaring war on another culture for allowing someone to receive asylum. Similarly, there would be nothing wrong with the Kaelons forcing members of their culture to unwillingly commit suicide. This sort of case can be found in history as well, with the sear number of examples being too numerous to list. This is a problem which will appear again later in this class, because it's more obvious inside another argument.  

\section{The Cultural Imperialism Argument and Its Problems}

There's one other argument which one could come across in favor of Moral Cultural Relativism, this is the Cultural Imperialism Argument. To start us off, Cultural Imperialism is the view that your culture as all of the right answers and that all of the other cultures are mistaken. In virtue of this, you are justified in forcing other cultures to change to your own. Take, for example the treatment of Native Americans in America. The treatment of the Indians in India by the British. The treatment of the Latin Americans by the Spanish, and so on. Basically, the colonial practices of Europeans (and other groups, but the Europeans were some of the worst offenders) all trace back to this view, which leads to oppressive practices and forced cultural change. 

Looking over world history, this practice was wrong, they should not have done it. So, we seem to be able to use this as a premise and, maybe, just maybe, be able to get some kind of Moral Relativism from it.

\subsection{The Cultural Imperialism Argument}
\begin{earg}
    \item[1]Cultural Imperialism is morally wrong. (Intuition).
    \item[2]If Cultural Imperialism is morally wrong, then it is wrong to judge the values of one culture against the values of another. (Consequence of it being wrong)
    \item[3]If it is wrong to judge the values of one culture against the values of another, then no person is ever justified in criticizing the moral norms of another culture. (Consequence of the consequence)
    \item[4]If no person is ever justified in criticizing the moral norms of another culture, there are no non-relative moral truths (they are all relative to culture). (Consequence)
    \item[5]Therefore, there are no non-relative moral truths (they are all relative to culture).
\end{earg}

This argument is a wee-bit stronger than the Cultural  Differences argument, but it too has its fair share of problems. Can you see the big one? I'll give you a hint, look at the first and the last lines (the intuition and the conclusion). The first line gives us a non-relative moral truth, that this practice is wrong, but the conclusion claims that there are none of those. The first line disproves the conclusion and the conclusion disproves the first line. They are contradictory, the argument is hypocritical, so to speak. There are ways of strengthening this argument, limiting the scope of the Relativism, yet again. For example, one could divide the practices into internal (those concerning members of the same culture) and external (those concerning members of different cultures). If you go this route, then you could, maybe, claim that cultural imperialism is wrong because it's an external practice and only internal practices are relative. This would be Cultural Relativism About Internal Norms (CRAIN). However, this too has its issues, not concerning the arguments for it, but concerning the facts of the world. Take this example, from Star Trek:


\factoidbox{\noindent  \fontsize{20pt}{0pt}\textbf{Half a Life}

For most of the world, the elderly are treated with respect and cared for until their natural death. The Kaelon people are different. In this society, the prevailing view is that it's the duty of the 'elderly' to leave their tasks to the next generation, that forcing the next generation to care for the elderly is cruel,  and that having death come for a person  seemingly randomly is heartless. So, when a Kaelon turns 60 years old, they undergo the Resolution. In it, a great party is thrown, celebrating their life and accomplishments, and afterwards, the Kaelon commits suicide. Living past this point is seen as greedy, their time is up and their accomplishments after this point are seen as stolen from the next generation. This is expected of all Kaelons, refusing to kill yourself at the allotted time will cause even family members to be ashamed of you. If a Kaelon seeks asylum to avoid the Resolution, the Kaelons will declare war on the other culture in order to kill the person.\autocite{HalfaLife}}

We can say that the Kaelons declaring war on another culture over this is wrong, but the Kaelons forcing members of their culture to commit suicide is an internal practice, it only concerns those which are members of the same culture, so, if you think that this practice is wrong and should be stopped, we have an example of when cultural imperialism might actually be the right thing to do.  And, in fact, if we look into history, we find some cases where cultural imperialism was, in fact, the right thing to do (as in, forcing a change in a culture). For example, any culture which sided with the Allies during WW2 in order to stop the Nazis from killing the Jewish people was engaged in this sort of imperialism.  Figuring out when it's right to force a culture to change its practices is a bit tricky, but we can say that there are some cases where it needs to be done. 

\chapter{Part 3: Problems for Moral Relativism}

Even with the arguments for Moral Cultural Relativism not being good, there's still a chance, no matter how slim, that it's the correct view of the world. It could be possible that people smarter than me just haven't figured out a good argument for it. So, to really disprove moral relativism, we need to look at the consequences and see whether they are the sort of things which actually line up with realty. For this, we will look at 3 different consequences of taking Relativism seriously. If you are willing to accept all of these, then Relativism might be for you.

\section{The Criticism Problem}

\begin{center}
\textbf{We could no longer say that the customs of other societies are morally inferior to our own}
\end{center}

This is the point that cultural relativists really want to stress and it is one that people who like this avenue believe. But here are some of the things that follow:
\begin{earg}
    \item[]We cannot criticize cultures with slavery (pre-Civil War South).
    \item[]We cannot criticize cultures with anti-Semitic views (the Nazis).
    \item[]We cannot criticize cultures with rigid caste systems (found the world over, throughout history). 
\end{earg}
If we took cultural relativism seriously, we would have to regard these social practices as also immune from criticism. Any country entering WW2 would have done wrong (so long as they did so to stop the practices). This issue was brought up with the Kaelons example before and does warrant retelling.\footnote{There are several other examples which can be found where forcing a change in a culture seems like the right thing to do. If you take an anthropology class, you may encounter examples involving female circumcision or, by another name, female genital mutilation. Forcing them to abandon this practice seems like the right thing to do. Forcing the change, though the right thing to do, will often require that it's done in the right way. It should be done gradually, recognizing the reasons why the culture does the practice and then helping to eliminate those reasons.}

\subsection{The Argument for The Criticism Problem}

\begin{earg}
    \item[1] If MCR is true, then morality is relative to culture. (This is basically the definition)
    \item[2] If morality is relative to the culture, then there is no standard to gauge the morality of a culture.
    \item[3] If there is no such standard, there is no way to compare one culture to another (morally speaking)
    \item[4] If there is no way to compare them, then there is no way to say that the Nazis were bad.
    \item[5] Therefore, if MCR is true, then there is no way to say that the Nazis were bad.
\end{earg}

So, Moral Relativism implies that one culture could never, accurately, pass judgement on the practices of another, for good or for ill.

\section{The Sheeple Problem}

\begin{center}
\textbf{We could decide whether actions are right or wrong just by consulting the standards of our society.}
\end{center}

Cultural Relativism suggests a simple test for determining what is right and what is wrong: All one need do is ask whether the action is in accordance with the code of one's society. To help picture this, imagine that whenever the standards for being a culture are met (that's another problem all together, "what does it take for a group to be a culture?"), a big book falls from the heavens with all of the norms of that culture written down. So, if you ever have a question about whether an action is right or wrong, you would just need to open the book, find the relevant rule and follow it like a sheep.  For example:  Suppose that a person in the deep south (Pre-Civil War) was curious about whether slavery was permissible. All they would need to do is ask whether it fit with the code of the society. If it did, then they would be OK with having slaves. People often think that believing in Moral Relativism is in some way liberating when in fact, it's imprisoning.  Not only can’t we criticize other cultural codes, but we can’t criticize our own. (All human rights movements would be incorrect according to this).

\subsection{The Argument for The Sheeple Problem}
\begin{earg}
    \item[1] If MCR is true, then morality is relative to culture. (This is basically the definition)
    \item[2] If morality is relative to the culture, then whether an action is right or wrong is determined by your cultural norms.
    \item[3] If those are determined by the norms, then all you need to do is consult the standards of your culture to determine morality.
    \item[4] So, if MCR is true, then all you need to do is consult the standards to determine morality. (Intermediary conclusion)
    \item[5] If morality is determined by the standards, then acting against those standards is morally wrong.
    \item[6] All Human/Civil Rights movements act against the standards of the culture.
    \item[7] Therefore, If MCR is true, then all Human/Civil Rights movements were/are morally wrong.
\end{earg}
So, Moral Relativism implies that you need to be a sheep to your culture, that you shouldn't object to a practice and blindly follow it, no matter how bad.\footnote{To clarify this, there is a general norm which says that it's wrong to criticize a person for doing something morally permissible. In seeking to change a norm of a culture, the person is criticizing them for doing something which, by their culture, is permissible. This makes seeking to change the norm wrong. In a more applied case, the protests against segregation in the US, according to Moral Relativism, were wrong because they violated the norm, segregation, which they were trying to change. There has never, and likely could never, be a case where trying to change a norm in a culture does not violate a norm in that culture.}

\section{The Progress Problem}

\begin{center}
\textbf{The idea of moral progress is called into doubt}
\end{center}

We often claim that certain societal changes are for the better and of course, we often claim that some of them are for the worse. Take, for example, the history of Women's Rights. This is such a universal problem that I really don't need to point to any individual one. Throughout most of Western history the place of women in society was narrowly circumscribed. They could not own property; they could not vote or hold political office; and generally they were under the almost absolute control of their husbands. Recently much of this has changed, and most people think of it as progress. Progress is where the changes move something closer to an objective standard (in this context). We say that we are making progress towards fixing a car, finishing an essay, or any other project with a fixed end-point. If Moral Relativism, of any sort, is correct, then there's no objective standard which we are moving towards, so there's no way of saying (correctly) that we have made progress. Similarly, different people at different times are, for all intents and purposes, different cultures (so long as standards have shifted). So, like with the first consequence, we can't say (correctly) that the standards had at one time are better/worse than the standards had at a different time. So, in the same way, if MCR is correct, then there would be no progress, at all.\footnote{For another way to think about this, we judge progress by comparing how the situation is and how it was according to some end goal. If Moral Relativism is correct, then there is no end goal for the comparison and there is no means of comparison even if we devise some other scale. For example, take two pieces of paper and cut one of them to be a different length than it was. Putting them side-by-side, you could say that one is shorter than the other. But could you say that one is shorter than the other if you were only ever able to see one of them at a time (and you didn't cut them)? Could you do so if you were incapable of remembering the other piece? You couldn't. Moral Relativism takes away all abilities to compare the moral standings of cultures, not only different cultures at the same time but the same culture at different times.}

\subsection{The Argument for The Progress Problem}
\begin{earg}
    \item[1] If MCR is true, then morality is relative to culture. (This is basically the definition)
    \item[2] If morality is relative to the culture, then there is no standard to the morality of a culture.
    \item[3] If there is no such standard, there is no way to compare one culture to another (morally speaking)
    \item[4] Different people at different times, if the norms changed, are, for all intents and purposes, different cultures.
    \item[5] So (from 1, 2, 3, and 4), if MCR is true, then there is no way to compare people from one time to another (morally).
    \item[6] If there is no way to compare people from different times, there can be no way for us to say that we are better than we were (made progress).
    \item[7] Therefore, if MCR is true, then there is no way to say that we have made progress.
\end{earg}

\chapter{Part 4: Other Problems for Moral Relativism}

\section{Why There is Less Disagreement Than There Seems}

The Cultural Differences Argument centers around the idea that the differences between cultures are insurmountable. It holds that the differences are fundamental differences in values not just differences in perspectives or views about the nature of the world. This, however, is a rather strong claim and can be easily debunked once we do a little digging. Many robust claims like this are the product of what I will call intellectual laziness. This is where the person holds the stance because they are too lazy to learn something more about the subject and then that laziness lets it fester into an unquestioned belief. So, let's use a simple example to illustrate this. Suppose that news broke that some far off culture holds that birds are never to be eaten. Eating or killing a bird is equivalent to killing or eating a human person. On the face of it, without doing much further digging, one could think that this amounts to a fundamental difference in how they place values on the world. But, it would be intellectually lazy to let it rest there. If we do a little digging, we find that the culture actually believes in reincarnation and that good people are reincarnated as birds so that they can hear prayers and fly back and forth between Earth and Heaven. This changes the perspective we should have about the culture. It's not that they have a fundamentally different value system, rather they disagree with us about the role of birds and the nature of the soul. For them, killing a bird is the same as killing a good person and we disagree about whether a bird was once a good person. 

\section{There are Norms Shared by All Cultures}

One mark against Moral Relativism, aside from the lack of real disagreement about values, is that there are certain norms which are shared by all cultures. These norms could come from various different sources, but typically, for a culture to survive through time, they are going to need to have these. There is sort of a natural selection process which weeds out cultures which lack or discredit these features. The first of these features is that any culture is going to need to place a high value on raising or caring for their young. Even with the Inuit cultures, which practice The Dip, if an infant survives, then it is very well cared for. This is because a culture which does not care for its children will likely not survive past a couple of generations or those who adhere to that custom will not survive past a few generations and the culture will change to one which cares for the young. Failing to care for the young or infanticide will need to be the exception rather than the rule. Next, all societies will need to place some value on honesty. Lying, cheating, and deceit, if allowed to propagate in a culture will quickly be the undoing of that culture and those who are honest (in the right times and conditions) will survive to continue a new culture which values the honesty. And, for a third example, all cultures will need to have some sort of ban on needlessly killing other members of that culture. This should seem obvious but any culture which allows for murder and killing to go unchecked will likely not survive through time and will eventually be replaced by one which has such a ban on murder. 

\section{But What is a Culture?}

So far, I have danced around the notion of being a culture. I use the term and expect a certain intuitive understanding. But, there are questions which we should ask. Moral Relativism bases its claims about morality on the notion of a culture, so we should ask what it takes to be a culture. One could go a few different routes about what it takes for some group of people to be a culture and we could even ask about whether there even are cultures. The first way we could define a culture is to have it be a collection, or set, of norms, or rules of behavior which are followed by a certain group of people. We have further worries about the number of people and so on, but this is a reasonable place to start. Defining a culture as a set of norms, however, runs into a bunch of issues with vagueness.

Suppose that there is a group of people, say 1000, who all believe in a certain collection of norms. We will say, just assume, that this is a culture, C. Now suppose that as time progresses some people disagree with one of the norms in C, for example, about whether you should pour the milk or the tea first. This movement grows and now 500 believe that the milk should go first and 500 believe the tea should go first. Do we say that C has split into two different cultures or do we say that the rule isn’t settled? If we say that C has split over one rule, then any disagreement would mean that the Cs would split again, and again, and again, eventually leading to 1000 different cultures. Every person would be a culture of one and, morally, one could never criticize or correct another because they are a different culture.  In real life, this is far more messy and even more disastrous for Moral Relativism.  If we say that the rule in the culture is undecided, we get a similar result. Any disagreement within a culture would result in that culture having the rule be undecided. So, eventually, there would be no norms of the culture at all which would destroy the culture from ever existing. As a result, if we say that disagreement in a culture is allowed, then there are no cultures at all and if we say that any disagreement splits the culture, then all people would be a culture of one. 

That example is an issue inside of the culture, but if we expand it and use it to compare cultures, as is common, then we run into similar worries. Suppose that C meets another culture D. D has many of the same norms as C except one, which is different. If we say that, despite appearances, they are actually the same culture, C’, we get an interesting snowball. C’ shares many of the same norms as another culture, E, so they are the same culture. This slowly builds until there’s a point where there aren’t different cultures, but rather there is only one culture, the human culture, with minor regional variations.

One could, attempting to escape this, define a culture merely as a certain group of people when they hit a certain number. The notion here is that once you have a group of a relevant size, then the norms will organically emerge. There are issues with this such as geographic distribution and stuff like that, but we can put those aside.  We could say that a culture is a group of so many people. But, is there a hard line for the number of people it takes to be a culture? What if a group has exactly enough people to be a culture and one or two die? Did the culture die too? Morality should not be so fragile.

\chapter{Part 5: If Moral Relativism is wrong, What's Next?}

If you are willing to say that moral cultural relativism is wrong, and, given the general responses to the various problems with moral relativism, you likely are willing to say that it's wrong, we have some problems. Namely, what is moral? Who decides who's correct in a moral conflict? Where do moral truths come from? Here I will cover some of the basic questions which former relativists tend to give and the general responses (sometimes, the response will need to be more particular).
\section{If culture's don't choose, what is morality?}

There are a few ways to go about replying to this question. Morality/ethics centers around 'should' questions of a certain kind. When we ask questions like these, for example "what should I do?", "should you call a doctor?", "should you get the assignment in on time?",  there are two different senses, which may or may not overlap, depending on context and relevance. Something is moral/ethical when it's the correct answer to the question "what should I do?", when we aren't talking about practical cases (a practical case is one where you are asking about the correct way to perform some task, like changing the oil on your car). The cultural moral relativist would claim that the correct answer to the question in non-practical cases (for example, should I flip the switch in the trolley problem?)\footnote{You will see me use the Trolley Problem as an example in a few different places. It essentially boils down to asking whether it is permissible to kill one person in order to save 5. There are many versions of it out there and some might make you think that it's wrong while others will make you think it's the right thing to do. That's why it's a problem, after all.} will depend on your culture and the norms associated with it. The moral objectivist, a moral realist who is also a non-relativist, would claim that there's an actual answer to this question, which is not determined by your culture. The next module of this class covers some of the ways which philosophers have answered how to answer that question, without being relativist. Ethics is, at it's core, trying to find the correct answer to this question.

You will notice that when I give examples of cultural norms, most of the time, these are cases where the culture got the answer wrong. Ethical theories are basically hypotheses about the correct way to get the answer to the question. Some theories point to absolute, objective duties which a person has (typically, it's wrong to be irrational; acting in a certain way is irrational, therefore those ways of acting are wrong). Others point to the well-being of those affected (if an action makes the affected better off than otherwise, that's the right action). Others still point to some exemplar of morality, some person who has the perfect character and then asks "what would that person do?". You may have heard that kind of thinking before with 'WWJD'.  
\section{Who decides who's right in moral conflict?}

A moral conflict is a case where two or more people/groups disagree about the answer to the "What should I do?" question, in the relevant sense. The moral cultural relativist has a simple way to answer it, almost too simple, "if it's two groups in the same culture, the cultural norms settles the conflict, if it's two cultures, there's no real conflict." But this just does not seem right, if you recall the last page, were this true, all civil rights movements would wrong, because the cultural norms would settle the conflict in favor of the oppressing group. Very few of us would want this. So, that's out, but is there a non-cultural way to get the answer? This is the quest for the moral realist. It's not going to be based on belief (as the relativist points to).

Recall the Cultural Differences Argument. This argument relies on moral conflicts/disagreements. But, do we always settle debates like that? There may be disagreement about something, anything (for that matter), but it doesn't follow that there's no fact of the matter. If there's a conflict about math, we don't say that there's no answer, we consult the rules of mathematics (as they were discovered, not invented). When there's a conflict in science, we don't say that there's no answer, we perform experiments and discover the truth. For the moral realist, settling moral conflicts is more like settling a conflict in math or science than one in art. Realists perform experiments and consult the rules of morality to settle the debates (and the experiments either further support the rules or give examples of amendments which should be made). The quest to get the non-relativist rule to settle moral conflicts is hard, but it does not seem impossible, we do make progress in it. There's even a scientific-style method for figuring out which ethical theories should be applied and/or how they should be amended (as we will see in the next module).

In short, no one decides who's right in a moral conflict, it's just a fact that one is right and the other is wrong.
\section{Where do moral truths come from?}

Moral truths are the answers to the "what should I do?" question (in the relevant sense). The moral cultural relativist has a simple, again almost too simple, answer "from the beliefs of the culture." Yet again, how can we possibly say that a culture has the wrong answer to the question when the culture chooses the answer? Like the issues before, this just can't be right. So, the question for the realist is just that, where do they come from? 

For the realist, this is like asking "where do scientific truths come from?" or "where do mathematical truths come from?". Some moral objectivists point to God and say that all truths flow through Him, moral truths included, saying that God made them (an easy way to get all-knowing and all-good).\footnote{We will be encountering this theory, called Divine Command Theory, later in the course. But, for a primer, that theory claims that an action is right if and only if God commands it and wrong if and only if God forbids it. There are many problems with this theory, which we will see.} Others point to the sort of creatures we are and our place in nature, these give us moral truths. Others still point to abstract notions of well-being or rationality.  And even more say that moral truths are things which just could not have been otherwise, they just always are, they didn't "come from" anything. There is some debate about where they come from.

But, don't lose heart! There's just as much debate about where mathematical and scientific truths come from. For example, in the case of math, we could claim that the truths are necessary, they could not have been otherwise. We could claim, with just as much evidence, that the truths are constructs of the definitions of the terms which we are using overlapping in consistent ways (1+2=3 because of the definitions of the terms and operations). And we could also claim that the truths came from God.  As it turns out, in philosophy, the most common stance (for where mathematical truths come from and where moral truths come from) is that they always were and always will be.
\section{What about forcing them onto another culture?}

This is a hard problem, because one of the core intuitions behind the Moral Relativist, and one of the big reasons why the stance is contradictory, is that it’s wrong to force your morals onto another. The Moral Objectivist will claim that under the right circumstances, it’s morally required that you force another to change. To some, this feels wrong. This feeling, often, gets traced back to the horrors of history, which I used as examples for the intuition behind the Cultural Imperialism Argument, where one group imposed themselves on another. At the same point, we need to look at history as a whole. There have been times when it was actually correct for one group to come in and force a change. In those cases, the imposing groups had real morality on their side and in the cases where it was wrong, we can say that they were mistaken about morality. For example, the Nazis in WW2 were just wrong about morality and the groups which came in were right. So, the Moral Objectivist just needs to be sure that their moral theory is getting the right answer in the case. This is an epistemological worry for the objectivist, not an ethical, meta-ethical, or metaphysical one. 

Interestingly, when we look at our own history, we find things which we are ashamed of, things which we condemn the previous generation for. How and when this is correct is easy for us to see. Since different people at different times are, basically, different cultures, the exact same, or a similar, thought process applies when condemning or appreciating the behaviors of another culture. Kwame Appiah in What Will Future Generations Condemn Us For? gives us some compelling tests for when future generations will be ashamed of our practices. Appiah gives us three tests. These tests don't tell us what we should do, rather they tell us that something is wrong in the practice we have and it needs to change. We can use the hindsight of the future generations, which is 20-20, to tell us what we need to do today.\autocite{Appiah1} 

The first sign that future generations will be ashamed of our behavior is that we already know the arguments against the practice. Most of the time, when there is a morally questionable practice, the arguments against it are actually quite old. Very, very, rarely is there a moment of moral clarity which changes the world. For example, take the consumption of meat or the meat production industry. The arguments against the various practices in farming meat are centuries old and this is a sign that something is off morally and future generations will feel shame for the practices we engage in. For another example, arguments against police brutality and the militarization of the police force are centuries old. In fact, many of those arguments are why most officers in England do not carry firearms. This is a sign that something needs to be done about police brutality. 

The second sign is that defenders of the practice use non-moral reasons to justify or defend the practice. The defenders point to religion or tradition, they point to 'human nature', or they point to some sort of economic or cultural necessity. Moral reasons should always trump non-moral reasons. So, the defenders are, in a sense, floundering. In the case of tradition, the defenders will say things like "it's always been this way, why should we change it?" Future generations will likely read about these arguments in textbooks and feel shame because of how nearsighted their ancestors were and proud of those who fought to change it. In the case of  'human nature', very rarely is that a good mark for morality and we should strive to better ourselves rather than be fatalistic about out lot in life. And with cultural or economic necessity, they make claims like "the economy couldn't survive such a change!" Future generations will likely express that any system which only survives on injustice has no right to survive. 

The third and final test is that of strategic ignorance. Most of the time, also, when people are accustomed to a system and like it, they are resistant to changing it. Not only will they resist changing it, but they will intentionally or subconsciously ignore some of the factors which point to the practice being bad. These factors will become harder and harder to ignore for future generations and eventually they will change the system and feel shame for our inaction. For example, economic inequality is strategically ignored by many or they emphasize the rags-to-riches stories which support the system. They ignore the fact that the odds of pulling one up by their boot straps is decreasing with every generation and they ignore the fact that the opportunities to pull oneself up are not evenly distributed. This is a sign that there is something in this system which future generations will be ashamed of.