\part{Feminist Ethics}
\label{ch.mod9}
\addtocontents{toc}{\protect\mbox{}\protect\hrulefill\par}
\chapter{Part 26: Feminist Ethics}

This week, we will be discussing one of the newer perspectives on ethics, Feminist Ethics. Before we start, however, ask yourself what every theory which we have covered in this class has in common. The answer might be a little surprising. Every theory which we have covered was kicked off by men. The vast majority of them lived in societies which were very patriarchal. This type of society lead them to make some rather disturbing claims about women. For example, with Virtue Ethics, Aristotle wrote ``the male is superior, the female, inferior." For Divine Command Theory, Aquinas wrote “as regards to her individual nature, each woman is defective and misbegotten.” And for Kantianism, Kant wrote ``Laborious learning or painful pondering, even if a woman should greatly succeed at it, destroy the merits proper to her sex… Her philosophy is not to reason but to sense."

These statements could be read as a product of their time and we could think that these views did not bleed into the theories. And in a sense, this is correct, none of the authors really wrote much about women (with the exception of J.S. Mill (Utilitarianism)).\footnote{John Stuart Mill was an avid supporter of women's rights. His wife, Harriet Taylor Mill, who I mention later in other footnotes, was a heavy hitter in early British feminism and was very influential on him (those who claim otherwise are why we need feminist ethics, as we will see later). His work ``The Subjection of Women" and Harriet Taylor Mill's work ``The Enfranchisement of Women" are very often sited in feminist literature today. After the passing of Harriet Taylor, J.S. Mill ran for parliament and won. He was the first person in the history of that body to argue for women's suffrage and would passionately engage with anyone on the topic. He also argued for extensive cultural reforms to enfranchise women, recognizing that it was not just legal hurtles standing in their way but also cultural. For example, if there were no laws blocking you from having a certain position but you were brought up with the belief that those roles just `weren't for you', it's very unlikely for you to try and get the position and then there's those selecting others for the position who think that the position just isn't `for you'. J.S. Mill recognized that not only the laws needed to change but also the culture.} But there’s a sense which it would be wrong to downplay these quotes. There are two ways philosophers have not given women their due: First, they make false or damaging claims about women (those quotes). Second, they do not value or totally ignore the experiences and perspectives of women. Feminist ethics seeks to change this.

Some people think of Feminist Ethics as a stance itself, and there has been a lot of advancement in trying to make an ethical theory `from scratch' in the feminist framework, however, Feminist Ethics, like Virtue Ethics and some of the more `meta' theories, is better seen as a family of different theories which all share a certain family resemblance. It is possible to make versions of the various theories inside of the feminist ethics framework, though they will be a little different (just like how it’s possible to make a consequentialist theory inside of the virtue ethics framework).\footnote{The closer the theory is to having the core features already, the less the ethicist will need to modify the theory to fit in with the Feminist Ethics family. In much of this module, you will see me reference Feminist Consequentialism and the optional reading for this module is about that. This is because Consequentialism needs the least modification to fit in with the family, as you can probably tell from the previous footnote and great strides have been made in making such a theory in the last 15 years.} One way to think of this is that feminist ethics is a meta-ethical stance about the correct way of doing ethics, like what the methodology should be, or one could see this as a perspective on ethics. For example, Julia Driver (Virtue Consequentialism) sought to make a Feminist Consequentialism. There are several (4) features which makes an ethical theory fit into this family, these features are, in a sense, part of the family resemblance which links them all together. Also, these features, or tenets, if applied differently, are used to handle the ethics of race and sexual orientation today. This is, in part, why feminist philosophy has sort of adopted those topics in recent years. 

\section{The Tenets of Feminist Ethics}

Generally, when deciding whether a theory is of one kind or another, not just in Ethics, but in any field, we look at the general features which the theory has. For example, with Consequentialism, the main feature is that the consequences matter morally or that there is a basic obligation to leave the world a better place.\footnote{As Samantha Brennan puts it in her 2020 article `` The Love-Hate Relationship Between Feminism and Consequentialism", quote, ``it's my view that to say we have a moral reason to do x because x would make the world better is to engage in consequentialist moral reasoning."\autocite{Brennan1}} With Feminist Ethics, there are 4 such tenets or features. It should be noted, however, that many contemporary philosophers working in this field have limited it to 2 tenets, namely the first two.\footnote{Alison Jaggar in her article ``Feminist Ethics" (1991) says that there are two features to a Feminist Ethical theory, first that it must state that the subordination of women is morally wrong and second that the moral experiences of women need to be treated with equal consideration to those had by men.\autocite{Jaggar1} Samantha Brennan in ``Recent Work in Feminist Ethics" (1999) says that a Feminist Theory must have two aims; first, to achieve a theoretical understanding of women's oppression with the purpose of ending that oppression and second, to develop an account of morality based on women's experiences.\autocite{Brennan2} In these two accounts of the tenets of feminist ethics, there's nothing, directly, about the traits traditionally associated with women nor is there anything, directly, about the methods of moral reasoning. However, one could derive the third tenet from the first and the fourth from the second, without adding or detracting very much. } Some ethical theories could, potentially have these features by design or because of a consequence accidentally.

\subsection{Tenet 1: The Equality of Men and Women}

The first feature of a feminist ethical theory is that it must, either by design or consequence, have built in that men and women are equal morally and intellectually. Any theory which justifies the subjugation of women or downplay their interests is wrong. This plays right into my initial note about how all of the authors, with the exception of Mill,\footnote{For a historical note, John Stuart Mill's wife, Harriet Taylor Mill (often cited as Harriot Taylor), was very influential on J.S. Mill. She was a `heavy hitter', so to speak, in early Feminism and a great philosopher in her own right. Many of his works, he later claimed, were actually hers, or at the very least her ideas. Several of her works were applications of the consequentialist theories which she and her husband developed. We could also, maybe, credit her with pushing J.S. Mill towards a more Rule Utilitarian approach, which has similarities to the general approaches in Feminist Ethics today. It makes sense, then, that J.S. Mill would be the exception to the general claim about the authors of classical ethical theories.} we have seen thus far have downplayed or ignored the interests of women. We can think of this as a methodological point. In constructing an ethical theory, one must take the interests of women into account and give them equal weight to the other voices in the debate.
\begin{center}
People of different sexes are morally equal; any theory which justifies the subjugation of a group or downplays their interests because of sex is wrong.
\end{center}
As I mentioned before, a very similar tenet is found when one is developing an ethical theory with other groups in mind. In the case of race, this tenet would read as something along the lines of ``people of different races are morally equal; any theory which justifies the subjugation of a group or downplays their interests because of race is wrong." Similarly, in the case of the LGBTQ+ community, this tenet would be phrased as ``all people are morally equal, regardless of their gender of sexual orientation; any theory which justifies the subjugation of a group or downplays their interests because of gender or orientation is wrong." This tenet can be found throughout the history of Feminism, both in Academia and elsewhere. Any theory or policy which does not recognize the equality of women and men in the relevant regard is just plain wrong. 

\subsection{Tenet 2: The Experiences of Women are Vital}

This next tenet plays right into the previous and adds a little bit to it. This tenet claims that the experiences of women deserve our respect and are vital to a complete moral picture. Any theory which does not take these into account (ignores them) will be incomplete and likely will be biased. The core idea here is that the perspectives of women when formulating a moral theory have often been a blind spot in the history of ethical frameworks. This seeks to make the perspectives of women on equal footing to those of men.
\begin{center}
The perspectives of women are essential to a complete moral picture. Those which do not that those into consideration will be incomplete and biased.
\end{center}
This feature applied to race/racism looks like this “the perspectives of members of different races are essential to a complete moral picture; any theory which does not take them into account will be incomplete and likely biased.” Similarly, for the LGBTQ+ community, we have something along the lines of “the perspectives of members of the LGBTQ+ community are essential to a complete moral picture; any theory which does not take them into account will be incomplete and likely biased.” As before, we see this sort of tenet found in various Feminist movements today. For example, in many of the contemporary arguments against Anti-Abortion Laws (the pro-choice arguments), we are quick to point out that those making the laws in question are not affected by the laws, in that they are all men. The lack of a woman's input makes the likelihood of a biased or incomplete law too high to ignore. 

\subsection{Tenet 3: Masculine vs Feminine Traits}

We notice in interacting with others, there are certain character traits which are stereotypically feminine and those which are stereotypically masculine. For example, some feminine traits include sympathy, empathy, caring, mercy, compassion, and cooperation. On the other side, some of the more masculine traits include things like competitiveness, independence, willingness to resort to violence, demanding one’s fair share, and the defense of personal honor. Many ethical theories over-emphasize the masculine traits in a person and many patriarchal cultures value those traits over the feminine ones. This tenet wishes to draw attention to this and claims that the feminine traits are, at least, as important as the masculine ones, if not more so.\footnote{The mention of traits and character dispositions like this should remind you of Virtue Ethics, and you would be correct in spotting this. Though I am unaware of advancements down this path, but placing `feminine' traits on equal footing as the masculine and amending the methods for determining which traits are virtues might lead to interesting advancements into a feminist virtue theory.}
\begin{center}
Feminine traits are at the very least as important as masculine ones.
\end{center}
For this one, we are talking about traditional traits associated with men and women. No one will claim that all women have all of these traits, nor will they claim that all men lack those traits. However, our cultures have long cultivated the idea that men have these traits and women have those traits. In the case of applying this tenet to race and gender/sexual orientation, it would be far more culturally relative to assign traits to the communities and, as a result, far more difficult. In general, the initial version should be able to handle those variations without need for modification, though I may be wrong.

\subsection{Tenet 4: Feminine Moral Reasoning}

In the previous theories which we have seen, there has been an emphasis on abstraction, impartiality, and strict adherence to some principle rules. Consequentialism gets around the latter because of its flexibility, which is why feminist consequentialism is possible. But, traditionally feminine moral reasoning has features emphasizing cooperation, flexibility, openness to competing ideas, and relationships between people.
\begin{center}
Those traditionally feminine features are superior to the traditionally masculine features in a moral theory.
\end{center}
Feminist moral theories will, in a sense, be more grounded in the cases, because of their flexibility. They will also be more open to seeking a common ground with other people in cases where there are competing interests, because of the emphasis on relationships and competing ideas. The emphasis on the relationships does lead to a rejection of impartiality. For the theories which we have seen, they are built up to be impartial, that is, they treat all people equally, regardless of the relationship to the person. This rejection is very interesting to me, at least. For example, take this thought experiment:
\factoidbox{\noindent  \fontsize{20pt}{0pt}\textbf{Drowning Children}

    You are at the park with your child and their two friends. They choose to go swimming. As you watch, they begin to drown. If you dive in, you can either save your child or their friends, you can't save all three. What do you do?}

An impartial ethical theory would treat the three people the same, regardless of their relation to you. This, however, does not sit well with some people.\footnote{With this case, also, there is something interesting. In my family, those who are related to me, man or woman, tend to be the `bread-winners'. When I pose this question to the people who actually raised the children (the moms or mr. moms), I tended to get that they went with `save my kid', while those not directly involved in the raising tended to have a harder time. This difference is an example of how the feminine traits are not associated with sex, rather with a role they play.} They claim that you have a greater moral duty to save your child than the other two kids because your child is, well, your child. That is an example of this built in bias at work. If your initial reaction is to call for help, if there is no one, dive in, the idea of cooperation is also exemplified.\footnote{Harriot Taylor Mill, who I spoke about in the third footnote, once said to her husband ``[religion and superstition] must be superseded
by morality deriving its power from sympathies and benevolence and its reward from the approbation of those we respect."  This was in a Letter to J.S. Mill dated February 15th 1854. The nature of the letter as well as the content surrounding this quote, leads one to believe that she was influencing him towards both emphasizing the feminine traits which we saw before as will as placing greater emphasis on personal relationships.\autocite{Hayek1}}

There are two additional features which tend to naturally arise when developing an ethical theory with these tenets at the heart. Some authors have even treated these as fundamental features, tenets, in their own right. 

\subsection{Against Unification}

Most ethical theories try to get one rule to determine morality. For example Utilitarianism had the Principle of Utility, Kantianism had the Categorical Imperative, and so on. Feminist Ethics doesn’t like that. There isn’t one formula for right actions, rather several. This is sometimes called ethical pluralism. We see this all the time in life, especially in the home-life, we have conflicting demands from various family members and each one needs to be treated differently. In which case, we need to apply different rules in different cases. This shares most of its features with virtue ethics.

\subsection{Against Impartiality and Abstraction}

There are many reasons why philosophers want one overarching rule for morality. One of them is that the more general the rule, the less likely it will have bias. Feminist ethics thinks that being at least a little bias and down to earth (not abstract) illuminates the nature of things like justice and our moral duties far better. For example, take pragmatic encroachment.  As far as I am aware, this particular concept has not been applied to Ethics, it has mostly been a problem for Epistemology (the study of knowledge). In the case of knowledge claims, this is where we would say that we know something on an abstract level, but that fails to transfer to the applied level. Pragmatic, real world, considerations encroach on it. Here is an example: Suppose that you know that 99\% of people in a town don’t have a high school diploma. Now, I take one of them aside and put them in front of you and ask “do you know that this person doesn’t have a high school diploma?” Given normal levels of certainty, on the abstract level, you would say that you know, or are highly certain, but something is changing the case.

When we take this idea and apply it to ethics, we get something interesting. On an abstract level, we might be willing to say that certain things are right or wrong, make general rules, but something about the cases, when we apply it, comes out odd and we aren’t willing to say that the theory gets it right. For an example, take one of the ethical theories which I have presented to you and that you liked, now think about whether you agreed with the counter examples.
 

\chapter{Part 27: History and The Ethics of Care}

Feminist ethics really came into its own in the 1980s. In 1982, a Harvard psychologist, Carol Gilligan published a book, In A Different Voice. In it, she argued that women experience and interact with the world differently than men (kind of a no-duh). But the difference was that she rejected the idea that this was inferior to men. Gilligan’s teacher, Kohlberg, defined six stages of moral development. Gilligan noticed that women hardly ever move beyond the third stage. She saw that claiming that women are in this stage and this stage is inferior, as a bias. Here is a chart of the different stages of moral development according to Kohlberg:
\noindent
%\begin{landscape}
\begin{tabular}{p{1.5in}|p{1.5in}|p{1in}|p{1in}|p{1in}|p{1in}|}
\tiny{Moral Development}&\tiny{Cognative Prerequisite}&\tiny{Who Decides?}&\tiny{How Do I Know?}&\tiny{Why Act Morally?}\\\hline
\tiny{Level 1: Preconventional Morality}& & & & \\
\tiny{Stage 1: Punishment}&\tiny{Egocentrism}&\tiny{Authority}&\tiny{Reward and punishment}&\tiny{Reward and punishment}\\
\tiny{Stage 2: Instrumental}&\tiny{Individualistic}&\tiny{I do based on my needs.}&\tiny{Self-interest}&\tiny{If I act good, then I will get something in return.}\\\hline
\tiny{Level 2: Conventional Morality}& & & & \\
\tiny{Stage 3: Interpersonal Expectations}&\tiny{Can think about other's wants and needs.}&\tiny{Social Roles}&\tiny{They are what a good person in my society would do.}&\tiny{Others expect me to act like a good person.}\\
\tiny{Stage 4: Social System}&&\tiny{Social laws and conventions.}&\tiny{The Law}&\tiny{If I don't follow the laws, society will break down.}\\
\tiny{Level 3: Postconventional Morality}&&&&\\
\tiny{Stage 4.5: Transition}&\tiny{Considers many different perspectives at once.}&\tiny{Each individual must decide what is moral.}&\tiny{Moral behaviors are what's right for each person.}&\tiny{You need to do what you think is right.}\\\hline
\tiny{Stage 5}&\tiny{Considers many different perspectives at once including an objective impartial outside perspective.}&\tiny{An abstract, impartial moral principle. Social laws are contracts which should be changed when necessary to reflect that principle.}&\tiny{Morality is what is right for all people, this is may or may not be legal.}&\tiny{One must promote justice and equality.}\\\hline
\tiny{Stage 6}&&\tiny{I define what is right or wrong based on my own self-evident principles.}&\tiny{Morality is abstract and universal, transcending social conventions.}&\tiny{All people deserve equality and justice. It's our duty to promote those.} \\\hline
\end{tabular}
%\end{landscape}

In testing the stages of moral development for various people, Kohlberg and Gilligan presented people with what is known as the Heinz Dilemma. It comes in various different forms, but the basic principle behind the case is the same:

\factoidbox{\noindent  \fontsize{20pt}{0pt}\textbf{The Heinz Dilemma}

In Europe, a woman was near death from cancer. One drug might save her, a form of radium that a druggist in the same town had recently discovered. The druggist was charging \$2,000, ten times what the drug cost him to make. The sick woman’s husband, Heinz, went to everyone he knew to borrow the money, but he could only get together about half of what it cost. He told the druggist that his wife was dying and asked him to sell it cheaper or let him pay later. But the druggist said, “No.” The husband got desperate and broke into the man’s store to steal the drug for his wife. Was this morally right or wrong? Why?
}

According to studies involving this case, women tended to say that the husband should not steal the drug. While men tended towards claiming that he should. For the study, they, prior to conducting, made a hierarchy of moral development. According to them, the opinions of women for this case were less developed than those of men. This caused women philosophers to analyze the case and come up with the reasons why women went this direction and why the study would, in error, think that women were less morally developed.

\section{Woman's Experience}

In her studies, Gilligan did not argue that all women share these feminine traits, nor did she argue that all men have the masculine. Also, it is very difficult to define the ‘female perspective’, as women are just as diverse as men are in their ideas and thinking. But, there are certain experiences which can only be had by females (childbirth for example) and there are many experiences which are mostly had by women, and men less-so, pay inequality, and near total exclusion from certain professions (airline pilot). These experiences, according to Gilligan and others, like Nel Noddings, lead women to tend to have a different perspective. This perspective, according to the authors, centers around concrete particulars, rather than abstract cases, focusing on relationships, rather than impartial judgements. These aspects of this perspective have evolved into the fourth tenet of feminist ethics.

Women aren’t the only ones who suffer those things, men do too, but it is a far smaller percentage. Philosophers before the 80s largely ignored these issues. One of the goals of Feminist Ethics is to talk about these things. Another point worth bringing up is that women, the world over, experience decreased autonomy and increased dependence on their spouse. All of these points lead to a different outlook on ethics which makes the theories in feminist ethics so different.
\section{The Ethics of Care}

Most of the theories which we have spoken about have not been made with the familial roles in mind. But since most of our treasured moments are spent with our family and many of our choices concern close relationships. What would a theory look like if that was the starting point? In particular, feminist ethicists have argued that the maternal love for a child should be the model for ethics. This is the ethics of care. It’s not egoist and it’s not Kantian. Justice is absent in the mother-child relationship. It’s not utilitarian because it has at its heart a bias. Mothers are biased towards their children.

The Ethics of Care has, at its heart, an important feature. It places great emphasis on caring and emotion. Care is an emotion, or at least a set of reinforcing emotions like love, sympathy, and empathy. Like all emotions, care involves both thinking and feeling. The relevant thoughts for care are towards the wants and needs of the being cared for. The feelings are positive, like love. Care makes us more attuned to the needs of others. Parents often know their kids better than anyone else. Both utilitarians and Kantians don’t place much stock on emotions. But feminists hold that care is central to moral motivation and discovery.

\subsection{But What is Care?}

We have seen this idea of care before in the Mohist Consequentialism section, but we still need to explore it in depth in order to fully appreciate the theory which is being proposed here. As I have mentioned, care is an emotion. It is a disposition or an attitude towards the well-being of another thing.\footnote{it is possible to care for a rock or a person or an animal.} We need to be careful, however, because there are, at least, two general ways in which this may manifest. One could \emph{care about} another thing or one could \emph{care for} another thing. Caring-about is where you have a desire for the betterment of the well-being for a thing. For example, I can care about the survivors of some disaster, care about the health of a friend, or care about the state of the economy. Caring-about does not require me to actually do anything.\footnote{Though, if you really do care about something then you are motivated to do something.} Caring-for is where you are actively taking steps to promote the well-being of the thing in question. It is certainly possible to care for something without caring about them, though such a dispassionate approach might be seen as monsterous. It is also possible to care about something without caring for it, such as caring about the well-being of disaster survivors without taking steps to help them. 

When one actually cares about something, there are certain general things which arise in that person, all of which can give some moral guidance. First, we have attentiveness. If you care about another person, then you are attentive to their needs. You understand what they need and want and are able to recognize when something is off. You could be ignorant of another person’s needs, but being inattentive to a dependent's needs seems far worse. There is a difference between doing something for them when needed and being attentive. When you are merely doing something for them, then it’s likely prompted by external expectation, like it’s your job. When you are attentive, the same action is prompted internally. You recognize the need. Second, when we really care about/for another, we are willing to take on responsibilities. One could be attentive to the needs of another, as in keeping track of their needs, but to really care about them, you need to take on the responsibilities involved in getting them those needs. If you really care about something, you willingly take on the obligations to see it through. We all have responsibilities. Some of these are thrust upon us, others we take onto ourselves. When you care about something, you place the responsibility to attend and act on their behave without external pressures. You willingly take it on because you care for them. Third, when we care about/for something, we need to recognize our own competency. You can take on the responsibility and be attentive, but you are also obligated to be competent in fulfilling those obligations. You need to fill them adequately. If you care about someone, then likely the standards for adequacy will seem bigger than they actually are, and this is a good thing. When you care for something, it should feel wrong to only do as much as necessary for them/it. It should also feel wrong to fail to fulfill the responsibilities. This means that you are obligated to fill the role with competence and, maybe, go above and beyond. And finally, we have responsiveness. You can take on the responsibility, be competent, and attentive, but there’s the problem of your knowledge. When you care about someone, you are responsive to their actual needs, the sort of condition they are in. This is different than ‘putting yourself in their shoes’, you are responding to their condition and wanting to help from outside, recognizing what they would want, not what you would want in their position.

\section{Some Overarching Issues with Feminist Ethics}

Since Feminist Ethics is a family of theories, if we are going to pose challenges or problems for it, we will need to focus on the core similar features which all share. These would be the tenets which they all have in common. These issues mostly arise from the fourth tenet in a feminist ethical theory. The heart of the theories generated will be against abstraction, impartiality, and unification.

\subsection{Issue 1: The scope of the moral community}

Feminist ethics threatens to limit the scope of who we have moral duties towards to just those we care about. Early feminist ethicists argued that we only have duties towards those we care about. This isn’t argued for any more but the idea is still central. For example, do I have some obligation to a stranger? What about a person I can’t stand? Similarly, if I am morally expected to be biased towards certain individuals, then is it permissible for me to hoard resources for them and leave very little for others?

A potential way out is to pull some ideas from other theories when it comes to caring. For example, the Feminist Ethicist could argue that we need to care about the well-being of all people but at the same time only need to care for the well-being of those we \emph{ought} to care for.\footnote{This is the line of reasoning which is found in Mohism.} This would mean that I have some obligations towards others, as I ought to care about all people's well-being, I just have a greater obligation towards some people.

\subsection{Issue 2: The Role of Emotions}

While emotions certainly give us insight into morality, the exact role of it needs to be fleshed out. In some cases, our emotions will cloud our judgement and make the right move hidden to us. So, more research into this role of emotions is needed, what emotions are the guiding ones? What exactly is this emotion ‘care’? And so on.

\subsection{Issue 3: Bias has its costs}

What is the justification for treating the perspectives of another group equally to your own? Well, it would be because it would be biased and incomplete not to include those perspectives. However, at the same point, Feminist Ethics is opposed to this kind of impartiality. For example, with the Ethics of Care, it would make sense for a mother to treat the interests of her child with a greater weight than the interests of another child. Being against impartiality, allowing for a bias like this, detracts from the very motivation behind the theory itself. 

More contemporary researchers in Feminist Ethics have recanted their rejection of impartiality, at least a little bit, and have stated that being impartial, at least some of the time, is the right thing to do. This naturally expands the scope of the moral community beyond those which we care about. There are certain social roles and situations which demand impartiality and moral consideration of others, even beyond those you care for. Feminist Ethics' emphasis on relationships and social roles makes adding a dash of impartiality a natural addition, though it will be open to bias when one is in certain roles. This addition is covered in the optional reading, Consequentialism and Feminist Ethics by Julia Driver, along with other minor amendments. This development also can expand the scope of the moral community, which removes the first issue.\footnote{The thing to note is that the addition of at least some impartiality, in the right contexts, is very recent and I don't know whether it has `picked up steam' and became the prevailing view. Treat this as a remaining issue for Feminist Ethics to tackle until such a time as this, or another solution is found.}

\subsection{Issue 4: Conflicts}

Like with virtue ethics, feminist ethics doesn’t have a hard and fast rule for morality. Without it, it is very difficult to actually use the stances. This is an issue with just about any form of ethical pluralism. The best route to take is to, at least in spirit, accept some manner of unification and then use the tenets as methodological principles to either create an ethical theory in the model of the normative theories which we have seen or use the tenets to modify existing theories into the feminist family (such as a feminist virtue theory or feminist consequentialism).\footnote{Some examples of this line of work, incorporating Feminist tenets into existing theories, can be found in the  optional reading. There has, recently, been work to show that the reasoning and principles in Feminist Ethics are justified and established using consequentialist principles. This can be seen in The Love-Hate Relationship between Feminism and Consequentialism by Samantha Brennan. The justification for allowing bias (in certain circumstances) into a consequentialist framework is a bit sophisticated, functioning in a similar way to Rule Consequentialism.}

\subsection{Issue 5: Uncooperative people}

Feminist Ethics places a very heavy emphasis on cooperation. While cooperation is awesome, we need strategies for dealing with uncooperative people and governments. Sometimes, good faith efforts and a helping hand are met with an iron fist. This can be seen in the Heinz Dilemma from earlier. Sometimes, even with an uneven weighing system for interests and consideration, the best option would be to fight and act on some of the more masculine traits which we have seen in order to solve the problems. 

Similarly, people often claim that cooperation is the opposite of competition. Competition can be a good thing, though I will admit that the benefits are often over exaggerated in more patriarchal cultures. It’s motivating in that the drive to win in a game or the drive to be better at something than another gives us reason to do it. Engaging in competitions among ourselves teach us particular life lessons which, in turn, make us better people. Since a competitive drive is a masculine trait and Feminist Ethics claims that feminine traits are at least as important as the masculine ones, we need a system in place for determining which of the traits is better to act on or have given a situation. This would be a wonderful place to incorporate some of the developments in Virtue Ethics. Being hyper-competitive all the time is not a virtue but neither is being submissive and overly cooperative. There is a middle ground here which would do wonders in solving this issue.\footnote{One could also use the work in Virtue Consequentialism to determine which of the traits is better to act on and thereby virtuous.}

\subsection{Issue 6: Justice and Rights}

There are two general ways in which people think about justice in American Political Philosophy. For more left-leaning philosophers, justice is equality. That justice is being treated fairly and un-equal treatment is only justified if it makes everyone better off. There is no room in Feminist Ethics, as the tenets lay out, for this kind of impartiality. Fair and equal treatment for all people seems to be a core feature of the moral picture and something which should not be easily swept aside. The picture is just as incomplete and one which ignores the views of women.


\chapter{Part 28: Feminist Consequentialism}

Through out this content about Feminist Ethics, I have been careful to note that this is a family of theories or a methodology for doing Ethics, Feminist Ethics is not a theory in and of itself. As a result, there could be Feminist versions of the other theories which we have seen in this class. There could be a Feminist Virtue Ethics or a Feminist Kantianism, for example. While I do see great promise in a Feminist Virtue Ethics, one which has caught my attention and seems worthy of doing into detail about is \gls{Feminist Consequentialism} (Utilitarianism).\autocite{Driver2} As an approach to Ethics, Consequentialism in general and Utilitarianism in particular is very open to taking many different viewpoints into consideration and is very flexible in generating ever more robust versions of itself from those considerations. It seems strange, then, that Feminist Consequentialism wasn't taken as the original theory at the beginning.

\newglossaryentry{Feminist Consequentialism}
{
  name=Feminist Consequentialism,
  description={A form of Consequentialism which has all of the tenets of Feminist Ethics. Such a theory is likely to be indirect, rather than direct},
  plural=Feminist Consequentialist
}


\section{Feminist Historical Opposition to Consequentialism}

Though, from my perspective, calling this `historical' feels wrong, but as I have mentioned, Feminist Ethics is relatively young. In the early 1990s, many philosophers thought that the bedrock principles of Utilitarianism were opposed to the same for Feminist Ethics. Utilitarianism seems to be a highly abstract theory with a very deeply seeded commitment to impartiality. Feminist Ethics, on the other hand, is supposed to be a very down-to-Earth theory with a deeply held commitment to bias (at least to those we care about or have some relationship with). This was noted by Virginia Held in this passage:\autocite{Held1}

\factoidbox{Utilitarians suppose that one highly abstract principle, the Principle of Utility, can be applied to every moral problem no matter what the context. A genuinely universal or gender-neutral moral theory would be one that would take account of the experience and concerns of women as fully as it would take account of the experience and concerns of men. When we focus on women's experience of moral problems, however, we find that they are especially concerned with actual relationships between embodied persons and with what these relationships seem to require.}

This passage illustrates that at least a Naïve Act Utilitarianism would be committed to some principles opposed to the features of Feminist Ethics.  Carol Gilligan and Nel Noddings both also noted that women’s experience of morality tended to be primarily about concrete, real world, cases rather than abstract and distant moral principles and also focused on close caring relationships rather than impartiality. These notes too place Naïve Act Utilitarianism in opposition to Feminist Ethics. For most ethical theorists, impartiality, a lack of bias, is a good thing for a theory to have. Feminist Ethicists, on the other hand, don't think that this is so. Many think that the correct ethical theory would have the person following it lead a good life. Our friendships/relationships are of great importance to a good life and those relationships require a certain degree of bias on the part of those in them. For example, we have special obligations towards our children and we should care about them more that other people and we owe our families more consideration than others. Naïve Act-Utilitarianism would not have room for this. So, this means that Utilitarianism, at least a Naïve Act Utilitarianism, could not be correct. Take this case for example:

\factoidbox{\noindent  \fontsize{20pt}{0pt}\textbf{Genius vs Your Child}

You are having a picknick lunch by a lake, your kid and their friend had just finished eating and decided to go swimming. They swim out pretty far and both begin to drown. Being attentive, you notice this immediately. They are so far out that you can only save one of them, either your child or their friend. You know full well that your child is a bit dull and would likely lead a pretty average life. Their friend, on the other hand, is a genius and stands to cause great things in the world. 

Who do you save?}

This example is one of a family of issues which arrise from a completely impartial theory of Ethics. We call them Special Obligation Problems.\footnote{or just The Special Obligation Problem} We want to say that we have a greater duty to our children than we do to other kids, we have \glspl{Special Obligation} towards family members or those we have a relationship with. Any reasonable ethical theory will say that we have some obligations to other people, but a truly impartial theory will apply those obligations evenly, regardless of their social or familial relation to us. For Naïve Act Utilitarianism, I have just as much an obligation to my child as I do to a child on the other side of the globe. In the above case, Na\"ive Utilitarianism would have it that I should save the friend rather than my child. But, there is a sense in which I have a greater obligation towards my child, some sort of special obligation to them which is not had towards others. If that is the case, then we have a problem. For Consequentialism, in general, to survive this objection, there must be a way to create this sort of bias from the impartial system. In other versions of this problem, we see that Naïve Act Utilitarianism doesn't have room for some kind of special obligation to ourselves (like taking on hobbies or going on vacation). We could only do this once we have ensured that everyone else was living the best life they could. It seems to call for great self-sacrifice. To many feminist writers, this will be a problem because such routine sacrifice has traditionally been asked of women and a Feminist ethical theory should be opposed to or mitigate this sacrifice. For an example of this thought, take this passage from Virginia Woolf, called the Angel of the House:\autocite{Woolf1}

\factoidbox{\noindent  \fontsize{20pt}{0pt}\textbf{The Angel of the House}

[She was] intensely sympathetic. She was immensely charming. She was utterly unselfish. . .. She sacrificed herself daily. If there was chicken, she took the leg: if there was a draught, she sat in it-in short she was so constituted that she never had a mind or a wish of her own, but preferred to sympathize always with the minds and wishes of others.}

\newglossaryentry{Special Obligation}
{
  name=Special Obligation,
  description={A relationship between two beings where one of them has a greater duty or obligation towards the other than they do towards a third party; a parent has a special obligation towards their child in that they have a greater duty to care for them than the parent may have towards other children who are not theirs},
  plural=special obligations
}


This passage illustrates, on a more local level, what the Naïve Act Utilitarian would hold as a good life or a good person. She is constantly sacrificing her own happiness and betterment for that of others. She is ignoring the special obligations she has to care for herself and have a mind of her own.  Something does not seem right here. 

At the same time,  the truly dedicated Naïve Act Utilitarian would say that being the Angel of the House is too small of a scale, she should be the Angel of the World. The Angel of the World is much like the Angel of the House but is far more global, making the problem even worse. Naïve Utilitarianism would have us always act for the betterment of the world as a whole, not just those we have special obligations towards and do so impartially. The Angel of the World would not carve out time for special obligations nor would she really carve out time for her individuality. She would subsume this for the betterment of the world.

From the Naïve Act Utilitarian perspective, it would be hard to justify any sort of friendship at all, let alone the special obligations necessary to maintain them. This is because it would be hard to see how friendship with some particular person would serve to promote the good. If the idea is to promote the good for others, and this governs all aspects of one's life, then she should be cultivating friendships that serve the greater good. She tries to get others to give to Oxfam, or volunteer at the soup kitchen, or protest against polluting business practices, and so forth. She thinks of the people she interacts with as potential beneficiaries of benevolence, but also as potential contributors to overall happiness in the world. Rather than thinking of friendships as good or thinking of the special obligations maintaining those friendships require, she thinks of friendship as a mere-means to the goal of making the world overall better. Even at her own detriment.

This illustrates to us that  Naïve Act Utilitarianism simply could not fit into the Feminist Ethics family. But maybe, just maybe, there are other, more sophisticated version of Consequentialism which could fit in with the family. 

\section{Naïve VS Sophisticated Utilitarianism}

While it is certainly the case that  Naïve Act Utilitarianism is committed to the view that we should all be the Angel of the World, this does not hold for all Consequentialist theories.  Naïve Act Utilitarianism falls into this conclusion because it conflates the decision making procedure with the criterion for right action. As we have seen previously, every Consequentialist theory is going to need to answer these two questions: 

\begin{enumerate}
\item What is the good?
\item How should I promote that?
\end{enumerate}

Utilitarianism answers the first question by saying that it's happiness or well-being. A more sophisticated form of Consequentialism (or more particularly, Utilitarianism) would answer the second question by \emph{not} claiming that we need to promote it in every action we take, but rather by other, indirect means, such as in how we make choices about what actions we take.\footnote{This should remind you of Rule Consequentialism and Mohist Consequentialism.} The decision making procedure for a theory is how we actually use it in the real world. While on an abstract level, it would seem like the decision procedure should map closely with the criterion for right action, it doesn't need to be and maybe, just maybe, on a more applied level, they could easily come apart. A more sophisticated Consequentialism would have our actions indirectly cause the best outcomes rather than directly. It could say that we should follow rules which, if followed generally, would lead to the best outcomes or it could say that we should do as the virtuous person would do and then define virtues using Virtue Consequentialism. In such a theory, we would praise and blame based on the decision procedure used and the actual moral rectitude would be based on the actual consequences of the action. To see this distinction in action, take this example:
\factoidbox{\noindent  \fontsize{20pt}{0pt}\textbf{Juan and Linda}

    Juan and Linda are a happily married couple but due to work and other factors, they rarely see each other (constantly on business trips) to the point that Juan has an apartment in another city. Because of a bonus Juan received, he could choose to forgo going on one of the business trips and spend the money to visit his wife when she would be home. Juan could also, without much distress, donate the money to charity.}

From the Naïve perspective, the right thing to do would be to donate the money to charity. This is because the act in isolation  would be for the betterment of the world as a whole, even if it displays some less than admirable character trait. But the more sophisticated Utilitarian/Consequentialist would argue that using the money to visit his wife is the correct course of action because it displays or ingrains a disposition to act in a caring way towards those you care about. Having this character trait is good because it encourages similar actions and makes the world better that way. Juan using the money to visit would be a case of blameless wrong-doing.\footnote{For the Na\"ive Utilitarian/Consequentialist, this is wrong and blameworthy, for the more sophisticated form of the theory, it's right because of the indirect consequences.} This would be a case where what they did was technically the wrong thing to do, but they used the correct decision procedure to arrive at it and we want to encourage those kinds of choices. A good consequentialist decision procedure would have these sorts of cases.

Continuing to use this example, there are two different kinds of sophisticated consequentialism which have the general tenets for Feminist Ethics and get the general intuitions correct. First, we have a form of Virtue Consequentialism.\footnote{Which we have seen previously in this textbook.} From a Virtue Consequentialist perspective, we have an obligation to fulfill these special obligations because they engrain or encourage certain character traits which systematically produce the greatest good, all things considered. Though the individual act did not produce the best outcome, in the long run, having this character trait does.\footnote{One could also say that Consequentialism says that one ought to act in such a away as to bring about a world they ought to want. It is worth noting that `ought to want' concerns the dispositions or traits of a person. Caring, like the Ethics of Care and Mohist Consequentialism, might, maybe be the guiding emotion which tells one what they ought to want.} It can also be shown that the generally considered feminine traits, like compassion, are virtues and therefore more desirable than the traditionally considered masculine ones, like aggression. The feminine moral reasoning aspect would also play a part in that the traits which come out as virtues would generate this sort of reasoning. 

The next would be Rule Consequentialism. From an Rule Consequentialist perspective, though the individual act of visiting Juan’s wife might not produce the greatest good, if everyone followed this sort of rule, spelled out in terms of caring for those we have special obligations towards, then the world would be better. While this theory does not deal with traits and dispositions, only with actions, there is potential room for a rule to be made about the short of traits which should be used and the general moral reasoning used will be more concrete and less abstract than the one found in the Na\"ive Act Utilitarianism. 

Both of these approaches have one thing in common. Neither of them tells the agent to consciously try for the best outcome on a universal scale, they are indirect in this way. They do not hold that we should think of people as mere-means. In a way, the impartial abstract rules and proceedures generate the obligations to be partial and concrete in more practical cases.\footnote{This can get the Feminist Ethicist's opposition to impartiality and abstraction into the picture nicely.} So, from this perspective, it is at least possible that the applied versions of Virtue Consequentialism or Rule Consequentialism will behave the same as a good Feminist Ethics theory in all of the positive ways while avoiding most of the issues which may arise. 

\section{Some Concluding Remarks}

Since this stance is even more of an infant than regular Feminist Ethics, I am not comfortable spelling this out in terms of a biconditional, like I normally do, but I am willing to give some general features: Feminist Consequentialism will be an indirect Utilitarian stance, meaning that it will see a distinction between blameworthiness and rectitude for actions. It will most likely say that an action is right if it stems from a virtue and the virtues are defined using Virtue Consequentialism. Feminist Consequentialism would promote friendships and social relationships as instrumentally good from an abstract level but encourage seeing them as intrinsically good on a practical level. This is because the belief would make the world better indirectly. Feminist Consequentialism will see degrees of obligation reflecting those found in Feminist Ethics generally. For example, I will have a greater obligation towards my family and those I care about that I would towards those who I have little to no relationship, but they are still worthy of my consideration. As more ideas and other voices enter into this discussion and perspective on Ethics, the tenets of Feminist Ethics may change, but Consequentialism is powerful and flexible enough to give justification for the additions and amendments. 


