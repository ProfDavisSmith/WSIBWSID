\chapter{The Moral and Legal Status of Abortion by Mary Warren}\autocite{Warren1}
We will be concerned with both the moral status of
abortion, which for our purposes we may define as the act
that a woman performs in voluntarily terminating, or
allowing another person to terminate, her pregnancy, and
the legal status that is appropriate for this act. I will argue
that, while it is not possible to produce a satisfactory
defense of a woman’s right to obtain an abortion without
showing that a fetus is not a human being, in the morally
relevant sense of that term, we ought not to conclude that
the difficulties involved in determining whether or not a
fetus is human make it impossible to produce any
satisfactory solution to the problem of the moral status of
abortion. For it is possible to show that, on the basis of
intuitions which we may expect even the opponents of
abortion to share, \emph{a fetus is not a person, hence not the
sort of entity to which it is proper to ascribe full moral
rights.}

Of course, while some philosophers would deny the
possibility of any such proof,\footnote{For example. Roger Wertheimer, who in "Understanding the
Abortion Argument" (Philosophy and Public Affairs I:1) argues
that the problem of the moral status of abortion is insoluble, in
that the dispute over the status of the fetus is not a question of
fact at all, but only a question of how one responds to the facts.} others will deny that there
is any need for it, since the moral permissibility of
abortion appears to them to be too obvious to require
proof. But the inadequacy of this attitude should he
evident from the fact that both the friends and foes of
abortion consider their position to he morally self-evident.
Because proabortionists have never adequately come to
grips with the conceptual issues surrounding abortion,
most if not all, of the arguments which they advance in
opposition to laws restricting access to abortion fail to
refute or even weaken the traditional antiabortion
argument, i.e., that a fetus is a human being, and therefore
abortion is murder.

These arguments are typically of one of two sorts. Either
they point to the terrible side effects of the restrictive
laws, e.g., the deaths due to illegal abortions, and the fact
that it is poor women who suffer the most as a result of
these laws, or else they state that to deny a woman access
to abortion is to deprive her of her right to control her
own body. Unfortunately, however, the fact that
restricting access to abortion has tragic side effects does
not, in itself, show that the restrictions are unjustified,
since murder is wrong regardless of the consequences of
prohibiting it; and the appeal to the right to control ones
body, which is generally construed as a property right, is
at best a rather feeble argument for the permissibility of
abortion. Mere ownership does not give me the right to
kill innocent people whom I find on my property, and
indeed I am apt to he held responsible if such people
injure themselves while on my property. It is equally
unclear that I have any moral right to expel an innocent
person from my property when I know that doing so will
result in his death.

John Noonan is correct in saying that “the fundamental
question in the long history of abortion is, How do you
determine the humanity of a being?”. \autocite{Noonan1} He summarizes his
own antiabortion argument, which is a version of the
official position of the Catholic Church, as follows:\autocite{Noonan2}
\factoidbox{... it is wrong to kill humans, however poor, weak,
defenseless, and lacking in opportunity to develop
their potential they may he. It is therefore morally
wrong to kill infants. Similarly, it is morally wrong to
kill embryos.} 

Noonan bases his claim that fetuses are human upon what
he calls the theologians’ criterion of humanity: that
whoever is conceived of human beings is human. But
although he argues at length for the appropriateness of
this criterion, he never questions the assumption that if a
fetus is human then abortion is wrong for exactly the
same reason that murder is wrong.

Judith Thomson is, in fact, the only writer I am aware of
who has seriously questioned this assumption; she has
argued that, even if we grant the antiabortionist his claim
that a fetus is a human being, with the same right to life as
any other human being, we can still demonstrate that, in at
least some and perhaps most cases, a woman is under no
moral obligation to complete an unwanted pregnancy.\autocite{Thomson1}
Her argument is worth examining, since if it holds up it
may enable us to establish the mural permissibility of
abortion without becoming involved in problems about
what entitles an entity to be considered human, and
accorded full mural rights. To be able to do this would he
a great gain in the power and simplicity of the proabortion
position, since, although I will argue that these problems
can be salved at least as decisively as can any other moral
problem, we should certainly be pleased to be able to
avoid having to solve them as part of the justification of
abortion. 

On the other hand, even if Thomson’s argument does not
hold up, her insight, i.e., that it requires arguments to show that if fetuses are human then abortion is properly
classified as murder, is an extremely valuable one. The
assumption she attacks is particularly invidious, for it
amounts to the decision that it is appropriate, in deciding
the moral status of abortion, to leave the rights of the
pregnant woman out of consideration entirely, except
possibly when her life is threatened. Obviously, this will
not do; determining what moral rights, if any, a fetus
possesses is only the first step in determining the moral
status of abortion. Step two, which is at least equally
essential, is finding a just solution to the conflict between
whatever rights the fetus may have, and the rights of the
woman who is unwillingly pregnant. While the historical
error has been to pay far too little attention to the second
step, Thomson’s suggestion is that if we look at the
second step first, we may find that a woman has a right to
obtain an abortion regardless of what rights the fetus has.

Our own inquiry will also have two stages. In Section I,
we will consider whether or not it is possible to establish
that abortion is morally permissible even on the
assumption that a fetus is an entity with a full-fledged
right to life. I will argue that in fact this cannot he
established, at least not with the conclusiveness which is
essential to our hopes of convincing those who are
skeptical about the morality of abortion, and that we
therefore cannot avoid dealing with the question of
whether or not a fetus really does have the same right to
life as a (more fully developed) human being.

In Section II, I will propose an answer to this question,
namely, that a fetus cannot he considered a member of the
moral community, the set of beings with full and equal
moral rights, for the simple reason that it is not a person,
and that it is personhood, and not genetic humanity, i.e.,
humanity as defined by Noonan, which is the basis for
membership in this community. I will argue that a fetus,
whatever its stage of development, satisfies none of the
basic criteria of personhood, and is not even enough like a
person to he accorded even some of the same rights on the
basis of this resemblance. Nor, as we will see, is a fetus’s
potential personhood a threat to the morality of abortion,
since, whatever the rights of potential people may be, they
are invariably overridden in any conflict with the moral
rights of actual people.

\section{Part I}

We turn now to Professor Thomson’s case for the claim
that even if a fetus has full moral rights, abortion is still
morally permissible, at least sometimes, and for some
reasons other than to save the woman’s life. Her argument
is based upon a clever, but I think faulty, thinking. She
asked us to picture ourselves waking up one day, in bed
with a famous violinist. Imagine that you have been
kidnapped, and your bloodstream hooked up to that of the
violinist, who happens to have an ailment that will
certainly kill him unless he is permitted to share your
kidneys for a period of nine months. No one else can save
him, since you alone have the right type of blood. He will
he unconscious all that time, and you will have to stay in
bed with him, but after the nine months are over he may
be unplugged, completely cured, that is provided that you
have cooperated.

Now then, she continues, what are your obligations in this
situation? The antiabortionist, if he is consistent, will have
to say that you are obligated to stay in bed with the
violinist: for all people have a right to life, and violinists
are people, and therefore it would be murder for you to
disconnect yourself from him and let him die.\autocite{Thomson1} But this is
outrageous, and so there must he something wrong with
the same argument when it is applied to abortion. It would
certainly be commendable of you to agree to save the
violinist, but it is absurd to suggest that your refusal to do
so would be murder. His right to life does not obligate
you to do whatever is required to keep him alive; nor does
it justify anyone else forcing you to do so. A law that
required you to stay in bed with the violinist would
clearly be an unjust law, since it is no proper function of
the law to force unwilling people to make huge sacrifice
for the sake of other people toward whom they have no
such prior obligation. Thomson concludes that, if this
analogy is an apt one, then we can grant the
antiabortionist his claim that a fetus is a human being, and
still hold that it is at least sometimes the case that a
pregnant woman has the right to refuse to be a Good
Samaritan towards the fetus, i.e., to obtain an abortion.
For there is a great gap between the claim that x has a
right to life, and the claim that y is obligated to do
whatever is necessary to keep x alive, let alone that he
ought to be forced to do so. It is y’s duty to keep x alive
only if he somehow contracted a special obligation to do
so; a woman who is unwillingly pregnant, e.g., who was
raped, has done nothing which obligates her to make the
enormous sacrifice which is necessary to preserve the
conceptus.

This argument is initially quite plausible, and in the
extreme case of pregnancy due to rape, it is probably
conclusive. Difficulties arise, however, when we try to
specify more exactly the range of cases in which abortion
is clearly justifiable even on the assumption that the fetus
is human. Professor Thomson considers it a virtue of her
argument that it does not enable us to conclude that
abortion is always permissible. It would, she says, be
“indecent” for a woman in seventh month to obtain an
abortion just to avoid having to postpone a trip to Europe.
On the other hand, her argument enables us to see that “a
sick and desperately frightened schoolgirl pregnant due to
rape may of course choose abortion, and that any law
which rules this out is an insane law” (p. 65). So far, so
good, but what are we to say about the woman who
becomes pregnant not through rape but as a result of her
own carelessness, or because of contraceptive failure, or
who gets pregnant intentionally and then changes her
mind about wanting a child? With respect to such cases,
the violinist analogy is of much less use to the defender of
the woman’s right to obtain an abortion.

Indeed, the choice of a pregnancy due to rape, as an
example of a case in which abortion is permissible even if
a fetus is considered a human being, is extremely
significant; for it is only in the case of pregnancy due to
rape that the woman’s situation is adequately analogous to
the violinist case for our intuitions about the latter to
transfer convincingly. The crucial difference between a
pregnancy due to rape and the normal case of an
unwanted pregnancy is that in the normal case we cannot
claim that the woman is in no way responsible for her
predicament; she could have remained chaste, or taken
her pills more faithfully or abstained on dangerous days,
and so on. If on the other hand, you are kidnapped by
strangers, and hooked up to a strange violinist, then you
are free of any shred of responsibility for the situation, on
the basis of which it would he argued that you are
obligated to keep the violinist alive. Only when her
pregnancy is due to rape is a woman clearly just as
nonresponsible.\footnote{We may safely ignore the fact that she might have avoided
getting raped, e.g., by carrying a gun, since by similar means
you might likewise have avoided getting kidnapped, and in
neither case does the victim's failure to take all possible
precautions against a highly unlikely event (as opposed to
reasonable precautions against a rather likely event) mean that
he is morally responsible for what happens.}

Consequently, there is room for the antiabortionist to
argue that in the normal case of unwanted pregnancy a
woman has, by her own actions, assumed responsibility of
the fetus. For if x behaves in a way which he could have
avoided, and which he knows involves, let us say, a 1
percent chance of bringing into existence a human being,
with a right to life, and does so knowing that if this should
happen then that human being will perish unless x does
certain things to keep him alive, then it is by no means
clear that when it does happen x is free of any obligation
to what he knew in advance would he required to keep
that human being alive.

The plausibility of such an argument is enough to show
that the Thomson analogy can provide a clear and
persuasive defense of a woman’s right to obtain an
abortion only with respect to those cases in which the
woman is in no way responsible for her pregnancy, e.g.,
where it is due to rape. In all other cases, we would
almost certainly conclude that it was necessary to look
carefully at the particular circumstances in order to
determine the extent of the woman’s responsibility and
hence the extent of her obligation. This is an extremely
unsatisfactory outcome, from the viewpoint of the
opponents of restrictive abortion laws, most of whom are
convinced that a woman has a right to obtain an abortion
regardless of how and why she got pregnant.

Of course, a supporter of the violinist analogy might point
out that it is absurd to suggest that forgetting her pill one
day might be sufficient to obligate a woman to complete
an unwanted pregnancy. And indeed, it is absurd to
suggest this. As we will see, the moral right to obtain an
abortion is not in the least dependent upon the extent to
which a woman is responsible for her pregnancy. But
unfortunately, once we allow the assumption that a fetus
has full moral rights, we cannot avoid taking this absurd
suggestion seriously. Perhaps we can make this point
more clear by altering the violinist story just enough to
make it more analogous to a normal unwanted pregnancy
and less to a pregnancy due to rape, and then seeing
whether it is still obvious that you are not obligated to
stay in bed with the fellow.

Suppose, then, that violinists are peculiarly prone to the
sort of illness the only cure for which is the use of
someone else’s bloodstream for nine months, and that
because of this there has been formed a society of music
lovers who agree that whenever a violinist is stricken they
will draw lots and the loser will, by some means, be made
the one and only person capable of saving him. Now then,
would you be obligated to cooperate in curing the
violinist if you had voluntarily joined this society,
knowing the possible consequences, and then your name
had been drawn and you had been kidnapped?
Admittedly, you did not promise ahead of time that you
would, but you did deliberately place yourself in a
position in which it might happen that a human life would
be lost if you did not. Surely, this is at least a prima facie
reason for supposing that you have an obligation to stay in
bed with the violinist. Suppose that you had gotten your
name drawn deliberately; surely that would be quite a
strong reason for thinking that you had such an obligation.

It might be suggested that there is one important
disanalogv between the modified violinist case and the
case of an unwanted pregnancy, which makes the
woman’s responsibility significantly less, namely, the fact
that the fetus comes into existence as the result of the
woman’s actions. This fact might give her a right to
refuse to keep it alive, whereas she would not have had
this right had it existed previously, independently, and
then as a result of her actions become dependent upon her
for its survival.

My own intuition, however, is that x has no more right to
bring into existence, either deliberately or as a foreseeable
result of actions he could have avoided, a being with full
moral rights y, and then refuse to do what he knew
beforehand would be required to keep that being alive,
than he has to enter into an agreement with an existing
person, whereby he may be called upon to save that
person’s life, and then refuse to do so when so called 
upon. Thus x’s responsibility for y’s existence does not
seem to lessen his obligation to keep y alive, if he is also
responsible for y’s being in a situation in which only he
can save him.

Whether or not this intuition is entirely correct, it brings
us back once again to the conclusion that once we allow
the assumption that a fetus has full moral rights it
becomes an extremely complex and difficult question
whether and when abortion is justifiable. Thus the
Thomson analogy cannot help us produce a clear and
persuasive proof of the moral permissibility of abortion.
Nor will the opponents of the restrictive laws thank us for
anything less; for their conviction (for the must part) is
that abortion is obviously not a morally serious and
extremely unfortunate, even though sometimes justified
act, comparable to killing in self-defense or to letting the
violinist die, but rather is closer to being a morally neutral
act, like cutting one’s hair.

The basis of this conviction, I believe, is the realization
that a fetus is not a person, and thus does not have a
full-fledged right to life. Perhaps the reason why this
claim has been so inadequately defended is that it seems
self-evident to those who accept it. And so it is, insofar as
it follows from what I take to be perfectly obvious claims
about the nature of personhood, and about the proper
grounds for ascribing moral rights, claims which ought,
indeed, to be obvious to both the friends and foes of
abortion. Nevertheless, it is worth examining these
claims, and showing how they demonstrate the moral
innocuousness of abortion, since this apparently has not
been adequately done before.

\section{Part II}

The question which we must answer in order to produce a
satisfactory solution to the problem of the moral status of
abortion is this: How are we to define the moral
community, the set of beings with full and equal moral
rights, such that we can decide whether a human fetus is a
member of this community or not? What sort of entity,
exactly, has the inalienable rights to life, liberty, and the
pursuit of happiness? Jefferson attributed these rights to
all men ... If so, then we arrive, first, at Noonan’s
problem of defining what makes a being human, and,
second, at the equally vital question which Noonan does
not consider, namely, What reason is there for identifying
the moral community with the set of all human beings, in
whatever way we have chosen to define that term?

\subsection{1. On the Definition of “Human”}

One reason why this vital second question is so frequently
overlooked in the debate over the moral status of abortion
is that the term “human” has two distinct, but not often
distinguished, senses. This fact results in a slide of
meaning, which serves to conceal the fallaciousness of the
traditional argument that since (1) it is wrong to kill
innocent human beings, and (2) fetuses are innocent
human beings, then (3) it is wrong to kill fetuses. For if
“human” is used in the same sense in both (1) and (2)
then, whichever of the two uses is meant, one of these
premises is question-begging. And if it is used in two
different senses then of course the conclusion doesn’t
follow.

Thus, (1) is a self-evident moral truth, \footnote{Of course, the principle that it is (always) wrong to kill
innocent human beings is in need of many other modifications,
e.g., that it may he permissible to do so to save a greater number
of other innocent human beings, but we may safely ignore these
complications here.} and avoids
begging the question about abortion, only if “human
being” is used to mean something like “a full-fledged
member of the moral community.” (It may or may not
also be meant to refer exclusively to members of the
species Homo sapiens.) We may call this the moral sense
of “human.” It is not to be confused with what we will
call the genetic sense; i.e., the sense in which a member of
the species is a human being, and no member of any other
species could be. If (1) is acceptable only if the moral
sense is intended, (2) is non-question-begging only if
what is intended is the genetic sense.

In “Deciding Who Is Human,” Noonan argues for the
classification of fetuses with human beings by pointing to
the presence of the full genetic code, and the potential
capacity for rational thought (p. 135). It is clear that what
he needs to show, for his version of the traditional
argument to be valid, is that fetuses are human in the
moral sense, the sense in which it is analytically true that
all human beings have full moral rights. But, in the
absence of any argument showing that whatever is
genetically human is also morally human, and he gives
none, nothing more than genetic humanity can be
demonstrated by the presence of the human genetic code.
And, as we will see, the potential capacity for rational
thought can at most show that an entity has the potential
for becoming human in the moral sense.

\subsection{2. Defining the Moral Community}

Can it be established that genetic humanity is sufficient
for moral humanity’? I think that there are very good
reasons for not defining the moral community in way. I
would like to suggest an alternative way of defining the
moral community, which I will argue for only to the
extent of explaining why it is, or should be, self-evident.
The suggestion is simply that the moral community
consists of all and only people, rather than all and only
human beings;\footnote{From here on, we will use "human" to mean genetically
human, since the moral sense seems closely connected to, and perhaps derived from, the assumption that genetic humanity is
sufficient for membership in the moral community.} and probably the best way of
demonstrating its self-evidence is by considering the
concept of personhood, to see what sorts of entity are and
are not persons, and what the decision that a being is or is
not a person implies about its moral rights.

What characteristics entitle an entity to be considered a
person? This is obviously not the place to attempt a
complete analysis of the concept of personhood, but we
do not need such a fully adequate analysis just to
determine whether and why a fetus is or isn’t a person.
All we need is a rough and approximate list of the most
basic criteria of personhood, and some idea of which, or
how many, of these an entity must satisfy in order to
properly be considered a person.
28 In searching for such criteria, it is useful to look beyond
the set of people with whom we are acquainted, and ask
how we would decide whether a totally alien being was a
person or not. (For we have no right to assume that
genetic humanity is necessary for personhood.) Imagine a
space traveler who lands on an unknown planet and
encounters a race of beings utterly unlike any he has ever
seen or heard of. If he wants to be sure of behaving
morally toward these beings, he has to somehow decide
whether they are people, and hence have full moral rights,
or whether they are the sort of thing which he need not
feel guilty about treating as, for example, a source of
food.

How should he go about making this decision? If he has
some anthropological background, he might look for such
things as religion, art, and the manufacturing of tools,
weapons, or shelters, since these factors have been used to
distinguish our human from our prehuman ancestors, in
what seems to be closer to the moral than the genetic
sense of “human.” And no doubt he would be right to
consider the presence of such factors as good evidence
that the alien beings were people, and morally human. It
would, however, be overly anthropocentric of him to take
the absence of these things as adequate evidence that they
were not, since we can imagine people who have
progressed beyond, or evolved without ever developing
these cultural characteristics.

I suggest that the traits which are most central to the
concept of personhood, or humanity’ in the moral sense,
are, very roughly; the following:

\begin{enumerate}
\item[1] consciousness (of objects and events external
and/or internal to the being), and in particular the
capacity to feel pain;
\item[2] reasoning (the developed capacity to solve new and
relatively complex problems);
\item[3] self-motivated activity (activity which is relatively
independent of either genetic or direct external
control);
\item[4] the capacity to communicate, by whatever means,
messages of an indefinite variety of types, that is, not
just with an indefinite number of possible contents,
but on indefinitely many possible topics;
\item[5] the presence of self-concepts, and self-awareness,
either individual or racial, or both.
\end{enumerate}

Admittedly, there are apt to he a great many problems
involved in formulating precise definitions of these
criteria, let alone in developing universally valid
behavioral criteria for deciding when they apply. But I
will assume that both we and our explorer know
approximately what (1)-(5) mean, and that he is also able
to determine whether or not they apply. How, then, should
he use his findings to decide whether or not the alien
beings are people? We needn’t suppose that an entity
must have oil of these attributes to he properly considered
a person; (1) and (2) alone may well he sufficient for
personhood, and quite probably (1)-(3), if “activity” is
construed so as to include the activity of reasoning.

All we need to claim, to demonstrate that a fetus is not a
person, is that any being which satisfies none of (1)-(5) is
certainly not a person. I consider this claim to be so
obvious that I think anyone who denied it, and claimed
that a being which satisfied none of (1)-(5) was a person
all the same, would thereby demonstrate that he had no
notion at all of what a person is—perhaps because he had
confused the concept of a person with that of genetic
humanity. If the opponents of abortion were to deny the
appropriateness of these five criteria, I do not know what
further arguments would convince them. We would
probably have to admit that our conceptual schemes were
indeed irreconcilably different, and that our dispute could
not be settled objectively.

I do not expect this to happen, however, since I think that
the concept of a person is one which is very nearly
universal (to people), and that it is common to both
proabortionists and antiabortionists, even though neither
group has fully realized the relevance of this concept to
the resolution of their dispute. Furthermore, I think that
on reflection even the antiabortionists ought to agree not
only that (1) - (5) are central to the concept of
personhood, but also that it is a part of this concept that
all and only people have full moral rights. The concept of
a person is in part a moral concept; once we have
admitted that x is a person we have recognized, even if we
have not agreed to respect, x’s right to he treated as a
member of the moral community. It is true that the claim
that x is a human being is more commonly voiced as part
of an appeal to treat x decently than is the claim that x is a
person, but this is either because “human being” is here
used in the sense which implies personhood, or because
the genetic and moral sense of “human” have been
confused.

Now if (1)-(5) are indeed the primary criteria of
personhood, then it is clear that genetic humanity is
neither necessary nor sufficient for establishing that an
entity is a person. Some human beings arc not people, and
there may well be people who are not human beings. A
man or woman whose consciousness has been
permanently obliterated but who remains alive is a human
being which is no longer a person; defective human
beings, with no appreciable mental capacity, are not and
presumably never will be people; and a fetus is a human
being which is not yet a person, and which therefore
cannot coherently be said to have full moral rights.
Citizens of the next century should be prepared to
recognize highly advanced, self-aware robots or
computers, should such he developed, and intelligent
inhabitants of other worlds, should such he found, as
people in the fullest sense, and to respect their moral
rights. But to ascribe full moral rights to an entity which
is not a person is as absurd as to ascribe moral obligations
and responsibilities to such an entity.

\subsection{3. Fetal Development and the Right to Life}

Two problems arise in the application of these
suggestions for the definition of the moral community to
the determination of the precise moral status of a human
fetus. Given that the paradigm example of a person is a
normal adult being, then (I) How like this paradigm, in
particular how far advanced since conception, does a
human being need to be before it begins to have a right to
life by virtue, not of being fully a person as of vet, but of
being like a person? and (2) To what extent, if any does
the fact that a fetus has the potential for becoming a
person endow it with some of the same rights? Each of
these questions requires some comment.

In answering the first question, we need not attempt a
detailed consideration of the moral rights of organisms
which are not developed enough, aware enough,
intelligent enough, etc., to be considered people, but
which resemble people in some respects. It does seem
reasonable to suggest that the more like a person, in the
relevant respects, a being is, the stronger is the case for
regarding it as having a right to life, and indeed the
stronger its right to life is. Thus we ought to take seriously
the suggestion that, insofar as “the human individual
develops biologically in a continuous fashion ... the rights
of a human person might develop in the same way”. \autocite{Hayes1} But
we must keep in mind that the attributes which are
relevant in determining whether or not an entity is enough
like a person to be regarded as having some of the same
moral rights are no different from those which are
relevant to determining whether or not it is fully a
person—i.e., are no different from (1)-(5)—and that being
genetically human, or having recognizably human facial
and other physical features, or detectable brain activity, or
the capacity to survive outside the uterus, are simply not
among these relevant attributes.

Thus it is clear that even though a seven- or eight-month
fetus has features which make it apt to arouse in us almost
the same powerful protective instinct as is commonly
aroused by a small infant, nevertheless it is not
significantly more personlike than is a very small embryo.
It is somewhat more personlike; it can apparently feel and
respond to pain, and it may even have a rudimentary form
of consciousness, insofar as its brain is quite active.
Nevertheless, it seems safe to say that it is not fully
conscious, in the way that an infant of a few months is,
and that it cannot reason, or communicate messages of
indefinitely many sorts, does not engage in self-motivated
activity; and has no self-awareness. Thus, in the relevant
respects, a fetus, even a fully developed one, is
considerably less personlike than is the average mature
mammal, indeed the average fish. And I think that a
rational person must conclude that if the right to life of a
fetus is to be based upon its resemblance to a person, then
it cannot be said to have any more right to life then, let us
say, a newborn guppy (which also seems to be capable of
feeling pain), and that a right of that magnitude could
never override a woman’s right to obtain an abortion, at
any stage of her pregnancy.

There may, of course, he other arguments in favor of
placing legal limits upon the stage of pregnancy in which
an abortion may he performed. Given the relative safety
of the new techniques of artificially inducing labor during
the third trimester, the danger to the woman’s life or
health is no longer such an argument.
39 Neither is the fact that people tend to respond to the
thought of abortion in the later stages of pregnancy with
emotional repulsion, since mere emotional responses
cannot take the place of moral reasoning in determining
what ought to he permitted. Nor, finally, is the frequently
heard argument that legalizing abortion, especially late in
the pregnancy, may erode the level of respect for human
life, leading, perhaps, to an increase in unjustified
euthanasia and other crimes. For this threat, if it is a
threat, can be better met by educating people to the kinds
of moral distinctions which we are making here than by
limiting access to abortion (which limitation may, in its
disregard for the rights of women, be just as damaging to
the level of respect for human rights).

Thus, since the fact that even a fully developed fetus is not
personlike enough to have any significant right to life on
the basis of its personlikeness shows that no legal
restrictions upon the stage of pregnancy in which an
abortion may be performed can be justified on the
grounds that we should protect the rights of the older
fetus. And once there is no other apparent justification for
such restrictions, we may conclude that they are entirely
unjustified. Whether or not it would be indecent
(whatever that means) for a woman in her seventh month
to obtain an abortion just to avoid having a to postpone a
trip to Europe, it would not, in itself, be immoral, and
therefore it ought to be permitted.

\subsection{4. Potential Personhood and the Right to Life}

We have seen that a fetus does not resemble a person in
any way that can support the claim that it has even some
of the same rights. But what about its potential, the fact
that if nurtured and allowed to develop naturally it will
very probably become a person? Doesn’t that alone give it
at least some right to life? It is hard to deny that the fact
that an entity is a potential person is a strong prima facie
reason for not destroying it, but we need not conclude
from this that a potential person has a right to life, by
virtue of that potential. It may be that our feeling that it is
better, other things being equal, not to destroy a potential
person is better explained by the fact that potential people
are still (felt to be) an invaluable resource, not to be
lightly squandered. Surely, if every speck of dust were a
potential person, we would be much less apt to conclude
that every potential person has a right to become actual.

Still, we do not need to insist that a potential person has
no right to life whatever. There may well be something
immoral, and not just imprudent, about wantonly
destroying potential people, when doing so isn’t necessary
to protect anyone’s rights. But even if a potential person
does have some prima facie right to life, such a right
could not possibly outweigh the right of a woman to
obtain an abortion, since the rights of any actual person
invariably outweigh those of any potential person,
whenever the two conflict. Since this may not be
immediately obvious in the case of a human fetus, let us
look at another ease.

Suppose that our space explorer falls into the hands of an
alien culture, whose scientists decide to create a few
hundred thousand or more human beings, by breaking his
body into its component cells, and using these to create
fully developed human beings, with, of course, his genetic
code. We may imagine that each of these newly created
men will have all of the original man’s abilities, skills,
knowledge, and so on, and also have an individual
self-concept, in short that each of them will be a bona fide
(though hardly unique) person. Imagine that the whole
project will take only seconds, and that its chances of
success are extremely high, and that our explorer knows
all of this, and also knows that these people will be treated
fairly. I maintain that in such a situation he would have
every right to escape if he could, and thus to deprive all of
these potential people of their potential lives; for his right
to life outweighs all of theirs together, in spite of the fact
that they are all genetically human, all innocent, and all
have a very high probability of becoming people very
soon, if only he refrains from acting.

Indeed, I think he would have a right to escape even if it
were not his life which the alien scientists planned to take,
but only a year of his freedom, or, indeed, only a day. Nor
would he be obligated to stay if he had gotten captured
(thus bringing all these people-potentials into existence)
because of his own carelessness, or even if he had done so
deliberately knowing the consequences. Regardless of
how he got captured, he is not morally obligated to
remain in captivity for any period of time for the sake of
permitting any number of potential people to come into
actuality, so great is the margin by which one actual
person’s right to liberty outweighs whatever right to life
even a hundred thousand potential people have. And it
seems reasonable to conclude that the rights of a woman
will outweigh by a similar margin whatever right to life a
fetus may have by virtue of its potential personhood.

Thus, neither a fetus’s resemblance to a person, nor its
potential for becoming a person, provides any basis
whatsoever for the claim that it has any significant right to
life. Consequently, a woman’s right to protect her health,
happiness, freedom, and even her life, \footnote{That is, insofar as the death rate, for the woman, is higher for
childbirth than for early abortion.} by terminating an
unwanted pregnancy will always override whatever right
to life it may be appropriate to ascribe to a fetus, even a
fully developed one. And thus, in the absence of any
overwhelming social need for every possible child, the
laws which restrict the right to obtain an abortion, or limit
the period of pregnancy during which an abortion maybe
performed, are a wholly unjustified violation of a
woman’s most basic moral and constitutional rights.\footnote{My thanks to the following people, who were kind enough to
read and criticize an earlier version of this paper: Herbert Gold,
Gene Glass, Anne Lauterbach, Judith Thomson, Mary
Mothersill, and Timothy Binkley.}

\section{Postscript on Infanticide, February 26, 1982}

One of the most troubling objections to the argument
presented in this article is that it may appear to justify not
only abortion but infanticide as well. A newborn infant is
not a great deal more personlike than a ninemonth fetus,
and thus it might seem that if late-term abortion is
sometimes justified, then infanticide must also be
sometimes justified. Yet most people consider that
infanticide is a form of murder, and thus never justified.

While it is important to appreciate the emotional force of
this objection, its logical force is far less than it may seem
at first glance. There are many reasons why infanticide is
much more difficult to justify than abortion, even though
if my argument is correct neither constitutes the killing of
a person. In this country, and in this period of history, the
deliberate killing of viable newborns is virtually never
justified. This is in part because neonates are so very
close to being persons that to kill them requires a very
strong moral justification as does the killing of dolphins,
whales, chimpanzees, and other highly personlike
creatures. It is certainly wrong to kill such beings just for
the sake of convenience, or financial profit, or “sport.”

Another reason why infanticide is usually wrong, in our
society, is that if the newborn’s parents do not want it, or
are unable to care for it, there are (in most cases) people
who are able and eager to adopt it and to provide a good
home for it. Many people wait years for the opportunity to
adopt a child, and some are unable to do so even though
there is every reason to believe that they would be good
parents. The needless destruction of a viable infant
inevitably deprives some person or persons of a source of
great pleasure and satisfaction, perhaps severely
impoverishing their lives. Furthermore, even if an infant
is considered to be adoptable (e.g., because of some
extremely severe mental or physical handicap) it is still
wrong in most cases to kill it. For most of us value the
lives of infants, and would prefer to pay taxes to support
orphanages and state institutions for the handicapped
rather than to allow unwanted infants to be killed. So long
as most people feel this way, and so long as our society
can afford to provide care for infants which are unwanted
or which have special needs that preclude home care, it is
wrong to destroy any infant which has a chance of living
a reasonably satisfactory life.

If these arguments show that infanticide is wrong, at least
in this society, then why don’t they also show that late-
term abortion is wrong? After all, third trimester fetuses
are also highly personlike, and many people value them
and would much prefer that they be preserved; even at
some cost to themselves. As a potential source of pleasure
to some family, a viable fetus is just as valuable as a
viable infant. But there is an obvious and crucial
difference between the two cases: once the infant is born,
its continued life cannot (except, perhaps, in very
exceptional cases) pose any serious threat to the woman’s
life or health, since she is free to put it up for adoption, or,
where this is impossible, to place it in a state-supported
institution. While she might prefer that it die, rather than
being raised by others, it is not clear that such a
preference would constitute a right on her part. True, she
may suffer greatly from the knowledge that her child will
be thrown into the lottery of the adoption system, and that
she will be unable to ensure its well-being, or even to
know whether it is healthy, happy, doing well in school,
etc.: for the law generally does not permit natural parents
to remain in contact with their children, once they are
adopted by another family. But there are surely better
ways of dealing with these problems than by permitting
infanticide in such cases. (It might help, for instance, if
the natural parents of adopted children could at least
receive some information about their progress, without
necessarily being informed of the identity of the adopting
family.)

In contrast, a pregnant woman’s right to protect her own
life and health clearly outweighs other people’s desire that
the fetus be preserved-just as, when a person’s life or limb
is threatened by some wild animal, and when the threat
cannot be removed without killing the animal, the
person’s right to self-protection outweighs the desires of
those who would prefer that the animal not be harmed.
Thus, while the moment of birth may not mark any sharp
discontinuity in the degree to which an infant possesses a
right to life, it does mark the end of the mother’s absolute
right to determine its fate. Indeed, if and when a late-term
abortion could be safely performed without killing the
fetus, she would have no absolute right to insist on its
death (e.g., if others wish to adopt it or pay for its care),
for the same reason that she does not have a right to insist
that a viable infant be killed.

It remains true that according to my argument neither
abortion nor the killing of neonates is properly considered
a form of murder. Perhaps it is understandable that the
law should classify infanticide as murder or homicide,
since there is no other existing legal category which
adequately or conveniently expresses the force of our
society’s disapproval of this action. But the moral
distinction remains, and it has several important
consequences.

In the first place, it implies that when an infant is born
into a society which-unlike ours-is so impoverished that it
simply cannot care for it adequately without endangering
the survival of existing persons, killing it or allowing it to
die is not necessarily wrong-provided that there is no
other society which is willing and able to provide such
care. Most human societies, from those at the hunting and
gathering stage of economic development to the highly
civilized Greeks and Romans, have permitted the practice
of infanticide under such unfortunate circumstances, and I
would argue that it shows a serious lack of understanding
to condemn them as morally backward for this reason
alone.

In the second place, the argument implies that when an
infant is born with such severe physical anomalies that its
life would predictably be a very short and/or very
miserable one, even with the most heroic of medical
treatment, and where its parents do not choose to bear the
often crushing emotional, financial and other burdens
attendant upon the artificial prolongation of such a tragic
life, it is not morally wrong to cease or withhold
treatment, thus allowing the infant a painless death. It is
wrong (and sometimes a form of murder) to practice
involuntary euthanasia on persons, since they have the
right to decide for themselves whether or not they wish to
continue to live. But terminally ill neonates cannot make
this decision for themselves, and thus it is incumbent
upon responsible persons to make the decision for them,
as best they can. The mistaken belief that infanticide is
always tantamount to murder is responsible for a great
deal of unnecessary suffering, not just on the part of
infants which are made to endure needlessly prolonged
and painful deaths, but also on the part of parents, nurses,
and other involved persons, who must watch infants
suffering needlessly, helpless to end that suffering in the
most humane way.

I am well aware that these conclusions, however modest
and reasonable they may seem to some people, strike
other people as morally monstrous, and that some people
might even prefer to abandon their previous support for
women’s right to abortion rather than accept a theory
which leads to such conclusions about infanticide. But all
that these facts show is that abortion is not an isolated
moral issue; to fully understand the moral status of
abortion we may have to reconsider other moral issues as
well, issues not just about infanticide and euthanasia, but
also about the moral rights of women and of nonhuman
animals. It is a philosopher’s task to criticize mistaken
beliefs which stand in the way of moral understanding,
even when-perhaps especially when-those beliefs are
popular and widespread. The belief that moral strictures
against killing should apply equally to all genetically
human entities, and only to genetically human entities, is
such an error. The overcoming of this error will
undoubtedly require long and often painful struggle; but it
must be done.