\part{Moral Anti-Realism}
\label{ch.mod11}
\addtocontents{toc}{\protect\mbox{}\protect\hrulefill\par}

Moral Relativism and Moral Objectivism (Realism) are not the only meta-ethical stances, there are two others worth noting (though this does go beyond the necessary scope of this class). Like Moral Relativism, these stances deny that there are absolute moral truths. But, unlike Relativism, these theories claim that they aren't relative. They claim that there's no truth to them at all, regardless of context. Theories which do this fall into a category called `Moral Nihilism'. Like Relativism, Nihilism comes as a family and you can pick and choose accordingly. Skepticism is most like Nihilism in how the family tree is organized. 

Moral Nihilism, like Moral Relativism, is limited, it only makes claims about moral claims. Like Relativism, if you find one absolute, objective truth, then Global Nihilism is false, but for Limited Nihilism, you would need to find it in the context it's limited to. The two major theories which call Moral Nihilism home are `Error Theory' and `Expressivism'. Of the two, Error Theory is the strongest, but it can be quite counter-intuitive. 

\chapter{Error Theory}

Error Theory, applied differently, is a concept and stance which can be found in other areas of philosophy, though I have personally mostly encountered it in Metaphysics. It is a lot like Skepticism, but rather than saying that we will never know about the thing, it says that we will always be mistaken.  For Meta-Ethics, this is the stance that all of our moral judgments will be mistaken. We will always be in error, hence the name. This is true even if we were to randomly chose an answer. For the \glspl{error theory}, we will ask questions about a subject, sincerely try to answer them, but will always get the wrong answer. There is no right answer.  Applied in other fields, it claims that all of our judgments in those fields will always be in error. They come to this conclusion from three assumptions, or starting points, which get them off the ground:

First, they need to have the standard-issue feature of a moral nihilistic stance, this is that morality is not a feature of the world. In other words, there are no moral facts. Some could say that morality is a useful fiction or say that morality is self-contradictory or some other means. The second feature is a bit more precise to this stance. This is the claim that no moral judgments will ever be correct/accurate.  This follows, using some fairly simple reasoning, from the first feature. For our judgments to be accurate/correct, they must correctly/accurately depict something. However, because of the core premise to Moral Nihilism, there's nothing for the judgment to correctly depict, so it can never be accurate/correct. The third aspect follows suit, they claim that it's pretty obvious that people try to use moral judgments, we act on them, and we think about them, but those judgments will always be in error, hence the name. Since our judgments can never be correct, they will always be wrong. The error theorist is making a pretty big claim. They aren’t just trying to bash on social policies and individual actions. They are going all out and claiming that morality is a fiction.

\newglossaryentry{error theory}
{
  name=error theory,
  description={A general claim that all other claims within a field of study or context will always be mistaken, try as we might to say something truthful, we cannot because there are no truths in that area or context},
plural=Error Theorist
}


They are essentially the atheists of ethics. Atheists hold that there are no religious features in the world, the error theorist holds the same about ethical features. The atheist holds that, try as you might, all religious doctrines are mistaken because (for example) God does not exist. The error theorist holds that, try as you might, ethical judgments/beliefs are all mistaken because morality does not exist.

\section{For Error Theory to be Correct}

All error theorists hold that the basic mistake in moral thinking is that it depends on some absolute or objective truth about morality. This truth applies to all of us, regardless of who or what we are. These are what some refer to this as ‘objective morality’ and ‘categorical reasons’.

To get error theory off of the ground, they need to convince us of 2 things. First, morality really does depend on these absolute standards (they would need to come up with the correct moral theory, examples are in Module 8). Second, supposing that it does, they have to show us that at least one of these is false, which would mean that the correct theory to describe morality is fundamentally and irreparably flawed. So, for the Error Theorist to prove themselves, they need to solve all problems in Ethics and then show that the stance which did it couldn't be right. This is quasi-contradictory.

\chapter{Expressivism}

Expressivism could, in principle, be a stance in other areas of philosophy and maybe, for some of them, it could make a solid stance, but I have not encountered it so much in my time and, when I have, it has almost exclusively been in Ethics. As I mentioned before, given the core flaws in trying to prove Error Theory (in the case of Ethics, in other fields this issue is not present), 
\Gls{expressivism} is the stronger of the two. As it is a Nihilistic stance, it does come with two assumptions, but it differs from Error Theory in the last aspect.

\newglossaryentry{expressivism}
{
  name=expressivism,
  description={A general claim that all other claims within a field of study or context are not intended to and do not express anything true or false, rather those claims are expressing emotions, commands, or questions. In ethics, saying something is wrong, according to this stance, is the same as saying something like `boo' and saying that something is right is the same as saying `yay'},
plural=expressivist
}


First, as with the Error Theorist, the Expressivist claims that morality is not a feature of the world. But, it doesn't say that it's a useful fiction or anything like that. It also claims, but the same reasoning, that moral judgments are never accurate. But, it does not say that they are always wrong. I know that this last bit seems odd, so let me explain: The Moral Objectivist (Realist), Relativist, and Error Theorist all share one thing in common: When we make moral claims, we are, at least, trying to describe the world, pick out some feature in it. The Expressivist is different: They claim that when we make moral claims our intent is to express something, not describe. They think that our ethical statements/judgments aren't the sort of things which can be true/false, correct/incorrect, or accurate/inaccurate. Take these two examples: 

\noindent
\begin{tabular}{p{2.75in}|p{2.75in}}
Example A&Example B\\\hline
    Murder is wrong&Blue is a color
\end{tabular}

Although it may look like these examples are stating the same sort of thing, and, in fact, were we to  do a sentence/semantic diagram for these examples, they would look freakishly similar, according to the Expressivist, these are fundamentally different sorts of sentences. In linguistics, profanity is often claimed to not add any meaning to a sentence, but rather is an emotional or rhetorical expression; the Expressivist says, basically, that moral claims, like (A), are the same sort of thing, they are only emotionally/rhetorically meaningful. In the case of (B), it is expressing something true about the world, there's a collection of things, colors, and it's putting `blue' in that collection (by one theory) or there's a thing in the world, `blue', and assigning a feature to `is a color' (by another theory, this is what I really work in). But, in the case of (A), we aren’t assigning a feature to murder, according to the Expressivist. Rather it is like we are saying something like:

\begin{earg}
    \item[]BOO murder!
    \item[]Don’t murder!
    \item[]Let’s not murder!
    \item[]Wouldn’t it be nice if we didn’t murder?
\end{earg}
Those sorts of statements aren’t the kinds of things which are true or false, they are more like actions or commands or questions (for interesting content on statements as actions, check out J.L. Austin's How To Do Things With Words). If you have ever been to a highly emotionally charged (peaceful) protest, people screaming statements which boil down to ``this is wrong" will feel more like ``boo this thing" rather than something based on an intellectual process, which is evidence in support of this sort of stance. 

\section{Three Concerns for Expressivism:}

There are three major worries for Expressivism. Although, I will say, Expressivism is the most popular, it is by far the weaker of the two stances and it will need some time in the gym, so to speak, to really hold its own. The first worry for the Expressivist concerns how we really reason about morality. Take the following arguments:

    \begin{earg}
    \item[]All cases of hurting people are immoral.
    \item[]Torture hurts people.
    \item[]Therefore, torture is immoral.
   \end{earg}

\begin{earg}
    \item[]All men are mortal.
    \item[]Socrates is a man.
    \item[]Therefore, Socrates is mortal.
\end{earg}

In fact, we use the exact same style of reasoning in both cases, agree or disagree with it, the logic is exactly the same. But, according to the Expressivist, the first argument looks like this:
\begin{earg}
    \item[]Hurting people - YUCKY
    \item[]Torture hurts people.
    \item[]Therefore, BOO torture. 
\end{earg}
We reason about morality logically, not emotionally (sometimes we do both, but when we are thinking clearly, emotions play a supporting role, they aren't the main character). According to Expressivism, however, we wouldn't do this, it would only be emotional. This observation, if correct, means that Expressivism can't be accurate. 

The second concern for Expressivism comes from Amoralism. An amoralist is a person who sincerely makes moral claims but is completely unmoved into action by them. These people might be out there, this would be like Data from Star Trek, in that they would not feel the emotions associated with ethical claims. If they do exist, this is a serious problem for the Expressivist. Expressivism claims that moral claims are emotional expressions. But, emotional expressions, most of the time, move us to action (even if we don’t actually act, we still kind of want to). How can a person sincerely make a moral claim, which is supposed to be an emotional expression, without being moved to action?  Basically, if a person can sincerely make a moral claim without being moved by it, then Expressivism is incorrect. 

The final concern for the Expressivist concerns how we actually make these moral judgments, much like the first. The absolutist (objectivist/realist), the relativist, and the error theorist disagree on just about everything, but they do agree on one thing, that moral claims are, at least, trying to describe the world. The Expressivist denies this. They have to paraphrase the moral claims to not have them attempt to describe the world. But, if this were really the case, why wouldn’t we just express our emotions and not conceal them? There are many statements, which are ethical in nature, which the Expressivist can't handle, many of which I have heard people make in the real world, for example:
\factoidbox{
    I’m not sure whether torture is moral, but I think people smarter than me would know.
    
    There’s a difference between being moral, being required, and being praise worthy.
    
    How much we punish should match how bad the crime was.
    
    If war is immoral, then generals are not as good as they seem.}

If we read these like the other stances would, then we have no problems, all are clearly understandable, but if we read them as the Expressivist thinks they should be, we are up a creek. We could all be lying or mistaken, but when we think about what we are trying to do when we make these statements, we see that we are really trying to describe the world. 