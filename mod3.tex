\part{Divine Command Theory And Natural Law Theory}
\label{ch.mod3}
\addtocontents{toc}{\protect\mbox{}\protect\hrulefill\par}
\chapter{Part 9: Morality and Religion}
Religious views have always been popular in determining morality/ethics. In times of moral conflict, religious leaders like Rabbis, Imams, and Priests have always been there to guide the flock. Since hundreds of millions of people tend to view morality through a religious lens, it’s important to cover the connection carefully. We are not going to try and prove that God or gods exist (note the grammar there) (for that, in depth, take Philosophy of Western Religion). Nor are we going to look at the teaching of any one religious text (though my examples will often be from the King James Bible, Old Testament).  Rather, we are going to look at the assumptions (there are 2) which are needed for the view that morality depends on religion and the end resulting theory about morality from those assumptions.
\section{Assumption 1: Religious belief is necessary to get people to do their duty}

Get the chuckle about the word `duty' out now, you will be hearing it a lot. A very popular argument against atheism (the stance that God doesn’t exist) states that being an atheist prevents a person from seeing why they should be moral. Assuming that this is true, why would that be? By far the most popular reason given is that fear of God and the promise of heaven is a good to keep us in line. God rewards good deeds and punishes the bad. Eternal damnation for the bad and eternal bliss for the good is a great reason to keep on the right side of the tracks. However, this might not be all that great of reasoning. First, it doesn’t tell us why religious people would be more likely to do what’s right. All it tells us is that religious people are more likely to be conscientious (wanting to do what’s right) but this doesn’t translate to actually doing it. For example, the Spanish Inquishion, which no one expects, was trying to do the right thing, but failed gloriously at it. The WBC (Westborough Baptist Church) is trying to do what’s right, but are also failing horribly at it. It all depends on whether the religious teachings are in fact moral to start with.
\section{Assumption 2: Morality must have been created by someone/thing and God (or gods) is (are) the best candidate(s) for the job.}

Another way of saying this is that God created morality. This assumption will lead us to our first normative theory of Ethics (coming later on this page). The underlining idea to this is that morality is a set of norms (common phrase ‘norms of behavior’, which are the standards we should live up to). Without God, according to these believers, it would be up to humans to make up these norms and we just aren’t up to the task. So ‘without God, all is permitted’. Here is the argument:
\begin{earg}
    \item[1] Laws require a lawmaker.
    \item[2] So, moral laws require a maker.
    \item[3] Humans could not have made these laws (since we are imperfect).
    \item[4] If humans didn’t make them, some perfect being made them.
    \item[5] God is that perfect being.
    \item[6] Therefore, God is the author of moral laws (therefore, God made moral laws).
\end{earg}

\section{Divine Command Theory}

This leads us directly to what is called the \gls{Divine Command Theory} of Ethics (DCT). It is:
\begin{center}
Something is moral if and only if God commands it.
\end{center}
Here are some equivalent ways of phrasing it:
\begin{earg}
    \item[1] If God commands it, then it is moral, and if it is moral, God commands it.
    \item[2] If something is immoral, God forbids it, and if God forbids it, then it is immoral.
    \item[3] Something is immoral IFF God forbids it.
    \item[4] Something is moral because God commands it and immoral because God forbids it.
\end{earg}

\newglossaryentry{Divine Command Theory}
{
  name=Divine Command Theory,
  description={The stance that an action is moral if and only if God or the gods command it}
}

This stance is not without some backing. For example, it completely removes all of the problems with Relativism. Second, it has built into it that the standards which make behaviors right/wrong are written by a perfect author (God) so they can’t be wrong (so long as God exists). And third, it ties the reward/punishment which people think is tied to morality (through God) directly to the reason why those things are moral. There is even a very simple kind of reasoning which can be used to determine whether an action is right or wrong:

\begin{earg}
    \item[1] If God forbids X, then X is immoral (a no go).
    \item[2] God forbids X.
    \item[3] Therefore, X is immoral.
\end{earg}

Here are a couple of examples:

\begin{tabular}{p{1.5in}|p{1.5in}}
Homosexuality&Abortion\\\hline
If God forbids homosexuality, then it’s immoral.& If God forbids abortion, then it’s immoral.\\
God forbids homosexuality.&    God forbids abortion.\\
Therefore, homosexuality is immoral.&Therefore, abortion is immoral.
\end{tabular}

\begin{tabular}{p{1.5in}|p{1.5in}}
The Binding of Issac&A Bible Verse Tattoo\\\hline
God commanded Abraham to kill his only son.&God forbids tattoos.\\
If God commands it, it is moral.&If God forbids it, then it’s immoral.\\
Therefore, Abraham killing his son is moral.&Therefore, getting tattoos is immoral.
\end{tabular}

\section{Problems for Divine Command Theory}

This theory is not without its problems. Just about every early ethical theory will have some issues, problems which need to be buffed out or some reason to reject it whole hog. The major problem for this theory is the Euthyphro Dilemma.  But there are some other ones, such as:

\factoidbox{\noindent  \fontsize{20pt}{0pt}\textbf{The Atheism Problem}

Since morality is tied to God, if God doesn’t exist or if God doesn’t actually command or forbid, morality would be a sham.}

\factoidbox{\noindent  \fontsize{20pt}{0pt}\textbf{The Which-Book Problem}

What religious teaching is the right one? Some say that eating pork is damning, while others say beef, others say all meat, some say getting a tattoo is damning, others seem cool with it.}


\subsection{The Euthyphro Dilemma}

Remember what the DCT states:

\begin{center}
\noindent Something is moral if and only if God commands it.
\end{center}

So, what if we ask why God commands us to do those things? Are they good/moral because He commands them? Or does He command them because they are good/moral? This little puzzle dates back to Socrates and it is a real head scratcher. Other versions include: Is it tasty because I like it, or do I like it because it’s tasty? Do the gods love certain acts because they are pious or are they pious because the gods love them?

There are two horns or prongs to this argument, and it really is a ton of fun to get people in this situation. There are two options, you gotta choose one, but both lead to a major problem for your stance. The first option in this argument is ‘they are moral/good because God commands them’. The second option is ‘God commands them because they are moral/good’. If you go down one path, then you get that God is imperfect and if you go down the other, we get that DCT is false. For this argument, we will be working from ‘either God has reason for His commands or He lacks reason’. Basically, if God commands because it’s good, He has reason. If it’s good because God commands it, he lacks reason.  Here is the argument:
\begin{earg}
    \item[1] Either God has reason for His commands or He lacks reason.
    \item[2] If God lacks reason for His commands, then they are arbitrary.
    \item[3] If they are arbitrary, then God’s imperfect.
    \item[4] If God has reason for his commands, then it’s those reasons, not the commands, which makes actions right or wrong.
    \item[5] If it’s those reasons, then DCT is false.
    \item[6] So, either God’s imperfect or DCT is false.
    \item[7] God is perfect (definition)
    \item[8] Therefore, DCT is false. 
\end{earg}
Another argument follows a similar line of reasoning:

\subsection{The Divine Perfection Argument}
\begin{earg}
    \item[1] If DCT is true, then a morally perfect god could have created a flawless morality which required us to rape, steal, kill, and all that but also forbade any acts of kindness/generosity.
    \item[2] A morally perfect god could not do such a thing.
    \item[3] Therefore, DCT is false
\end{earg}

\chapter{Part 10: Natural Law Theory}

You’re an animal. So am I. So are all people. All animals have certain needs, food, water, security, freedom from pain, and so on. All animals have the same basic problems too. Someday, we’re going to die and we are prone to suffering in the meantime. Maybe, just maybe, figuring out what makes something moral has to do with our place in nature. Many have thought so.

\section{What Makes a Good Human Life?}

Trying to answer that question with the idea that we’re animals seems to lead us down an interesting path. What makes an animal’s life good? The common answer given is something along the lines of ‘when its nature is fulfilled.’ Basically, when it fills its role in nature. A chameleon is built to camouflage, most birds are built to fly, wolves are built to be in packs, and so on. Life wouldn’t be good for a chameleon if it couldn’t change color, a parrot who couldn’t fly would have a bad life, and a lone wolf would just cry to the moon. Applying this to humans, we have an interesting theory. What makes a human’s life good is one which fulfills their nature. This gives us \gls{Natural Law Theory}:
\begin{center}
An action is moral because it is natural and immoral when it is unnatural.
\end{center}
I am using ‘natural’ here in the sense that it is inline with what we are built to do. Another phrasing is ‘an action is moral IFF it’s natural.’

\newglossaryentry{Natural Law Theory}
{
  name=Natural Law Theory,
  description={The stance that an action is moral if and only if it is natural}
}

\section{What are some reasons to think this is right?}

First, there isn't any relativism. Also, we don’t (necessarily) need to bring God or gods in the picture to get it. Basically, so long as there is such a thing as ‘human nature’, there’s grounds for morality. Second, NLT explains why humans can be moral, but not necessarily anything else. Human beings are, in general, rational. No other creature on earth approaches us in our abilities there. Since we are rational, we can better choose whether to go with our nature or against it. This makes it so that humans are the sort of things which get praise and blame. NLT gives us a foundation for morality. Since humans are animals, we have a human nature, and this nature gives us the foundation for morality. Fourth, NLT gives us an answer to a really hard question: How do we learn about ethics? This is an argument for moral skepticism, which was a problem with DCT (how do we know what God wants?). This is often called ‘Hume’s Argument’ after the Scottish Philosopher David Hume.
\begin{earg}
    \item[1] We can only know two kinds of things: conceptual truths or empirical truths.
    \item[2] Moral claims are neither conceptual truths nor empirical truths.
    \item[3] Therefore, we can’t know anything about moral claims.
\end{earg}

There are two terms used here which require some explanation. First, we have the terms `conceptual truth'. These are statements which you can know are true just by knowing the meanings of the words involved. There is much more which can be said about this, but that's enough to serve our purposes. The second term is `empirical truth'. These are statements which we need to have experience in order to know are true. For example, `it is cloudy in Western Washington today' or `water is dihydrogen monoxide'. 

The natural law theorist thinks that morality is empirical, we learn it through experience. To know something about morality, we need two things: First, we need to know what human nature is. Second, we need to know what actions are needed to fulfil that nature. Both of these require us to have experience in the world to know.

\section{Problems with Natural Law Theory}

As before, this theory, too, has its problems. NLT centers around the idea that morality is based on nature. So to know what is moral for humans, we need to know what our nature is. There are 3 different answers to this and each one has its own problems.

\subsection{Human Nature is Animal Nature}

The first way to try and encapsulate the idea of human nature is to claim that humans are animals. So our nature is to act as other animals do. All animals need protection, food, water, and so on. This is why we don’t criticize people for self-defense (when it’s really self-defense). This seems plausible. But, other animals kill and eat other animals. Just because others do something doesn’t mean you should (take disrespecting the dead of enemy combatants as an example, Human Nature is what all humans have in common because they do to ours doesn’t mean we should do it to theirs). Other animals kill their young, others eat their young, still others will brutally kill off weaker members of their own species. The fact that we share traits with other animals doesn’t give us an answer to what human nature is.

\subsection{Human Nature is Innate}

Something is innate when it’s something you are born with, it’s like the programs which come pre-installed in a computer which you can’t delete. Some philosophers (Jean-Jacques Rousseau) have held that humans are innately angelic, we care about each other. Before society corrupts us, we are angels, so to speak. Our nature is to be kind, cooperative, and considerate. Other philosophers (Thomas Hobbes) have held otherwise. Our innate nature is to be selfish, competitive, and distrustful. We are born that way and, generally, stay that way. (In Phil 238, we cover this in detail as it applies to economics). If human nature is the stuff we’re born with, innate, then we would need to settle the debate about whether something is in our nature or is nurtured into us. But this seems wrong. We are very confident that things like rape, theft, and others are wrong. We don’t need to check to see whether it’s in human nature to distrust, compete, or stab each other in the face.  We can be very sure that killing people because they don’t look like us is wrong, but we don’t need to check whether it’s in our nature to do so.

\subsection{Human Nature is what all humans have in common}

Many think that human nature is the traits which all of us share. These are the universal human features which make up ‘humanity’ (“where’s your humanity?!?”). This allows science to figure it out. The data won’t be easy to get, but by looking at enough people, we can figure out what features all of us share. There are two general problems with this view.

\begin{center}
First Problem: There may be no human universals.
\end{center}

It might be silly to think that there aren’t any human universals. Don’t all humans need food, want to live, and have a sex-drive? Can’t all humans make complex plans for the future? Some humans don’t want to live anymore, some people are indifferent to sex, and some even are so mentally impaired that they can’t plan for the future. For just about any trait claimed to be universal, there’s some case where it doesn’t work out. But these theorists do have a reply. Let’s take an example: It is the nature of a cheetah to be fast, to hunt, to have four legs, and be a certain color. But some are albino, others have 3 legs, others still can’t hunt, and still others are slow. We might say that these cheetahs aren’t real cheetahs, but faux-cheetahs. It’s not that the traits are shared by all, but most. If that sounds right, then maybe human nature is what is shared by most of us, not every last one of us. But this too has a problem, how many humans having some trait does it take? 

\begin{center}
The second problem: Whatever these human universals are may not provide moral guidance or ‘good rules’.
\end{center}

Let’s suppose that NLT has a number for that last question, or a percentage. The real problem here is that most human traits don’t give us moral guidance or are irrelevant. Suppose that all of us (or at least most of us) are cruel and greedy. Does it follow that we should be cruel and greedy? Remember the what if test: What if all of us were cruel and greedy, would being cruel and greedy still be wrong? This is a problem, just because something is the case doesn’t mean it ought be the case. Just because the climate is changing at an accelerated rate, doesn’t mean it ought to be changing at an accelerated rate.

\subsection{If human nature isn’t what’s innate (all/most of us share), then what is it?}

The natural response (pun intended) is this:

\begin{center}
Human nature is what we (All Humans) were designed to be/do. This is the function were perform, the role we play, the end we were built for. Remember, that this applies to all people, it is not relative.
\end{center}

This seems to put human nature outside of the realm of science and into the realm of religion. In fact, many natural law theorists have claimed that God was our designer and gave us this purpose. Our Natural Purpose. Since God is all knowing and all good, going against the purpose of humans which God laid out is immoral. (See a link to DCT?) It is not quite the same as DCT, but it is close enough to get all of the problems. So we will look at a more secular version.

\section{Natural Purpose}

It is strange to talk in terms of purpose or other things like that without having some divine architect, however it is useful to think that way. Nature itself lacks the ability to make plans and all that, so it doesn’t have a ‘purpose’ for us. But the mechanism of nature and evolution can serve as a way to get a natural purpose. For instance, the purpose of our kidneys is to filter our blood. Although nothing actually assigned their purpose, we can say what they were meant to do, without having a consciousness which made them for it. This makes us ask the questions: What is the natural purpose of humans? What are we for?

\subsection{The Efficiency Model}

To answer ‘what are humans for?’, some have looked to examples, keeping with the kidneys, we know that they are for filtering blood because their the best at it. So, applying this to humans, what are we the best at. Our natural purpose, our human nature, is, according to this model, what we are better at than any other critter. Just to use an example, humans are better at language or solving puzzles than any other creature on the planet. But, are you a bad human if you can’t solve a puzzle or don’t have great linguistic skills. Remember, according to NLT, something is moral because it’s natural. By this model, something is natural when it fulfills what we, as humans, are the best at. So, if you aren’t that great at puzzles, you are going something wrong…

\subsection{The Fitness Model}

According to this model, our organs have purposes because they enhance our fitness. The heart, lungs, kidneys, and so on have the purposes they do and are good at it because they help us survive and reproduce (those are the classic examples). Nature gave them those purposes because they increased the odds of our survival. Our natural purpose, what is natural for us, is to survive and pass down our genes. Something is moral IFF it is natural, something is natural IFF it fits with our natural purpose, something fits with our natural purpose IFF it enhances our chances of survival and reproduction. This makes NLT boil down to:

\begin{center}
Something is moral IFF it enhances out chances of survival and reproduction.
\end{center}

Both of these have their problems, but we will look most closely at the Fitness Model, mostly because it is the most fun.

\subsubsection{Morality = Increases the chances of survival and reproduction}

This is wrong on many different levels, but here are some examples: Would the most moral act a man could do be to rape as many women as he could in order to produce as many children as he could? See Genghis Khan. According to this account, Genghis Khan was the most moral human. Now, lets take a look at this quote from Primo Levi, an Auschwitz prisoner: “The worst – that is the fittest – survived. The best all died.” Sometimes, the most schooled in violence and the most treacherous will live to survive another day. This means that NTL would endorse violence and brutality in order to be moral, which is certainly not correct.

\subsubsection{Immorality = Decreases the chances of survival and reproduction}

This is also wrong for similar reasons. Not every act which decreases our odds of survival/reproduction is immoral, for example: Wearing headphones to listen to music blocks out the sound of attackers coming to get you (your ears are made for this), decreases your odds of survival. But is wearing headphones and listening to music immoral? Is choosing to only have one partner and just a few children, or none at all, immoral? Choosing that decreases your odds of reproduction. So is wearing protection, so are homosexual acts, so is any sex act without the goal or ability to produce a child… 