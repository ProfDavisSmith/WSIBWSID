\thispagestyle{empty}

\pagestyle{empty}

\vspace*{80pt}

\begin{raggedleft}
\fontsize{30pt}{24pt}\sffamily
\selectfont
  \textbf{Who Should I Be?}
  
  \textbf{What Should I Do?}

\medskip\fontsize{18pt}{20pt}\selectfont

\textbf{An Introduction to Ethics}

\vfill
\fontsize{12pt}{16pt}\selectfont \textit{By }  \textbf{Davis A. Smith}\\
\end{raggedleft}

\newpage

\noindent\fontsize{12pt}{16pt}\selectfont
\emph{Who Should I Be? What Should I Do?} by Davis Smith is, for the most part, an original introduction to Ethics textbook. It covers the most common content in an introductory level Ethics course but also covers less common topics and perspectives, such as Feminist Consequentialism, Mohist Consequentialism, and pre-colonial Nahuatl Virtue Ethics. Aside from the original content provided by Davis Smith, this text also includes primary source material:

\begin{enumerate}
\item The Challenge of Cultural Relativism by James Rachels
\item Why I am an Objectivist about Ethics (and Why You Are, Too) by David Enoch
\item The Moral Status of Bloodbending by Davis Smith
\item The Moral and Legal Status of Abortion by Mary Warren
\item I was Once a Fetus by Alexander Pruss
\end{enumerate}

\subsubsection{Credits for Some of the Content, Images, and Typesetting}
Due to the overlap between an Introduction to Philosophy course and an Introduction to Ethics course, some of the content in Questions We Have by Davis Smith has been repurposed and included in this work. The content is used under the \href{https://creativecommons.org/licenses/by/4.0/}{CC BY 4.0} license. All of the images used in this textbook are AI generated, this includes the cover art. They were generated by Bing's Designer, which is powered by DALL-E 3. This textbook was typeset using \LaTeX. The \LaTeX code was based on the code used to generate Questions We Have by Davis A. Smith, used under the \href{https://creativecommons.org/licenses/by/4.0/}{CC BY 4.0} license. The typesetting for Questions We Have was, in turn, based on the code used for \forallx: \emph{$R^3$} by Davis A. Smith, used under a 
\href{https://creativecommons.org/licenses/by/4.0/}{CC BY 4.0} license , which, finally, was based on the typesetting code used for \href{https://forallx.openlogicproject.org/}{\forallx: \emph{Calgary}}, 
by Aaron Thomas-Bolduc and Richard Zach, used under a 
\href{https://creativecommons.org/licenses/by/4.0/}{CC BY 4.0} license.


\subsubsection{Copyright Status}

This work is licensed under a \href{https://creativecommons.org/licenses/by/4.0/}{Creative Commons Attribution 4.0} license.
You are free to copy and redistribute the material in any medium or format, and  remix, transform, and build upon the material for any purpose, even commercially, under the following terms:
\begin{itemize}
\item You must give appropriate credit, provide a link to the license, and indicate if changes were made. You may do so in any reasonable manner, but not in any way that suggests the licensor endorses you or your use.
\item You may not apply legal terms or technological measures that legally restrict others from doing anything the license permits.
\end{itemize}

\subsubsection{Notes For Instructors}

This book was designed for a quarter-long (10-11 weeks) Introduction to Ethics. In my classes, I (Davis A. Smith) cover all modules, spending 1 week per module. The CC BY license gives you the right to download and distribute the book yourself. In order to ensure that all your students have the same version of the book throughout the term you’re using it, you should upload the PDF you decide to use to your LMS rather than merely give your students a link to the source. You are also free to have the PDFs printed by your bookstore.

\subsubsection{Funding}

The production the 2022 edition of this work was made possible by Professional Development funds provided by \href{https://www.pierce.ctc.edu/elad}{Pierce College's Employee Learning and Development (ELAD)}. 


\includegraphics{marcom-PierceCollege-Logo.png}



\bigskip
