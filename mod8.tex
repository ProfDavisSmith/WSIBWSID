\part{Applied Ethics: The Abortion Debate}
\label{ch.mod8}
\addtocontents{toc}{\protect\mbox{}\protect\hrulefill\par}
\chapter{The Moral and Legal Status of Abortion by Mary Warren}\autocite{Warren1}
We will be concerned with both the moral status of
abortion, which for our purposes we may define as the act
that a woman performs in voluntarily terminating, or
allowing another person to terminate, her pregnancy, and
the legal status that is appropriate for this act. I will argue
that, while it is not possible to produce a satisfactory
defense of a woman’s right to obtain an abortion without
showing that a fetus is not a human being, in the morally
relevant sense of that term, we ought not to conclude that
the difficulties involved in determining whether or not a
fetus is human make it impossible to produce any
satisfactory solution to the problem of the moral status of
abortion. For it is possible to show that, on the basis of
intuitions which we may expect even the opponents of
abortion to share, \emph{a fetus is not a person, hence not the
sort of entity to which it is proper to ascribe full moral
rights.}

Of course, while some philosophers would deny the
possibility of any such proof,\footnote{For example. Roger Wertheimer, who in "Understanding the
Abortion Argument" (Philosophy and Public Affairs I:1) argues
that the problem of the moral status of abortion is insoluble, in
that the dispute over the status of the fetus is not a question of
fact at all, but only a question of how one responds to the facts.} others will deny that there
is any need for it, since the moral permissibility of
abortion appears to them to be too obvious to require
proof. But the inadequacy of this attitude should he
evident from the fact that both the friends and foes of
abortion consider their position to he morally self-evident.
Because proabortionists have never adequately come to
grips with the conceptual issues surrounding abortion,
most if not all, of the arguments which they advance in
opposition to laws restricting access to abortion fail to
refute or even weaken the traditional antiabortion
argument, i.e., that a fetus is a human being, and therefore
abortion is murder.

These arguments are typically of one of two sorts. Either
they point to the terrible side effects of the restrictive
laws, e.g., the deaths due to illegal abortions, and the fact
that it is poor women who suffer the most as a result of
these laws, or else they state that to deny a woman access
to abortion is to deprive her of her right to control her
own body. Unfortunately, however, the fact that
restricting access to abortion has tragic side effects does
not, in itself, show that the restrictions are unjustified,
since murder is wrong regardless of the consequences of
prohibiting it; and the appeal to the right to control ones
body, which is generally construed as a property right, is
at best a rather feeble argument for the permissibility of
abortion. Mere ownership does not give me the right to
kill innocent people whom I find on my property, and
indeed I am apt to he held responsible if such people
injure themselves while on my property. It is equally
unclear that I have any moral right to expel an innocent
person from my property when I know that doing so will
result in his death.

John Noonan is correct in saying that “the fundamental
question in the long history of abortion is, How do you
determine the humanity of a being?”. \autocite{Noonan1} He summarizes his
own antiabortion argument, which is a version of the
official position of the Catholic Church, as follows:\autocite{Noonan2}
\factoidbox{... it is wrong to kill humans, however poor, weak,
defenseless, and lacking in opportunity to develop
their potential they may he. It is therefore morally
wrong to kill infants. Similarly, it is morally wrong to
kill embryos.} 

Noonan bases his claim that fetuses are human upon what
he calls the theologians’ criterion of humanity: that
whoever is conceived of human beings is human. But
although he argues at length for the appropriateness of
this criterion, he never questions the assumption that if a
fetus is human then abortion is wrong for exactly the
same reason that murder is wrong.

Judith Thomson is, in fact, the only writer I am aware of
who has seriously questioned this assumption; she has
argued that, even if we grant the antiabortionist his claim
that a fetus is a human being, with the same right to life as
any other human being, we can still demonstrate that, in at
least some and perhaps most cases, a woman is under no
moral obligation to complete an unwanted pregnancy.\autocite{Thomson1}
Her argument is worth examining, since if it holds up it
may enable us to establish the mural permissibility of
abortion without becoming involved in problems about
what entitles an entity to be considered human, and
accorded full mural rights. To be able to do this would he
a great gain in the power and simplicity of the proabortion
position, since, although I will argue that these problems
can be salved at least as decisively as can any other moral
problem, we should certainly be pleased to be able to
avoid having to solve them as part of the justification of
abortion. 

On the other hand, even if Thomson’s argument does not
hold up, her insight, i.e., that it requires arguments to show that if fetuses are human then abortion is properly
classified as murder, is an extremely valuable one. The
assumption she attacks is particularly invidious, for it
amounts to the decision that it is appropriate, in deciding
the moral status of abortion, to leave the rights of the
pregnant woman out of consideration entirely, except
possibly when her life is threatened. Obviously, this will
not do; determining what moral rights, if any, a fetus
possesses is only the first step in determining the moral
status of abortion. Step two, which is at least equally
essential, is finding a just solution to the conflict between
whatever rights the fetus may have, and the rights of the
woman who is unwillingly pregnant. While the historical
error has been to pay far too little attention to the second
step, Thomson’s suggestion is that if we look at the
second step first, we may find that a woman has a right to
obtain an abortion regardless of what rights the fetus has.

Our own inquiry will also have two stages. In Section I,
we will consider whether or not it is possible to establish
that abortion is morally permissible even on the
assumption that a fetus is an entity with a full-fledged
right to life. I will argue that in fact this cannot he
established, at least not with the conclusiveness which is
essential to our hopes of convincing those who are
skeptical about the morality of abortion, and that we
therefore cannot avoid dealing with the question of
whether or not a fetus really does have the same right to
life as a (more fully developed) human being.

In Section II, I will propose an answer to this question,
namely, that a fetus cannot he considered a member of the
moral community, the set of beings with full and equal
moral rights, for the simple reason that it is not a person,
and that it is personhood, and not genetic humanity, i.e.,
humanity as defined by Noonan, which is the basis for
membership in this community. I will argue that a fetus,
whatever its stage of development, satisfies none of the
basic criteria of personhood, and is not even enough like a
person to he accorded even some of the same rights on the
basis of this resemblance. Nor, as we will see, is a fetus’s
potential personhood a threat to the morality of abortion,
since, whatever the rights of potential people may be, they
are invariably overridden in any conflict with the moral
rights of actual people.

\section{Part I}

We turn now to Professor Thomson’s case for the claim
that even if a fetus has full moral rights, abortion is still
morally permissible, at least sometimes, and for some
reasons other than to save the woman’s life. Her argument
is based upon a clever, but I think faulty, thinking. She
asked us to picture ourselves waking up one day, in bed
with a famous violinist. Imagine that you have been
kidnapped, and your bloodstream hooked up to that of the
violinist, who happens to have an ailment that will
certainly kill him unless he is permitted to share your
kidneys for a period of nine months. No one else can save
him, since you alone have the right type of blood. He will
he unconscious all that time, and you will have to stay in
bed with him, but after the nine months are over he may
be unplugged, completely cured, that is provided that you
have cooperated.

Now then, she continues, what are your obligations in this
situation? The antiabortionist, if he is consistent, will have
to say that you are obligated to stay in bed with the
violinist: for all people have a right to life, and violinists
are people, and therefore it would be murder for you to
disconnect yourself from him and let him die.\autocite{Thomson1} But this is
outrageous, and so there must he something wrong with
the same argument when it is applied to abortion. It would
certainly be commendable of you to agree to save the
violinist, but it is absurd to suggest that your refusal to do
so would be murder. His right to life does not obligate
you to do whatever is required to keep him alive; nor does
it justify anyone else forcing you to do so. A law that
required you to stay in bed with the violinist would
clearly be an unjust law, since it is no proper function of
the law to force unwilling people to make huge sacrifice
for the sake of other people toward whom they have no
such prior obligation. Thomson concludes that, if this
analogy is an apt one, then we can grant the
antiabortionist his claim that a fetus is a human being, and
still hold that it is at least sometimes the case that a
pregnant woman has the right to refuse to be a Good
Samaritan towards the fetus, i.e., to obtain an abortion.
For there is a great gap between the claim that x has a
right to life, and the claim that y is obligated to do
whatever is necessary to keep x alive, let alone that he
ought to be forced to do so. It is y’s duty to keep x alive
only if he somehow contracted a special obligation to do
so; a woman who is unwillingly pregnant, e.g., who was
raped, has done nothing which obligates her to make the
enormous sacrifice which is necessary to preserve the
conceptus.

This argument is initially quite plausible, and in the
extreme case of pregnancy due to rape, it is probably
conclusive. Difficulties arise, however, when we try to
specify more exactly the range of cases in which abortion
is clearly justifiable even on the assumption that the fetus
is human. Professor Thomson considers it a virtue of her
argument that it does not enable us to conclude that
abortion is always permissible. It would, she says, be
“indecent” for a woman in seventh month to obtain an
abortion just to avoid having to postpone a trip to Europe.
On the other hand, her argument enables us to see that “a
sick and desperately frightened schoolgirl pregnant due to
rape may of course choose abortion, and that any law
which rules this out is an insane law” (p. 65). So far, so
good, but what are we to say about the woman who
becomes pregnant not through rape but as a result of her
own carelessness, or because of contraceptive failure, or
who gets pregnant intentionally and then changes her
mind about wanting a child? With respect to such cases,
the violinist analogy is of much less use to the defender of
the woman’s right to obtain an abortion.

Indeed, the choice of a pregnancy due to rape, as an
example of a case in which abortion is permissible even if
a fetus is considered a human being, is extremely
significant; for it is only in the case of pregnancy due to
rape that the woman’s situation is adequately analogous to
the violinist case for our intuitions about the latter to
transfer convincingly. The crucial difference between a
pregnancy due to rape and the normal case of an
unwanted pregnancy is that in the normal case we cannot
claim that the woman is in no way responsible for her
predicament; she could have remained chaste, or taken
her pills more faithfully or abstained on dangerous days,
and so on. If on the other hand, you are kidnapped by
strangers, and hooked up to a strange violinist, then you
are free of any shred of responsibility for the situation, on
the basis of which it would he argued that you are
obligated to keep the violinist alive. Only when her
pregnancy is due to rape is a woman clearly just as
nonresponsible.\footnote{We may safely ignore the fact that she might have avoided
getting raped, e.g., by carrying a gun, since by similar means
you might likewise have avoided getting kidnapped, and in
neither case does the victim's failure to take all possible
precautions against a highly unlikely event (as opposed to
reasonable precautions against a rather likely event) mean that
he is morally responsible for what happens.}

Consequently, there is room for the antiabortionist to
argue that in the normal case of unwanted pregnancy a
woman has, by her own actions, assumed responsibility of
the fetus. For if x behaves in a way which he could have
avoided, and which he knows involves, let us say, a 1
percent chance of bringing into existence a human being,
with a right to life, and does so knowing that if this should
happen then that human being will perish unless x does
certain things to keep him alive, then it is by no means
clear that when it does happen x is free of any obligation
to what he knew in advance would he required to keep
that human being alive.

The plausibility of such an argument is enough to show
that the Thomson analogy can provide a clear and
persuasive defense of a woman’s right to obtain an
abortion only with respect to those cases in which the
woman is in no way responsible for her pregnancy, e.g.,
where it is due to rape. In all other cases, we would
almost certainly conclude that it was necessary to look
carefully at the particular circumstances in order to
determine the extent of the woman’s responsibility and
hence the extent of her obligation. This is an extremely
unsatisfactory outcome, from the viewpoint of the
opponents of restrictive abortion laws, most of whom are
convinced that a woman has a right to obtain an abortion
regardless of how and why she got pregnant.

Of course, a supporter of the violinist analogy might point
out that it is absurd to suggest that forgetting her pill one
day might be sufficient to obligate a woman to complete
an unwanted pregnancy. And indeed, it is absurd to
suggest this. As we will see, the moral right to obtain an
abortion is not in the least dependent upon the extent to
which a woman is responsible for her pregnancy. But
unfortunately, once we allow the assumption that a fetus
has full moral rights, we cannot avoid taking this absurd
suggestion seriously. Perhaps we can make this point
more clear by altering the violinist story just enough to
make it more analogous to a normal unwanted pregnancy
and less to a pregnancy due to rape, and then seeing
whether it is still obvious that you are not obligated to
stay in bed with the fellow.

Suppose, then, that violinists are peculiarly prone to the
sort of illness the only cure for which is the use of
someone else’s bloodstream for nine months, and that
because of this there has been formed a society of music
lovers who agree that whenever a violinist is stricken they
will draw lots and the loser will, by some means, be made
the one and only person capable of saving him. Now then,
would you be obligated to cooperate in curing the
violinist if you had voluntarily joined this society,
knowing the possible consequences, and then your name
had been drawn and you had been kidnapped?
Admittedly, you did not promise ahead of time that you
would, but you did deliberately place yourself in a
position in which it might happen that a human life would
be lost if you did not. Surely, this is at least a prima facie
reason for supposing that you have an obligation to stay in
bed with the violinist. Suppose that you had gotten your
name drawn deliberately; surely that would be quite a
strong reason for thinking that you had such an obligation.

It might be suggested that there is one important
disanalogv between the modified violinist case and the
case of an unwanted pregnancy, which makes the
woman’s responsibility significantly less, namely, the fact
that the fetus comes into existence as the result of the
woman’s actions. This fact might give her a right to
refuse to keep it alive, whereas she would not have had
this right had it existed previously, independently, and
then as a result of her actions become dependent upon her
for its survival.

My own intuition, however, is that x has no more right to
bring into existence, either deliberately or as a foreseeable
result of actions he could have avoided, a being with full
moral rights y, and then refuse to do what he knew
beforehand would be required to keep that being alive,
than he has to enter into an agreement with an existing
person, whereby he may be called upon to save that
person’s life, and then refuse to do so when so called 
upon. Thus x’s responsibility for y’s existence does not
seem to lessen his obligation to keep y alive, if he is also
responsible for y’s being in a situation in which only he
can save him.

Whether or not this intuition is entirely correct, it brings
us back once again to the conclusion that once we allow
the assumption that a fetus has full moral rights it
becomes an extremely complex and difficult question
whether and when abortion is justifiable. Thus the
Thomson analogy cannot help us produce a clear and
persuasive proof of the moral permissibility of abortion.
Nor will the opponents of the restrictive laws thank us for
anything less; for their conviction (for the must part) is
that abortion is obviously not a morally serious and
extremely unfortunate, even though sometimes justified
act, comparable to killing in self-defense or to letting the
violinist die, but rather is closer to being a morally neutral
act, like cutting one’s hair.

The basis of this conviction, I believe, is the realization
that a fetus is not a person, and thus does not have a
full-fledged right to life. Perhaps the reason why this
claim has been so inadequately defended is that it seems
self-evident to those who accept it. And so it is, insofar as
it follows from what I take to be perfectly obvious claims
about the nature of personhood, and about the proper
grounds for ascribing moral rights, claims which ought,
indeed, to be obvious to both the friends and foes of
abortion. Nevertheless, it is worth examining these
claims, and showing how they demonstrate the moral
innocuousness of abortion, since this apparently has not
been adequately done before.

\section{Part II}

The question which we must answer in order to produce a
satisfactory solution to the problem of the moral status of
abortion is this: How are we to define the moral
community, the set of beings with full and equal moral
rights, such that we can decide whether a human fetus is a
member of this community or not? What sort of entity,
exactly, has the inalienable rights to life, liberty, and the
pursuit of happiness? Jefferson attributed these rights to
all men ... If so, then we arrive, first, at Noonan’s
problem of defining what makes a being human, and,
second, at the equally vital question which Noonan does
not consider, namely, What reason is there for identifying
the moral community with the set of all human beings, in
whatever way we have chosen to define that term?

\subsection{1. On the Definition of “Human”}

One reason why this vital second question is so frequently
overlooked in the debate over the moral status of abortion
is that the term “human” has two distinct, but not often
distinguished, senses. This fact results in a slide of
meaning, which serves to conceal the fallaciousness of the
traditional argument that since (1) it is wrong to kill
innocent human beings, and (2) fetuses are innocent
human beings, then (3) it is wrong to kill fetuses. For if
“human” is used in the same sense in both (1) and (2)
then, whichever of the two uses is meant, one of these
premises is question-begging. And if it is used in two
different senses then of course the conclusion doesn’t
follow.

Thus, (1) is a self-evident moral truth, \footnote{Of course, the principle that it is (always) wrong to kill
innocent human beings is in need of many other modifications,
e.g., that it may he permissible to do so to save a greater number
of other innocent human beings, but we may safely ignore these
complications here.} and avoids
begging the question about abortion, only if “human
being” is used to mean something like “a full-fledged
member of the moral community.” (It may or may not
also be meant to refer exclusively to members of the
species Homo sapiens.) We may call this the moral sense
of “human.” It is not to be confused with what we will
call the genetic sense; i.e., the sense in which a member of
the species is a human being, and no member of any other
species could be. If (1) is acceptable only if the moral
sense is intended, (2) is non-question-begging only if
what is intended is the genetic sense.

In “Deciding Who Is Human,” Noonan argues for the
classification of fetuses with human beings by pointing to
the presence of the full genetic code, and the potential
capacity for rational thought (p. 135). It is clear that what
he needs to show, for his version of the traditional
argument to be valid, is that fetuses are human in the
moral sense, the sense in which it is analytically true that
all human beings have full moral rights. But, in the
absence of any argument showing that whatever is
genetically human is also morally human, and he gives
none, nothing more than genetic humanity can be
demonstrated by the presence of the human genetic code.
And, as we will see, the potential capacity for rational
thought can at most show that an entity has the potential
for becoming human in the moral sense.

\subsection{2. Defining the Moral Community}

Can it be established that genetic humanity is sufficient
for moral humanity’? I think that there are very good
reasons for not defining the moral community in way. I
would like to suggest an alternative way of defining the
moral community, which I will argue for only to the
extent of explaining why it is, or should be, self-evident.
The suggestion is simply that the moral community
consists of all and only people, rather than all and only
human beings;\footnote{From here on, we will use "human" to mean genetically
human, since the moral sense seems closely connected to, and perhaps derived from, the assumption that genetic humanity is
sufficient for membership in the moral community.} and probably the best way of
demonstrating its self-evidence is by considering the
concept of personhood, to see what sorts of entity are and
are not persons, and what the decision that a being is or is
not a person implies about its moral rights.

What characteristics entitle an entity to be considered a
person? This is obviously not the place to attempt a
complete analysis of the concept of personhood, but we
do not need such a fully adequate analysis just to
determine whether and why a fetus is or isn’t a person.
All we need is a rough and approximate list of the most
basic criteria of personhood, and some idea of which, or
how many, of these an entity must satisfy in order to
properly be considered a person.
28 In searching for such criteria, it is useful to look beyond
the set of people with whom we are acquainted, and ask
how we would decide whether a totally alien being was a
person or not. (For we have no right to assume that
genetic humanity is necessary for personhood.) Imagine a
space traveler who lands on an unknown planet and
encounters a race of beings utterly unlike any he has ever
seen or heard of. If he wants to be sure of behaving
morally toward these beings, he has to somehow decide
whether they are people, and hence have full moral rights,
or whether they are the sort of thing which he need not
feel guilty about treating as, for example, a source of
food.

How should he go about making this decision? If he has
some anthropological background, he might look for such
things as religion, art, and the manufacturing of tools,
weapons, or shelters, since these factors have been used to
distinguish our human from our prehuman ancestors, in
what seems to be closer to the moral than the genetic
sense of “human.” And no doubt he would be right to
consider the presence of such factors as good evidence
that the alien beings were people, and morally human. It
would, however, be overly anthropocentric of him to take
the absence of these things as adequate evidence that they
were not, since we can imagine people who have
progressed beyond, or evolved without ever developing
these cultural characteristics.

I suggest that the traits which are most central to the
concept of personhood, or humanity’ in the moral sense,
are, very roughly; the following:

\begin{enumerate}
\item[1] consciousness (of objects and events external
and/or internal to the being), and in particular the
capacity to feel pain;
\item[2] reasoning (the developed capacity to solve new and
relatively complex problems);
\item[3] self-motivated activity (activity which is relatively
independent of either genetic or direct external
control);
\item[4] the capacity to communicate, by whatever means,
messages of an indefinite variety of types, that is, not
just with an indefinite number of possible contents,
but on indefinitely many possible topics;
\item[5] the presence of self-concepts, and self-awareness,
either individual or racial, or both.
\end{enumerate}

Admittedly, there are apt to he a great many problems
involved in formulating precise definitions of these
criteria, let alone in developing universally valid
behavioral criteria for deciding when they apply. But I
will assume that both we and our explorer know
approximately what (1)-(5) mean, and that he is also able
to determine whether or not they apply. How, then, should
he use his findings to decide whether or not the alien
beings are people? We needn’t suppose that an entity
must have oil of these attributes to he properly considered
a person; (1) and (2) alone may well he sufficient for
personhood, and quite probably (1)-(3), if “activity” is
construed so as to include the activity of reasoning.

All we need to claim, to demonstrate that a fetus is not a
person, is that any being which satisfies none of (1)-(5) is
certainly not a person. I consider this claim to be so
obvious that I think anyone who denied it, and claimed
that a being which satisfied none of (1)-(5) was a person
all the same, would thereby demonstrate that he had no
notion at all of what a person is—perhaps because he had
confused the concept of a person with that of genetic
humanity. If the opponents of abortion were to deny the
appropriateness of these five criteria, I do not know what
further arguments would convince them. We would
probably have to admit that our conceptual schemes were
indeed irreconcilably different, and that our dispute could
not be settled objectively.

I do not expect this to happen, however, since I think that
the concept of a person is one which is very nearly
universal (to people), and that it is common to both
proabortionists and antiabortionists, even though neither
group has fully realized the relevance of this concept to
the resolution of their dispute. Furthermore, I think that
on reflection even the antiabortionists ought to agree not
only that (1) - (5) are central to the concept of
personhood, but also that it is a part of this concept that
all and only people have full moral rights. The concept of
a person is in part a moral concept; once we have
admitted that x is a person we have recognized, even if we
have not agreed to respect, x’s right to he treated as a
member of the moral community. It is true that the claim
that x is a human being is more commonly voiced as part
of an appeal to treat x decently than is the claim that x is a
person, but this is either because “human being” is here
used in the sense which implies personhood, or because
the genetic and moral sense of “human” have been
confused.

Now if (1)-(5) are indeed the primary criteria of
personhood, then it is clear that genetic humanity is
neither necessary nor sufficient for establishing that an
entity is a person. Some human beings arc not people, and
there may well be people who are not human beings. A
man or woman whose consciousness has been
permanently obliterated but who remains alive is a human
being which is no longer a person; defective human
beings, with no appreciable mental capacity, are not and
presumably never will be people; and a fetus is a human
being which is not yet a person, and which therefore
cannot coherently be said to have full moral rights.
Citizens of the next century should be prepared to
recognize highly advanced, self-aware robots or
computers, should such he developed, and intelligent
inhabitants of other worlds, should such he found, as
people in the fullest sense, and to respect their moral
rights. But to ascribe full moral rights to an entity which
is not a person is as absurd as to ascribe moral obligations
and responsibilities to such an entity.

\subsection{3. Fetal Development and the Right to Life}

Two problems arise in the application of these
suggestions for the definition of the moral community to
the determination of the precise moral status of a human
fetus. Given that the paradigm example of a person is a
normal adult being, then (I) How like this paradigm, in
particular how far advanced since conception, does a
human being need to be before it begins to have a right to
life by virtue, not of being fully a person as of vet, but of
being like a person? and (2) To what extent, if any does
the fact that a fetus has the potential for becoming a
person endow it with some of the same rights? Each of
these questions requires some comment.

In answering the first question, we need not attempt a
detailed consideration of the moral rights of organisms
which are not developed enough, aware enough,
intelligent enough, etc., to be considered people, but
which resemble people in some respects. It does seem
reasonable to suggest that the more like a person, in the
relevant respects, a being is, the stronger is the case for
regarding it as having a right to life, and indeed the
stronger its right to life is. Thus we ought to take seriously
the suggestion that, insofar as “the human individual
develops biologically in a continuous fashion ... the rights
of a human person might develop in the same way”. \autocite{Hayes1} But
we must keep in mind that the attributes which are
relevant in determining whether or not an entity is enough
like a person to be regarded as having some of the same
moral rights are no different from those which are
relevant to determining whether or not it is fully a
person—i.e., are no different from (1)-(5)—and that being
genetically human, or having recognizably human facial
and other physical features, or detectable brain activity, or
the capacity to survive outside the uterus, are simply not
among these relevant attributes.

Thus it is clear that even though a seven- or eight-month
fetus has features which make it apt to arouse in us almost
the same powerful protective instinct as is commonly
aroused by a small infant, nevertheless it is not
significantly more personlike than is a very small embryo.
It is somewhat more personlike; it can apparently feel and
respond to pain, and it may even have a rudimentary form
of consciousness, insofar as its brain is quite active.
Nevertheless, it seems safe to say that it is not fully
conscious, in the way that an infant of a few months is,
and that it cannot reason, or communicate messages of
indefinitely many sorts, does not engage in self-motivated
activity; and has no self-awareness. Thus, in the relevant
respects, a fetus, even a fully developed one, is
considerably less personlike than is the average mature
mammal, indeed the average fish. And I think that a
rational person must conclude that if the right to life of a
fetus is to be based upon its resemblance to a person, then
it cannot be said to have any more right to life then, let us
say, a newborn guppy (which also seems to be capable of
feeling pain), and that a right of that magnitude could
never override a woman’s right to obtain an abortion, at
any stage of her pregnancy.

There may, of course, he other arguments in favor of
placing legal limits upon the stage of pregnancy in which
an abortion may he performed. Given the relative safety
of the new techniques of artificially inducing labor during
the third trimester, the danger to the woman’s life or
health is no longer such an argument.
39 Neither is the fact that people tend to respond to the
thought of abortion in the later stages of pregnancy with
emotional repulsion, since mere emotional responses
cannot take the place of moral reasoning in determining
what ought to he permitted. Nor, finally, is the frequently
heard argument that legalizing abortion, especially late in
the pregnancy, may erode the level of respect for human
life, leading, perhaps, to an increase in unjustified
euthanasia and other crimes. For this threat, if it is a
threat, can be better met by educating people to the kinds
of moral distinctions which we are making here than by
limiting access to abortion (which limitation may, in its
disregard for the rights of women, be just as damaging to
the level of respect for human rights).

Thus, since the fact that even a fully developed fetus is not
personlike enough to have any significant right to life on
the basis of its personlikeness shows that no legal
restrictions upon the stage of pregnancy in which an
abortion may be performed can be justified on the
grounds that we should protect the rights of the older
fetus. And once there is no other apparent justification for
such restrictions, we may conclude that they are entirely
unjustified. Whether or not it would be indecent
(whatever that means) for a woman in her seventh month
to obtain an abortion just to avoid having a to postpone a
trip to Europe, it would not, in itself, be immoral, and
therefore it ought to be permitted.

\subsection{4. Potential Personhood and the Right to Life}

We have seen that a fetus does not resemble a person in
any way that can support the claim that it has even some
of the same rights. But what about its potential, the fact
that if nurtured and allowed to develop naturally it will
very probably become a person? Doesn’t that alone give it
at least some right to life? It is hard to deny that the fact
that an entity is a potential person is a strong prima facie
reason for not destroying it, but we need not conclude
from this that a potential person has a right to life, by
virtue of that potential. It may be that our feeling that it is
better, other things being equal, not to destroy a potential
person is better explained by the fact that potential people
are still (felt to be) an invaluable resource, not to be
lightly squandered. Surely, if every speck of dust were a
potential person, we would be much less apt to conclude
that every potential person has a right to become actual.

Still, we do not need to insist that a potential person has
no right to life whatever. There may well be something
immoral, and not just imprudent, about wantonly
destroying potential people, when doing so isn’t necessary
to protect anyone’s rights. But even if a potential person
does have some prima facie right to life, such a right
could not possibly outweigh the right of a woman to
obtain an abortion, since the rights of any actual person
invariably outweigh those of any potential person,
whenever the two conflict. Since this may not be
immediately obvious in the case of a human fetus, let us
look at another ease.

Suppose that our space explorer falls into the hands of an
alien culture, whose scientists decide to create a few
hundred thousand or more human beings, by breaking his
body into its component cells, and using these to create
fully developed human beings, with, of course, his genetic
code. We may imagine that each of these newly created
men will have all of the original man’s abilities, skills,
knowledge, and so on, and also have an individual
self-concept, in short that each of them will be a bona fide
(though hardly unique) person. Imagine that the whole
project will take only seconds, and that its chances of
success are extremely high, and that our explorer knows
all of this, and also knows that these people will be treated
fairly. I maintain that in such a situation he would have
every right to escape if he could, and thus to deprive all of
these potential people of their potential lives; for his right
to life outweighs all of theirs together, in spite of the fact
that they are all genetically human, all innocent, and all
have a very high probability of becoming people very
soon, if only he refrains from acting.

Indeed, I think he would have a right to escape even if it
were not his life which the alien scientists planned to take,
but only a year of his freedom, or, indeed, only a day. Nor
would he be obligated to stay if he had gotten captured
(thus bringing all these people-potentials into existence)
because of his own carelessness, or even if he had done so
deliberately knowing the consequences. Regardless of
how he got captured, he is not morally obligated to
remain in captivity for any period of time for the sake of
permitting any number of potential people to come into
actuality, so great is the margin by which one actual
person’s right to liberty outweighs whatever right to life
even a hundred thousand potential people have. And it
seems reasonable to conclude that the rights of a woman
will outweigh by a similar margin whatever right to life a
fetus may have by virtue of its potential personhood.

Thus, neither a fetus’s resemblance to a person, nor its
potential for becoming a person, provides any basis
whatsoever for the claim that it has any significant right to
life. Consequently, a woman’s right to protect her health,
happiness, freedom, and even her life, \footnote{That is, insofar as the death rate, for the woman, is higher for
childbirth than for early abortion.} by terminating an
unwanted pregnancy will always override whatever right
to life it may be appropriate to ascribe to a fetus, even a
fully developed one. And thus, in the absence of any
overwhelming social need for every possible child, the
laws which restrict the right to obtain an abortion, or limit
the period of pregnancy during which an abortion maybe
performed, are a wholly unjustified violation of a
woman’s most basic moral and constitutional rights.\footnote{My thanks to the following people, who were kind enough to
read and criticize an earlier version of this paper: Herbert Gold,
Gene Glass, Anne Lauterbach, Judith Thomson, Mary
Mothersill, and Timothy Binkley.}

\section{Postscript on Infanticide, February 26, 1982}

One of the most troubling objections to the argument
presented in this article is that it may appear to justify not
only abortion but infanticide as well. A newborn infant is
not a great deal more personlike than a ninemonth fetus,
and thus it might seem that if late-term abortion is
sometimes justified, then infanticide must also be
sometimes justified. Yet most people consider that
infanticide is a form of murder, and thus never justified.

While it is important to appreciate the emotional force of
this objection, its logical force is far less than it may seem
at first glance. There are many reasons why infanticide is
much more difficult to justify than abortion, even though
if my argument is correct neither constitutes the killing of
a person. In this country, and in this period of history, the
deliberate killing of viable newborns is virtually never
justified. This is in part because neonates are so very
close to being persons that to kill them requires a very
strong moral justification as does the killing of dolphins,
whales, chimpanzees, and other highly personlike
creatures. It is certainly wrong to kill such beings just for
the sake of convenience, or financial profit, or “sport.”

Another reason why infanticide is usually wrong, in our
society, is that if the newborn’s parents do not want it, or
are unable to care for it, there are (in most cases) people
who are able and eager to adopt it and to provide a good
home for it. Many people wait years for the opportunity to
adopt a child, and some are unable to do so even though
there is every reason to believe that they would be good
parents. The needless destruction of a viable infant
inevitably deprives some person or persons of a source of
great pleasure and satisfaction, perhaps severely
impoverishing their lives. Furthermore, even if an infant
is considered to be adoptable (e.g., because of some
extremely severe mental or physical handicap) it is still
wrong in most cases to kill it. For most of us value the
lives of infants, and would prefer to pay taxes to support
orphanages and state institutions for the handicapped
rather than to allow unwanted infants to be killed. So long
as most people feel this way, and so long as our society
can afford to provide care for infants which are unwanted
or which have special needs that preclude home care, it is
wrong to destroy any infant which has a chance of living
a reasonably satisfactory life.

If these arguments show that infanticide is wrong, at least
in this society, then why don’t they also show that late-
term abortion is wrong? After all, third trimester fetuses
are also highly personlike, and many people value them
and would much prefer that they be preserved; even at
some cost to themselves. As a potential source of pleasure
to some family, a viable fetus is just as valuable as a
viable infant. But there is an obvious and crucial
difference between the two cases: once the infant is born,
its continued life cannot (except, perhaps, in very
exceptional cases) pose any serious threat to the woman’s
life or health, since she is free to put it up for adoption, or,
where this is impossible, to place it in a state-supported
institution. While she might prefer that it die, rather than
being raised by others, it is not clear that such a
preference would constitute a right on her part. True, she
may suffer greatly from the knowledge that her child will
be thrown into the lottery of the adoption system, and that
she will be unable to ensure its well-being, or even to
know whether it is healthy, happy, doing well in school,
etc.: for the law generally does not permit natural parents
to remain in contact with their children, once they are
adopted by another family. But there are surely better
ways of dealing with these problems than by permitting
infanticide in such cases. (It might help, for instance, if
the natural parents of adopted children could at least
receive some information about their progress, without
necessarily being informed of the identity of the adopting
family.)

In contrast, a pregnant woman’s right to protect her own
life and health clearly outweighs other people’s desire that
the fetus be preserved-just as, when a person’s life or limb
is threatened by some wild animal, and when the threat
cannot be removed without killing the animal, the
person’s right to self-protection outweighs the desires of
those who would prefer that the animal not be harmed.
Thus, while the moment of birth may not mark any sharp
discontinuity in the degree to which an infant possesses a
right to life, it does mark the end of the mother’s absolute
right to determine its fate. Indeed, if and when a late-term
abortion could be safely performed without killing the
fetus, she would have no absolute right to insist on its
death (e.g., if others wish to adopt it or pay for its care),
for the same reason that she does not have a right to insist
that a viable infant be killed.

It remains true that according to my argument neither
abortion nor the killing of neonates is properly considered
a form of murder. Perhaps it is understandable that the
law should classify infanticide as murder or homicide,
since there is no other existing legal category which
adequately or conveniently expresses the force of our
society’s disapproval of this action. But the moral
distinction remains, and it has several important
consequences.

In the first place, it implies that when an infant is born
into a society which-unlike ours-is so impoverished that it
simply cannot care for it adequately without endangering
the survival of existing persons, killing it or allowing it to
die is not necessarily wrong-provided that there is no
other society which is willing and able to provide such
care. Most human societies, from those at the hunting and
gathering stage of economic development to the highly
civilized Greeks and Romans, have permitted the practice
of infanticide under such unfortunate circumstances, and I
would argue that it shows a serious lack of understanding
to condemn them as morally backward for this reason
alone.

In the second place, the argument implies that when an
infant is born with such severe physical anomalies that its
life would predictably be a very short and/or very
miserable one, even with the most heroic of medical
treatment, and where its parents do not choose to bear the
often crushing emotional, financial and other burdens
attendant upon the artificial prolongation of such a tragic
life, it is not morally wrong to cease or withhold
treatment, thus allowing the infant a painless death. It is
wrong (and sometimes a form of murder) to practice
involuntary euthanasia on persons, since they have the
right to decide for themselves whether or not they wish to
continue to live. But terminally ill neonates cannot make
this decision for themselves, and thus it is incumbent
upon responsible persons to make the decision for them,
as best they can. The mistaken belief that infanticide is
always tantamount to murder is responsible for a great
deal of unnecessary suffering, not just on the part of
infants which are made to endure needlessly prolonged
and painful deaths, but also on the part of parents, nurses,
and other involved persons, who must watch infants
suffering needlessly, helpless to end that suffering in the
most humane way.

I am well aware that these conclusions, however modest
and reasonable they may seem to some people, strike
other people as morally monstrous, and that some people
might even prefer to abandon their previous support for
women’s right to abortion rather than accept a theory
which leads to such conclusions about infanticide. But all
that these facts show is that abortion is not an isolated
moral issue; to fully understand the moral status of
abortion we may have to reconsider other moral issues as
well, issues not just about infanticide and euthanasia, but
also about the moral rights of women and of nonhuman
animals. It is a philosopher’s task to criticize mistaken
beliefs which stand in the way of moral understanding,
even when-perhaps especially when-those beliefs are
popular and widespread. The belief that moral strictures
against killing should apply equally to all genetically
human entities, and only to genetically human entities, is
such an error. The overcoming of this error will
undoubtedly require long and often painful struggle; but it
must be done.
\chapter{Part 23:What is an Abortion?}

This question is the start of a very heated debate which can get nasty, so please keep that in mind. I have used other heated examples before, so you should know how to handle this. For this chunk, we are just looking at what it is, not the moral status of it. Here, we are looking at the metaphysical question concerning abortion, what is it? Later, once this is settled, we will look at the ethical question, is it OK to have one? 

One essential feature to an abortion, it would seem, is that there needs to be the ending of a pregnancy. But, that certainly can't be it. Take this case as an example:
\begin{earg}
    \item[] Birth is the ending of a pregnancy. 
    \item[] By our definition, the ending of a pregnancy is an abortion. 
    \item[Therefore,] birth is an abortion.
\end{earg}
So, we could add in something about the pre-mature nature of the termination, but that would make pre-mature births abortions, which also seems just as wrong (as in misfitting). Glossing over some of the more graphic examples I could give, the core, missing feature which makes an act an abortion and not birth or some crime against another person seems to be that it needs to be voluntary. The woman needs to, with informed consent, want to terminate the pregnancy early, without resulting in a child. There can be interesting cases, worth thinking about, where the woman gives consent, but not informed consent (she may have been misinformed about what exactly it entails, which might make her not want it).   For this module, we will be defining an abortion as one of these two things (could be both, but that's a weird case), these two features fit for both the pro-choice side and the pro-life side of the debate, we will be covering both:
\begin{enumerate}
\item A woman voluntarily terminating her own pregnancy.  	
\item A woman allowing another to terminate her pregnancy (referring to the subject).
\end{enumerate}
If you only have the first one, then you will not get cases of, say, doctor assisting the woman in terminating her pregnancy. If you only have the second one, then you will not get cases of self-administered abortions.  To avoid the cases where birth could be defined as an abortion, we need to say that 'terminating a pregnancy' does involve the ending of a fetus. The moral status of that fetus is where the debate is.  

\subsection{Is it morally permissible to have an abortion?}

I get that this is a hot issue, and if I have not already, I guess that I will need to be far more active in the comments in the discussion for this one than I already have been, please remember to be civil. For Warren, the moral status of abortion hinges on the answer to the following question:

\begin{center}
The Fetus Question: Is a fetus a person, in the morally relevant sense?
\end{center}

The main tie-in, and one which you will read me say several times, is that if a fetus is a person, then abortion is wrong, if a fetus is not a person, then abortion is permissible. The Fetus Question moves the ethical debate regarding abortion, "is abortion murder?" to a metaphysical debate regarding person-hood. Typically, our moral intuitions are gut responses, which come from mental shortcuts, when we analyse that shortcut, we can get down to a metaphysical question which we have grounds to prove or disprove. The author of the reading is arguing that it is possible to show that a fetus is not a person, making abortion permissible.\footnote{Some students, especially those who use translation dictionaries, will have issues with two different words, these words are "person" and "human". These do not mean the same thing, and we will see how these come apart later. For now, a human is a member of our species and a person is a being with moral worth equal to you or me. We will see later how there can be non-human persons.} 

\subsection{Framing the Problem}

Of course, while some philosophers and others would deny the possibility of a proof that a fetus is not a person, claiming that to do so would be to prove a contradiction or that it's not possible to prove either way (this would lead to skepticism about fetal person-hood). By the same token, others will claim that there's no need for a proof. These people claim that the moral status of a fetus, its person-hood, is too obvious to need justification. But, both sides of the debate, pro-choice and the pro-life, take their evidence and reasoning to be obvious, to an equal degree. This is much like a belief in God. Some Atheists claim that the non-existence of God is clear and obvious, while at the same time, Theists claim that the existence of God is equally obvious. This disagreement means that we can't trust our gut instincts on this issue, we need to use logic to show that one of the sides is faulty.

Through this module, we will see the best arguments on both sides of the abortion debate. But, to start us off, we will look at the commonplace, pro-choice arguments. Though Warren agrees with their conclusions, she disagrees with how they got there. There are glaring issues in their reasoning. 

\noindent \begin{tabular}{p{2in}|p{2.2in}}
Argument A&Argument B\\
    \begin{earg}
    \item[] Restrictive Laws regarding abortion cause more harm than the lack of those laws.
    \item[] Causing more harm than otherwise is always wrong.
    \item[] Therefore, restrictive laws regarding abortions are wrong.
\end{earg}&\begin{earg}
    \item[] Restrictive laws regarding abortion deny women the ability to control their reproduction.
    \item[] Denying women the right to control their reproduction violates their right to control their body.
    \item[] Violating their right to control their body is always wrong.
    \item[] Therefore, restrictive laws regarding abortions are wrong.
\end{earg}\\
\end{tabular}

Pro-choicers (I made that term up) have never adequately come to grips with the conceptual issues surrounding abortion. As a result, their arguments miss the mark when they try to attack the pro-life side. Their arguments avoid the fight rather than engaging in it. You can think of it as the pro-lifers are in a castle, and the pro-choicers are attacking it, but their catapults always miss.  Most, if not all, of the arguments which they give in favor of legal abortions fail to refute or even weaken the traditional pro-life argument. This is that a fetus is a human being and, therefore, abortion is murder.

The pro-choice arguments tend to fall into two different categories. First, the arguments use consequentialist/utilitarian style reasoning to show that having abortions be illegal is wrong. This is exemplified by Argument A. For example, these arguments point to the terrible results of having the restrictive laws. These include things like the deaths caused by unsafe abortions, the fact that they unfairly result in harder hardship on poorer women, the fact that the lack of access results in emotional hardship, and so on. But, the pro-life side has an easy reply to this. They can claim that the tragic side effects don't, by themselves, show that the laws aren't justified (they can still be justified even with the results). This is were we can get these argument: 

\begin{tabular}{p{2in}|p{2in}}
The Pro-Life Response to Argument A:&The Pro-Life Response to Argument B:\\
\begin{earg}
    \item[1 ] A fetus is a being worthy of our moral consideration, same as you or me.
    \item[2 ] If a fetus is a being worthy of our moral consideration, then abortion is murder.
    \item[3 ] Murder is always wrong (regardless of the consequences).
    \item[4 ] So, from (1) and (2), abortion is murder.
    \item[5 ] Therefore, from (3) and (4), abortion is always wrong (regardless of the consequences). 
\end{earg}&
\begin{earg}
    \item[1 ] Violating a person's rights is always wrong.
    \item[2 ] A person's rights extend so far as they do not violate another's stronger right.
    \item[3 ] A person's right to life is stronger than another person's bodily rights.
    \item[4 ] Having an abortion (a claimed case of bodily right) is violating a fetus' right to life (as a fetus is a person).
    \item[5 ] Therefore, having an abortion is always wrong.
\end{earg}
\end{tabular}
The Pro-Life Response to Argument A , in their eyes, takes out the reasoning for the second line of Argument A. This is basically showing that there are some cases where the moral status of some behavior is not determined by the consequences. This can be supported by the very definition of murder, which is wrongful intentional killing. This kind of argument falls into the non-consequentialist style thinking and the basis there is that something are wrong regardless of the consequences. This will show up several times, the pro-life side of the debate tends to give reasons rooted in non-consequentialist style thinking. There is another pro-choice reply to this, which debates the idea that the abortion is murder in this case, further deepening the divide between the intuitions (because the reasonable response is more strongly consequentialist) 

The second argument given by the pro-choicers is exemplified by Argument B. This one uses a more non-consequentialist/Kantian style reasoning, pointing out that denying a woman the ability is have an abortion is to deprive her of some manner of bodily right. For example, the right to choose when and how one bears young. This one also falls short. The Pro-Life Response to Argument B shows where this falls short. They are basically saying there that while a person has rights, those rights can't imply that it's OK to violate another's rights. For example, take property rights. It seems clear that a person has the right to remove another from their property and has the right to defend their property, but how far does that right extend? Take this example, which really did happen, but I have exaggerated for our purposes:

\factoidbox{A man owns a large piece of property, land, and there's a group of young hooligans who ride their motor bikes on the trails through his land. So, one day, he put up a line of piano wire across one of the trails, at a particular height. As one of the young people rides through the trail, the wire, tight, catches them on the neck and decapitates them.}

This was wrong of him, or so many have argued, and the pro-lifers can use the sort of reasoning here to support themselves. In putting up the piano wire and killing the hooligan, the man violated that person's right to life. This right is stronger than the property rights which the man would have otherwise had, meaning that in this case, he did not have the right to protect his property in this way.  From this line of thought, the pro-lifer can say that, because a person's right to life is stronger than a person's bodily rights, abortion is still wrong.  Since we have these strong competing intuitions and because the pro-lifer, the serious ones at least, won't give an inch for the consequentialist considerations, we are going to need to approach the abortion debate from the non-consequentialist perspective and show that the system, in fact, allows for abortion.
\subsection{The Fundamental Question}

The most basic question which we need to answer is not about the results of having legal abortions but rather about what it takes to be a person. This is the Fetus Question. The pro-life responses to the commonplace pro-choice arguments show that they take a fetus to be a person. So, there are two routes we can take. First, we could show that there are some cases such that, even if we assume that a fetus is a person, abortion is still permissible (this is the first part of the pro-choice response). Second, we could show that a fetus is not a person, thereby making abortion permissible (before a certain point). So, for that second half, we need to show what features it takes to be a person. If a fetus has the features, then it is a person and abortion is murder, if it does not, then it is not. Warren's case is that there are certain cases where it's permissible to have an abortion, even if a fetus is a person, and then she moves on to show that a fetus is not a person, which entails that abortion is permissible.

\chapter{Part 24: The Abortion Debate (Pro-Choice)}
\section{Assume That a Fetus is a Person}
As I mentioned at the end of the previous page, Warren has two parts to her paper. First, she will assume, just for the sake of argument, that a fetus is a person, and then from that show that if a fetus is a person, there are cases where killing it is permissible. The second section is where she shows that a fetus is not a person, so it's not entitled to the same moral rights as you or me, which means that abortion is permissible. This page is the start of that first section. If there's a way to get that at least some abortions are OK in this case, then you can't have the all out ban on them which is some times proposed. Rather, morally speaking, you would need to have some exceptions. It is worth noting, and we will return to this point, that the morality of an abortion, proved in this section, assuming that a fetus is a person, is limited to a very select range of cases. This range of cases is defined, roughly, by how much consent the woman had in the actions resulting in pregnancy.  This is where Warren, through Judith Thomson, gets the Violinist Thought Experiment:\autocite{Thomson1}

    \factoidbox{Imagine that you have been kidnapped, and your bloodstream hooked up to that of the violinist, who happens to have an ailment that will certainly kill him unless he is permitted to share your kidneys for a period of nine months. You are a human dialysis machine. No one else can save him, since you alone have the right type of blood. He will be unconscious all that time, and you will have to stay in bed with him, but after the nine months are over he may be unplugged, completely cured, that is provided that you have cooperated. The violinist themselves had no knowledge that this would happen to them.}

A common point stated about this thought experiment is that it says that you are stuck in bed for the 9-months. While this is not true in most cases of pregnancy (as in the woman can move around), there are plenty of cases of pregnancy where this is the case (as in they are stuck in bed for most of it), especially if the woman is quite small (my mom, for example, is 4' 10'') and the father is quite large (my father, for example, is 6' 2'').  But, moving on,  if the person on the pro-life side of this debate is consistent in their beliefs, if they don't have any contradictions in their reasoning, then they will need to say that you would need to go to term and be there for the full 9-months.  Despite you being forced into the situation, you will need to keep the violinist alive. They come to this from the following reasoning, which is much like their response to Argument B in the previous part:

\begin{tabular}{p{2in}|p{2in}}
The Pro-Life Response to Argument B:&The Pro-Life Case to Stay Plugged In:\\
\begin{earg}
    \item[0 ] (hidden line) A fetus is a person.
    \item[1 ] Violating a person's rights is always wrong.
    \item[2 ] A person's rights extend so far as they do not violate another's stronger right.
    \item[3 ] A person's right to life is stronger than another person's bodily rights.
    \item[4 ] Having an abortion (a claimed case of bodily right) is violating a fetus' right to life (as a fetus is a person).
    \item[5 ] Therefore, having an abortion is always wrong.
\end{earg}&\begin{earg}
    \item[0 ] (hidden line) A violinist is a person.
    \item[1 ] Violating a person's rights is always wrong.
    \item[2 ] A person's rights extend so far as they do not violate another's stronger right.
    \item[3 ] A person's right to life is stronger than another person's bodily rights.
    \item[4 ] Unplugging from the violinist is violating the violinist's right to life.
    \item[5 ] Therefore, unplugging from the violinist is always wrong.
\end{earg}\end{tabular}
This, as we have seen before, is very strongly non-consequentialist style thinking. But, the vast majority of people would think that it's outrageous to think that there's the moral obligation here to keep the violinist alive. The claim here, roughly, boils down to the idea that the rights of another can't force a person to go above and beyond, take extreme measures, to ensure it. The violinist case shows that there are some cases where a person's bodily rights are stronger than a person's right to life. If there are cases like this for pregnancy, then the pro-lifer, morally, can't hold their position absolutely. It's really good of a person to agree to take on such a sacrifice, especially it was thrust upon them like this, but it seems wrong to say that your refusal is murder. Though he certainly has the right to life, something about this case must be off, removing the obligation. His right to life, in this case, does not force you, morally speaking, to to keep him alive by what ever means necessary; nor does it justify anyone else forcing you to do so. A law that required you to stay in bed with the violinist would clearly be an unjust law, since it is no proper function of the law to force unwilling people to make huge sacrifice for the sake of other people toward whom they have no such prior obligation. The key feature, for the Violinist case, is that you did not give informed consent to be plugged into the violinist. 
\subsection{What does this case get us? }

Well, it does get us something to get started on. There are a few similarities and differences between this case and pregnancies. First, we have a person (assuming that a fetus is a person) who is dependent on another for survival. In both cases, if the aware party does certain actions, then the other will die. The other aspect is that the dependency causes a drain on the aware party. In this case, there's a sense in which the other does not have a moral obligation to keep them alive. But, what removed that obligation? Some would say, as Warren does, that the key feature which negates the obligation is the kidnapping aspect. The person did not knowingly and voluntarily enter into this arrangement. They did not consent to taking on the risk.  If a woman does not enter into this arrangement willingly or without knowing the risks (without informed consent), then her situation is sufficiently similar to the violinist's case.

\subsection{The Results of the Violinist Case and the Problems}
The Violinist Thought-Experiment is initially quite plausible. It gives us a grounding to have that we aren't always obligated to keep people alive by any means necessary. If there are cases where a pregnancy is sufficiently like this case, then we can get the permissibility of abortion in those cases. But, for cases where they aren't relevantly similar to the Violinist case, we don't get the windy-side of morality, necessarily. The only real cases which are sufficiently like the violinist case are cases of pregnancy due to rape. In those cases, the woman did not voluntarily take on the risks. But, there could be some vagueness on how much give the Violinist Case gives us. If we extend it too far from the bounds of the Violinist case, we could run into wrinkles. For example, take this case: 

    \factoidbox{A woman, who is 7-months pregnant, finds herself unable to travel to Europe because of the pregnancy. She really does not want to postpone the trip. But, if she has an abortion, then the trip will go off without a hitch. Is it permissible for her to have an abortion?}

There seems to be a relevant difference between this case and the violinist case. For this one, people will often make a few different claims. First, some will claim that the woman is too far into the game to quit now, saying that the time for the abortion has passed. Others might claim that the trip to Europe is not a good enough reason to want an abortion, the case just doesn't make her bodily right strong enough. And others still will say that (assuming this is not a case of rape) that she entered into this knowing the risk and has the obligation.

In the case of pregnancies not caused by rape, there are other things which the woman could have done to prevent her In other cases, there are some things which the woman could have done. These are Warren's examples, so if they aren't correct or in some way off, be mad at her.  First, she could have remained chaste. In other words, she could have denied her partner the relations. If the partner acted anyway, this would be rape and fall into the violinist case. Remember, informed consent is absolutely key. Her second option, if she chooses to have sex, is to have taken her pills more faithfully. I know from the life experiences of friends, family, and former students, that this is hardly a 100\% sure-fire way to prevent pregnancy as it's often believed to be.  Personally, I am in favor of the development of the male-birth control, as it's better to take the bullets out of the gun than put on a bullet proof vest. The third option, if all else fails, is for her to abstain on dangerous days. But this option, also, is not reliable as some might claim. I encourage all people to research these options, but only get your research from non-religious, non-abstinence only, scientific sources.  

\subsubsection{The Pro-Life Response}

Consequently, there is room for the antiabortionist to argue that in the normal case of unwanted pregnancy a woman has, by her own actions, assumed responsibility of the fetus.
\factoidbox{If x behaves in a way which he could have avoided, and which he knows involves a 1\% chance of bringing into existence a human being, with a right to life, and does so knowing that if this should happen then that human being will perish unless x does certain things to keep him alive, then, when it does happen, x is not free of any obligation to what he knew in advance would he required to keep that human being alive.}

 To make this into a case, something which we can imagine and use for the analogy, I have created this thought-experiment, based on the Violinist Case, Violinist Cult Thought Experiment:

    \factoidbox{Suppose that you are a member of a cult along with 99 other people. I know that cult has a negative stigma to it, but bear with me. All of you have voluntarily and with full reasonable consent, entered into a lottery where one of you will be chosen at random to take on the role of being this violinist's human dialysis machine. Imagine that your name is drawn and you have the violinist hooked up. What's your obligation like now?}

Most people, from my experience, say that in this case, there is the obligation to keep the violinist alive. You signed up knowing the risks and you lost the lottery, so to speak. So, what's the difference between the violinist case and the violinist cult case? The first seems to remove the obligation, but the second seems to have it. 

\subsubsection{Restricting the Outcome}

The plausibility of such an argument is enough to show that the Violinist analogy can provide a clear and persuasive defense of a woman’s right to obtain an abortion only when the woman is in no way responsible for her pregnancy. In all other cases, we would almost certainly conclude that it was necessary to look carefully at the particular circumstances in order to determine the extent of the woman’s responsibility and hence the extent of her obligation.

\section{Prove That a Fetus is Not a Person}
\begin{center}Is a fetus a person?\end{center}

As I have mentioned before, the second section of this paper concerns whether or not a fetus is a person. The point of the previous section was to show that there are some cases where abortion is OK, even if a fetus is a person. The point here is to show that a fetus is not a person, which means that abortion isn't murder, and therefore is not wrong. This is where the Fetus Question comes in very strong, this is why I also noted that we need to distinguish between 'person' and 'human'. Questions regarding personhood are metaphysical questions, does a thing have certain features? Similarly, questions regarding 'human-hood' are metaphysical questions. From this, as I have mentioned, we are able to move from an ethical question to a metaphysical one. So, let's look at the standard, non-religious, argument from the pro-life side and another argument, which doesn't look similar, but I will explain how these relate:

\begin{tabular}{p{2in}|p{2in}}
The Standard Pro-Life Argument&The Cheese Sandwich Fallacy\\
\begin{earg}
    \item[] It is wrong to kill innocent human beings
    \item[] Fetuses are innocent human beings
    \item[] Therefore, it is wrong to kill fetuses
\end{earg}&
\begin{earg}
    \item[] Nothing is better than God.
    \item[] A cheese sandwich is better than nothing.
    \item[] Therefore, a cheese sandwich is better than God. 
\end{earg}
\end{tabular}
These two arguments might look completely different, but both of them fall into the same logical fallacy, equivocation. Remember, I mentioned that there's a distinction between 'human' and 'person'. With that in mind, it becomes clear that there's something wrong with the Standard Pro-Life Argument. Glossing over that distinction leads to the equivocation. The Cheese Sandwich Fallacy is a great example of this fallacy. An equivocation is where a person uses the same word in two different ways in an argument, this is meant to mislead the reader.  The Standard Pro-Life Argument has the same error as this one. Now, we aren't the Dark Brotherhood, so the equivocation is not in the word 'innocent'. Rather, the phrase "human being" is being used in two different ways.

In the sentence "fetuses are innocent human beings", the term 'human being' is being used to talk about a member of our species, and up until this point, I have been very consistent in the use of the term 'human' to talk about members of our species, things with the same sort of genetic make up as us. If we use this species interpretation of human, which has a long and solid history, we see that this is correct, fetuses are humans.

The other time we encounter this term is in the first line "it's wrong to kill innocent human beings." If we use the case we use the genetic, or species, interpretation of the phrase 'human being', we can quickly find cases where this is wrong, even by the non-consequentialist's lights. For example, if a brain dead human has a living will saying that they should pull the plug after a few days. But, if this was said without any context, we would likely accept it, so what makes it different? Well, the use of 'human being' in this sentence, normally, means, in the most generous interpretation, where the line makes sense,  “a full-fledged member of the moral community.” We may call this the moral sense of “human” and I have been using the term ‘person’ to demark this.\footnote{There's a certain idiosyncratic grammar which I have the habit of using in the case of the word 'person'. In English, there are, it would seem, two acceptable plurals, 'people' and 'persons'. How I use them, 'people' refers to a collection of persons and 'persons' refers to beings with person-hood individually. 'People' is a collective or group sense of the plural and 'persons' is a more individualistic sense.} In general, even when we are dealing with certain interesting legal cases, this distinction between 'human' and 'person' is overlooked. For example, a person has rights, but a human may or may not have rights. Though I don't like this example personally, but corporate person-hood is an example of this. We have a non-human entity, a corporation, seen as a person. Now, I would argue that 'corporations are people' is a legal fiction, it's not actual. But, it's certainly possible, as we will see later, that there are actual, non-fictitious, non-human persons. Having a clear distinction between these in our everyday speech makes many thorny moral questions disappear, or, at the very least, makes them more intelligible. If we remove the equivocations in both of the arguments we get the following:

\begin{tabular}{p{2in}|p{2in}}
The Standard Pro-Life Argument&The Cheese Sandwich Fallacy\\
\begin{earg}
    \item[] It is wrong to kill innocent persons
    \item[] Fetuses are innocent human beings
    \item[] Therefore, it is wrong to kill fetuses
\end{earg}&
\begin{earg}
    \item[] No existing thing is better than God.
    \item[] A cheese sandwich is better than not having anything at all.
    \item[] Therefore, a cheese sandwich is better than God. 
\end{earg}
\end{tabular}

As you can see, this doesn’t work. But, if we have that all humans are persons, as in that being a generic human is enough to be a person, the argument would work. This would be to say that all humans are persons. On the other hand, however, if there are cases where a human is not a person, then the argument fails. If we can show that all fetuses (before a certain point of development) and not persons, then we have that abortion is permissible (before a certain point in development), by the pro-life style reasoning. Showing that no fetus (before that point) is a person is Warren's next step.  

\section{Warren's Criteria for Personhood}
Warren argues that there are 5 features which are needed to be a person, and if a fetus has these features (after a certain point), then abortion would be murder (after that point). These features are listed here:

\begin{enumerate}
\item Consciousness 	
\item Reasoning
\item Self-Motivated Activity 
\item Communication
\item Self-Awareness
\end{enumerate}

\subsection{Feature 1: Consciousness}

The first of these features seem to be the most intuitive. This is consciousness. We have touched on this before, in the Mind-Body Problem module. Although there is much debate about some of the features of consciousness, we do have a general understanding of when it's had and when it's not. For example, does it react to external stimulus? Is there some behavioral or other kind of evidence that shows that this thing is thinking, has an internal life? 

\subsubsection{Is this thing conscious?}

There are a few tests to tell whether a being is conscious. The relevant one here awareness (as in reaction to external stimulus) and evidence of internal thoughts. Many other tests include the self-awareness, which is not necessarily the same thing, but that one is another feature. We will limited this test to merely something like "does it feel pain?". If it reacts, it's conscious.

\subsubsection{Are we the only conscious things?}

There are several creatures which have consciousness (by this definition) aside from humans. It is worth noting also that not all humans are conscious. Consciousness in humans is only there when the human is developed beyond a certain point and without certain impairments. In fact, most animals do have this and so do most fish. It's possible for some plants to even have this, though that is easily debated against.

\subsubsection{Why does this matter?}

If Warren is correct, several beings, including some humans, are immediately excluded from person-hood. Plants are (more than likely) taken out of the moral community, severely disabled humans, clams, and some animals. Even with this feature alone, it could be argued that fetuses aren't persons (before a certain point, we will see this later). One way to think about this is that you take the set of all things out there, and then slowly add criteria to whittle the total down to just persons. It is important to realize that even in international law, 'human' is not the same as 'person'.  Framing the question in this way moves it out of ethics and into metaphysics. So, "is abortion permissible?" is an ethical question, while "is a fetus a person?" is a metaphysical one. The answer to the second gives us the answer to the first.

\subsection{Feature 2: Reasoning}

The second feature which seems to be necessary for person-hood according to Warren is reasoning. This is the developed capacity to solve new and relatively complex problems. This, too, is not found in all humans. Most people might think that this is a necessary part of consciousness, so it should not have its own section. But this is not quite true. Consciousness, as we are using it, is having a 'what-it's-like'-ness. Having sensations. A being can feel pleasure and pain without having the ability to reason. It also should be noted that we are worried about the mental capacity to solve the problems, not the physical ability. Infant humans, if they are not disabled to certain degrees, have the mental ability to solve the problems, but not the physical ability (strength, dexterity, so on).

\subsubsection{Does this thing have reasoning?}

There are some basic tests to tell whether a being has reasoning. These tests are likely going to be more involved than merely trying to tell whether the thing can feel pain/pleasure. The task is to give the creature a puzzle and see whether it can solve the puzzle. For example, in the case of a raven, put some food just out of its reach and see whether it can come up with a way to get the food.  

\subsubsection{Are we the only reasoning creatures?}

Just like with consciousness, we also find this capacity in many non-human animals (chimpanzees are an easy example, same with dolphins). We also find this capacity in some fish (octopuses are a great example). In general, if the being can figure something out, or at least shows that it's thinking about a problem in a more abstract way, then we can say that it has reasoning. But, it's also true that some humans lack this feature. There are some which are severely mentally handicapped, those in the later stages of dementia, and so on. These humans are certainly human but they are not, according to Warren, persons.

\subsubsection{Why does this matter?}

As before, if reasoning is an essential part to being a person, we can further whittle down our list of potential persons. As before, humans developed to a certain point have reasoning, but before that we don't. So, fetuses don't count here if they are prior to a certain stage of development. Also, many non-human animals do stay in the list. For example, we have more complexly intelligent animals and some fish (octopuses, cetaceans, new world monkeys, apes, chimpanzees, bears, otters and so on, basically, if you can train them, they have this), but plants are now certainly out. 

\subsection{Feature 3: Self-Motivated Activity}

The third feature is a bit more restrictive. Self-motivated activity is closely tied with reasoning and consciousness. This is activity which is relatively independent of either genetic or direct external control. Some may think that this requires some kind of libertarian free will, which seems to require some kind of soul. Warren, however, gets around this worry by adding in that it's independent of 'direct' external control. This allows for indirect external control to have a part in it. To see the difference, direct external control would be a case where a mad scientist puts a microchip in your brain and controls you with a remote. This would not make you free, your actions would not come from you at all. On the other hand, indirect external control would be the sort of thing which you, more than likely, are experiencing right now. You have been heavily influenced by the past and your experiences, your choices are dictated by those, they are not necessarily instinctual. Both the hard determinists and the compatibilists would be fine with saying that actions can be independent of direct external control, but both will claim that they are indirectly controlled, by the past and the laws of nature. The hard determinist, however, would not be cool with the idea of morality, however, they would say that (although it's there) it doesn't count.

\subsubsection{Is this thing 'free'?}

In general, the tests which we apply to figure out whether a being has reasoning will apply here. We could call those tests a 'two for one'. When it comes to the tests for reasoning, we are asking whether they can solve puzzles, and to even engage with a puzzle to solve it, without really strange external factors being included (such as, being a remote-controlled robot), requires you to have self-motivated activity.

There are potential ways for isolating this feature and testing only it. For example, we would need to make a scenario where the creature (human, animal, robot) is denied external motivation for acting, it would have no instictual reason to act. If the creature still engages in the activity, then it would be self motivated. For example, a spy-camera in your house watching your pet. If we see that the pet acts without direct external interaction, then we could say that the action is self-motivated. 


\subsubsection{Are we the only 'free' things?}

Some, like Descartes and Kant, will claim that humans are the only creatures which can be free, and even some humans (non-person humans) lack this. This mostly stems from their Libertarian Free Will intuitions. But, if we limit the scope of self-motivation to something within the range of a compatibilist, then we have that there's no reason to think we are the only ones with it. In the case of Descartes, as we will see later, he claimed that animals lacked a soul, so (as a consequence, though not the one he was shooting for), non-human animals can't have self-motivated activity. 

As with the previous, we see again that we aren't the only beings which count as having self-motivated activity. Much of the same beings from the reasoning section remain, but this is mostly because I don't know of a way to tell that a being has reasoning without getting that it has self-motivated activity.

We also have that some humans lack self-motivated activity. This list is much the same as before. Fetuses, yet again, lack this feature prior to a certain point in development.

\subsubsection{Why does this matter?}

Intuitively, it seems clear that for a being to have moral rights and be in the moral community (be a person), they would need to be able to act. Reasoning and consciousness can only get you so far. We also need assurance that these beings are acting freely. Without this, it just does not seem to have the kind of weight needed. If we say that a being is a person, then they must have moral responsibility, to at least some degree, which gets us, by definition, self-motivated activity.  

Similarly to the previous two, there are some humans which lack this ability. For example, the severely disabled or humans prior to a certain point in development. But there are certain other, non-human, animals which have this feature. In fact, it could be argued that most animals have this and fish. As before, also, some great examples are chimpanzees, dolphins, various new world monkeys, and octopuses. As with the previous two, fetuses, prior to a certain point, lack these. 

\subsection{Feature 4: Communication}

This fourth feature is where person-hood becomes far more restrictive for Warren. Communication is the ability to express messages, by whatever means (not just vocal, but signing counts, and so do other methods), messages of various types and with a large variation of contents. 

Warren goes, I think, a little too far in her definition of communication; claiming "by whatever means, messages of an indefinite variety of types, that is, not just with an indefinite number of possible contents, but on indefinitely many possible topics." This seems very extreme to me, and I doubt that I even qualify here. But, limiting it to as I have said above helps. This criterion, as she phrases it, would mean that I would need to be able to talk about anything and  be able to do so in indefinite number of ways, which I think is just not possible, my brain is just not that big.
\subsubsection{Can this thing communicate?}

The real discrepancy here, and the reason Warren makes such and extreme standard for her communication, is that we don't want it to be just simple messages. The messages need to be more complicated than, for example, the chemical trails which ants leave or the dancing movements of bees. Rather these messages need to convey complicated information and they need to be able to convey it in several different ways.  The real point is to raise the bar on how smart the being needs to be to make the cut for personhood. The tests here are going to be a little more relative to the kind of creature which we are testing. For example, with some particularly primitive human languages, the messages might not be able to be expressed in several different ways, but the speakers can learn the different ways (though the older members will have more difficulty). The basic test would be to watch the beings interact with each other and notice the kinds of messages they are able to understand and convey to each other. Is there a grammar? Is the communication structure learned or instinctual? Can they understand abstract concepts?   
\subsubsection{Are we the only talkers?}

Despite what my friends who study linguistics might think (claiming that humans are the only language-users), when it comes to communication, we are far from the only ones. Animal communication is very wide-spread, with, I would argue, the complexity necessary for personhood. For example, dolphins have communication, and this is not instinctual but learned, we have even figured out some words in at least one of the variety of languages spoken.\autocite{tedtalksdirector2013} Chimpanzees have this capacity, but it does not seem to have one naturally arising. Some new world monkeys certainly have this capacity and have a naturally arising and learned languages (my personal favorite example is the cotton-top tamarin)\autocite{tededucation2014}. It may be also the case that octopuses are in this category.  If we treat this as a standard for intelligence, then individuals with the metal capacity to communicate, but not the physical ability, would qualify. That being said, some humans lack the mental capacity to communicate, not just the physical ability. This can be due to a variety of reasons.  And, as it applies to the relevant topic, fetuses before a certain point, lack this capacity all together.
\subsubsection{Why does this matter?}

In general, the ability to communicate our thoughts and intentions is the biggest sign of intelligence but it's also a bench-mark for degrees of intelligence. My father, who is mono-lingual, has often claimed that speaking multiple languages is a sign that the person is really smart, encouraging me to learn several (which I have). Though I have met very smart mono-lingual people and not-so-smart multilingual people, the ability to express oneself is a fairly intuitive standard. Similarly, if a being which had this mental capacity and later lost it, we often, in the real world, hold them to a different moral standard. For example, a person with sever mental disabilities is not held as responsible for their actions as a person without those disabilities, even if they behave the same way (in this case). And, it seems that they should not be held to those standards. This criterion limits the scope of person-hood yet again. Sure fish can reason, but they certainly aren't smart enough to be persons. Similarly, cats and dogs are in the same boat. So, the examples I have been giving thus far remain as potential non-human persons, but some are excluded. Yet again, fetuses are not included in this list. 
\subsection{Feature 5: Self-Awareness}

This is the fifth and final feature of person-hood for Warren. This is self-awareness. This one, one could think, should be earlier than communication, as there are many creatures which have this but lack communication. To be self-aware one must have self-concepts, understand that they are a different being from others. You are an individual, not a 'hive-mind'. You can think about you, what you want.
\subsubsection{Is this thing self-aware?}

This is where I give the tests which one can use to tell whether a being is self-aware. In the sciences, the commonly used test is to show whether the being can recognize themselves in a mirror, know that the reflection is not another being, then they have this self-awareness.  But this method is flawed in several ways. There are some creatures who clearly have self-concepts but lack interest in mirrors. In the previous feature, I gave cotton-top tamarins as being talkers, and there's some solid evidence that the nature of their communication does require self-concepts, but they have failed the mirror test (when tested at least once, but there may have been something wrong with the testing perimeters). For example, when I was paying for my community college, o so long ago, I was working in Dementia Care. In trying to put an elderly woman to bed, she saw herself in a mirror. She called out to the reflection, asking  it to leave her room and consistently looking back and trying to get it to leave. Eventually, I had to cover up the mirror. But, it's clear that, though her dementia was extreme enough for her not to have the same moral privileges as you or I, she still had self-concepts.  So, though the mirror test works in most cases, we need to be careful about calling it a definitive test. A more holistic testing model is appropriate, so that we can weed out false positives and negatives.  
\subsubsection{Are we the only ones self-aware?}

It is certainly not the case that we are the only ones self-aware. Other creatures certainly are too, even the ones which can't communicate. For example, various cetaceans (whales and dolphins), primates, and some other creatures, such as some new world monkeys.  Some octopuses have failed to show that they are self-aware in a way which I am willing to accept, namely in the presence of a mirror, they behaved as if there was another octopus present. But the jury is out on this for more intelligent species of octopus. In general, if a being is able to communicate to the degree necessary for person-hood, then it's going to have this feature, but not the other way around. It's worth noting that not all humans are self-aware. We have the extreme cases of humans in irreversible comas or humans born without certain portions of the brain. Fetuses, yet again, lack this feature. 
\subsubsection{Why does it matter?}

As with the other standards, this raises the bar on what it takes to be a full-fledged person. Lacking the presence of self-concepts does not entail the sort of moral duties to them which we would place on ourselves for beings with these concepts. Persons are individuals, we have duties to them as individuals. If something can't identify as an individual, then we don't have the same duties to it. 

\subsection{An Interesting Tangent}

As we have seen through our analysis of these features which make a being a person, there are some humans which are not persons. Warren herself does not go down this rabbit hole, but using her requirements for person-hood gives us a very interesting tangent:
\begin{center}Is it possible for a non-human to be a person?\end{center}

This question is very interesting for the discussions of Animal Rights. To have rights, to have aspects which others have a moral duty to ensure, you certainly need to be a person. If you are a person, then you have rights. Though it is possible for something to have rights and not be a person in the fullest sense, but those rights would be limited. Applying Warren's 5 criteria to other creatures in the world, we see that there are non-fictional, real, actual, non-human persons on the planet right now. Now, I am not an Area 51 conspiracy theories, saying that their are aliens in the base. But, looking at these, we have some fascinating arguments to show that humans aren't the only persons. Proving and establishing legally that these beings are non-human persons will give animal rights activists a strong argument and a more powerful footing in making their case. As persons, these creatures will need to be given the same moral consideration as you or me. 
Below, I will give the 5 (five) categories of creatures where at least some of them have person-hood. For some, the category will have more than one example in them. The first are the least contentious and the last are the most contentious. 
\subsubsection{Primates and (some) New-World Monkeys}

Of course, humans are primates, but we are human, so not exactly non-human persons. But we do find the capacities for the aspects of person-hood in the non-human members. To start, they clearly react to external stimulation, so there's something going on upstairs, that's the first box ticked. Second, they can solve reasonably complex puzzles and even can make tools, so that's a second aspect met. Third, through their ability to solve puzzles and makes choices, we can see that their actions are not merely instinctual, they do have a moral compass, so to speak. Fourth, though they don't have a naturally arising language, they can be taught it and will use it even when not prompted to speak with each other if they know that the other will understand. Also, once the primate understands a language, it will teach its children the skill, with them even, sometimes, inventing new words. For more information on this, check out Washoe the Chimp. And fifth, primates, by and large, do have the ability to recognize themselves in the mirror and they do have self-directed thoughts and awareness. All of these features make it so that at least some primates, not just humans, are persons and, from that reasoning, deserve the same moral consideration as we would a human with the same mental capacity. 

When it comes to new-world monkeys, we have a very interesting case, these are the cotton top tamarins. These little guys are quite amazing. Not only do they have all of the features of person-hood, like the primates, but they have an added aspect. They have their own, learned, naturally arising language. Their language was not taught to them by us. They react to external stuff, and so forth. The only area which requires proving is whether they have self-concepts, but their language aspects seem to show that they would. 
\subsubsection{Cetaceans}

These are your dolphins and some whales. At present, some countries have recognized these creatures as non-human persons and have granted them various rights, even though they did not follow the same kind of reasoning given here. Dolphins, in particular, do have all of these features. First off, they react to external stimuli, which is going to be true for all animals worth mentioning. Second, they can solve problems and learn from each other. For example, one pod of dolphins independently learned to use a sponge to root the sea floor and others have learned this behavior from them. Third, their behaviors are not always instinctual and are independent of direct external control. For example, they play and will even engage in behaviors contrary to what we would think they would instinctively. Fourth, they do have communication, and a rather sophisticated one. The language is learned by the children, meaning that different pods might not understand another and we have even learned aspects of it able to communicate with them. Similarly, they are able to learn more than one language. Dolphins which we have trained will learn the whistle patterns for various tricks (like a dog), but will also mimic them to attempt at communication with other dolphins and even understand that the whistles in different orders will mean different things. The fifth aspect is self-awareness. This can be seen in both how they will pass the mirror test and also how they have naming customs for their young and how they introduce themselves. These make some dolphins non-human persons and worthy of our protection.   
\subsubsection{Extraterrestrials and Advanced AI}

Both of these are at the bottom merely because their existence is controversial. No, I do not think that such beings are currently on the planet, rather they are worthy of mention because of their possibility. The section regarding the Mind-Body Problem shows us that proving person-hood for an AI will be tough, it will need to have Strong AI, if they are possible at all. However, a Strong AI would obviously be able to speak, have consciousness, self-awareness, self-motivated activity, and reasoning to the same degree as human persons, if they are possible. Not recognizing that an AI machine has reached the level of person-hood is the base-line for pretty much every robot-uprising Sci-Fi. 

Extraterrestrials will likely have an easier time proving their person-hood than the machines. This is because they will have likely came from a process much like the one humans did. But, it would not surprise me at all if, on the day the first contact is made, there's a group out there who think that they aren't persons, not worthy of our consideration, because they have built into it the idea that person-hood is exclusively human. This, it would seem, would be mistaken. 
\chapter{I Was Once a Fetus by Alexander Pruss}\autocite{Pruss1}

\section{INTRODUCTION}
 
I am going to give an argument showing that abortion is wrong in exactly 
the same circumstances in which it is wrong to kill an adult.  To argue 
further that abortion is always wrong would require showing that it is 
always wrong to kill an adult or that the circumstances in which it is not 
wrong–say, capital punishment–never befall a fetus.  Such an argument 
will be beyond the scope of this paper, but since it is uncontroversial that 
it is wrong to kill an adult human being for the sorts of reasons for which 
most abortions are performed, it follows that most abortions are wrong. 
The argument has three parts, of decreasing difficulty.  The most 
difficult will be the first part where I will argue that I was once a fetus 
and  before  that  I  was  an  embryo.    This  argument  will  rest  on  simple 
considerations  of  the  metaphysics  of  identity.    The  next  part  of  the 
argument will be to show that it would have been at least as wrong to 
have killed me before I was born as it would be to kill me now.  I will 
argue for this in more than one way, but the guiding intuition is clear: if 
you kill me earlier, the victim is the same but the harm is greater since I 
am  deprived  of  more  the  earlier  I  die.    Finally,  the  easiest  part  of  the 
argument will be that I am not relevantly different from anybody else and 
the fetus that I was was not relevantly different from any other human 
fetus, and so the argument applies equally well to all fetuses. 
The advantage of this argument over others is that it avoids talking 
of personhood, except in one of the several independent arguments in 
part two. 
 
 
\section{1. I WAS ONCE A FETUS} 
The first part seems innocuous.  After all, is it not biologically evident 
that first I was an embryo, then a fetus, then a neonate, then an infant, 
then a toddler, then a child, then an adolescent, and then an adult?  Does 
not my mother talk of the time when she was “pregnant with me” and 
thereby  imply  that  it  was  I  who  was  in  her  womb  when  she  was 
pregnant?    Is  not  the  sonogram  of  my  daughter  the  sonogram  of  that 
daughter of mine who will be born?  Evident as it might be that I was 
once a fetus and given how clear it will be that abortion is wrong if I was 
once a fetus, it is obvious, however, that the opponent will have to focus 
his attack on this part of the argument.  So more needs to be said. 

About thirty years ago, nine months before I was born, a conception 
occurred.  A sperm from my father fertilized an ovum from my mother.  
Within  twenty-four  hours,  or  sooner,  a  new  organism  came    into 
existence, an organism that was neither a part of my mother nor of my 
father.  For one, this organism was genetically distinct from both.  For 
another,  this  organism’s  functioning  was  directed  towards  its  own 
benefit–selfishly, the organism colonized the womb, released hormones 
that trigger changes in the woman beneficial to the organism, and so on.  
It certainly did not behave like a body part of either my mother or my 
father.  Moreover, it clearly was not a part of my father–it need no longer 
have had any interaction with him.  But it could not really be a part of my 
mother since the genetic contribution from my father was equal to that 
from  the  mother,  so  it  was  either  a  part  of  both  or  of  neither.    Thus, 
indeed, it was not a part of either.  Besides, we can see that in the earliest 
days  of  this  organism  before  implantation,  the  organism  floated  free, 
independently seeking nutrition in my mother’s womb.  This organism 
certainly was not a part of my mother. 

Hence, we have on the scene a new individual organism, one that 
did not exist before.  Let’s give this organism a name: call it Bob.  If we 
have a camera and look at what was happening in the womb in which 
Bob is living, we will see an embryo developing, cells differentiating, a 
fetus forming, growing, and finally a birth.  If we keep watching, we see 
a neonate, then an infant, then a toddler, then a child, then an adolescent 
and then an adult.  It’s all a continuous history.  But recall what I am out 
to prove.  I am out to prove that I was once a fetus and indeed an embryo 
against an opponent who will not grant this.  My opponent will thus have 
to deny that I and Bob are one and the same entity.  He will have to say 
that “Bob” and “Alex” name two different entities, rather than being two 
names for one and the same entity at different stages of its life. 

In any case, we have on the scene Bob the embryo.  And then all 
this development happens.  I now need a simple metaphysical principle.  
If an organism that once existed has never died, then this organism still 
exists.    I  will  not  argue  for  this  principle.    Someone  who  thinks  that 
something  can  exist  at  time  A  and  not  exist  at  a  later  time  B,  without 
having ceased to exist in between, is beyond the reach of argument.  The 
crucial  question  now  is:  Has  Bob  the  embryo  ever  died?    This  is  a 
question to which the biologists can tell us the answer.  Bob’s cells have 
divided,  differentiated,  and  Bob  has  developed.    But  nowhere  in  the 
continuous history just described have we seen anything we could iden-
tify as “the death of Bob.”  In fact, the whole process is the very opposite 
of the process of death: we have a process of growth.  That embryo that 
was conceived nine months before my birth never died.  True, it ceased 
to be an embryo, and at the end of the nine months it ceased to be a fetus. 
 But this is no more a literal death than my passing from childhood to 
adolescence or from adolescence to adulthood was. 

Indeed, if Bob died, we would be mystified as to when he died.  All 
we have in its life history is a process of growth and development.  Now, 
it is true that not all deaths are alike–not all deaths involve an evident 
destruction.  For instance, some philosophers think that the right way to 
describe an amoeba’s splitting is to say that the original amoeba dies and 
from its ashes there arise two new amoebae.  Likewise, some 
philosophers  think  that  when  two  entities  merge  into  a  single  unified 
entity, the original entities perish and a new one is formed.  That in fact 
may be how we should understand the process of conception: the egg and 
sperm perish, and a new thing results.  But again, for as long as Bob has 
existed,  he  has  always,  in  fact,  been  a  single  unified  organism,  and 
nothing like that happened in Bob’s life history–Bob never split in two 
and never merged with anything else so as to lose its own identity.  If I 
were an identical twin, matters would be slightly different as an argument 
could then be made that the pre-twinning embryo has indeed perished 
when it split in two.  But that’s not what happened to Bob.  It is clear that 
Bob has not died in the prosaic way of having his organic functioning 
disrupted,  and  hasn’t  even  died  in  these  two  more  outré  ways  that 
philosophers discuss.  

Furthermore,  the  very  continuity  in  Bob’s  development  speaks 
against  the  hypothesis  that  he  died.    When  did  that  momentous  event 
happen?  When did Bob cease to exist?  Could there have been some 
moment  in  Bob’s  growth  where  one  millisecond  Bob was alive and a 
millisecond later Bob was no longer around?  Surely not. 

Therefore, it is sufficiently established that Bob, that embryo who 
came into existence nine months before my birth, has never died.  But by 
my metaphysical principle, if he has never died, he is still alive.  Where, 
then, is Bob?  But surely there is no mystery there.  Every part of Bob–
other  than  the  cells  in  the  placenta  and  the  umbilical  cord  that  were 
shedi–developed continuously into a part of me, and every part of me has 
developed ultimately out of a part of Bob.  It is thus quite futile to look 
for Bob outside of me.  If Bob is anywhere, he is right here, where I am.  
It may be true that most of the original cells in Bob are no longer around, 
but  that  does  not  stop  the  survival  of  an  organism:  organisms  replace 
their cells regularly and do not perish thereby. 

Now, Bob can’t be a mere part of my body, because all of my body 
has continuously come from Bob’s body.  Therefore, one can’t set aside 
some  special  part  of  my  body  and  say  “that  part  of  me  is  Bob.”    So, 
where is Bob?  The answer is simple: Here.  I am Bob.  That embryo has 
grown to be a fetus, then to be a neonate, then an infant, then a child, 
then an adolescent and finally an adult.  Bob is I and I am Bob.  This was 
what I was trying to establish. 

But this is a little too quick.  I just said, vaguely, that Bob is here, 
and concluded that Bob is I.  We need the following argument.  Here 
where I stand there is only one large animal–Alexander Pruss.  Bob is 
presumably right here–there is nowhere else for him to be.  Bob has been 
growing for much of his life, and so Bob is also a large animal.  The only 
large  animal  here  is  Alexander  Pruss,  and  hence  Bob  and  Alexander 
Pruss are one and the same animal.  I, thus, am Bob.  If Bob is here, and 
if no part of me is a large animal, and if Bob is a large animal, Bob and I 
must be one and the same entity.   
 
Besides, given how organic development works, it is easy to see that 
every  organ  of  mine  is  an  organ  of  Bob’s  since  Bob’s  organs  have 
developed into being my organs, and yet without any transplant 
happening.  Thus, I and Bob are organisms having all of our organs in 
common.    But  the  only  way  that  can  be  is  if  I  and  Bob  are  the  same 
organism, i.e., I am Bob.  “Bob” and “Alex” are just different names for 
one and the same being: Alexander Robert Pruss. 

There is only one way of countering this argument, and this is to 
deny that I am an animal, that I am an organism.  This response seems 
absurd on the face of it, and it is right that we should see it as absurd.  I 
am a rational animal.  But there are three seemingly plausible ways of 
making this objection work.  They are not the only ones, but they will be 
representative. 

The  first  form  that  this  objection  can  take  is  Cartesian  dualism.  
Souls and bodies are separate substances.  What I really am is a soul, a 
spiritual  substance.  The body is simply a tool that my soul owns and 
uses, much as I might use a hammer.  My body is an organism, indeed an 
animal, but I am not myself an organism or animal.  Thus, what Bob is is 
my  body:  an  animal  that  I  own.    This  dualistic  view  has  various 
paradoxical consequences.  My wife has never kissed me–she has only 
kissed Bob, my body.  You cannot touch me–you can only touch Bob.  
Likewise, rape is then a mere property crime. Making philosophical sense 
of the meaning of sexuality is a lost cause: two persons’ having sexual 
intercourse is nothing but the intercourse between the animals owned by 
each of the persons.  My body is simply my property, and so stealing one 
of my kidneys is a mere property crime–it is not stealing a part of me.  
These consequences are ethically unacceptable.  After all, the 
government can morally take away some of my property for the greater 
good  and  does  so  in  taxes.    If  my  body  were  mere  property,  then  the 
government would in principle have a right, when necessary, to extract a 
kidney  from  me  as  a  tax  payment.    Finally,  if  this  is  right,  then  the 
traditional rallying cry of abortion supporters that “it’s my body” is no 
different in principle from the silly argument that I can do whatever I like 
in my house because my house is my property. 

There is too much absurdity there, and so this Cartesian view fails.  

But  even  if  it  did  not  fail,  it  could  only  be  used  by  the  proponent  of 
abortion if he had good reason to deny that the soul substance was united 
with the embryo from conception–otherwise, the safer thing is to refrain 
from killing what might be I.  But since the soul substance is 
unobservable, no such grounds are possible, apart from revelation-based 
religious  arguments,  and  those  should  not  be  brought  into  a  secular 
societal context. 

The arguments against the Cartesian view are not arguments against 
the existence of a soul.  The Cartesian view that the soul is a separate 
substance, distinct from the body, is not the only view of the soul.  The 
Aristotelian or Thomistic view is that the soul is that which makes an 
organism to be the organism it is and to develop as it does.  Thus, the 
soul is not something over and beyond the organism–it constitutes the 
organism  as  what  it  is,  and  what  we  are  are  organisms,  organisms 
constituted by our souls.  Thus, as soon as there is a unitary organism, 
there  is  a  soul.    (Admittedly,  Aristotle  and  Thomas  believed  that  the 
conceptus  did  not  have  the same kind of soul that I do–but they were 
theorizing  in  the  absence  of  empirical  evidence  about  the  conceptus 
being an animal that continuously grows and develops into me, or else 
they were going against what they should have said by their own lights.)  
The Cartesian view is rather unpopular these days in secular circles. 
 But there is a secular version of it, that replaces body-soul duality with 
body-brain  duality:  I  am  not  my  body  and  I  am  not  an animal–I am a 
brain.  This kind of a view will not help the abortion supporter all that 
much since the brain develops relatively early in pregnancy—around six 
weeks after conception.  But in fact the most trenchant objections against 
the  “I  am  a  soul”  view  can  be  made  against  the  “I  am  a  brain”  view.  
Only in the course of brain surgery can my wife kiss me if I am a brain.  
Rape, still, is only a property crime.  My kidneys are not parts of me but 
mere  property,  and  hence  can  be  expropriated  by  the  government  if 
necessary. 

And there is a further objection.  My brain developed out of earlier 
cells guided by the genetic information already present in the embryo. 
There was, first, a neural tube, and earlier there were precursors to that.  
Brain development was gradual, cells specializing more and more and 
arranging themselves.  At which point did I come to exist?  And why 
should the cells that were the precursors of the brain cells not be counted 
as having been the same organ as the brain, albeit in inchoate form?  If 
so, then perhaps I was there from conception, even on this view. 

The third response to my argument is that I am not my body or my 
brain, but what I am is my body’s intellectual functioning.  This response 
requires a metaphysical answer.  On this view, I do not think.  Rather, I 
am nothing else than thought itself, or more precisely, I am nothing else 
than a process of thinking.  We would do well to reject this view just 
because it contradicts the commonsensical fact that we think.  But we can 
also reject this view for a deeper reason.  If I am a particular process of 
thought, then it follows that if that process of thought were not to have 
occurred, I would not have existed.  Thus, when asleep, I do not exist.  
Moreover, were I not to have engaged in the processes of thought that I 
have engaged in over my lifetime, but instead were I to have engaged in 
different processes of thought, then I would not have existed–there would 
then have been a different process of thought, and hence someone else, if 
what I am is the process of thought that I am.  It follows that we cannot 
think otherwise than we do because our very identity is defined by the 
process of thought we engage in.  This fatalism, this deprivation of free 
will, is unacceptable. 

As I said, there are views of who I am that compete with the view 
that I am an animal and that are not the same as these three, but they tend 
to be variations of these three.  For instance, some think that what I am is 
a whole made up of two parts, a Cartesian soul and a body-animal.  This 
view  is  open  to  the  simple  objection  that  two  interacting  parts  do  not 
automatically make for a whole.  Moreover, there is the objection that 
surely I think, and yet my soul thinks, and since I am not a part of me, it 
follows absurdly that there are two thinkers here: I and my soul. 

We see thus that I am Bob.  I was once an embryo and a fetus.  The 
embryo or fetus that was there was just I–in an earlier stage of my life.  
This completes the first and hardest step of the argument. 

An  objection.    In  the  first  two  weeks  or  so  after  conception,  the 
blastocyst was not an individual, and hence in particular is not the same 
individual as I am, because it was capable of twinning–of splitting into 
two or more individuals–which in fact it does in about once every 260 
cases.  While what is normally called “abortion” is not likely to be done 
at  this  time  since  the  woman  at  this  time  rarely  knows  herself  to  be 
pregnant,  nonetheless  there  are  abortifacients  that  act  this  early–for 
instance  the  IUD,  Emergency  Birth  Control  or  the  Pill  in  those  cases 
where these act through an abortifacient effect–and hence the question is 
not merely of theoretical interest. 

This objection rests on the false principle that if it is merely possible 
that an organism will split in the future, then we do not have a genuine 
individual on the scene.  But this is plainly false: amoebae are certainly 
individuals,  but  they  are  capable  of  splitting.    What  happens  to  the 
individuality  when  they  split  is  disputed  by  philosophers.    One  might 
hold that the old amoeba continues existing as one of the two new ones, 
but  we  simply  do  not  know  which  one.    Or  one  might  hold,  more 
plausibly, that the old perishes and a new one comes to be in its place.  In 
the latter case, if I had had an identical twin, then I would have come to 
exist about two weeks after conception, not at conception, and the human 
being who came to exist at conception would no longer be alive. 

But if we have an amoeba  in front of us for a period of time during 
which it does not split, then it is the same amoeba, the same organism, 
over all of this time.  This judgment is unaffected even should we learn 
that the amoeba could have split during this period of time, just as our 
judgment that someone is alive is unaffected by learning that she could 
have died yesterday.  As long as the amoeba does not in fact split, it is 
one and the same individual as we had on the scene earlier. 

One might argue that if one could know in the first two weeks that 
twinning  was  going  to  occur,  then  one  would  thereby  know  that  the 
conceived  embryo  would  cease  to  exist  at  two  weeks  of  age,  and  one 
could abort it earlier, since one would not be depriving it of a long and 
meaningful  life.    Whether  this  argument  is  correct  or  not–and  I  am 
inclined to think it is not, since I think how good the life that one is being 
deprived of should not affect whether it is wrong for someone deprive 
one  of  it–it  does  not  matter  in  practice.    We  just  cannot  tell  at  the 
moment.  And as in 259 out of 260 cases twinning will not occur, one 
needs to act on the presumption that it will not in fact occur. 

\section{2. IF I WAS A FETUS, IT WOULD HAVE BEEN WRONG TO KILL THAT FETUS}

There  are  several  paths  to  the  conclusion  of  the  second  part  of  the 
argument, that if I was once a fetus (or an embryo for that matter), then it 
would  have  been  wrong  to  kill  that  fetus,  under  exactly  the  same 
circumstances under which it would be wrong to kill me now. 

The most powerful argument is to look at what is wrong with killing 
me  now.    Killing  me  now  is  a  paradigmatic  crime-with-a-victim,  the 
victim being me.  What would make killing me now wrong is the harm it 
would do to me: it would deprive me, who am juridically innocent, of 
life, indeed of the rest of my life.  Now, consider the hypothetical killing 
of the fetus that I once was.  This killing would have exactly the same 
victim as killing me now would.  Moreover, the harm inflicted on the 
victim  would  have  been  strictly  greater,  in  the  sense  that  any  harm 
inflicted on me by killing me now would likewise have been inflicted on 
me by killing me when I was a child.  I am now 29 years old.  Suppose 
that left to nature’s resources, I would die at 65.  Then, killing me now 
would deprive me of years 29 through 65 of my life.  However, killing 
me when I was a fetus would also deprive me of years 29 through 65 of 
my  life–as  well  as  the  years  from  the  moment  of  the killing up to 29.  
Given that murder is a crime whose wrongness comes from the harm to 
the  victim,  it  is  clear  that  when  the  victim  is  the  same,  and  the  harm 
greater, killing is if anything more wrong. 

Of course, there may be circumstances in which it is acceptable to 
kill  me  now.    It  might  be  that  under  some  circumstances  capital 
punishment is justified.  If so, then it might be acceptable to kill the fetus 
under  the  same  circumstances.    However,  it  is  also  clear  that  the 
circumstances involved in capital punishment do not apply in the case of 
the fetus.  Whether there are any other circumstances in which it would 
be acceptable to kill me now is a question that is beyond the scope of this 
paper, although I believe that the answer is basically negative.ii  In any 
case, we see that the wrongfulness of killing me when I was a fetus is at 
least as great as the wrongfulness of killing me now in relevantly similar 
circumstances.  Thus, my moral status when I was a fetus with respect to 
being killed is the same, or more favorable to me than, my status now. 

The reason for the “more favorable than now” option is that we have
an intuition that it is particularly wrong to kill people earlier.  Although 
there may be no duty thus to sacrifice one’s life, we see nothing irrational 
in an older person sacrificing his life for a younger on the grounds that 
the older has literally less to lose by death.  When I was a fetus, I had 
more to lose by death than I do now.  Thus, to have killed me then would, 
strictly speaking, have been to inflict a greater harm. 

Observe that nothing is said here about whether I was a person when 
I was a fetus.  That issue is irrelevant.  Whether I was a person then or 
not,  killing  me  would  have  had  the  same  victim  and  involved  greater 
harm as killing me now.  Observe that if I was not a person when I was a 
fetus, then the harm in killing me then would have been even greater than 
if I was a person then.  For killing me when I was not a person would 
thus have deprived me of all of my personhood as lived out on earth, and 
this radical deprivation would have been a greater crime than killing me 
now which would not deprive me of ever having had a personhood lived 
out on earth. 

That  said,  an  independent  argument  shows  that  in  fact  I  was  a 
person when I was a fetus.  This gives a second argument for why killing 
a fetus is wrong, and it is the only argument I give that depends on issues 
of  personhood.    The argument turns on the metaphysical notion of an 
“essential  property.”    The  essential  property  of  a  being  is  a  property 
which that being cannot lack as long as that being exists.  For instance, 
many philosophers think that being a horse is an essential property of a 
horse.    If  you  take  a  horse  like  Silver  Blaze  and  modify  it  to  such  a 
degree that it is no longer a horse, Silver Blaze will cease to exist and 
something  else  will  come  to  exist  in  his  place.    Being  material  is  an 
essential property of a rock: it could not exist without being material. 
Now,  it  is  likewise  plausible  that  being  a  person  is  an  essential 
property of every person.  If someone were a person and if personhood 
were removed from her, she would cease to exist.  If this is correct, then 
the fetus that I was truly was a person since I am a person.  If the fetus 
that I was were not a person, then it would be the case that I could have 
existed without being a person–which is impossible. 

Even  more  plausibly,  it  is  an  essential  property  of  me  to  have  a 
property that I will call human dignity.  Human dignity is a property of
me that makes it wrong for another human being to set out to kill meiii 
when I am juridically innocent.  As before, I leave capital punishment as 
an open question.  Human dignity is an essential property: it is part of the 
essence of who I am. Were I to lack this intrinsic dignity, I would not be 
myself; I would not exist.  But if human dignity understood in this way is 
an essential property and I have it, then the fetus that I was also had it–
otherwise it wouldn’t be an essential property. 

Finally, there is a very different argument for the wrongfulness of 
killing  the fetus that I was, based on John Rawls’s concept of justice.  
Even though I take this concept to be incorrect, the more bases on which 
our argument can rest, the better the argument. Rawls bids us to imagine 
that we do not know which role in society we fill–imagining this is called 
entering under the “veil of ignorance.”  What kind of a society I would I 
come  up  with,  and  what  kinds  of  rules  would  I  rationally  devise  on 
selfish grounds, if I did not know which role in this society I am going to 
live in?  Rawls says that that kind of society is the just society, and its 
rules are the rules of justice.  In such a society, for instance, we would 
forbid racism because under the veil of ignorance we would not know 
whether we would end up having the role of victim or inflicter of racism, 
and we would not want to take the risk of being on the victim.  Likewise, 
we would prohibit the murder of adults. 

Would we forbid the killing of fetuses?  This question depends on 
just how much we are to be ignorant of under the veil of ignorance.  If we 
know  that  we  are  not  fetuses,  then  we  might  not  forbid  the  killing  of 
fetuses when it is convenient to non-fetuses because we would have no 
selfish  reason  to  prohibit  it.    So,  is  the  fact  of  us  not  being  fetuses 
something that is under the veil of ignorance or not?  Well, we must be 
careful  not to take too much out from under the veil.  For instance, if 
racism is to end up being deemed unjust, our race must lie under the veil. 
Moreover,  even  our  being  conscious  must  fall  under  the  veil–thereby 
showing how much the veil is just a figure of speech since we cannot 
really be ignorant of our consciousness.  The reason our being conscious 
must fall under the veil is that otherwise we might well enact that it is 
right to kill the unconscious for the sake of the conscious–to use the man 
in a coma for medical experiments, say.  But at the same time, we cannot
put  too  much  under  the  veil.    We  had  better  have  an  awareness  of 
ourselves  as  human  since  otherwise  our  “just  society”  will  end  up 
prohibiting  all  killing  of  animals,  and  this  would  make  even  most 
vegetarian  farming  wrong  because  of  the  moles  and  voles  and  other 
animals killed in the process of farming, as someone has once argued. 

So,  where  do  we  draw  the  line?    I  would  propose  this  simple 
criterion.  Under the veil, we are aware of which social roles it would be 
logically possible for us to fill, but not aware which of those roles we do 
in fact fill.  It would not be logically possible for me to fill the role of a 
mole in the ground–I would not be myself then.  So I know, even under 
the veil, that I am not a mole.  However, it plainly is logically possible for 
me to fill the role of a fetus–it is possible because I did fill the role of a 
fetus once!   Thus, whether I am a fetus or not is something that must fall 
under the veil of ignorance, and hence the killing of fetuses will end up 
being  prohibited  in  exactly  the  same  way  as  that  of  adults:  we  just 
wouldn’t want to take the risk that we might end up being a fetus that is 
being killed.  Hence, justice requires a prohibition on killing fetuses in 
exactly the way in which it requires a prohibition on killing adults. 

Later,  Rawls  modified  his  criterion  by  talking  of  an  unselfish 
caretaker  for  someone  making  the  decision  under  a  veil  of  ignorance 
about what role her charge would fill.  This takes care of the problem that 
we can hardly be ignorant of whether we are conscious, while a caretaker 
can be ignorant of whether her charge is conscious.  But it does not affect 
the  rest  of  the  argument.    The  caretaker  needs  to be ignorant of some 
properties of her charge, such as the charge’s profession in life, but not of 
others, such as that her charge is not an insect.  Again, I would suggest 
that  a  natural  way  to  draw  the  line  is  apt  to  make  the  caretaker  be 
ignorant of whether her charge is a fetus or not.  For at the very least the 
caretaker should be ignorant as to which of the roles her charge could fill 
she  in  fact  fills,  and  certainly  her  charge  could  fill the role of a fetus.  
And if that were so, the caretaker, truly loving whoever is entrusted into 
her care, would not want to take the risk of enacting a system whereby 
her charge could be killed. 

\section{3. IF IT WAS WRONG TO KILL ME WHEN I WAS A FETUS, IT WAS WRONG TO KILL ANYONE WHEN HE IS A FETUS} 
If you cut me, do I bleed any more than the next guy?  No.  I was not and 
am  not  special.    If  it  was  wrong  to  kill  me  when  I was a fetus, it was 
likewise wrong to kill anyone else when he was a fetus.  

It might be argued that there are some special differences between 
the fetus that I was, which we have seen it would have been wrong to 
  kill  me  when  I was a fetus, it was 
likewise wrong to kill anyone else when he was a fetus.  
It might be argued that there are some special differences between 
the fetus that I was, which we have seen it would have been wrong to 
kill, and some other fetuses.  For instance, I was wanted.  But that I was 
wanted  did  not  anywhere  enter  into  my  arguments  against  killing  me 
when I was a fetus.  It is wrong to kill me now no matter whether I am 
wanted  by  others  or  not.    Killing  me  earlier,  I  have  argued,  is  not 
significantly different from killing me now, and so whether I was wanted 
or not is irrelevant. 

A  different  objection  would  be  that,  as  far  as  I  know,  I  did  not 
endanger my mother’s life.  However, my arguments would continue to 
apply even if I did: the fetus needs to be protected at least to the extent to 
which we would protect an adult under relevantly similar circumstances.  
If  the  fetus  endangers  the  mother’s  life,  it  does  so    unintentionally.  
Whether  it  is  acceptable  to  kill  the  fetus  under  those  circumstances 
depends  on  whether  it  would  be  acceptable  to  kill  me  now  were  I  to 
endanger my mother’s life unintentionally. As I announced, the aim of 
this paper is limited: it is to argue that killing fetuses is wrong under the 
same circumstances under which it is wrong to kill adults, but it is not the 
paper of the paper to discuss the circumstances, if any, under which it is 
permissible to kill adults.  I think it would not be acceptable to kill me 
were I endangering my mother’s life unintentionally: I will simply say in 
support  of  this  that  were  I  alone  in  a  space  capsule,  three  days  from 
rescue, with my mother, with only enough air for 1.5 days each, it would 
not be acceptable for my mother or her agent to kill me. 

A yet different objection is: I was a healthy fetus, but some others 
are not.  The wrong in killing me when I was a fetus would have been 
depriving me of a meaningful and long future life.  But what if the fetus 
cannot be expected to have such a life?  Again, I respond that the purpose 
of this paper is limited: I am not going to settle issues of euthanasia here. 
 It is acceptable to kill such a fetus only if it is acceptable to kill an adult 
 who  cannot  be  expected  to  have  a  meaningful  and  long  future  life.  
Again, I think it is not acceptable to kill an adult under such 
circumstances.    Human  life  is  intrinsically  worthwhile  and  always 
meaningful.  But this is not a paper about euthanasia.  If I have shown 
that  the  fetus  is  worthy  of  at  least  the  same  respect  as  an  adult  in 
comparable circumstances, I have done my task.
\chapter{Part 25: The Abortion Debate (Pro-Life)}
For this module, thus far, we have been looking at one of the best arguments which can be made on the pro-choice side. This argument goes after the core beliefs of the pro-life side. Namely, that if a fetus is a person, then abortion is murder (and thereby wrong) and that a fetus is in fact a person. The argument does not rely on consequentialist reasoning nor on a woman's bodily rights.

For the sake of fairness, I feel that it's appropriate for us to also look at one of the best arguments from the pro-life side.  This is a purely optional reading as you can complete the assignments without reading it. However, it may serve you well if you ever encounter puzzles like this or if you are on the pro-life side of the debate, this paper will give you a far stronger argument for your stance. The paper is called I Was Once a Fetus: That is Why Abortion is Wrong by Alexander Pruss. His argument does not rely on religious beliefs nor on any notions of person-hood (per se). The argument goes like this:
\begin{earg}
    \item[1 ]I was once a fetus.
    \item[2 ]If I was once a fetus, it would have been wrong to kill that fetus.
    \item[3 ]If it was wrong to kill me when I was a fetus (that fetus), then it would be wrong to kill anyone while they were a fetus.
    \item[4 ]Therefore, it is wrong to kill anyone while they are a fetus.
\end{earg}
The conclusion, the fourth line, makes all cases of abortion morally wrong. It may seem a bit strong, but that is what is argued. We will go through the paper point by point.
\section{I was once a fetus}
This is the first line of his argument and it is the most seemingly obvious of the lines. Pruss starts off by asking us a few rhetorical questions (avoid them in philosophy papers, they just lead to confusion if they are not well crafted and obvious). The point of them is to show that at some point in the far past 'I' was an embryo, then a fetus, then a neonate, then an infant, then a toddler, a child, an adolescent, and then, finally, an adult. There are several ways in which a person can show that they were once a fetus. Pruss does this in an interesting way, one which I would not have thought to apply to this debate. he uses what we in the philosophy biz call a 'continuum', or an indiscrete series. To make these philosophically interesting, you need to have it such that something clearly holds on one side of the spectrum and clearly doesn't hold on the other. These are often called Sorites Paradoxes. 'Sorites' is Greek for 'heap' or 'pile'. These sort of cases are where we get the stereotypical philosophical question "how many grains of sand does it take to make a heap?" So, imagine that you have a pile of sand, a large pile, something which is clearly a heap of sand. If you take 1 singular grain of sand off of the top, is it still a heap? Clearly and obviously it is. Now, I remove another, and another, and another. All the while, 1 grain of sand doesn't make a difference, all the way down until I have only 1 grain of sand left. This grain of sand is clearly not a heap, but the reasoning shows that it is. For another example, take a spectrum of colors, from blue to red. The far side is clearly blue and the other is clearly red. What if I start at the blue  end and move over, ever so slightly, is it still blue? Yes, yes it is. Now, what if I move over a little more, still blue, a little more, still blue, and so on. In the middle, we will likely find something which we would call 'purple', but, by the reasoning, it would still be blue. As we continue, the responses get seemingly more and more absurd, until we are saying that red is blue.   

Any line which we draw in this spectrum, saying that before this point, it's blue and after this point it's red or some other color would, frankly, be unnatural and arbitrary. In a third example, one which was given to me when I learned about this paper in the equivalent of 101 which I took (that course was exclusively epistemology and metaphysics, no ethics), concerns baldness. My professor, at the time, was balding, and was the example person for the entire department for this concept. On one side, you have him when he was young, full head of glorious hair. On the other side you have what he will become, a chrome-dome. Now, where is he now? Somewhere in the middle, is he bald? No, but is he 'haired'? No. There's the paradox, something is both bald and not bald. 

Now, at this point, the same sort of reasoning can be used for the developing embryos through pre-birth infants all the way to you reading this in-front of some screen. You, now, are clearly you. You 1 second ago is also clearly you. You 2 seconds ago is also clearly you.\footnote{ I am not sure about the grammar of these two sentences.} We can keep going back, and back, and back, deducting seconds off the clock, until we get to your birth. Before this point, it gets a little more tricky to imagine it. The continuous nature of the transitions are even more evident in this case. We have, now, that at your birth, that's you, and on the opposite side, at conception, we have a zygote, or some such. But this is again, a continuous spectrum. There's no hard-line, no non-arbitrary position where we could, within reason, say that you appeared and the zygote disappeared.  Without such a line, then you are that zygote. So, this first line is pretty obviously true. I was once a fetus.

\section{If I was a fetus, it would have been wrong to kill that fetus}

The last line of the argument showed that I was once a fetus. It is worth noting, and this will come up again later, that I could have used any one of you in place of myself. Now, our job is to show that if I was once a fetus (I was), then it would have been wrong to kill that fetus (me). We need to show that there's a causal connection between me (or you, or anyone) being a fetus and the wrongness of killing that fetus. There are two ways which I would go in trying to show this connection, if I were Pruss. Here's the route which he takes  (along with my connections to previous content in this course) and then there's my personal preferred route. We will start by looking at the route which Pruss takes and then move on to the route which I would take. Both involve an intermediary proof for the stance that it's wrong to kill me now. 
\subsection{Pruss' way of showing that it's wrong to kill that fetus}

If you remember way back in the course, we covered Nagel's work Death. In that work, Nagel came from the idea that death is always bad for the person who died because something of great value is lost. That thing must be of such great value that loosing it is worse than anything which can come from having it. For Nagel, this is experience. Having experience is so valuable that death is always worse than whatever experiences may come from the experiences. Connecting this into Pruss, my untimely death now would be cutting off experiences which I would have otherwise had. This means that killing me has deprived me of some good, and that makes it wrong. This is, interestingly, in line with the typical Consequentialist kind of thinking (as in the way of approaching it is how the Consequentialist would do so). This is not Utilitarian thinking, however, because experience is included as an ultimate, super, good, which Utilitarianism does not incorporate.

So, we have that it's wrong to kill me now, which is awesome, but how do we move from this to killing me as a fetus? Well, we need to look at why it's wrong. As I have likely mentioned several times now, the why is the big aspect, this is the aspect which we can use in debate, either for or against, and make headway. To reiterate, it's wrong because I was deprived of experience. Suppose that I'm 28 and I would have otherwise lived to be 65. Killing me now would deprive me of 37 years of experience. Had I been terminated as a fetus, I would have been deprived 65 years of experience. From this, because 65 is greater than 37, we can say that killing me as a fetus is at least on par with killing me now, if not worse than killing me now. Because of this, we get that if I was a fetus, then it would be wrong to kill that fetus. Putting this as a more formal argument, we have the following:
\begin{enumerate}
    \item Depriving someone of a good which they would have otherwise had is wrong. 
    \item Killing someone deprives them of a good which they would have otherwise had (namely experience). 
    \item If I was a fetus, then killing that fetus would have deprived me of a good which I would have otherwise had.
    \item Therefore, if I was a fetus, then killing that fetus would have been wrong. 
\end{enumerate}
\subsection{An Argument Which He Could Have Used}

Pruss' way of getting to the premise relies on the intuition which Nagel uses. This is not shared by many, but there are those who have it. An alternative way which Pruss could have used to get this line, but does not (for reasons which we will see later), is to reapply the Sorites style thinking to this problem. All one needs to make this reasoning work is that it's wrong to kill me now and that the morality of killing another person does not change with time (which is Non-consequentialist style thinking).  The first half, that it's wrong to kill me now, can be gotten through your preferred method, though non-consequentialist thinking will likely be preferred. To start, if it's wrong to kill me now, then it would have been wrong to kill me a second ago. This is because the morality of killing someone does not change with time. This also means that if it's wrong to kill me now, then it would have been wrong to kill me two seconds ago, and three seconds, four, and so on. This goes all the way back to when I was a fetus. This means that if I was once a fetus, then it would have been wrong to kill that fetus. Put as a more formal argument, we get the following:
\begin{enumerate}
    \item It is wrong to kill me now (at T).
    \item If it is wrong to kill me now (T), then it was wrong to kill me a second ago (T-1).
    \item If something is wrong at T-n, then it is wrong at T-(n+1).
    \item Therefore, If I was fetus, then it would have been wrong to kill that fetus.
\end{enumerate}
There is a very innocent jump between lines 3 and 4 which can be glossed over. Basically, to make that move, there are some additional, veiled, premises which concern the time I was a fetus and the identity between myself and that fetus. The main point of this argument is that there's no hard line which makes it OK to kill me before a certain point, and wrong to kill me after that point. 

\section{If it was wrong to kill me when I was a fetus, then it was wrong to kill anyone when they were a fetus}

The author's original line used the gendered pronoun which I have replaced, but the point remains the same. Proving this line in a way that makes the entire thing still valid is a bit more difficult than normal. You see, this will appear to be inductive, which is not strong enough for philosophy, but it's not. As I have mentioned a few times in this proof, I am not special. I could have put that you were once a fetus and ran the argument in the same way and I could have put any person in this argument in place of myself and gotten the same result. Since there are no cases where the validity of this argument does not transfer to another, we are safe in making the generalization to all people. Take these two arguments, which are oddly similar:

\begin{tabular}{p{1.75in}|p{1.75in}}
    I Argument&Sally Argument\\\hline
    I was once a fetus.&Sally was once a fetus.\\\hline
    If I was once a fetus, then it would have been wrong to kill me when I was a fetus.&If Sally was once a fetus, then it would have been wrong to kill Sally when Sally was a fetus.\\\hline 
    Therefore, it would have been wrong to kill me when I was a fetus.&    Therefore, it would have been wrong to kill Sally when Sally was a fetus.
 \end{tabular}

The first argument is the one which we currently have a proof for. But, if we take the reasoning and replace the references to myself with references to Sally, we get the same result. Now, what if I was to replace the references to Sally with some generic, like, I don't know, a human person. Take these for the comparison:

\begin{tabular}{p{1.75in}|p{1.75in}}
    Sally Argument&Human Person Argument\\\hline
    Sally was once a fetus.&A human person was once a fetus.\\\hline
    If Sally was once a fetus, then it would have been wrong to kill Sally when Sally was a fetus.& If a human person was once a fetus, then it would have been wrong to kill them when they were a fetus.\\\hline 
    Therefore, it would have been wrong to kill Sally when Sally was a fetus.& Therefore, it would have been wrong to kill a human person when they were a fetus. 
\end{tabular}

But, a human person is just a generic stand in, if this reasoning works (and it does, soundness is a different story), then the reasoning applies to all, real or potential, human persons. This means that I can move from the individual case, that it would have been wrong to kill me when I was a fetus, to the stance that it would have been wrong to kill anyone when they were a fetus, because the justification applies just as well to them as it does me. This gives us the final premise of the argument, namely, if it was wrong to kill me when I was a fetus, then it would have been wrong to kill anyone when they were a fetus. This last line connects everything together, and with some pretty easy reasoning, we have that killing anyone when they were a fetus is wrong, which makes abortion always wrong. 
\begin{earg}
    \item[1] I was once a fetus.
    \item[2] If I was once a fetus, then it would have been wrong to kill that fetus.
    \item[3] If it would have been wrong to kill that fetus, then it would have been wrong to kill anyone when they were a fetus. 
    \item[4] So, it would have been wrong to kill anyone when they were a fetus.
    \item[5] If it would have been wrong to kill anyone when they were a fetus, then abortion is wrong (this is implied from the definition of an abortion). 
    \item[6] Therefore, abortion is wrong.
\end{earg}
With that, we have what is likely the best pro-life argument which you will find. Your typical pro-life argument which we saw earlier relied on an equivocation between 'human' and 'person'. This one, however, relies on sorites reasoning and, maybe, some intuition about experience. This could, maybe, also be used to make a hole in Warren's reasoning as well. But, that being said, there are some objections to this argument which need to be addressed. It should be noted that Pruss' replies rely on the Nagel intuition which we have seen previously. To reply to these objections without the value of experience being so much would require a different line of thinking than the one which Pruss uses. 
\section{But I was wanted!}

This is rather cold, I admit, but there may be one important difference between fetus-me and some other fetuses, namely that I was wanted. But, you have to notice that no-where in the arguments for the impermissibility of abortion were the claims that I was wanted. It solely relies on the fact that I am a person and that I was a fetus. There are some cases where the pregnancy and child-rearing will result in undue and heavy burdens on the mother. In these cases, it seems like abortion should be available (according to some). This is not to say that those burdens will always be present in that person’s life, but rather that they would not be removed or be very difficult to remove if she went to term. This sort of reasoning falls in line with one of the popular pro-choice arguments which I have given for Warren, namely the ones which rely on the undue burdens on poor women. 
\subsection{Pruss' Reply}

It is worth noting that nowhere in the argument was there something about the want of a child, nor was there anything about the life of the child after birth. Even if I wasn’t wanted, there’s still a loss, namely the fact that I would have otherwise had experiences. A person is a person, no matter how small. Morality, often, will force us into situations which we don't want to do, which might not be in line with our self-interest. 
\section{But I didn't endanger my mommy!}

That was me making a joke of the objection a bit. The point of it is that, as far as I am aware, there were no complications with my birth/time in the womb which added risk to my mother's life, which may make a relevant difference between me and other fetuses. There are some cases where a person must choose between the life of the mother and the life of the child. There are other cases, even more extreme, where the process (not just child-birth) of going to term will kill both. Such as cases of unidentified ectopic pregnancies. These cases make up 10\% the deaths during pregnancy.
\subsection{Pruss' Reply}

But, you have to notice, that we have to chose between the life of one and another person from time to time, though it is not common. More over, in cases where a fetus is endangering the life of the mother, it does so unintentionally. In cases where it is acceptable to kill a person who is endangering another, it is intentional. So, since it is unintentional, it is not acceptable to kill the fetus. Also, if it comes to a choice between the mother and the child, if we were to apply Nagel’s thoughts, we should choose the child. Namely because the experiences had by the mother up until that point plus those which will be had by the child is greater than the amount had by the mother if she were to continue living. This, however, does not work in cases where the fetus is not able to develop and the process of development will only kill both. For example, extreme cases of ectopic pregnancies. 
\section{But I was a healthy fetus!}

This becomes a debate about euthanasia (the killing of a person painlessly when they have some extreme health problem). Killing me as a fetus would have deprived me of a long life, but killing an unhealthy fetus would not have deprived them of much. For example, Peter Singer, who I have used as examples before, is a rather extreme Utilitarian, both in his work as well as in his actions. His charity work and donation practices are top-tier. However, when Singer was directing a bioethics center in Australia (which is a universal healthcare country), he was contacted by doctors concerning their ethical dilemmas. In one case, he was contacted concerning issues in the Neonatal Intensive Care Units, ICUs for newborns.  Many of these newborns had a range of very serious medical conditions which often lead to a very slow and painful death. If the newborns actually did survive, they would need multiple extremely expensive surgeries which would drain the resources from the collective system (it's even worse in a system without universal healthcare) and they would be very disabled in various ways. In the extreme cases of these conditions, the ones which the doctors were contacting Singer about, all people concerned, the  doctors, nurses, and even the parents, believed that the babies should not survive. In these cases, all the newborns could do is experience pain. As a result, many infants were being left in the ICU untreated, with minor adjustments to alleviate the suffering where possible.  This was an very painful experience for all those concerned, on an emotional level and took up resources which would have otherwise been provided to healthy newborns. Almost all cases like these die before they are 6-months old.  Peter Singer advised that the newborns with the extreme cases of this condition be given a quick and painless death. Would it not be better for the parents to have this condition identified early and be able to have an abortion?  
\subsection{Pruss' Reply}

This becomes a debate about euthanasia (the killing of a person painlessly when they have some extreme health problem). The connection is that, for the author, killing an unhealthy fetus is the same as killing a person with a terminal illness. So, if you think that killing a person with a terminal illness is permissible, then you must also think that killing an unhealthy fetus is permissible. The author thinks that euthanasia is wrong, full stop. Again, this is a lot like Nagel's reasoning. The core principle is that no experience can make life not worth it. Because of this, every human life, for Pruss, must be worthwhile. 
\subsubsection{A Reply to this Reply}

There is an issue with this reply. Namely, if experience is what makes human life always worthwhile, what if the fetus is going to be brain-dead? For example, what if the child has a certain genetic abnormality or some other factor which results in anencephaly (Greek for no-in-head (AKA brain)). This is a real thing, the baby, if it has the main brain stem, will be alive, but will not have any experiences whatsoever. No consciousness at all. The life expectancy for a baby born with this is between a few hours and a few days. This means that the human's life would not have anything to it, there would be none of what makes life always worth it, according to Pruss. This means that he does not have a way of saying that abortion would be wrong in this case. 
